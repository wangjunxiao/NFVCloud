\begin{quote}
Finds the common standard cache directory. \end{quote}


Recently the \href{https://www.npmjs.com/package/nyc}{\tt {\ttfamily nyc}} and \href{https://www.npmjs.com/package/ava}{\tt {\ttfamily A\+VA}} projects decided to standardize on a common directory structure for storing cache information\+:


\begin{DoxyCode}
# nyc
./node\_modules/.cache/nyc

# ava
./node\_modules/.cache/ava

# your-module
./node\_modules/.cache/your-module
\end{DoxyCode}


This module makes it easy to correctly locate the cache directory according to this shared spec. If this pattern becomes ubiquitous, clearing the cache for multiple dependencies becomes easy and consistent\+:


\begin{DoxyCode}
rm -rf ./node\_modules/.cache
\end{DoxyCode}


If you decide to adopt this pattern, please file a PR adding your name to the list of adopters below.

\subsection*{Install}


\begin{DoxyCode}
$ npm install --save find-cache-dir
\end{DoxyCode}


\subsection*{Usage}


\begin{DoxyCode}
const findCacheDir = require('find-cache-dir');

findCacheDir(\{name: 'unicorns'\});
//=> /user/path/node-modules/.cache/unicorns
\end{DoxyCode}


\subsection*{A\+PI}

\subsubsection*{find\+Cache\+Dir(\mbox{[}options\mbox{]})}

Finds the cache dir using the supplied options. The algorithm tries to find a {\ttfamily package.\+json} file, searching every parent directory of the {\ttfamily cwd} specified (or implied from other options). It returns a {\ttfamily string} containing the absolute path to the cache directory, or {\ttfamily null} if {\ttfamily package.\+json} was never found.

\paragraph*{options}

\subparagraph*{name}

{\itshape Required} ~\newline
Type\+: {\ttfamily string}

This should be the same as your project name in {\ttfamily package.\+json}.

\subparagraph*{files}

Type\+: {\ttfamily array} of {\ttfamily string}

An array of files that will be searched for a common parent directory. This common parent directory will be used in lieu of the {\ttfamily cwd} option below.

\subparagraph*{cwd}

Type\+: {\ttfamily string} ~\newline
Default {\ttfamily process.\+cwd()}

The directory to start searching for a {\ttfamily package.\+json} from.

\subparagraph*{create}

Type\+: {\ttfamily boolean} ~\newline
Default {\ttfamily false}

If {\ttfamily true}, the directory will be created synchronously before returning.

\subparagraph*{thunk}

Type\+: {\ttfamily boolean} ~\newline
Default {\ttfamily false}

If {\ttfamily true}, this modifies the return type to be a function that is a thunk for {\ttfamily path.\+join(the\+Found\+Cache\+Directory)}.


\begin{DoxyCode}
const thunk = findCacheDir(\{name: 'foo', thunk: true\});

thunk();
//=> /some/path/node\_modules/.cache/foo

thunk('bar.js')
//=> /some/path/node\_modules/.cache/foo/bar.js

thunk('baz', 'quz.js')
//=> /some/path/node\_modules/.cache/foo/baz/quz.js
\end{DoxyCode}


This is helpful for actually putting actual files in the cache!

\subsection*{Adopters}


\begin{DoxyItemize}
\item \href{https://www.npmjs.com/package/nyc}{\tt {\ttfamily N\+YC}}
\item \href{https://www.npmjs.com/package/ava}{\tt {\ttfamily A\+VA}}
\end{DoxyItemize}

\subsection*{License}

M\+IT © \href{http://github.com/jamestalmage}{\tt James Talmage} 