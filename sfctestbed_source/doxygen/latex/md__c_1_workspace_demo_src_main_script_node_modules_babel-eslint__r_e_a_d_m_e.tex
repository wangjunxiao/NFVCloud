{\bfseries babel-\/eslint} allows you to lint {\bfseries A\+LL} valid Babel code with the fantastic \href{https://github.com/eslint/eslint}{\tt E\+S\+Lint}.

\subsubsection*{Why Use babel-\/eslint}

You only need to use babel-\/eslint if you are using types (Flow) or experimental features not supported in E\+S\+Lint itself yet. Otherwise try the default parser (you don\textquotesingle{}t have to use it just because you are using Babel). 



\begin{quote}
If there is an issue, first check if it can be reproduced with the regular parser or with the latest versions of {\ttfamily eslint} and {\ttfamily babel-\/eslint}! \end{quote}


For questions and support please visit the \href{https://babeljs.slack.com/messages/linting/}{\tt {\ttfamily \#linting}} babel slack channel (sign up \href{https://babel-slack.herokuapp.com}{\tt here})!

\begin{quote}
Note that the {\ttfamily ecma\+Features} config property may still be required for E\+S\+Lint to work properly with features not in E\+C\+M\+A\+Script 5 by default. Examples are {\ttfamily global\+Return} and {\ttfamily modules}). \end{quote}


\subsection*{Known Issues}

Flow\+: \begin{quote}
Check out \href{https://github.com/gajus/eslint-plugin-flowtype}{\tt eslint-\/plugin-\/flowtype}\+: An {\ttfamily eslint} plugin that makes flow type annotations global variables and marks declarations as used. Solves the problem of false positives with {\ttfamily no-\/undef} and {\ttfamily no-\/unused-\/vars}. \end{quote}

\begin{DoxyItemize}
\item {\ttfamily no-\/undef} for global flow types\+: {\ttfamily React\+Element}, {\ttfamily React\+Class} \href{https://github.com/babel/babel-eslint/issues/130#issuecomment-111215076}{\tt \#130}
\begin{DoxyItemize}
\item Workaround\+: define types as globals in {\ttfamily .eslintrc} or define types and import them `import type React\+Element from './types\textquotesingle{}{\ttfamily  -\/}no-\/unused-\/vars/no-\/undef{\ttfamily with Flow declarations (}declare module A \{\}\`{}) \href{https://github.com/babel/babel-eslint/issues/132#issuecomment-112815926}{\tt \#132}
\end{DoxyItemize}
\end{DoxyItemize}

Modules/strict mode
\begin{DoxyItemize}
\item {\ttfamily no-\/unused-\/vars\+: \mbox{[}2, \{vars\+: local\}\mbox{]}} \href{https://github.com/babel/babel-eslint/issues/136}{\tt \#136}
\end{DoxyItemize}

Please check out \href{https://github.com/yannickcr/eslint-plugin-react}{\tt eslint-\/plugin-\/react} for React/\+J\+SX issues
\begin{DoxyItemize}
\item {\ttfamily no-\/unused-\/vars} with jsx
\end{DoxyItemize}

Please check out \href{https://github.com/babel/eslint-plugin-babel}{\tt eslint-\/plugin-\/babel} for other issues

\subsection*{How does it work?}

E\+S\+Lint allows custom parsers. This is great but some of the syntax nodes that Babel supports aren\textquotesingle{}t supported by E\+S\+Lint. When using this plugin, E\+S\+Lint is monkeypatched and your code is transformed into code that E\+S\+Lint can understand. All location info such as line numbers, columns is also retained so you can track down errors with ease.

Basically {\ttfamily babel-\/eslint} exports an \href{/index.js}{\tt {\ttfamily index.\+js}} that a linter can use. It just needs to export a {\ttfamily parse} method that takes in a string of code and outputs an A\+ST.

\subsection*{Usage}

\begin{quote}
E\+S\+Lint 1.\+x $\vert$ Use $<$= 5.\+x \end{quote}


\begin{quote}
E\+S\+Lint 2.\+x $\vert$ Use $>$= 6.\+x \end{quote}


\subsubsection*{Supported E\+S\+Lint versions}

\tabulinesep=1mm
\begin{longtabu} spread 0pt [c]{*{2}{|X[-1]}|}
\hline
\rowcolor{\tableheadbgcolor}\textbf{ E\+S\+Lint  }&\textbf{ babel-\/eslint -\/-\/-\/---   }\\\cline{1-2}
\endfirsthead
\hline
\endfoot
\hline
\rowcolor{\tableheadbgcolor}\textbf{ E\+S\+Lint  }&\textbf{ babel-\/eslint -\/-\/-\/---   }\\\cline{1-2}
\endhead
3.\+x  &$>$= 6.\+x   \\\cline{1-2}
\end{longtabu}


\subsubsection*{Install}


\begin{DoxyCode}
$ npm install eslint@3.x babel-eslint@6 --save-dev
\end{DoxyCode}


\subsubsection*{Setup}

$\ast$$\ast$.eslintrc$\ast$$\ast$


\begin{DoxyCode}
\{
  "parser": "babel-eslint",
  "rules": \{
    "strict": 0
  \}
\}
\end{DoxyCode}


Check out the \href{http://eslint.org/docs/rules/}{\tt E\+S\+Lint docs} for all possible rules.

\subsubsection*{Configuration}

{\ttfamily source\+Type} can be set to `\textquotesingle{}module'{\ttfamily (default) or}\textquotesingle{}script\textquotesingle{}{\ttfamily if your code isn\textquotesingle{}t using E\+C\+M\+A\+Script modules. }allow\+Import\+Export\+Everywhere{\ttfamily can be set to true to allow import and export declarations to appear anywhere a statement is allowed if your build environment supports that. By default, import and export declarations can only appear at a program\textquotesingle{}s top level. }code\+Frame\`{} can be set to false to disable the code frame in the reporter. This is useful since some eslint formatters don\textquotesingle{}t play well with it.

$\ast$$\ast$.eslintrc$\ast$$\ast$


\begin{DoxyCode}
\{
  "parser": "babel-eslint",
  "parserOptions": \{
    "sourceType": "module",
    "allowImportExportEverywhere": false,
    "codeFrame": false
  \}
\}
\end{DoxyCode}


\subsubsection*{Run}


\begin{DoxyCode}
$ eslint your-files-here
\end{DoxyCode}
 