Runtime type checking for React props and similar objects.

You can use prop-\/types to document the intended types of properties passed to components. React (and potentially other libraries—see the check\+Prop\+Types() reference below) will check props passed to your components against those definitions, and warn in development if they don’t match.

\subsection*{Installation}


\begin{DoxyCode}
npm install --save prop-types
\end{DoxyCode}


\subsection*{Importing}


\begin{DoxyCode}
import PropTypes from 'prop-types'; // ES6
var PropTypes = require('prop-types'); // ES5 with npm
\end{DoxyCode}


If you prefer a {\ttfamily $<$script$>$} tag, you can get it from {\ttfamily window.\+Prop\+Types} global\+:


\begin{DoxyCode}
<script src="https://unpkg.com/prop-types/prop-types.js"></script>


<script src="https://unpkg.com/prop-types/prop-types.min.js"></script>
\end{DoxyCode}


\subsection*{Usage}

Prop\+Types was originally exposed as part of the React core module, and is commonly used with React components. Here is an example of using Prop\+Types with a React component, which also documents the different validators provided\+:


\begin{DoxyCode}
import React from 'react';
import PropTypes from 'prop-types';

class MyComponent extends React.Component \{
  render() \{
    // ... do things with the props
  \}
\}

MyComponent.propTypes = \{
  // You can declare that a prop is a specific JS primitive. By default, these
  // are all optional.
  optionalArray: PropTypes.array,
  optionalBool: PropTypes.bool,
  optionalFunc: PropTypes.func,
  optionalNumber: PropTypes.number,
  optionalObject: PropTypes.object,
  optionalString: PropTypes.string,
  optionalSymbol: PropTypes.symbol,

  // Anything that can be rendered: numbers, strings, elements or an array
  // (or fragment) containing these types.
  optionalNode: PropTypes.node,

  // A React element.
  optionalElement: PropTypes.element,

  // You can also declare that a prop is an instance of a class. This uses
  // JS's instanceof operator.
  optionalMessage: PropTypes.instanceOf(Message),

  // You can ensure that your prop is limited to specific values by treating
  // it as an enum.
  optionalEnum: PropTypes.oneOf(['News', 'Photos']),

  // An object that could be one of many types
  optionalUnion: PropTypes.oneOfType([
    PropTypes.string,
    PropTypes.number,
    PropTypes.instanceOf(Message)
  ]),

  // An array of a certain type
  optionalArrayOf: PropTypes.arrayOf(PropTypes.number),

  // An object with property values of a certain type
  optionalObjectOf: PropTypes.objectOf(PropTypes.number),

  // An object taking on a particular shape
  optionalObjectWithShape: PropTypes.shape(\{
    color: PropTypes.string,
    fontSize: PropTypes.number
  \}),

  // You can chain any of the above with `isRequired` to make sure a warning
  // is shown if the prop isn't provided.
  requiredFunc: PropTypes.func.isRequired,

  // A value of any data type
  requiredAny: PropTypes.any.isRequired,

  // You can also specify a custom validator. It should return an Error
  // object if the validation fails. Don't `console.warn` or throw, as this
  // won't work inside `oneOfType`.
  customProp: function(props, propName, componentName) \{
    if (!/matchme/.test(props[propName])) \{
      return new Error(
        'Invalid prop `' + propName + '` supplied to' +
        ' `' + componentName + '`. Validation failed.'
      );
    \}
  \},

  // You can also supply a custom validator to `arrayOf` and `objectOf`.
  // It should return an Error object if the validation fails. The validator
  // will be called for each key in the array or object. The first two
  // arguments of the validator are the array or object itself, and the
  // current item's key.
  customArrayProp: PropTypes.arrayOf(function(propValue, key, componentName, location, propFullName) \{
    if (!/matchme/.test(propValue[key])) \{
      return new Error(
        'Invalid prop `' + propFullName + '` supplied to' +
        ' `' + componentName + '`. Validation failed.'
      );
    \}
  \})
\};
\end{DoxyCode}


Refer to the \href{https://facebook.github.io/react/docs/typechecking-with-proptypes.html}{\tt React documentation} for more information.

\subsection*{Migrating from React.\+Prop\+Types}

Check out \href{https://facebook.github.io/react/blog/2017/04/07/react-v15.5.0.html#migrating-from-react.proptypes}{\tt Migrating from React.\+Prop\+Types} for details on how to migrate to {\ttfamily prop-\/types} from {\ttfamily React.\+Prop\+Types}.

There are also important notes below.

\subsection*{How to Depend on This Package?}

For apps, we recommend putting it in {\ttfamily dependencies} with a caret range. For example\+:


\begin{DoxyCode}
"dependencies": \{
  "prop-types": "^15.5.7" 
\}
\end{DoxyCode}


For libraries, we {\itshape also} recommend leaving it in {\ttfamily dependencies}\+:


\begin{DoxyCode}
"dependencies": \{
  "prop-types": "^15.5.7" 
\},
"peerDependencies": \{
  "react": "^15.5.0" 
\}
\end{DoxyCode}


{\bfseries Note\+:} there are known issues in versions before 15.\+5.\+7 so we recommend using it as the minimal version.

Make sure that the version range uses a caret ({\ttfamily $^\wedge$}) and thus is broad enough for npm to efficiently deduplicate packages.

For U\+MD bundles of your comoponents, make sure you {\bfseries don’t} include {\ttfamily Prop\+Types} in the build. Usually this is done by marking it as an external (the specifics depend on your bundler), just like you do with React.

\subsection*{Compatibility}

\subsubsection*{React 0.\+14}

This package is compatible with {\bfseries React 0.\+14.\+9}. Compared to 0.\+14.\+8 (which was released a year ago), there are no other changes in 0.\+14.\+9, so it should be a painless upgrade.


\begin{DoxyCode}
# ATTENTION: Only run this if you still use React 0.14!
npm install --save react@^0.14.9 react-dom@^0.14.9
\end{DoxyCode}


\subsubsection*{React 15+}

This package is compatible with {\bfseries React 15.\+3.\+0} and higher.


\begin{DoxyCode}
npm install --save react@^15.3.0 react-dom@^15.3.0
\end{DoxyCode}


\subsubsection*{What happens on other React versions?}

It outputs warnings with the message below even though the developer doesn’t do anything wrong. Unfortunately there is no solution for this other than updating React to either 15.\+3.\+0 or higher, or 0.\+14.\+9 if you’re using React 0.\+14.

\subsection*{Difference from {\ttfamily React.\+Prop\+Types}\+: Don’t Call Validator Functions}

First of all, {\bfseries which version of React are you using}? You might be seeing this message because a component library has updated to use {\ttfamily prop-\/types} package, but your version of React is incompatible with it. See the \href{#compatibility}{\tt above section} for more details.

Are you using either React 0.\+14.\+9 or a version higher than React 15.\+3.\+0? Read on.

When you migrate components to use the standalone {\ttfamily prop-\/types}, {\bfseries all validator functions will start throwing an error if you call them directly}. This makes sure that nobody relies on them in production code, and it is safe to strip their implementations to optimize the bundle size.

Code like this is still fine\+:


\begin{DoxyCode}
MyComponent.propTypes = \{
  myProp: PropTypes.bool
\};
\end{DoxyCode}


However, code like this will not work with the {\ttfamily prop-\/types} package\+:


\begin{DoxyCode}
// Will not work with `prop-types` package!
var errorOrNull = PropTypes.bool(42, 'myProp', 'MyComponent', 'prop');
\end{DoxyCode}


It will throw an error\+:


\begin{DoxyCode}
Calling PropTypes validators directly is not supported by the `prop-types` package.
Use PropTypes.checkPropTypes() to call them.
\end{DoxyCode}


(If you see {\bfseries a warning} rather than an error with this message, please check the \href{#compatibility}{\tt above section about compatibility}.)

This is new behavior, and you will only encounter it when you migrate from {\ttfamily React.\+Prop\+Types} to the {\ttfamily prop-\/types} package. For the vast majority of components, this doesn’t matter, and if you didn’t see \href{https://facebook.github.io/react/warnings/dont-call-proptypes.html}{\tt this warning} in your components, your code is safe to migrate. This is not a breaking change in React because you are only opting into this change for a component by explicitly changing your imports to use {\ttfamily prop-\/types}. If you temporarily need the old behavior, you can keep using {\ttfamily React.\+Prop\+Types} until React 16.

{\bfseries If you absolutely need to trigger the validation manually}, call {\ttfamily Prop\+Types.\+check\+Prop\+Types()}. Unlike the validators themselves, this function is safe to call in production, as it will be replaced by an empty function\+:


\begin{DoxyCode}
// Works with standalone PropTypes
PropTypes.checkPropTypes(MyComponent.propTypes, props, 'prop', 'MyComponent');
\end{DoxyCode}
 See below for more info.

{\bfseries You might also see this error} if you’re calling a {\ttfamily Prop\+Types} validator from your own custom {\ttfamily Prop\+Types} validator. In this case, the fix is to make sure that you are passing {\itshape all} of the arguments to the inner function. There is a more in-\/depth explanation of how to fix it \href{https://facebook.github.io/react/warnings/dont-call-proptypes.html#fixing-the-false-positive-in-third-party-proptypes}{\tt on this page}. Alternatively, you can temporarily keep using {\ttfamily React.\+Prop\+Types} until React 16, as it would still only warn in this case.

If you use a bundler like Browserify or Webpack, don’t forget to \href{https://facebook.github.io/react/docs/installation.html#development-and-production-versions}{\tt follow these instructions} to correctly bundle your application in development or production mode. Otherwise you’ll ship unnecessary code to your users.

\subsection*{Prop\+Types.\+check\+Prop\+Types}

React will automatically check the prop\+Types you set on the component, but if you are using Prop\+Types without React then you may want to manually call {\ttfamily Prop\+Types.\+check\+Prop\+Types}, like so\+:


\begin{DoxyCode}
const myPropTypes = \{
  name: PropTypes.string,
  age: PropTypes. number,
  // ... define your prop validations
\};

const props = \{
  name: 'hello', // is valid
  age: 'world', // not valid
\};

// Let's say your component is called 'MyComponent'

// Works with standalone PropTypes
PropTypes.checkPropTypes(myPropTypes, props, 'prop', 'MyComponent');
// This will warn as follows:
// Warning: Failed prop type: Invalid prop `age` of type `string` supplied to
// `MyComponent`, expected `number`.
\end{DoxyCode}
 