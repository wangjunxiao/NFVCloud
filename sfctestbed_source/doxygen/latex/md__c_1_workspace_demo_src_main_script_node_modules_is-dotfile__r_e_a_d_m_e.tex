\begin{quote}
Return true if a file path is (or has) a dotfile. Returns false if the path is a dot directory. \end{quote}


\subsection*{Install}

Install with \href{https://www.npmjs.com/}{\tt npm}\+:


\begin{DoxyCode}
$ npm install --save is-dotfile
\end{DoxyCode}


\subsection*{Usage}

To be considered a dotfile, it must be the last filename in the path, like {\ttfamily .gitignore}. Otherwise it\textquotesingle{}s a \href{https://github.com/jonschlinkert/is-dotdir}{\tt dot directory}, like {\ttfamily .git/} and {\ttfamily .github/}.


\begin{DoxyCode}
var isDotfile = require('is-dotfile');
\end{DoxyCode}


{\bfseries false}

All of the following return {\ttfamily false}\+:


\begin{DoxyCode}
isDotfile('a/b/c.js');
isDotfile('/.git/foo');
isDotfile('a/b/c/.git/foo');
//=> false
\end{DoxyCode}


{\bfseries true}

All of the following return {\ttfamily true}\+:


\begin{DoxyCode}
isDotfile('a/b/.gitignore');
isDotfile('.gitignore');
isDotfile('/.gitignore');
//=> true
\end{DoxyCode}


\subsection*{About}

\subsubsection*{Related projects}


\begin{DoxyItemize}
\item \href{https://www.npmjs.com/package/dotdir-regex}{\tt dotdir-\/regex}\+: Regex for matching dot-\/directories, like {\ttfamily .git/} $\vert$ \href{https://github.com/regexps/dotdir-regex}{\tt homepage}
\item \href{https://www.npmjs.com/package/dotfile-regex}{\tt dotfile-\/regex}\+: Regular expresson for matching dotfiles. $\vert$ \href{https://github.com/regexps/dotfile-regex}{\tt homepage}
\item \href{https://www.npmjs.com/package/is-dotdir}{\tt is-\/dotdir}\+: Returns true if a path is a dot-\/directory. $\vert$ \href{https://github.com/jonschlinkert/is-dotdir}{\tt homepage}
\item \href{https://www.npmjs.com/package/is-glob}{\tt is-\/glob}\+: Returns {\ttfamily true} if the given string looks like a glob pattern or an extglob pattern… \href{https://github.com/jonschlinkert/is-glob}{\tt more} $\vert$ \href{https://github.com/jonschlinkert/is-glob}{\tt homepage}
\end{DoxyItemize}

\subsubsection*{Contributing}

Pull requests and stars are always welcome. For bugs and feature requests, \href{../../issues/new}{\tt please create an issue}.

\subsubsection*{Contributors}

\tabulinesep=1mm
\begin{longtabu} spread 0pt [c]{*{2}{|X[-1]}|}
\hline
\rowcolor{\tableheadbgcolor}\multicolumn{2}{|p{(\linewidth-\tabcolsep*2-\arrayrulewidth*1)*2/2}|}{\cellcolor{\tableheadbgcolor}\textbf{ $\ast$$\ast$\+Commits$\ast$   }}\\\cline{1-2}
\endfirsthead
\hline
\endfoot
\hline
\rowcolor{\tableheadbgcolor}\multicolumn{2}{|p{(\linewidth-\tabcolsep*2-\arrayrulewidth*1)*2/2}|}{\cellcolor{\tableheadbgcolor}\textbf{ $\ast$$\ast$\+Commits$\ast$   }}\\\cline{1-2}
\endhead
13  &\href{https://github.com/jonschlinkert}{\tt jonschlinkert}   \\\cline{1-2}
1  &\href{https://github.com/Lykathia}{\tt Lykathia}   \\\cline{1-2}
\end{longtabu}


\subsubsection*{Building docs}

\+\_\+(This project\textquotesingle{}s readme.\+md is generated by \href{https://github.com/verbose/verb-generate-readme}{\tt verb}, please don\textquotesingle{}t edit the readme directly. Any changes to the readme must be made in the .verb.\+md \char`\"{}.\+verb.\+md\char`\"{} readme template.)\+\_\+

To generate the readme, run the following command\+:


\begin{DoxyCode}
$ npm install -g verbose/verb#dev verb-generate-readme && verb
\end{DoxyCode}


\subsubsection*{Running tests}

Running and reviewing unit tests is a great way to get familiarized with a library and its A\+PI. You can install dependencies and run tests with the following command\+:


\begin{DoxyCode}
$ npm install && npm test
\end{DoxyCode}


\subsubsection*{Author}

{\bfseries Jon Schlinkert}


\begin{DoxyItemize}
\item \href{https://github.com/jonschlinkert}{\tt github/jonschlinkert}
\item \href{https://twitter.com/jonschlinkert}{\tt twitter/jonschlinkert}
\end{DoxyItemize}

\subsubsection*{License}

Copyright © 2017, \href{https://github.com/jonschlinkert}{\tt Jon Schlinkert}. Released under the \mbox{[}M\+IT License\mbox{]}(L\+I\+C\+E\+N\+SE).





{\itshape This file was generated by \href{https://github.com/verbose/verb-generate-readme}{\tt verb-\/generate-\/readme}, v0.\+6.\+0, on May 30, 2017.} 