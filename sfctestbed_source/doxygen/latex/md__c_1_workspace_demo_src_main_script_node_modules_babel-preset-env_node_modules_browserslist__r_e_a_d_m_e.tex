

Library to share supported browsers list between different front-\/end tools. It is used in\+:


\begin{DoxyItemize}
\item \href{https://github.com/postcss/autoprefixer}{\tt Autoprefixer}
\item \mbox{[}babel-\/preset-\/env\mbox{]} (no config support, only tool option)
\item \href{https://github.com/amilajack/eslint-plugin-compat}{\tt eslint-\/plugin-\/compat}
\item \href{https://github.com/ismay/stylelint-no-unsupported-browser-features}{\tt stylelint-\/no-\/unsupported-\/browser-\/features}
\item \href{https://github.com/jonathantneal/postcss-normalize}{\tt postcss-\/normalize}
\end{DoxyItemize}

All tools that rely on Browserslist will find its config automatically, when you add the following to {\ttfamily package.\+json}\+:


\begin{DoxyCode}
\{
  "browserslist": [
    "> 1%",
    "last 2 versions"
  ]
\}
\end{DoxyCode}


Or in {\ttfamily .browserslistrc} config\+:


\begin{DoxyCode}
# Browsers that we support

> 1%
Last 2 versions
IE 10 # sorry
\end{DoxyCode}


Developers set browsers list in queries like {\ttfamily last 2 version} to be free from updating browser versions manually. Browserslist will use \href{http://caniuse.com/}{\tt Can i Use} data for this queries.

Browserslist will take browsers queries from tool option, {\ttfamily browserslist} config, {\ttfamily .browserslistrc} config, {\ttfamily browserslist} section in {\ttfamily package.\+json} or environment variables.

You can test Browserslist queries in \href{http://browserl.ist/}{\tt online demo}.

\href{https://evilmartians.com/?utm_source=browserslist}{\tt $<$img src=\char`\"{}https\+://evilmartians.\+com/badges/sponsored-\/by-\/evil-\/martians.\+svg\char`\"{} alt=\char`\"{}\+Sponsored by Evil Martians\char`\"{} width=\char`\"{}236\char`\"{} height=\char`\"{}54\char`\"{} $>$ }

\subsection*{Queries}

Browserslist will use browsers query from one of this sources\+:


\begin{DoxyEnumerate}
\item Tool options. For example {\ttfamily browsers} option in Autoprefixer.
\item {\ttfamily B\+R\+O\+W\+S\+E\+R\+S\+L\+I\+ST} environment variable.
\item {\ttfamily browserslist} config file in current or parent directories.
\end{DoxyEnumerate}
\begin{DoxyEnumerate}
\item {\ttfamily .browserslistrc} config file in current or parent directories.
\item {\ttfamily browserslist} key in {\ttfamily package.\+json} file in current or parent directories.
\item If the above methods did not produce a valid result Browserslist will use defaults\+: {\ttfamily $>$ 1\%, last 2 versions, Firefox E\+SR}.
\end{DoxyEnumerate}

We recommend to write queries in {\ttfamily package.\+json}.

You can specify the versions by queries (case insensitive)\+:


\begin{DoxyItemize}
\item {\ttfamily last 2 versions}\+: the last 2 versions for each browser.
\item {\ttfamily last 2 Chrome versions}\+: the last 2 versions of Chrome browser.
\item {\ttfamily $>$ 5\%} or {\ttfamily $>$= 5\%}\+: versions selected by global usage statistics.
\item {\ttfamily $>$ 5\% in US}\+: uses U\+SA usage statistics. It accepts \href{http://en.wikipedia.org/wiki/ISO_3166-1_alpha-2#Officially_assigned_code_elements}{\tt two-\/letter country code}.
\item {\ttfamily $>$ 5\% in my stats}\+: uses \href{#custom-usage-data}{\tt custom usage data}.
\item {\ttfamily ie 6-\/8}\+: selects an inclusive range of versions.
\item {\ttfamily Firefox $>$ 20}\+: versions of Firefox newer than 20.
\item {\ttfamily Firefox $>$= 20}\+: versions of Firefox newer than or equal to 20.
\item {\ttfamily Firefox $<$ 20}\+: versions of Firefox less than 20.
\item {\ttfamily Firefox $<$= 20}\+: versions of Firefox less than or equal to 20.
\item {\ttfamily Firefox E\+SR}\+: the latest \mbox{[}Firefox E\+SR\mbox{]} version.
\item {\ttfamily i\+OS 7}\+: the i\+OS browser version 7 directly.
\item {\ttfamily not ie $<$= 8}\+: exclude browsers selected before by previous queries. You can add {\ttfamily not} to any query.
\end{DoxyItemize}

Browserslist works with separated versions of browsers. You should avoid queries like {\ttfamily Firefox $>$ 0}.

Multiple criteria are combined as a boolean {\ttfamily OR}. A browser version must match at least one of the criteria to be selected.

All queries are based on the \href{http://caniuse.com/}{\tt Can I Use} support table, e.\+g. {\ttfamily last 3 i\+OS versions} might select {\ttfamily 8.\+4, 9.\+2, 9.\+3} (mixed major and minor), whereas {\ttfamily last 3 Chrome versions} might select {\ttfamily 50, 49, 48} (major only).

\subsection*{Browsers}

Names are case insensitive\+:


\begin{DoxyItemize}
\item {\ttfamily Android} for Android Web\+View.
\item {\ttfamily Black\+Berry} or {\ttfamily bb} for Blackberry browser.
\item {\ttfamily Chrome} for Google Chrome.
\item {\ttfamily Chrome\+Android} or {\ttfamily and\+\_\+chr} for Chrome for Android
\item {\ttfamily Edge} for Microsoft Edge.
\item {\ttfamily Electron} for Electron framework. It will be converted to Chrome version.
\item {\ttfamily Explorer} or {\ttfamily ie} for Internet Explorer.
\item {\ttfamily Explorer\+Mobile} or {\ttfamily ie\+\_\+mob} for Internet Explorer Mobile.
\item {\ttfamily Firefox} or {\ttfamily ff} for Mozilla Firefox.
\item {\ttfamily Firefox\+Android} or {\ttfamily and\+\_\+ff} for Firefox for Android.
\item {\ttfamily i\+OS} or {\ttfamily ios\+\_\+saf} for i\+OS Safari.
\item {\ttfamily Opera} for Opera.
\item {\ttfamily Opera\+Mini} or {\ttfamily op\+\_\+mini} for Opera Mini.
\item {\ttfamily Opera\+Mobile} or {\ttfamily op\+\_\+mob} for Opera Mobile.
\item {\ttfamily Q\+Q\+Android} or {\ttfamily and\+\_\+qq} for QQ Browser for Android.
\item {\ttfamily Safari} for desktop Safari.
\item {\ttfamily Samsung} for Samsung Internet.
\item {\ttfamily U\+C\+Android} or {\ttfamily and\+\_\+uc} for UC Browser for Android.
\end{DoxyItemize}

\subsubsection*{Electron}

\href{https://www.npmjs.com/package/electron-to-chromium}{\tt {\ttfamily electron-\/to-\/chromium}} could return a compatible Browserslist query for your (major) Electron version\+:


\begin{DoxyCode}
const e2c = require('electron-to-chromium')
autoprefixer(\{
    browsers: e2c.electronToBrowserList('1.4') //=> "Chrome >= 53"
\})
\end{DoxyCode}


\subsection*{{\ttfamily package.\+json}}

If you want to reduce config files in project root, you can specify browsers in {\ttfamily package.\+json} with {\ttfamily browserslist} key\+:


\begin{DoxyCode}
\{
  "private": true,
  "dependencies": \{
    "autoprefixer": "^6.5.4"
  \},
  "browserslist": [
    "> 1%",
    "last 2 versions"
  ]
\}
\end{DoxyCode}


\subsection*{Config File}

Browserslist config should be named {\ttfamily .browserslistrc} or {\ttfamily browserslist} and have browsers queries split by a new line. Comments starts with {\ttfamily \#} symbol\+:


\begin{DoxyCode}
# Browsers that we support

> 1%
Last 2 versions
IE 8 # sorry
\end{DoxyCode}


Browserslist will check config in every directory in {\ttfamily path}. So, if tool process {\ttfamily app/styles/main.\+css}, you can put config to root, {\ttfamily app/} or {\ttfamily app/styles}.

You can specify direct path in {\ttfamily B\+R\+O\+W\+S\+E\+R\+S\+L\+I\+S\+T\+\_\+\+C\+O\+N\+F\+IG} environment variables.

\subsection*{Environments}

You can also specify different browser queries for various environments. Browserslist will choose query according to {\ttfamily B\+R\+O\+W\+S\+E\+R\+S\+L\+I\+S\+T\+\_\+\+E\+NV} or {\ttfamily N\+O\+D\+E\+\_\+\+E\+NV} variables. If none of them is declared, Browserslist will firstly look for {\ttfamily development} queries and then use defaults.

In {\ttfamily package.\+json}\+:


\begin{DoxyCode}
\{
  …
  "browserslist": \{
    "production": [
      "last 2 version",
      "ie 9"
    ],
    "development": [
      "last 1 version"
    ]
  \}
\}
\end{DoxyCode}


In {\ttfamily .browserslistrc} config\+:


\begin{DoxyCode}
[production]
last 2 version
ie 9

[development]
last 1 version
\end{DoxyCode}


\subsection*{Environment Variables}

If some tool use Browserslist inside, you can change browsers settings by \mbox{[}environment variables\mbox{]}\+:


\begin{DoxyItemize}
\item {\ttfamily B\+R\+O\+W\+S\+E\+R\+S\+L\+I\+ST} with browsers queries.
\end{DoxyItemize}


\begin{DoxyCode}
BROWSERSLIST="> 5%" gulp css
\end{DoxyCode}



\begin{DoxyItemize}
\item {\ttfamily B\+R\+O\+W\+S\+E\+R\+S\+L\+I\+S\+T\+\_\+\+C\+O\+N\+F\+IG} with path to config file.
\end{DoxyItemize}


\begin{DoxyCode}
BROWSERSLIST\_CONFIG=./config/browserslist gulp css
\end{DoxyCode}



\begin{DoxyItemize}
\item {\ttfamily B\+R\+O\+W\+S\+E\+R\+S\+L\+I\+S\+T\+\_\+\+E\+NV} with environments string.
\end{DoxyItemize}


\begin{DoxyCode}
BROWSERSLIST\_ENV="development" gulp css
\end{DoxyCode}



\begin{DoxyItemize}
\item {\ttfamily B\+R\+O\+W\+S\+E\+R\+S\+L\+I\+S\+T\+\_\+\+S\+T\+A\+TS} with path to the custom usage data for {\ttfamily $>$ 1\% in my stats} query.
\end{DoxyItemize}


\begin{DoxyCode}
BROWSERSLIST\_STATS=./config/usage\_data.json gulp css
\end{DoxyCode}



\begin{DoxyItemize}
\item {\ttfamily B\+R\+O\+W\+S\+E\+R\+S\+L\+I\+S\+T\+\_\+\+D\+I\+S\+A\+B\+L\+E\+\_\+\+C\+A\+C\+HE} if you want to disable config reading cache.
\end{DoxyItemize}


\begin{DoxyCode}
BROWSERSLIST\_DISABLE\_CACHE=1 gulp css
\end{DoxyCode}


\subsection*{Custom Usage Data}

If you have a website, you can query against the usage statistics of your site\+:


\begin{DoxyEnumerate}
\item Import your Google Analytics data into \href{http://caniuse.com/}{\tt Can I Use}. Press {\ttfamily Import…} button in Settings page.
\item Open browser Dev\+Tools on \href{http://caniuse.com/}{\tt Can I Use} and paste this snippet into the browser console\+:

\`{}\`{}`js var e=document.\+create\+Element(\textquotesingle{}a');e.\+set\+Attribute(\textquotesingle{}href\textquotesingle{}, \textquotesingle{}data\+:text/plain;charset=utf-\/8,\textquotesingle{}+encode\+U\+R\+I\+Component(J\+S\+O\+N.\+stringify(J\+S\+O\+N.\+parse(local\+Storage\mbox{[}\textquotesingle{}usage-\/data-\/by-\/id\textquotesingle{}\mbox{]})\mbox{[}local\+Storage\mbox{[}\textquotesingle{}config-\/primary\+\_\+usage\textquotesingle{}\mbox{]}\mbox{]})));e.\+set\+Attribute(\textquotesingle{}download\textquotesingle{},\textquotesingle{}stats.\+json\textquotesingle{});document.\+body.\+append\+Child(e);e.\+click();document.\+body.\+remove\+Child(e); \`{}\`{}{\ttfamily }
\item {\ttfamily Save the data to a}browserslist-\/stats.\+json\`{} file in your project.
\end{DoxyEnumerate}

Of course, you can generate usage statistics file by any other method. File format should be like\+:


\begin{DoxyCode}
\{
  "ie": \{
    "6": 0.01,
    "7": 0.4,
    "8": 1.5
  \},
  "chrome": \{
    …
  \},
  …
\}
\end{DoxyCode}


Note that you can query against your custom usage data while also querying against global or regional data. For example, the query {\ttfamily $>$ 1\% in my stats, $>$ 5\% in US, 10\%} is permitted.

\subsection*{JS A\+PI}


\begin{DoxyCode}
var browserslist = require('browserslist');

// Your CSS/JS build tool code
var process = function (source, opts) \{
    var browsers = browserslist(opts.browsers, \{
        stats: opts.stats,
        path:  opts.file,
        env:   opts.env
    \});
    // Your code to add features for selected browsers
\}
\end{DoxyCode}


Queries can be a string {\ttfamily \char`\"{}$>$ 5\%, last 1 version\char`\"{}} or an array `\mbox{[}'$>$ 5\textquotesingle{}, \textquotesingle{}last 1 version\textquotesingle{}\mbox{]}\`{}.

If a query is missing, Browserslist will look for a config file. You can provide a {\ttfamily path} option (that can be a file) to find the config file relatively to it.

For non-\/\+JS environment and debug purpose you can use C\+LI tool\+:


\begin{DoxyCode}
browserslist "> 1%, last 2 versions"
\end{DoxyCode}


\subsection*{Coverage}

You can get total users coverage for selected browsers by JS A\+PI\+:


\begin{DoxyCode}
browserslist.coverage(browserslist('> 1%')) //=> 81.4
\end{DoxyCode}



\begin{DoxyCode}
browserslist.coverage(browserslist('> 1% in US'), 'US') //=> 83.1
\end{DoxyCode}


Or by C\+LI\+:


\begin{DoxyCode}
$ browserslist --coverage "> 1%"
These browsers account for 81.4% of all users globally
\end{DoxyCode}



\begin{DoxyCode}
$ browserslist --coverage=US "> 1% in US"
These browsers account for 83.1% of all users in the US
\end{DoxyCode}


\subsection*{Internal caches}

Browserslist caches the configuration it reads from {\ttfamily package.\+json} and {\ttfamily browserslist} files, as well as knowledge about the existence of files, for the duration of the hosting process.

To clear these caches, use\+:


\begin{DoxyCode}
browserslist.clearCaches();
\end{DoxyCode}


To disable the caching altogether, set the {\ttfamily B\+R\+O\+W\+S\+E\+R\+S\+L\+I\+S\+T\+\_\+\+D\+I\+S\+A\+B\+L\+E\+\_\+\+C\+A\+C\+HE} environment variable. 