An extremely efficient, flexible and amazing evaluator for Math expression in Javascript.(\href{http://redhivesoftware.github.io/math-expression-evaluator/}{\tt Documentation})

\subsection*{Use cases}

\tabulinesep=1mm
\begin{longtabu} spread 0pt [c]{*{3}{|X[-1]}|}
\hline
\rowcolor{\tableheadbgcolor}\textbf{ Input  }&\textbf{ Result  }&\textbf{ Expl   }\\\cline{1-3}
\endfirsthead
\hline
\endfoot
\hline
\rowcolor{\tableheadbgcolor}\textbf{ Input  }&\textbf{ Result  }&\textbf{ Expl   }\\\cline{1-3}
\endhead
{\bfseries 2+3-\/1}  &4  &Addition and Subtraction operator   \\\cline{1-3}
{\bfseries 2$\ast$5/10}  &1  &Multiplication and Division operator   \\\cline{1-3}
{\bfseries tan45} {\itshape or} {\bfseries tan(45)}  &1  &Trigonometric Function ( tan in Degree mode)   \\\cline{1-3}
{\bfseries tan45} {\itshape or} {\bfseries tan(45)}  &1.\+619775190543862  &Trigonometric Function ( tan in Radian mode)   \\\cline{1-3}
{\bfseries Pi1,15,n} {\itshape or} {\bfseries Pi(1,15,n)}  &1307674368000  &Product of Sequence   \\\cline{1-3}
{\bfseries Sigma1,15,n} {\itshape or} {\bfseries Sigma(1,15,n)}  &120  &Sum of Sequence( also called summation)   \\\cline{1-3}
{\bfseries 2$^\wedge$3}  &8  &Exponent( note this operator is left associative like M\+S Office)   \\\cline{1-3}
{\bfseries 5\+P3}  &60  &Permutaion Method to calculate all the permutaions   \\\cline{1-3}
{\bfseries sincostan90} {\itshape or} {\bfseries sin(cos(tan(90)))}  &0.\+017261434031253  &Multiple functions with or without parenthesis (both works)   \\\cline{1-3}
\end{longtabu}


\subsubsection*{\href{http://jsbin.com/fuyowu/1/edit?html,output}{\tt Fiddle Yourself}}

\subsection*{Installation}

\subsubsection*{Node JS}

{\bfseries Using npm}

npm install math-\/expression-\/evaluator

\subsubsection*{Browser}

{\bfseries Using bower}

bower install math-\/expression-\/evaluator

\subsubsection*{How to run test}

\begin{DoxyVerb}npm test
\end{DoxyVerb}


\subsection*{Supported symbols}

\tabulinesep=1mm
\begin{longtabu} spread 0pt [c]{*{2}{|X[-1]}|}
\hline
\rowcolor{\tableheadbgcolor}\textbf{ Symbol  }&\textbf{ Expl   }\\\cline{1-2}
\endfirsthead
\hline
\endfoot
\hline
\rowcolor{\tableheadbgcolor}\textbf{ Symbol  }&\textbf{ Expl   }\\\cline{1-2}
\endhead
{\bfseries +$\ast$$\ast$  }&{\bfseries  Addition Operator eg. 2+3 results 5   }\\\cline{1-2}
{\bfseries  $\ast$$\ast$-\/$\ast$$\ast$  }&{\bfseries  Subtraction Operator eg. 2-\/3 results -\/1   }\\\cline{1-2}
{\bfseries  $\ast$$\ast$/}  &Division operator eg 3/2 results 1.\+5   \\\cline{1-2}
$\ast$$\ast$$\ast$$\ast$$\ast$  &Multiplication Operator eg. 2$\ast$3 results 6   \\\cline{1-2}
{\bfseries Mod}  &Modulus Operator eg. 3 Mod 2 results 1   \\\cline{1-2}
$\ast$$\ast$($\ast$$\ast$  &Opening Parenthesis   \\\cline{1-2}
$\ast$$\ast$)$\ast$$\ast$  &Closing Parenthesis   \\\cline{1-2}
{\bfseries Sigma}  &Summation eg. Sigma(1,100,n) results 5050   \\\cline{1-2}
{\bfseries Pi}  &Product eg. Pi(1,10,n) results 3628800   \\\cline{1-2}
{\bfseries n}  &Variable for Summation or Product   \\\cline{1-2}
{\bfseries pi}  &Math constant pi returns 3.\+14   \\\cline{1-2}
{\bfseries e}  &Math constant e returns 2.\+71   \\\cline{1-2}
{\bfseries C}  &Combination operator eg. 4\+C2 returns 6   \\\cline{1-2}
{\bfseries P}  &Permutation operator eg. 4\+P2 returns 12   \\\cline{1-2}
{\bfseries !}  &factorial operator eg. 4! returns 24   \\\cline{1-2}
{\bfseries log}  &logarithmic function with base 10 eg. log 1000 returns 3   \\\cline{1-2}
{\bfseries ln}  &natural log function with base e eg. ln 2 returns .3010   \\\cline{1-2}
{\bfseries pow}  &power function with two operator pow(2,3) returns 8   \\\cline{1-2}
$\ast$$\ast$$^\wedge$$\ast$$\ast$  &power operator eg. 2$^\wedge$3 returns 8   \\\cline{1-2}
{\bfseries root}  &underroot function root 4 returns 2   \\\cline{1-2}
{\bfseries sin}  &Sine function   \\\cline{1-2}
{\bfseries cos}  &Cosine function   \\\cline{1-2}
{\bfseries tan}  &Tangent function   \\\cline{1-2}
{\bfseries asin}  &Inverse Sine funtion   \\\cline{1-2}
{\bfseries acos}  &Inverse Cosine funtion   \\\cline{1-2}
{\bfseries atan}  &Inverse Tangent funtion   \\\cline{1-2}
{\bfseries sinh}  &Hyperbolic Sine funtion   \\\cline{1-2}
{\bfseries cosh}  &Hyperbolic Cosine funtion   \\\cline{1-2}
{\bfseries tanh}  &Hyperbolic Tangent funtion   \\\cline{1-2}
{\bfseries asinh}  &Inverse Hyperbolic Sine funtion   \\\cline{1-2}
{\bfseries acosh}  &Inverse Hyperbolic Cosine funtion   \\\cline{1-2}
{\bfseries atanh}  &Inverse Hyperbolic Tangent funtion   \\\cline{1-2}
\end{longtabu}


\subsection*{Features}

\subsubsection*{Amazing support for Sigma and Pi}

This is a fantastic feature of this calculator that it is capable of evaluating expressions containing {\bfseries Sigma and Pi}. ~\newline
Passing {\ttfamily Sigma(1,100,n)} will evaluate to 5050 as n is summationed from 1 to 100. and Pi(1,15,n) will evaluate to 1307674368000 as n is multiplied from 1 to 15 which is equal to 15!

\subsubsection*{Parenthesis less expression}

If a expression is readable by human then it is readable by this evaluator. There is no need to wrap every function inside parenthesis. For eg. sin90 will work totally fine instead of sin(90)

\subsection*{Changelog}

\subsubsection*{Removed lodash.\+indexof and used native Array.\+prototype.\+index\+Of hence dropping suppports for I\+E8 and below.}

This will reflect in next release named v1.\+2.\+16 