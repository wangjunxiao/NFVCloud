\subsection*{Filing issues}

Issues can be filed on existing {\bfseries caniuse support data}, {\bfseries site functionality} or to make new {\bfseries support data suggestions}. Support data suggestions can be voted on with {\ttfamily +1} comments and can be \href{http://caniuse.com/issue-list}{\tt viewed in order} of votes.

\subsection*{Caniuse data}

The {\ttfamily features-\/json} directory includes J\+S\+ON files for every feature found on \href{http://caniuse.com/}{\tt the caniuse.\+com website}. Maintaining these files on Git\+Hub allows anyone to update or contribute to the support data on the site.

{\bfseries Note\+:} when submitting a patch, don’t modify the minified {\ttfamily data.\+json} file in the root — that is done automatically. Only modify the contents of the {\ttfamily features-\/json} directory.

\subsubsection*{How it works}

The data on the site is stored in a database. This data is periodically exported to the J\+S\+ON files on Git\+Hub. Once a change or new file here has been approved, it is integrated back into the database and the subsequent export files should be the same as the imported ones. Not too confusing, I hope. \+:)

\subsubsection*{Supported changes}

Currently the following feature information can be modified\+:
\begin{DoxyItemize}
\item {\bfseries title} — Feature name (used for the title of the table)
\item {\bfseries description} — Brief description of feature
\item {\bfseries spec} — Spec \mbox{\hyperlink{namespace_u_r_l}{U\+RL}}
\item {\bfseries status} — Spec status, one of the following\+:
\begin{DoxyItemize}
\item {\ttfamily ls} -\/ W\+H\+A\+T\+WG Living Standard
\item {\ttfamily rec} -\/ W3C Recommendation
\item {\ttfamily pr} -\/ W3C Proposed Recommendation
\item {\ttfamily cr} -\/ W3C Candidate Recommendation
\item {\ttfamily wd} -\/ W3C Working Draft
\item {\ttfamily other} -\/ Non-\/\+W3C, but reputable
\item {\ttfamily unoff} -\/ Unofficial, Editor\textquotesingle{}s Draft or W3C \char`\"{}\+Note\char`\"{}
\end{DoxyItemize}
\item {\bfseries links} — Array of \char`\"{}link\char`\"{} objects consisting of \mbox{\hyperlink{namespace_u_r_l}{U\+RL}} and short description of link
\item {\bfseries bugs} — Array of \char`\"{}bug\char`\"{} objects consisting of a bug description
\item {\bfseries categories} — Array of categories, any of the following\+: (Note that some of these categories are put into a parent category on the caniuse site)
\begin{DoxyItemize}
\item {\ttfamily H\+T\+M\+L5}
\item {\ttfamily C\+SS}
\item {\ttfamily C\+S\+S2}
\item {\ttfamily C\+S\+S3}
\item {\ttfamily S\+VG}
\item {\ttfamily P\+NG}
\item {\ttfamily JS A\+PI}
\item {\ttfamily Canvas}
\item {\ttfamily D\+OM}
\item {\ttfamily Other}
\item {\ttfamily JS}
\item {\ttfamily Security}
\end{DoxyItemize}
\item {\bfseries stats} — The collection of support data for a given set of browsers/versions. Only the support value strings can be modified; additional versions {\itshape cannot be added}. Values are space-\/separated characters with these meanings, and must answer the question \char`\"{}$\ast$\+Can I use$\ast$ the feature by default?\char`\"{}\+:
\begin{DoxyItemize}
\item {\ttfamily y} -\/ ({\bfseries Y})es, supported by default
\item {\ttfamily a} -\/ ({\bfseries A})lmost supported (aka Partial support)
\item {\ttfamily n} -\/ ({\bfseries N})o support, or disabled by default
\item {\ttfamily p} -\/ No support, but has ({\bfseries P})olyfill
\item {\ttfamily u} -\/ Support ({\bfseries u})nknown
\item {\ttfamily x} -\/ Requires prefi({\bfseries x}) to work
\item {\ttfamily d} -\/ ({\bfseries D})isabled by default (need to enable flag or something)
\item {\ttfamily \#n} -\/ Where n is a number, starting with 1, corresponds to the {\bfseries notes\+\_\+by\+\_\+num} note. For example\+: {\ttfamily \char`\"{}42\char`\"{}\+:\char`\"{}y \#1\char`\"{}} means version 42 is supported by default and see note 1.
\end{DoxyItemize}
\item {\bfseries notes} — Notes on feature support, often to explain what partial support refers to
\item {\bfseries notes\+\_\+by\+\_\+num} -\/ Map of numbers corresponding to notes. Used in conjunction with the \#n notation under {\bfseries stats}. Each key should be a number (no hash), the value is the related note. For example\+: {\ttfamily \char`\"{}1\char`\"{}\+: \char`\"{}\+Foo\char`\"{}}
\item {\bfseries ucprefix} — Prefix should start with an uppercase letter
\item {\bfseries parent} — ID of parent feature
\item {\bfseries keywords} — Comma separated words that will match the feature in a search
\item {\bfseries ie\+\_\+id} — Comma separated I\+Ds used by \href{http://status.modern.ie}{\tt status.\+modern.\+ie} -\/ Each ID is the string in the feature\textquotesingle{}s \mbox{\hyperlink{namespace_u_r_l}{U\+RL}}
\item {\bfseries chrome\+\_\+id} — Comma separated I\+Ds used by \href{http://chromestatus.com}{\tt chromestatus.\+com} -\/ Each ID is the number in the feature\textquotesingle{}s \mbox{\hyperlink{namespace_u_r_l}{U\+RL}}
\item {\bfseries firefox\+\_\+id} -\/ Comma separated I\+Ds used by \href{https://platform-status.mozilla.org/}{\tt platform-\/status.\+mozilla.\+org} -\/ Each ID is the filename (minus the {\ttfamily .md} extension suffix) of the relevant file in \href{https://github.com/mozilla/platform-status/tree/master/features}{\tt the {\ttfamily /features/} directory of Mozilla\textquotesingle{}s Platform Status project on Git\+Hub}
\item {\bfseries webkit\+\_\+id} -\/ Comma separated I\+Ds used by \href{http://www.webkit.org/status.html}{\tt webkit.\+org/status.html} -\/ Each ID is the title of the feature\textquotesingle{}s box on the status webpage
\item {\bfseries shown} — Whether or not feature is ready to be shown on the site. This can be left as false if the support data or information for other fields is still being collected
\end{DoxyItemize}

\subsubsection*{Adding a feature}

To add a feature, simply add another J\+S\+ON file, following the \href{/sample-data.json}{\tt example}, to the {\ttfamily features-\/json} directory with the base file name as the feature ID (only alphanumeric characters and hyphens please).

New additions will always start out with {\ttfamily \char`\"{}shown\char`\"{}\+: false} (regardless of the initial value set in the PR). This is so the data can undergo a certain level of verification to guarantee the correctness of information shown on the site. This verification happens {\itshape after} the pull request has already been accepted because it allows the data to automatically be updated with newly released browser versions when necessary so the pull request won\textquotesingle{}t need to require manual updates during this period.

For the same reason, on some occasion pull requests for new features may be accepted at first, but then have the data be rejected later if it\textquotesingle{}s decided that the data is for whatever reason inappropriate for caniuse (e.\+g. it\textquotesingle{}s for some feature already widely supported by all browsers)

Good/preferred pull requests for new features meet the following criteria\+:
\begin{DoxyItemize}
\item Feature is on the higher end of the spectrum on the \href{http://caniuse.com/issue-list/}{\tt Feature suggestion list}
\item Feature is {\itshape not} already widely supported (e.\+g. since I\+E6+, Firefox 2+, Chrome 1+ etc). This is because caniuse is intended to answer questions about mixed support, not to provide complete information on all web technologies.
\item Feature is at least supported in one (possibly upcoming) browser.
\item PR includes a link to the test case(s) used to test support (can be codepen, jsfiddle, etc)
\item Support data was properly validated using either test cases or from information from reliable sources. If you don\textquotesingle{}t know be sure to use {\ttfamily u} for unknown support, though it may be fine to make the more obvious extrapolations like really old browsers not supporting the latest A\+P\+Is, etc.
\item The more actual support information, the better (rather than most data simply being {\ttfamily u}nknown). \href{https://www.browserstack.com}{\tt https\+://www.\+browserstack.\+com} and \href{http://saucelabs.com}{\tt http\+://saucelabs.\+com} are excellent tools for good cross-\/browser support testing. In order to keep caniuse useful, features won\textquotesingle{}t be included on the site until almost all included browsers have actual support information. This does not however apply to older and lesser used browser versions.
\end{DoxyItemize}

\subsubsection*{Unsupported changes}

Currently it is not possible to\+:
\begin{DoxyItemize}
\item Add a new browser or browser version
\item Add a test for any given feature (should also come later)
\item Add any object properties not already defined above
\item Modify the {\bfseries usage\+\_\+perc\+\_\+y} or {\bfseries usage\+\_\+perc\+\_\+a} values (these values are generated)
\end{DoxyItemize}

\subsubsection*{Testing}

Make sure you have Node\+JS installed on your system.

Run

{\ttfamily node validator/validate-\/jsons.\+js}

If something is wrong, it will throw an error. Everything is ok otherwise. 