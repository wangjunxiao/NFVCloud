\href{http://travis-ci.org/AriaMinaei/RenderKid}{\tt }

Render\+Kid allows you to use H\+T\+ML and C\+SS to style your C\+LI output, making it easy to create a beautiful, readable, and consistent look for your nodejs tool.

\subsection*{Installation}

Install with npm\+: 
\begin{DoxyCode}
$ npm install renderkid
\end{DoxyCode}


\subsection*{Usage}


\begin{DoxyCode}
RenderKid = require('renderkid')

r = new RenderKid()

r.style(\{
  "ul": \{
    display: "block"
    margin: "2 0 2"
  \}

  "li": \{
    display: "block"
    marginBottom: "1"
  \}

  "key": \{
    color: "grey"
    marginRight: "1"
  \}

  "value": \{
    color: "bright-white"
  \}
\})

output = r.render("
<ul>
  <li>
    <key>Name:</key>
    <value>RenderKid</value>
  </li>
  <li>
    <key>Version:</key>
    <value>0.2</value>
  </li>
  <li>
    <key>Last Update:</key>
    <value>Jan 2015</value>
  </li>
</ul>
")

console.log(output)
\end{DoxyCode}




\subsection*{Stylesheet properties}

\subsubsection*{Display mode}

Elements can have a {\ttfamily display} of either {\ttfamily inline}, {\ttfamily block}, or {\ttfamily none}\+: 
\begin{DoxyCode}
r.style(\{
  "div": \{
    display: "block"
  \}

  "span": \{
    display: "inline" # default
  \}

  "hidden": \{
    display: "none"
  \}
\})

output = r.render("
<div>This will fill one or more rows.</div>
<span>These</span> <span>will</span> <span>be</span> in the same <span>line.</span>
<hidden>This won't be displayed.</hidden>
")

console.log(output)
\end{DoxyCode}




\subsubsection*{Margin}

Margins work just like they do in browsers\+: 
\begin{DoxyCode}
r.style(\{
  "li": \{
    display: "block"

    marginTop: "1"
    marginRight: "2"
    marginBottom: "3"
    marginLeft: "4"

    # or the shorthand version:
    "margin": "1 2 3 4"
  \},

  "highlight": \{
    display: "inline"
    marginLeft: "2"
    marginRight: "2"
  \}
\})

r.render("
<ul>
  <li>Item <highlgiht>1</highlight></li>
  <li>Item <highlgiht>2</highlight></li>
  <li>Item <highlgiht>3</highlight></li>
</ul>
")
\end{DoxyCode}


\subsubsection*{Padding}

See margins above. Paddings work the same way, only inward.

\subsubsection*{Width and Height}

Block elements can have explicit width and height\+: 
\begin{DoxyCode}
r.style(\{
  "box": \{
    display: "block"
    "width": "4"
    "height": "2"
  \}
\})

r.render("<box>This is a box and some of its text will be truncated.</box>")
\end{DoxyCode}


\subsubsection*{Colors}

You can set a custom color and background color for each element\+:


\begin{DoxyCode}
r.style(\{
  "error": \{
    color: "black"
    background: "red"
  \}
\})
\end{DoxyCode}


List of colors currently supported are {\ttfamily black}, {\ttfamily red}, {\ttfamily green}, {\ttfamily yellow}, {\ttfamily blue}, {\ttfamily magenta}, {\ttfamily cyan}, {\ttfamily white}, {\ttfamily grey}, {\ttfamily bright-\/red}, {\ttfamily bright-\/green}, {\ttfamily bright-\/yellow}, {\ttfamily bright-\/blue}, {\ttfamily bright-\/magenta}, {\ttfamily bright-\/cyan}, {\ttfamily bright-\/white}.

\subsubsection*{Bullet points}

Block elements can have bullet points on their margins. Let\textquotesingle{}s start with an example\+: 
\begin{DoxyCode}
r.style(\{
  "li": \{
    # To add bullet points to an element, first you
    # should make some room for the bullet point by
    # giving your element some margin to the left:
    marginLeft: "4",

    # Now we can add a bullet point to our margin:
    bullet: '"-"'
  \}
\})

# The four hyphens are there for visual reference
r.render("
----
<li>Item 1</li>
<li>Item 2</li>
<li>Item 3</li>
----
")
\end{DoxyCode}
 And here is the result\+:

 