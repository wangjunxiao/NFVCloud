\href{http://travis-ci.org/veged/coa}{\tt }

\subsection*{Что это?}

C\+OA — парсер параметров командной строки, позволяющий извлечь максимум пользы от формального A\+PI вашей программы. Как только вы опишете определение в терминах команд, параметров и аргументов, вы автоматически получите\+:


\begin{DoxyItemize}
\item Справку для командной строки
\item A\+PI для использования программы как модуля в C\+O\+A-\/совместимых программах
\item Автодополнение для командной строки
\end{DoxyItemize}

\subsubsection*{Прочие возможности}


\begin{DoxyItemize}
\item Широкий выбор настроек для параметров и аргументов, включая множественные значения, логические значения и обязательность параметров
\item Возможность асинхронного исполнения команд, используя промисы (используется библиотека \href{https://github.com/kriskowal/q}{\tt Q})
\item Простота использования существующих команд как подмодулей для новых команд
\item Комбинированная валидация и анализ сложных значений
\end{DoxyItemize}

\subsection*{Примеры}


\begin{DoxyCode}
require('coa').Cmd() // декларация команды верхнего уровня
    .name(process.argv[1]) // имя команды верхнего уровня, берем из имени программы
    .title('Жутко полезная утилита для командной строки') // название для использования в справке и
       сообщениях
    .helpful() // добавляем поддержку справки командной строки (-h, --help)
    .opt() // добавляем параметр
        .name('version') // имя параметра для использования в API
        .title('Version') // текст для вывода в сообщениях
        .short('v') // короткое имя параметра: -v
        .long('version') // длинное имя параметра: --version
        .flag() // параметр не требует ввода значения
        .act(function(opts) \{  // действия при вызове аргумента
            // результатом является вывод текстового сообщения
            return JSON.parse(require('fs').readFileSync(\_\_dirname + '/package.json'))
                .version;
        \})
        .end() // завершаем определение параметра и возвращаемся к определению верхнего уровня
    .cmd().name('subcommand').apply(require('./subcommand').COA).end() // загрузка подкоманды из модуля
    .cmd() // добавляем еще одну подкоманду
        .name('othercommand').title('Еще одна полезная подпрограмма').helpful()
        .opt()
            .name('input').title('input file, required')
            .short('i').long('input')
            .val(function(v) \{ // функция-валидатор, также может использоваться для трансформации значений
       параметров
                return require('fs').createReadStream(v) \})
            .req() // параметр является обязательным
            .end() // завершаем определение параметра и возвращаемся к определению команды
        .end() // завершаем определение подкоманды и возвращаемся к определению команды верхнего уровня
    .run(process.argv.slice(2)); // разбираем process.argv и запускаем
\end{DoxyCode}



\begin{DoxyCode}
// subcommand.js
exports.COA = function() \{
    this
        .title('Полезная подпрограмма').helpful()
        .opt()
            .name('output').title('output file')
            .short('o').long('output')
            .output() // использовать стандартную настройку для параметра вывода
            .end()
\};
\end{DoxyCode}


\subsection*{A\+PI}

\subsubsection*{Cmd}

Команда — сущность верхнего уровня. У команды могут быть определены параметры и аргументы.

\paragraph*{Cmd.\+api}

Возвращает объект, который можно использовать в других программах. Подкоманды являются методами этого объекта.~\newline
 {\bfseries \begin{DoxyReturn}{返回}
$\ast$\{Object\}$\ast$
\end{DoxyReturn}
\paragraph*{Cmd.\+name}}

{\bfseries  Определяет канонический идентификатор команды, используемый в вызовах A\+PI.~\newline
 $\ast$$\ast$
\begin{DoxyParams}{参数}
{\em $\ast$$\ast$} & {\itshape String} {\ttfamily \+\_\+name} имя команды~\newline
 {\bfseries }\\
\hline
\end{DoxyParams}
\begin{DoxyReturn}{返回}
{\bfseries } {\itshape C\+O\+A.\+Cmd} {\ttfamily this} экземпляр команды (для поддержки цепочки методов)
\end{DoxyReturn}
\paragraph*{Cmd.\+title}}

{\bfseries  Определяет название команды, используемый в текстовых сообщениях.~\newline
 $\ast$$\ast$
\begin{DoxyParams}{参数}
{\em $\ast$$\ast$} & {\itshape String} {\ttfamily \+\_\+title} название команды~\newline
 {\bfseries }\\
\hline
\end{DoxyParams}
\begin{DoxyReturn}{返回}
{\bfseries } {\itshape C\+O\+A.\+Cmd} {\ttfamily this} экземпляр команды (для поддержки цепочки методов)
\end{DoxyReturn}
\paragraph*{Cmd.\+cmd}}

{\bfseries  Создает новую подкоманду или добавляет ранее определенную подкоманду к текущей команде.~\newline
 $\ast$$\ast$
\begin{DoxyParams}{参数}
{\em $\ast$$\ast$} & {\itshape C\+O\+A.\+Cmd} {\ttfamily \mbox{[}cmd\mbox{]}} экземпляр ранее определенной подкоманды~\newline
 {\bfseries }\\
\hline
\end{DoxyParams}
\begin{DoxyReturn}{返回}
{\bfseries } {\itshape C\+O\+A.\+Cmd} экземпляр новой или ранее определенной подкоманды
\end{DoxyReturn}
\paragraph*{Cmd.\+opt}}

{\bfseries  Создает параметр для текущей команды.~\newline
 {\bfseries \begin{DoxyReturn}{返回}
{\itshape C\+O\+A.\+Opt} {\ttfamily new} экземпляр параметра
\end{DoxyReturn}
\paragraph*{Cmd.\+arg}}}

{\bfseries {\bfseries  Создает аргумент для текущей команды.~\newline
 {\bfseries \begin{DoxyReturn}{返回}
{\itshape C\+O\+A.\+Opt} {\ttfamily new} экземпляр аргумента
\end{DoxyReturn}
\paragraph*{Cmd.\+act}}}}

{\bfseries {\bfseries {\bfseries  Добавляет (или создает) действие для текущей команды.~\newline
 $\ast$$\ast$
\begin{DoxyParams}{参数}
{\em $\ast$$\ast$} & {\itshape Function} {\ttfamily act} функция, выполняемая в контексте экземпляра текущей команды и принимающая следующие параметры\+:~\newline

\begin{DoxyItemize}
\item {\itshape Object} {\ttfamily opts} параметры команды~\newline

\item {\itshape Array} {\ttfamily args} аргументы команды~\newline

\item {\itshape Object} {\ttfamily res} объект-\/аккумулятор результатов~\newline
 Функция может вернуть проваленный промис из Cmd.\+reject (в случае ошибки) или любое другое значение, рассматриваемое как результат.~\newline
 $\ast$$\ast$
\end{DoxyItemize}\\
\hline
{\em $\ast$$\ast$} & $\ast$\{Boolean\}$\ast$ \mbox{[}force=false\mbox{]} флаг, назначающий немедленное исполнение вместо добавления к списку существующих действий~\newline
 {\bfseries }\\
\hline
\end{DoxyParams}
\begin{DoxyReturn}{返回}
{\bfseries } {\itshape C\+O\+A.\+Cmd} {\ttfamily this} экземпляр команды (для поддержки цепочки методов)
\end{DoxyReturn}
\paragraph*{Cmd.\+apply}}}}

{\bfseries {\bfseries {\bfseries  Исполняет функцию с переданными аргументами в контексте экземпляра текущей команды.~\newline
 $\ast$$\ast$
\begin{DoxyParams}{参数}
{\em $\ast$$\ast$} & {\itshape Function} {\ttfamily fn}~\newline
 $\ast$$\ast$\\
\hline
{\em $\ast$$\ast$} & {\itshape Array} {\ttfamily args}~\newline
 {\bfseries }\\
\hline
\end{DoxyParams}
\begin{DoxyReturn}{返回}
{\bfseries } {\itshape C\+O\+A.\+Cmd} {\ttfamily this} экземпляр команды (для поддержки цепочки методов)
\end{DoxyReturn}
\paragraph*{Cmd.\+comp}}}}

{\bfseries {\bfseries {\bfseries  Назначает кастомную функцию автодополнения для текущей команды.~\newline
 $\ast$$\ast$
\begin{DoxyParams}{参数}
{\em $\ast$$\ast$} & {\itshape Function} {\ttfamily fn} функция-\/генератор автодополнения, исполняемая в контексте текущей команды. Принимает параметры\+:~\newline

\begin{DoxyItemize}
\item {\itshape Object} {\ttfamily opts} параметры~\newline
 Может возвращать промис или любое другое значение, рассматриваемое как результат исполнения команды.~\newline
 {\bfseries }
\end{DoxyItemize}\\
\hline
\end{DoxyParams}
\begin{DoxyReturn}{返回}
{\bfseries } {\itshape C\+O\+A.\+Cmd} {\ttfamily this} экземпляр команды (для поддержки цепочки методов)
\end{DoxyReturn}
\paragraph*{Cmd.\+helpful}}}}

{\bfseries {\bfseries {\bfseries  Ставит флаг поддержки справки командной строки, т.\+е. вызов команды с параметрами -\/h --help выводит справку по работе с командой.~\newline
 {\bfseries \begin{DoxyReturn}{返回}
{\itshape C\+O\+A.\+Cmd} {\ttfamily this} экземпляр команды (для поддержки цепочки методов)
\end{DoxyReturn}
\paragraph*{Cmd.\+completable}}}}}

{\bfseries {\bfseries {\bfseries {\bfseries  Добавляет поддержку автодополнения командной строки. Добавляется подкоманда \char`\"{}completion\char`\"{}, которая выполняет все необходимые действия.~\newline
 Может быть добавлен только для главной команды.~\newline
 {\bfseries \begin{DoxyReturn}{返回}
{\itshape C\+O\+A.\+Cmd} {\ttfamily this} экземпляр команды (для поддержки цепочки методов)
\end{DoxyReturn}
\paragraph*{Cmd.\+usage}}}}}}

{\bfseries {\bfseries {\bfseries {\bfseries {\bfseries  Возвращает текст справки по использованию команды для текущего экземпляра.~\newline
 {\bfseries \begin{DoxyReturn}{返回}
{\itshape String} {\ttfamily usage} Текст справки по использованию
\end{DoxyReturn}
\paragraph*{Cmd.\+run}}}}}}}

{\bfseries {\bfseries {\bfseries {\bfseries {\bfseries {\bfseries  Разбирает аргументы из значения, возвращаемого Node\+JS process.\+argv, и запускает текущую программу, т.\+е. вызывает process.\+exit после завершения всех действий.~\newline
 $\ast$$\ast$
\begin{DoxyParams}{参数}
{\em $\ast$$\ast$} & {\itshape Array} {\ttfamily argv}~\newline
 {\bfseries }\\
\hline
\end{DoxyParams}
\begin{DoxyReturn}{返回}
{\bfseries } {\itshape C\+O\+A.\+Cmd} {\ttfamily this} экземпляр команды (для поддержки цепочки методов)
\end{DoxyReturn}
\paragraph*{Cmd.\+invoke}}}}}}}

{\bfseries {\bfseries {\bfseries {\bfseries {\bfseries {\bfseries  Исполняет переданную (или текущую) команду с указанными параметрами и аргументами.~\newline
 $\ast$$\ast$
\begin{DoxyParams}{参数}
{\em $\ast$$\ast$} & {\itshape String$\vert$\+Array} {\ttfamily cmds} подкоманда для исполнения (необязательно)~\newline
 $\ast$$\ast$\\
\hline
{\em $\ast$$\ast$} & {\itshape Object} {\ttfamily opts} параметры, передаваемые команде (необязательно)~\newline
 $\ast$$\ast$\\
\hline
{\em $\ast$$\ast$} & {\itshape Object} {\ttfamily args} аргументы, передаваемые команде (необязательно)~\newline
 {\bfseries }\\
\hline
\end{DoxyParams}
\begin{DoxyReturn}{返回}
{\bfseries } {\itshape Q.\+Promise}
\end{DoxyReturn}
\paragraph*{Cmd.\+reject}}}}}}}

{\bfseries {\bfseries {\bfseries {\bfseries {\bfseries {\bfseries  Проваливает промисы, возращенные в действиях.~\newline
 Используется в .act() для возврата с ошибкой.~\newline
 $\ast$$\ast$
\begin{DoxyParams}{参数}
{\em $\ast$$\ast$} & {\itshape Object} {\ttfamily reason} причина провала~\newline
 Вы можете определить метод to\+String() и свойство to\+String() объекта причины провала.~\newline
 {\bfseries }\\
\hline
\end{DoxyParams}
\begin{DoxyReturn}{返回}
{\bfseries } {\itshape Q.\+promise} проваленный промис
\end{DoxyReturn}
\paragraph*{Cmd.\+end}}}}}}}

{\bfseries {\bfseries {\bfseries {\bfseries {\bfseries {\bfseries  Завершает цепочку методов текущей подкоманды и возвращает экземпляр родительской команды.~\newline
 {\bfseries \begin{DoxyReturn}{返回}
{\itshape C\+O\+A.\+Cmd} {\ttfamily parent} родительская команда
\end{DoxyReturn}
\subsubsection*{Opt}}}}}}}}

{\bfseries {\bfseries {\bfseries {\bfseries {\bfseries {\bfseries {\bfseries  Параметр — именованная сущность. У параметра может быть определено короткое или длинное имя для использования из командной строки.~\newline
 $\ast$$\ast$ }}}}}}}