

Utility methods for the hapi ecosystem. This module is not intended to solve every problem for everyone, but rather as a central place to store hapi-\/specific methods. If you\textquotesingle{}re looking for a general purpose utility module, check out \href{https://github.com/lodash/lodash}{\tt lodash} or \href{https://github.com/jashkenas/underscore}{\tt underscore}.

\href{http://travis-ci.org/hapijs/hoek}{\tt }

Lead Maintainer\+: \href{https://github.com/nlf}{\tt Nathan La\+Freniere}

\section*{Table of Contents}


\begin{DoxyItemize}
\item \href{#introduction}{\tt Introduction}
\item \href{#object}{\tt Object}
\begin{DoxyItemize}
\item \href{#cloneobj}{\tt clone}
\item \href{#clonewithshallowobj-keys}{\tt clone\+With\+Shallow}
\item \href{#mergetarget-source-isnulloverride-ismergearrays}{\tt merge}
\item \href{#applytodefaultsdefaults-options-isnulloverride}{\tt apply\+To\+Defaults}
\item \href{#applytodefaultswithshallowdefaults-options-keys}{\tt apply\+To\+Defaults\+With\+Shallow}
\item \href{#deepequala-b}{\tt deep\+Equal}
\item \href{#uniquearray-key}{\tt unique}
\item \href{#maptoobjectarray-key}{\tt map\+To\+Object}
\item \href{#intersectarray1-array2}{\tt intersect}
\item \href{#containref-values-options}{\tt contain}
\item \href{#flattenarray-target}{\tt flatten}
\item \href{#reachobj-chain-options}{\tt reach}
\item \href{#reachtemplateobj-template-options}{\tt reach\+Template}
\item \href{#transformobj-transform-options}{\tt transform}
\item \href{#shallowobj}{\tt shallow}
\item \href{#stringifyobj}{\tt stringify}
\end{DoxyItemize}
\item \href{#timer}{\tt Timer}
\item \href{#bench}{\tt Bench}
\item \href{#binary-encodingdecoding}{\tt Binary Encoding/\+Decoding}
\begin{DoxyItemize}
\item \href{#base64urlencodevalue}{\tt base64url\+Encode}
\item \href{#base64urldecodevalue}{\tt base64url\+Decode}
\end{DoxyItemize}
\item \href{#escaping-characters}{\tt Escaping Characters}
\begin{DoxyItemize}
\item \href{#escapehtmlstring}{\tt escape\+Html}
\item \href{#escapeheaderattributeattribute}{\tt escape\+Header\+Attribute}
\item \href{#escaperegexstring}{\tt escape\+Regex}
\end{DoxyItemize}
\item \href{#errors}{\tt Errors}
\begin{DoxyItemize}
\item \href{#assertcondition-message}{\tt assert}
\item \href{#abortmessage}{\tt abort}
\item \href{#displaystackslice}{\tt display\+Stack}
\item \href{#callstackslice}{\tt call\+Stack}
\end{DoxyItemize}
\item \href{#function}{\tt Function}
\begin{DoxyItemize}
\item \href{#nexttickfn}{\tt next\+Tick}
\item \href{#oncefn}{\tt once}
\item \href{#ignore}{\tt ignore}
\end{DoxyItemize}
\item \href{#miscellaneous}{\tt Miscellaneous}
\begin{DoxyItemize}
\item \href{#uniquefilenamepath-extension}{\tt unique\+Filename}
\item \href{#isabsolutepathpath-platform}{\tt is\+Absolute\+Path}
\item \href{#isintegervalue}{\tt is\+Integer}
\end{DoxyItemize}
\end{DoxyItemize}

\section*{Introduction}

The {\itshape Hoek} library contains some common functions used within the hapi ecosystem. It comes with useful methods for Arrays (clone, merge, apply\+To\+Defaults), Objects (remove\+Keys, copy), Asserting and more.

For example, to use Hoek to set configuration with default options\+: 
\begin{DoxyCode}
var Hoek = require('hoek');

var default = \{url : "www.github.com", port : "8000", debug : true\};

var config = Hoek.applyToDefaults(default, \{port : "3000", admin : true\});

// In this case, config would be \{ url: 'www.github.com', port: '3000', debug: true, admin: true \}
\end{DoxyCode}


Under each of the sections (such as Array), there are subsections which correspond to Hoek methods. Each subsection will explain how to use the corresponding method. In each js excerpt below, the `var Hoek = require(\textquotesingle{}hoek');\`{} is omitted for brevity.

\subsection*{Object}

Hoek provides several helpful methods for objects and arrays.

\subsubsection*{clone(obj)}

This method is used to clone an object or an array. A {\itshape deep copy} is made (duplicates everything, including values that are objects, as well as non-\/enumerable properties).


\begin{DoxyCode}
var nestedObj = \{
        w: /^something$/ig,
        x: \{
            a: [1, 2, 3],
            b: 123456,
            c: new Date()
        \},
        y: 'y',
        z: new Date()
    \};

var copy = Hoek.clone(nestedObj);

copy.x.b = 100;

console.log(copy.y);        // results in 'y'
console.log(nestedObj.x.b); // results in 123456
console.log(copy.x.b);      // results in 100
\end{DoxyCode}


\subsubsection*{clone\+With\+Shallow(obj, keys)}

keys is an array of key names to shallow copy

This method is also used to clone an object or array, however any keys listed in the {\ttfamily keys} array are shallow copied while those not listed are deep copied.


\begin{DoxyCode}
var nestedObj = \{
        w: /^something$/ig,
        x: \{
            a: [1, 2, 3],
            b: 123456,
            c: new Date()
        \},
        y: 'y',
        z: new Date()
    \};

var copy = Hoek.cloneWithShallow(nestedObj, ['x']);

copy.x.b = 100;

console.log(copy.y);        // results in 'y'
console.log(nestedObj.x.b); // results in 100
console.log(copy.x.b);      // results in 100
\end{DoxyCode}


\subsubsection*{merge(target, source, is\+Null\+Override, is\+Merge\+Arrays)}

is\+Null\+Override, is\+Merge\+Arrays default to true

Merge all the properties of source into target, source wins in conflict, and by default null and undefined from source are applied. Merge is destructive where the target is modified. For non destructive merge, use {\ttfamily apply\+To\+Defaults}.


\begin{DoxyCode}
var target = \{a: 1, b : 2\};
var source = \{a: 0, c: 5\};
var source2 = \{a: null, c: 5\};

Hoek.merge(target, source);         // results in \{a: 0, b: 2, c: 5\}
Hoek.merge(target, source2);        // results in \{a: null, b: 2, c: 5\}
Hoek.merge(target, source2, false); // results in \{a: 1, b: 2, c: 5\}

var targetArray = [1, 2, 3];
var sourceArray = [4, 5];

Hoek.merge(targetArray, sourceArray);              // results in [1, 2, 3, 4, 5]
Hoek.merge(targetArray, sourceArray, true, false); // results in [4, 5]
\end{DoxyCode}


\subsubsection*{apply\+To\+Defaults(defaults, options, is\+Null\+Override)}

is\+Null\+Override defaults to false

Apply options to a copy of the defaults


\begin{DoxyCode}
var defaults = \{ host: "localhost", port: 8000 \};
var options = \{ port: 8080 \};

var config = Hoek.applyToDefaults(defaults, options); // results in \{ host: "localhost", port: 8080 \}
\end{DoxyCode}


Apply options with a null value to a copy of the defaults


\begin{DoxyCode}
var defaults = \{ host: "localhost", port: 8000 \};
var options = \{ host: null, port: 8080 \};

var config = Hoek.applyToDefaults(defaults, options, true); // results in \{ host: null, port: 8080 \}
\end{DoxyCode}


\subsubsection*{apply\+To\+Defaults\+With\+Shallow(defaults, options, keys)}

keys is an array of key names to shallow copy

Apply options to a copy of the defaults. Keys specified in the last parameter are shallow copied from options instead of merged.


\begin{DoxyCode}
var defaults = \{
        server: \{
            host: "localhost",
            port: 8000
        \},
        name: 'example'
    \};

var options = \{ server: \{ port: 8080 \} \};

var config = Hoek.applyToDefaultsWithShallow(defaults, options, ['server']); // results in \{ server: \{
       port: 8080 \}, name: 'example' \}
\end{DoxyCode}


\subsubsection*{deep\+Equal(b, a, \mbox{[}options\mbox{]})}

Performs a deep comparison of the two values including support for circular dependencies, prototype, and properties. To skip prototype comparisons, use {\ttfamily options.\+prototype = false}


\begin{DoxyCode}
Hoek.deepEqual(\{ a: [1, 2], b: 'string', c: \{ d: true \} \}, \{ a: [1, 2], b: 'string', c: \{ d: true \} \});
       //results in true
Hoek.deepEqual(Object.create(null), \{\}, \{ prototype: false \}); //results in true
Hoek.deepEqual(Object.create(null), \{\}); //results in false
\end{DoxyCode}


\subsubsection*{unique(array, key)}

Remove duplicate items from Array


\begin{DoxyCode}
var array = [1, 2, 2, 3, 3, 4, 5, 6];

var newArray = Hoek.unique(array);    // results in [1,2,3,4,5,6]

array = [\{id: 1\}, \{id: 1\}, \{id: 2\}];

newArray = Hoek.unique(array, "id");  // results in [\{id: 1\}, \{id: 2\}]
\end{DoxyCode}


\subsubsection*{map\+To\+Object(array, key)}

Convert an Array into an Object


\begin{DoxyCode}
var array = [1,2,3];
var newObject = Hoek.mapToObject(array);   // results in [\{"1": true\}, \{"2": true\}, \{"3": true\}]

array = [\{id: 1\}, \{id: 2\}];
newObject = Hoek.mapToObject(array, "id"); // results in [\{"id": 1\}, \{"id": 2\}]
\end{DoxyCode}


\subsubsection*{intersect(array1, array2)}

Find the common unique items in two arrays


\begin{DoxyCode}
var array1 = [1, 2, 3];
var array2 = [1, 4, 5];

var newArray = Hoek.intersect(array1, array2); // results in [1]
\end{DoxyCode}


\subsubsection*{contain(ref, values, \mbox{[}options\mbox{]})}

Tests if the reference value contains the provided values where\+:
\begin{DoxyItemize}
\item {\ttfamily ref} -\/ the reference string, array, or object.
\item {\ttfamily values} -\/ a single or array of values to find within the {\ttfamily ref} value. If {\ttfamily ref} is an object, {\ttfamily values} can be a key name, an array of key names, or an object with key-\/value pairs to compare.
\item {\ttfamily options} -\/ an optional object with the following optional settings\+:
\begin{DoxyItemize}
\item {\ttfamily deep} -\/ if {\ttfamily true}, performed a deep comparison of the values.
\item {\ttfamily once} -\/ if {\ttfamily true}, allows only one occurrence of each value.
\item {\ttfamily only} -\/ if {\ttfamily true}, does not allow values not explicitly listed.
\item {\ttfamily part} -\/ if {\ttfamily true}, allows partial match of the values (at least one must always match).
\end{DoxyItemize}
\end{DoxyItemize}

Note\+: comparing a string to overlapping values will result in failed comparison (e.\+g. `contain(\textquotesingle{}abc', \mbox{[}\textquotesingle{}ab\textquotesingle{}, \textquotesingle{}bc\textquotesingle{}\mbox{]}){\ttfamily ). Also, if an object key\textquotesingle{}s value does not match the provided value,}false{\ttfamily is returned even when}part\`{} is specified.


\begin{DoxyCode}
Hoek.contain('aaa', 'a', \{ only: true \});                           // true
Hoek.contain([\{ a: 1 \}], [\{ a: 1 \}], \{ deep: true \});               // true
Hoek.contain([1, 2, 2], [1, 2], \{ once: true \});                    // false
Hoek.contain(\{ a: 1, b: 2, c: 3 \}, \{ a: 1, d: 4 \}, \{ part: true \}); // true
\end{DoxyCode}


\subsubsection*{flatten(array, \mbox{[}target\mbox{]})}

Flatten an array


\begin{DoxyCode}
var array = [1, [2, 3]];

var flattenedArray = Hoek.flatten(array); // results in [1, 2, 3]

array = [1, [2, 3]];
target = [4, [5]];

flattenedArray = Hoek.flatten(array, target); // results in [4, [5], 1, 2, 3]
\end{DoxyCode}


\subsubsection*{reach(obj, chain, \mbox{[}options\mbox{]})}

Converts an object key chain string to reference


\begin{DoxyItemize}
\item {\ttfamily options} -\/ optional settings
\begin{DoxyItemize}
\item {\ttfamily separator} -\/ string to split chain path on, defaults to \textquotesingle{}.\textquotesingle{}
\item {\ttfamily default} -\/ value to return if the path or value is not present, default is {\ttfamily undefined}
\item {\ttfamily strict} -\/ if {\ttfamily true}, will throw an error on missing member, default is {\ttfamily false}
\item {\ttfamily functions} -\/ if {\ttfamily true} allow traversing functions for properties. {\ttfamily false} will throw an error if a function is part of the chain.
\end{DoxyItemize}
\end{DoxyItemize}

A chain including negative numbers will work like negative indices on an array.

If chain is {\ttfamily null}, {\ttfamily undefined} or {\ttfamily false}, the object itself will be returned.


\begin{DoxyCode}
var chain = 'a.b.c';
var obj = \{a : \{b : \{ c : 1\}\}\};

Hoek.reach(obj, chain); // returns 1

var chain = 'a.b.-1';
var obj = \{a : \{b : [2,3,6]\}\};

Hoek.reach(obj, chain); // returns 6
\end{DoxyCode}


\subsubsection*{reach\+Template(obj, template, \mbox{[}options\mbox{]})}

Replaces string parameters ({\ttfamily \{name\}}) with their corresponding object key values by applying the ({\ttfamily reach()})\mbox{[}\#reachobj-\/chain-\/options\mbox{]} method where\+:


\begin{DoxyItemize}
\item {\ttfamily obj} -\/ the context object used for key lookup.
\item {\ttfamily template} -\/ a string containing {\ttfamily \{\}} parameters.
\item {\ttfamily options} -\/ optional ({\ttfamily reach()})\mbox{[}\#reachobj-\/chain-\/options\mbox{]} options.
\end{DoxyItemize}


\begin{DoxyCode}
var chain = 'a.b.c';
var obj = \{a : \{b : \{ c : 1\}\}\};

Hoek.reachTemplate(obj, '1+\{a.b.c\}=2'); // returns '1+1=2'
\end{DoxyCode}


\subsubsection*{transform(obj, transform, \mbox{[}options\mbox{]})}

Transforms an existing object into a new one based on the supplied {\ttfamily obj} and {\ttfamily transform} map. {\ttfamily options} are the same as the {\ttfamily reach} options. The first argument can also be an array of objects. In that case the method will return an array of transformed objects.


\begin{DoxyCode}
var source = \{
    address: \{
        one: '123 main street',
        two: 'PO Box 1234'
    \},
    title: 'Warehouse',
    state: 'CA'
\};

var result = Hoek.transform(source, \{
    'person.address.lineOne': 'address.one',
    'person.address.lineTwo': 'address.two',
    'title': 'title',
    'person.address.region': 'state'
\});
// Results in
// \{
//     person: \{
//         address: \{
//             lineOne: '123 main street',
//             lineTwo: 'PO Box 1234',
//             region: 'CA'
//         \}
//     \},
//     title: 'Warehouse'
// \}
\end{DoxyCode}


\subsubsection*{shallow(obj)}

Performs a shallow copy by copying the references of all the top level children where\+:
\begin{DoxyItemize}
\item {\ttfamily obj} -\/ the object to be copied.
\end{DoxyItemize}


\begin{DoxyCode}
var shallow = Hoek.shallow(\{ a: \{ b: 1 \} \});
\end{DoxyCode}


\subsubsection*{stringify(obj)}

Converts an object to string using the built-\/in {\ttfamily J\+S\+O\+N.\+stringify()} method with the difference that any errors are caught and reported back in the form of the returned string. Used as a shortcut for displaying information to the console (e.\+g. in error message) without the need to worry about invalid conversion.


\begin{DoxyCode}
var a = \{\};
a.b = a;
Hoek.stringify(a);      // Returns '[Cannot display object: Converting circular structure to JSON]'
\end{DoxyCode}


\section*{Timer}

A Timer object. Initializing a new timer object sets the ts to the number of milliseconds elapsed since 1 January 1970 00\+:00\+:00 U\+TC.


\begin{DoxyCode}
var timerObj = new Hoek.Timer();
console.log("Time is now: " + timerObj.ts);
console.log("Elapsed time from initialization: " + timerObj.elapsed() + 'milliseconds');
\end{DoxyCode}


\section*{Bench}

Same as Timer with the exception that {\ttfamily ts} stores the internal node clock which is not related to {\ttfamily Date.\+now()} and cannot be used to display human-\/readable timestamps. More accurate for benchmarking or internal timers.

\section*{Binary Encoding/\+Decoding}

\subsubsection*{base64url\+Encode(value)}

Encodes value in Base64 or \mbox{\hyperlink{namespace_u_r_l}{U\+RL}} encoding

\subsubsection*{base64url\+Decode(value)}

Decodes data in Base64 or \mbox{\hyperlink{namespace_u_r_l}{U\+RL}} encoding. \section*{Escaping Characters}

Hoek provides convenient methods for escaping html characters. The escaped characters are as followed\+:


\begin{DoxyCode}
internals.htmlEscaped = \{
    '&': '&amp;',
    '<': '&lt;',
    '>': '&gt;',
    '"': '&quot;',
    "'": '&#x27;',
    '`': '&#x60;'
\};
\end{DoxyCode}


\subsubsection*{escape\+Html(string)}


\begin{DoxyCode}
var string = '<html> hey </html>';
var escapedString = Hoek.escapeHtml(string); // returns &lt;html&gt; hey &lt;/html&gt;
\end{DoxyCode}


\subsubsection*{escape\+Header\+Attribute(attribute)}

Escape attribute value for use in H\+T\+TP header


\begin{DoxyCode}
var a = Hoek.escapeHeaderAttribute('I said "go w\(\backslash\)\(\backslash\)o me"');  //returns I said \(\backslash\)"go w\(\backslash\)\(\backslash\)o me\(\backslash\)"
\end{DoxyCode}


\subsubsection*{escape\+Regex(string)}

Escape string for Regex construction

\`{}\`{}\`{}javascript

var a = Hoek.\+escape\+Regex(\textquotesingle{}4$^\wedge$f\$s.4$\ast$5+-\/\+\_\+?\%=\#!\+:$|$$\sim$\textbackslash{}/`\char`\"{}($>$)\mbox{[}$<$\mbox{]}d\{\}s,');  // returns 4\textbackslash{}$^\wedge$f\textbackslash{}\$s\textbackslash{}.\+4$\ast$5\textbackslash{}+\textbackslash{}-\/\+\_\+\textbackslash{}?\%\textbackslash{}=\#!\textbackslash{}\+:@\textbackslash{}$\vert$$\sim$\textbackslash{}\textbackslash{}\textbackslash{}/\`{}\char`\"{}($>$)\mbox{[}$<$\mbox{]}d 