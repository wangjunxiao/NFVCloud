{\bfseries encoding} is a simple wrapper around \href{https://github.com/bnoordhuis/node-iconv}{\tt node-\/iconv} and \href{https://github.com/ashtuchkin/iconv-lite/}{\tt iconv-\/lite} to convert strings from one encoding to another. If node-\/iconv is not available for some reason, iconv-\/lite will be used instead of it as a fallback.

\href{http://travis-ci.org/andris9/Nodemailer}{\tt } \href{http://badge.fury.io/js/encoding}{\tt }

\subsection*{Install}

Install through npm \begin{DoxyVerb}npm install encoding
\end{DoxyVerb}


\subsection*{Usage}

Require the module \begin{DoxyVerb}var encoding = require("encoding");
\end{DoxyVerb}


Convert with encoding.\+convert() \begin{DoxyVerb}var resultBuffer = encoding.convert(text, toCharset, fromCharset);
\end{DoxyVerb}


Where


\begin{DoxyItemize}
\item {\bfseries text} is either a Buffer or a String to be converted
\item {\bfseries to\+Charset} is the characterset to convert the string
\item {\bfseries from\+Charset} ({\itshape optional}, defaults to U\+T\+F-\/8) is the source charset
\end{DoxyItemize}

Output of the conversion is always a Buffer object.

Example \begin{DoxyVerb}var result = encoding.convert("ÕÄÖÜ", "Latin_1");
console.log(result); //<Buffer d5 c4 d6 dc>
\end{DoxyVerb}


\subsection*{iconv support}

By default only iconv-\/lite is bundled. If you need node-\/iconv support, you need to add it as an additional dependency for your project\+: \begin{DoxyVerb}...,
"dependencies":{
    "encoding": "*",
    "iconv": "*"
},
...
\end{DoxyVerb}


\subsection*{License}

{\bfseries M\+IT} 