\begin{quote}
A smaller version of caniuse-\/db, with only the essentials! \end{quote}


\subsection*{Why?}

The full data behind \href{http://caniuse.com/}{\tt Can I use} is incredibly useful for any front end developer, and on the website all of the details from the database are displayed to the user. However in automated tools, \href{https://github.com/Fyrd/caniuse/issues/1827}{\tt many of these fields go unused}; it\textquotesingle{}s not a problem for server side consumption but client side, the less Java\+Script that we send to the end user the better.

caniuse-\/lite then, is a smaller dataset that keeps essential parts of the data in a compact format. It does this in multiple ways, such as converting {\ttfamily null} array entries into empty strings, representing support data as an integer rather than a string, and using base62 references instead of longer human-\/readable keys.

This packed data is then reassembled (via functions exposed by this module) into a larger format which is mostly compatible with caniuse-\/db, and so it can be used as an almost drop-\/in replacement for caniuse-\/db for contexts where size on disk is important; for example, usage in web browsers. The A\+PI differences are very small and are detailed in the section below.

\subsection*{A\+PI}


\begin{DoxyCode}
import * as lite from 'caniuse-lite';
\end{DoxyCode}


\subsubsection*{{\ttfamily lite.\+agents}}

caniuse-\/db provides a full {\ttfamily data.\+json} file which contains all of the features data. Instead of this large file, caniuse-\/lite provides this data subset instead, which has the {\ttfamily browser}, {\ttfamily prefix}, {\ttfamily prefix\+\_\+exceptions}, {\ttfamily usage\+\_\+global} and {\ttfamily versions} keys from the original.

\subsubsection*{{\ttfamily lite.\+feature(js)}}

The {\ttfamily feature} method takes a file from {\ttfamily data/features} and converts it into something that more closely represents the {\ttfamily caniuse-\/db} format. Note that only the {\ttfamily title}, {\ttfamily stats} and {\ttfamily status} keys are kept from the original data.

\subsubsection*{{\ttfamily lite.\+features}}

The {\ttfamily features} index is provided as a way to query all of the features that are listed in the {\ttfamily caniuse-\/db} dataset. Note that you will need to use the {\ttfamily feature} method on values from this index to get a human-\/readable format.

\subsubsection*{{\ttfamily lite.\+region(js)}}

The {\ttfamily region} method takes a file from {\ttfamily data/regions} and converts it into something that more closely represents the {\ttfamily caniuse-\/db} format. Note that {\itshape only} the usage data is exposed here (the {\ttfamily data} key in the original files).

\subsection*{Contributors}

Thanks goes to these wonderful people (\href{https://github.com/kentcdodds/all-contributors#emoji-key}{\tt emoji key})\+:

$\vert$ \href{http://beneb.info}{\tt \textsubscript{Ben Briggs}}~\newline
\href{https://github.com/ben-eb/caniuse-lite/commits?author=ben-eb}{\tt 💻} \href{https://github.com/ben-eb/caniuse-lite/commits?author=ben-eb}{\tt 📖} 👀 \href{https://github.com/ben-eb/caniuse-lite/commits?author=ben-eb}{\tt ⚠️} $\vert$ \href{https://github.com/andyjansson}{\tt \textsubscript{Andy Jansson}}~\newline
\href{https://github.com/ben-eb/caniuse-lite/commits?author=andyjansson}{\tt 💻} $\vert$ $\vert$ \+:---\+: $\vert$ \+:---\+: $\vert$

This project follows the \href{https://github.com/kentcdodds/all-contributors}{\tt all-\/contributors} specification. Contributions of any kind welcome!

\subsection*{License}

The data in this repo is available for use under a CC BY 4.\+0 license (\href{http://creativecommons.org/licenses/by/4.0/}{\tt http\+://creativecommons.\+org/licenses/by/4.\+0/}). For attribution just mention somewhere that the source is caniuse.\+com. If you have any questions about using the data for your project please contact me here\+: \href{http://a.deveria.com/contact}{\tt http\+://a.\+deveria.\+com/contact} 