\begin{quote}
eslint loader for webpack \end{quote}


\subsection*{Install}


\begin{DoxyCode}
$ npm install eslint-loader --save-dev
\end{DoxyCode}


\subsection*{Usage}

In your webpack configuration


\begin{DoxyCode}
module.exports = \{
  // ...
  module: \{
    rules: [
      \{
        test: /\(\backslash\).js$/,
        exclude: /node\_modules/,
        loader: "eslint-loader",
        options: \{
          // eslint options (if necessary)
        \}
      \},
    ],
  \},
  // ...
\}
\end{DoxyCode}


When using with transpiling loaders (like {\ttfamily babel-\/loader}), make sure they are in correct order (bottom to top). Otherwise files will be check after being processed by {\ttfamily babel-\/loader}


\begin{DoxyCode}
module.exports = \{
  // ...
  module: \{
    rules: [
      \{
        test: /\(\backslash\).js$/,
        exclude: /node\_modules/,
        use: [
          "babel-loader",
          "eslint-loader",
        ],
      \},
    ],
  \},
  // ...
\}
\end{DoxyCode}


To be safe, you can use {\ttfamily enforce\+: \char`\"{}pre\char`\"{}} section to check source files, not modified by other loaders (like {\ttfamily babel-\/loader})


\begin{DoxyCode}
module.exports = \{
  // ...
  module: \{
    rules: [
      \{
        enforce: "pre",
        test: /\(\backslash\).js$/,
        exclude: /node\_modules/,
        loader: "eslint-loader",
      \},
      \{
        test: /\(\backslash\).js$/,
        exclude: /node\_modules/,
        loader: "babel-loader",
      \},
    ],
  \},
  // ...
\}
\end{DoxyCode}


\subsubsection*{Options}

You can pass \href{http://eslint.org/docs/developer-guide/nodejs-api#cliengine}{\tt eslint options} using standard webpack \href{https://webpack.js.org/configuration/module/#useentry}{\tt loader options}.

Note that the config option you provide will be passed to the {\ttfamily C\+L\+I\+Engine}. This is a different set of options than what you\textquotesingle{}d specify in {\ttfamily package.\+json} or {\ttfamily .eslintrc}. See the \href{http://eslint.org/docs/developer-guide/nodejs-api#cliengine}{\tt eslint docs} for more detail.

\paragraph*{{\ttfamily fix} (default\+: false)}

This option will enable \href{http://eslint.org/docs/user-guide/command-line-interface#fix}{\tt E\+S\+Lint autofix feature}.

{\bfseries Be careful\+: this option might cause webpack to enter an infinite build loop if some issues cannot be fixed properly.}

\paragraph*{{\ttfamily cache} (default\+: false)}

This option will enable caching of the linting results into a file. This is particularly useful in reducing linting time when doing a full build.

The cache file is written to the {\ttfamily ./node\+\_\+modules/.cache} directory, thanks to the usage of the \href{https://www.npmjs.com/package/find-cache-dir}{\tt find-\/cache-\/dir} module.

\paragraph*{{\ttfamily formatter} (default\+: eslint stylish formatter)}

Loader accepts a function that will have one argument\+: an array of eslint messages (object). The function must return the output as a string. You can use official eslint formatters.


\begin{DoxyCode}
module.exports = \{
  entry: "...",
  module: \{
    rules: [
      \{
        test: /\(\backslash\).js$/,
        exclude: /node\_modules/,
        loader: "eslint-loader",
        options: \{
          // several examples !

          // default value
          formatter: require("eslint/lib/formatters/stylish"),

          // community formatter
          formatter: require("eslint-friendly-formatter"),

          // custom formatter
          formatter: function(results) \{
            // `results` format is available here
            // http://eslint.org/docs/developer-guide/nodejs-api.html#executeonfiles()

            // you should return a string
            // DO NOT USE console.*() directly !
            return "OUTPUT"
          \}
        \}
      \},
    ],
  \},
\}
\end{DoxyCode}


\paragraph*{Errors and Warning}

{\bfseries By default the loader will auto adjust error reporting depending on eslint errors/warnings counts.} You can still force this behavior by using {\ttfamily emit\+Error} {\bfseries or} {\ttfamily emit\+Warning} options\+:

\subparagraph*{{\ttfamily emit\+Error} (default\+: {\ttfamily false})}

Loader will always return errors if this option is set to {\ttfamily true}.


\begin{DoxyCode}
module.exports = \{
  entry: "...",
  module: \{
    rules: [
      \{
        test: /\(\backslash\).js$/,
        exclude: /node\_modules/,
        loader: "eslint-loader",
        options: \{
          emitError: true,
        \}
      \},
    ],
  \},
\}
\end{DoxyCode}


\subparagraph*{{\ttfamily emit\+Warning} (default\+: {\ttfamily false})}

Loader will always return warnings if option is set to {\ttfamily true}.

\paragraph*{{\ttfamily quiet} (default\+: {\ttfamily false})}

Loader will process and report errors only and ignore warnings if this option is set to true


\begin{DoxyCode}
module.exports = \{
  entry: "...",
  module: \{
    rules: [
      \{
        test: /\(\backslash\).js$/,
        exclude: /node\_modules/,
        loader: "eslint-loader",
        options: \{
          quiet: true,
        \}
      \},
    ],
  \},
\}
\end{DoxyCode}


\subparagraph*{{\ttfamily fail\+On\+Warning} (default\+: {\ttfamily false})}

Loader will cause the module build to fail if there are any eslint warnings.


\begin{DoxyCode}
module.exports = \{
  entry: "...",
  module: \{
    rules: [
      \{
        test: /\(\backslash\).js$/,
        exclude: /node\_modules/,
        loader: "eslint-loader",
        options: \{
          failOnWarning: true,
        \}
      \},
    ],
  \},
\}
\end{DoxyCode}


\subparagraph*{{\ttfamily fail\+On\+Error} (default\+: {\ttfamily false})}

Loader will cause the module build to fail if there are any eslint errors.


\begin{DoxyCode}
module.exports = \{
  entry: "...",
  module: \{
    rules: [
      \{
        test: /\(\backslash\).js$/,
        exclude: /node\_modules/,
        loader: "eslint-loader",
        options: \{
          failOnError: true,
        \}
      \},
    ],
  \},
\}
\end{DoxyCode}


\subparagraph*{{\ttfamily output\+Report} (default\+: {\ttfamily false})}

Write the output of the errors to a file, for example a checkstyle xml file for use for reporting on Jenkins CI

The {\ttfamily file\+Path} is relative to the webpack config\+: output.\+path You can pass in a different formatter for the output file, if none is passed in the default/configured formatter will be used


\begin{DoxyCode}
module.exports = \{
  entry: "...",
  module: \{
    rules: [
      \{
        test: /\(\backslash\).js$/,
        exclude: /node\_modules/,
        loader: "eslint-loader",
        options: \{
          outputReport: \{
            filePath: 'checkstyle.xml',
            formatter: require('eslint/lib/formatters/checkstyle')
          \}
        \}
      \},
    ],
  \},
\}
\end{DoxyCode}


\subsection*{Gotchas}

\subsubsection*{No\+Errors\+Plugin}

{\ttfamily No\+Errors\+Plugin} prevents webpack from outputting anything into a bundle. So even E\+S\+Lint warnings will fail the build. No matter what error settings are used for {\ttfamily eslint-\/loader}.

So if you want to see E\+S\+Lint warnings in console during development using {\ttfamily Webpack\+Dev\+Server} remove {\ttfamily No\+Errors\+Plugin} from webpack config.

\subsubsection*{Defining {\ttfamily config\+File} or using {\ttfamily eslint -\/c path/.eslintrc}}

Bear in mind that when you define {\ttfamily config\+File}, {\ttfamily eslint} doesn\textquotesingle{}t automatically look for {\ttfamily .eslintrc} files in the directory of the file to be linted. More information is available in official eslint documentation in section \href{http://eslint.org/docs/user-guide/configuring#using-configuration-files}{\tt {\itshape Using Configuration Files}}. See \href{https://github.com/MoOx/eslint-loader/issues/129}{\tt \#129}. 



\subsection*{C\+H\+A\+N\+G\+E\+L\+O\+G.\+md \char`\"{}\+Changelog\char`\"{}}

\subsection*{\mbox{[}License\mbox{]}(L\+I\+C\+E\+N\+SE)}