\section*{connect-\/history-\/api-\/fallback}

Middleware to proxy requests through a specified index page, useful for Single Page Applications that utilise the H\+T\+M\+L5 \mbox{\hyperlink{interface_history}{History}} A\+PI.

\href{https://travis-ci.org/bripkens/connect-history-api-fallback}{\tt } \href{https://david-dm.org/bripkens/connect-history-api-fallback/master}{\tt }

\href{https://nodei.co/npm/connect-history-api-fallback/}{\tt }

\subsection*{Introduction}

Single Page Applications (S\+PA) typically only utilise one index file that is accessible by web browsers\+: usually {\ttfamily index.\+html}. Navigation in the application is then commonly handled using Java\+Script with the help of the \href{http://www.w3.org/html/wg/drafts/html/master/single-page.html#the-history-interface}{\tt H\+T\+M\+L5 History A\+PI}. This results in issues when the user hits the refresh button or is directly accessing a page other than the landing page, e.\+g. {\ttfamily /help} or {\ttfamily /help/online} as the web server bypasses the index file to locate the file at this location. As your application is a S\+PA, the web server will fail trying to retrieve the file and return a {\itshape 404 -\/ Not Found} message to the user.

This tiny middleware addresses some of the issues. Specifically, it will change the requested location to the index you specify (default being {\ttfamily /index.html}) whenever there is a request which fulfills the following criteria\+:


\begin{DoxyEnumerate}
\item The request is a G\+ET request
\item which accepts {\ttfamily text/html},
\item is not a direct file request, i.\+e. the requested path does not contain a {\ttfamily .} (D\+OT) character and
\item does not match a pattern provided in options.\+rewrites (see options below)
\end{DoxyEnumerate}

\subsection*{Usage}

The middleware is available through N\+PM and can easily be added.


\begin{DoxyCode}
npm install --save connect-history-api-fallback
\end{DoxyCode}


Import the library


\begin{DoxyCode}
var history = require('connect-history-api-fallback');
\end{DoxyCode}


Now you only need to add the middleware to your application like so


\begin{DoxyCode}
var connect = require('connect');

var app = connect()
  .use(history())
  .listen(3000);
\end{DoxyCode}


Of course you can also use this piece of middleware with express\+:


\begin{DoxyCode}
var express = require('express');

var app = express();
app.use(history());
\end{DoxyCode}


\subsection*{Options}

You can optionally pass options to the library when obtaining the middleware


\begin{DoxyCode}
var middleware = history(\{\});
\end{DoxyCode}


\subsubsection*{index}

Override the index (default {\ttfamily /index.html})


\begin{DoxyCode}
history(\{
  index: '/default.html'
\});
\end{DoxyCode}


\subsubsection*{rewrites}

Override the index when the request url matches a regex pattern. You can either rewrite to a static string or use a function to transform the incoming request.

The following will rewrite a request that matches the {\ttfamily /\textbackslash{}/soccer/} pattern to {\ttfamily /soccer.html}. 
\begin{DoxyCode}
history(\{
  rewrites: [
    \{ from: /\(\backslash\)/soccer/, to: '/soccer.html'\}
  ]
\});
\end{DoxyCode}


Alternatively functions can be used to have more control over the rewrite process. For instance, the following listing shows how requests to {\ttfamily /libs/jquery/jquery.1.\+12.\+0.\+min.\+js} and the like can be routed to {\ttfamily ./bower\+\_\+components/libs/jquery/jquery.1.\+12.\+0.\+min.\+js}. You can also make use of this if you have an A\+PI version in the \mbox{\hyperlink{namespace_u_r_l}{U\+RL}} path. 
\begin{DoxyCode}
history(\{
  rewrites: [
    \{
      from: /^\(\backslash\)/libs\(\backslash\)/.*$/,
      to: function(context) \{
        return '/bower\_components' + context.parsedUrl.pathname;
      \}
    \}
  ]
\});
\end{DoxyCode}


The function will always be called with a context object that has the following properties\+:


\begin{DoxyItemize}
\item {\bfseries parsed\+Url}\+: Information about the \mbox{\hyperlink{namespace_u_r_l}{U\+RL}} as provided by the \href{https://nodejs.org/api/url.html#url_url_parse_urlstr_parsequerystring_slashesdenotehost}{\tt U\+RL module\textquotesingle{}s} {\ttfamily url.\+parse}.
\item {\bfseries match}\+: An Array of matched results as provided by {\ttfamily String.\+match(...)}.
\end{DoxyItemize}

\subsubsection*{verbose}

This middleware does not log any information by default. If you wish to activate logging, then you can do so via the {\ttfamily verbose} option or by specifying a logger function.


\begin{DoxyCode}
history(\{
  verbose: true
\});
\end{DoxyCode}


Alternatively use your own logger


\begin{DoxyCode}
history(\{
  logger: console.log.bind(console)
\});
\end{DoxyCode}


\subsubsection*{html\+Accept\+Headers}

Override the default {\ttfamily Accepts\+:} headers that are queried when matching H\+T\+ML content requests (Default\+: `\mbox{[}\textquotesingle{}text/html', \textquotesingle{}$\ast$/$\ast$\textquotesingle{}\mbox{]}\`{}).


\begin{DoxyCode}
history(\{
  htmlAcceptHeaders: ['text/html', 'application/xhtml+xml']
\})
\end{DoxyCode}


\subsubsection*{disable\+Dot\+Rule}

Disables the dot rule mentioned above\+:

\begin{quote}
\mbox{[}…\mbox{]} is not a direct file request, i.\+e. the requested path does not contain a {\ttfamily .} (D\+OT) character \mbox{[}…\mbox{]} \end{quote}



\begin{DoxyCode}
history(\{
  disableDotRule: true
\})
\end{DoxyCode}
 