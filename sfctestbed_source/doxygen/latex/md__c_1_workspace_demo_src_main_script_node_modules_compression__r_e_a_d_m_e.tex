\href{https://npmjs.org/package/compression}{\tt } \href{https://npmjs.org/package/compression}{\tt } \href{https://travis-ci.org/expressjs/compression}{\tt } \href{https://coveralls.io/r/expressjs/compression?branch=master}{\tt } \href{https://www.gratipay.com/dougwilson/}{\tt }

Node.\+js compression middleware.

The following compression codings are supported\+:


\begin{DoxyItemize}
\item deflate
\item gzip
\end{DoxyItemize}

\subsection*{Install}


\begin{DoxyCode}
$ npm install compression
\end{DoxyCode}


\subsection*{A\+PI}


\begin{DoxyCode}
var compression = require('compression')
\end{DoxyCode}


\subsubsection*{compression(\mbox{[}options\mbox{]})}

Returns the compression middleware using the given {\ttfamily options}. The middleware will attempt to compress response bodies for all request that traverse through the middleware, based on the given {\ttfamily options}.

This middleware will never compress responses that include a {\ttfamily Cache-\/\+Control} header with the \href{https://tools.ietf.org/html/rfc7234#section-5.2.2.4}{\tt {\ttfamily no-\/transform} directive}, as compressing will transform the body.

\paragraph*{Options}

{\ttfamily compression()} accepts these properties in the options object. In addition to those listed below, \href{http://nodejs.org/api/zlib.html}{\tt zlib} options may be passed in to the options object.

\subparagraph*{chunk\+Size}

The default value is {\ttfamily zlib.\+Z\+\_\+\+D\+E\+F\+A\+U\+L\+T\+\_\+\+C\+H\+U\+NK}, or {\ttfamily 16384}.

See \href{http://nodejs.org/api/zlib.html#zlib_memory_usage_tuning}{\tt Node.\+js documentation} regarding the usage.

\subparagraph*{filter}

A function to decide if the response should be considered for compression. This function is called as {\ttfamily filter(req, res)} and is expected to return {\ttfamily true} to consider the response for compression, or {\ttfamily false} to not compress the response.

The default filter function uses the \href{https://www.npmjs.com/package/compressible}{\tt compressible} module to determine if `res.\+get\+Header(\textquotesingle{}Content-\/\+Type')\`{} is compressible.

\subparagraph*{level}

The level of zlib compression to apply to responses. A higher level will result in better compression, but will take longer to complete. A lower level will result in less compression, but will be much faster.

This is an integer in the range of {\ttfamily 0} (no compression) to {\ttfamily 9} (maximum compression). The special value {\ttfamily -\/1} can be used to mean the \char`\"{}default
compression level\char`\"{}, which is a default compromise between speed and compression (currently equivalent to level 6).


\begin{DoxyItemize}
\item {\ttfamily -\/1} Default compression level (also {\ttfamily zlib.\+Z\+\_\+\+D\+E\+F\+A\+U\+L\+T\+\_\+\+C\+O\+M\+P\+R\+E\+S\+S\+I\+ON}).
\item {\ttfamily 0} No compression (also {\ttfamily zlib.\+Z\+\_\+\+N\+O\+\_\+\+C\+O\+M\+P\+R\+E\+S\+S\+I\+ON}).
\item {\ttfamily 1} Fastest compression (also {\ttfamily zlib.\+Z\+\_\+\+B\+E\+S\+T\+\_\+\+S\+P\+E\+ED}).
\item {\ttfamily 2}
\item {\ttfamily 3}
\item {\ttfamily 4}
\item {\ttfamily 5}
\item {\ttfamily 6} (currently what {\ttfamily zlib.\+Z\+\_\+\+D\+E\+F\+A\+U\+L\+T\+\_\+\+C\+O\+M\+P\+R\+E\+S\+S\+I\+ON} points to).
\item {\ttfamily 7}
\item {\ttfamily 8}
\item {\ttfamily 9} Best compression (also {\ttfamily zlib.\+Z\+\_\+\+B\+E\+S\+T\+\_\+\+C\+O\+M\+P\+R\+E\+S\+S\+I\+ON}).
\end{DoxyItemize}

The default value is {\ttfamily zlib.\+Z\+\_\+\+D\+E\+F\+A\+U\+L\+T\+\_\+\+C\+O\+M\+P\+R\+E\+S\+S\+I\+ON}, or {\ttfamily -\/1}.

{\bfseries Note} in the list above, {\ttfamily zlib} is from `zlib = require(\textquotesingle{}zlib')\`{}.

\subparagraph*{mem\+Level}

This specifies how much memory should be allocated for the internal compression state and is an integer in the range of {\ttfamily 1} (minimum level) and {\ttfamily 9} (maximum level).

The default value is {\ttfamily zlib.\+Z\+\_\+\+D\+E\+F\+A\+U\+L\+T\+\_\+\+M\+E\+M\+L\+E\+V\+EL}, or {\ttfamily 8}.

See \href{http://nodejs.org/api/zlib.html#zlib_memory_usage_tuning}{\tt Node.\+js documentation} regarding the usage.

\subparagraph*{strategy}

This is used to tune the compression algorithm. This value only affects the compression ratio, not the correctness of the compressed output, even if it is not set appropriately.


\begin{DoxyItemize}
\item {\ttfamily zlib.\+Z\+\_\+\+D\+E\+F\+A\+U\+L\+T\+\_\+\+S\+T\+R\+A\+T\+E\+GY} Use for normal data.
\item {\ttfamily zlib.\+Z\+\_\+\+F\+I\+L\+T\+E\+R\+ED} Use for data produced by a filter (or predictor). Filtered data consists mostly of small values with a somewhat random distribution. In this case, the compression algorithm is tuned to compress them better. The effect is to force more Huffman coding and less string matching; it is somewhat intermediate between {\ttfamily zlib.\+Z\+\_\+\+D\+E\+F\+A\+U\+L\+T\+\_\+\+S\+T\+R\+A\+T\+E\+GY} and {\ttfamily zlib.\+Z\+\_\+\+H\+U\+F\+F\+M\+A\+N\+\_\+\+O\+N\+LY}.
\item {\ttfamily zlib.\+Z\+\_\+\+F\+I\+X\+ED} Use to prevent the use of dynamic Huffman codes, allowing for a simpler decoder for special applications.
\item {\ttfamily zlib.\+Z\+\_\+\+H\+U\+F\+F\+M\+A\+N\+\_\+\+O\+N\+LY} Use to force Huffman encoding only (no string match).
\item {\ttfamily zlib.\+Z\+\_\+\+R\+LE} Use to limit match distances to one (run-\/length encoding). This is designed to be almost as fast as {\ttfamily zlib.\+Z\+\_\+\+H\+U\+F\+F\+M\+A\+N\+\_\+\+O\+N\+LY}, but give better compression for P\+NG image data.
\end{DoxyItemize}

{\bfseries Note} in the list above, {\ttfamily zlib} is from `zlib = require(\textquotesingle{}zlib')\`{}.

\subparagraph*{threshold}

The byte threshold for the response body size before compression is considered for the response, defaults to {\ttfamily 1kb}. This is a number of bytes, any string accepted by the \href{https://www.npmjs.com/package/bytes}{\tt bytes} module, or {\ttfamily false}.

{\bfseries Note} this is only an advisory setting; if the response size cannot be determined at the time the response headers are written, then it is assumed the response is {\itshape over} the threshold. To guarantee the response size can be determined, be sure set a {\ttfamily Content-\/\+Length} response header.

\subparagraph*{window\+Bits}

The default value is {\ttfamily zlib.\+Z\+\_\+\+D\+E\+F\+A\+U\+L\+T\+\_\+\+W\+I\+N\+D\+O\+W\+B\+I\+TS}, or {\ttfamily 15}.

See \href{http://nodejs.org/api/zlib.html#zlib_memory_usage_tuning}{\tt Node.\+js documentation} regarding the usage.

\paragraph*{.filter}

The default {\ttfamily filter} function. This is used to construct a custom filter function that is an extension of the default function.


\begin{DoxyCode}
app.use(compression(\{filter: shouldCompress\}))

function shouldCompress(req, res) \{
  if (req.headers['x-no-compression']) \{
    // don't compress responses with this request header
    return false
  \}

  // fallback to standard filter function
  return compression.filter(req, res)
\}
\end{DoxyCode}


\subsubsection*{res.\+flush}

This module adds a {\ttfamily res.\+flush()} method to force the partially-\/compressed response to be flushed to the client.

\subsection*{Examples}

\subsubsection*{express/connect}

When using this module with express or connect, simply {\ttfamily app.\+use} the module as high as you like. Requests that pass through the middleware will be compressed.


\begin{DoxyCode}
var compression = require('compression')
var express = require('express')

var app = express()

// compress all requests
app.use(compression())

// add all routes
\end{DoxyCode}


\subsubsection*{Server-\/\+Sent Events}

Because of the nature of compression this module does not work out of the box with server-\/sent events. To compress content, a window of the output needs to be buffered up in order to get good compression. Typically when using server-\/sent events, there are certain block of data that need to reach the client.

You can achieve this by calling {\ttfamily res.\+flush()} when you need the data written to actually make it to the client.


\begin{DoxyCode}
var compression = require('compression')
var express     = require('express')

var app = express()

// compress responses
app.use(compression())

// server-sent event stream
app.get('/events', function (req, res) \{
  res.setHeader('Content-Type', 'text/event-stream')
  res.setHeader('Cache-Control', 'no-cache')

  // send a ping approx every 2 seconds
  var timer = setInterval(function () \{
    res.write('data: ping\(\backslash\)n\(\backslash\)n')

    // !!! this is the important part
    res.flush()
  \}, 2000)

  res.on('close', function () \{
    clearInterval(timer)
  \})
\})
\end{DoxyCode}


\subsection*{License}

\mbox{[}M\+IT\mbox{]}(L\+I\+C\+E\+N\+SE) 