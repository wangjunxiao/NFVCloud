Send cross platform native notifications using Node.\+js. Notification Center for mac\+OS, notify-\/osd/libnotify-\/bin for Linux, Toasters for Windows 8/10, or taskbar Balloons for earlier Windows versions. Growl is used if none of these requirements are met. \href{#within-electron-packaging}{\tt Works well with electron}.

 

\subsection*{Input Example mac\+OS Notification Center}



\subsection*{Quick Usage}

Show a native notification on mac\+OS, Windows, Linux\+:


\begin{DoxyCode}
const notifier = require('node-notifier');
// String
notifier.notify('Message');

// Object
notifier.notify(\{
  'title': 'My notification',
  'message': 'Hello, there!'
\});
\end{DoxyCode}


\subsection*{Requirements}


\begin{DoxyItemize}
\item {\bfseries mac\+OS}\+: $>$= 10.\+8 or Growl if earlier.
\item {\bfseries Linux}\+: {\ttfamily notify-\/osd} or {\ttfamily libnotify-\/bin} installed (Ubuntu should have this by default)
\item {\bfseries Windows}\+: $>$= 8, task bar balloon for Windows $<$ 8. Growl as fallback. Growl takes precedence over Windows balloons.
\item {\bfseries General Fallback}\+: Growl
\end{DoxyItemize}

See ./\+D\+E\+C\+I\+S\+I\+O\+N\+\_\+\+F\+L\+OW.md \char`\"{}documentation and flow chart for reporter choice\char`\"{}

\#\# Install 
\begin{DoxyCode}
npm install --save node-notifier
\end{DoxyCode}


\subsection*{Cross-\/\+Platform Advanced Usage}

Standard usage, with cross-\/platform fallbacks as defined in the ./\+D\+E\+C\+I\+S\+I\+O\+N\+\_\+\+F\+L\+OW.md \char`\"{}reporter flow chart\char`\"{}. All of the options below will work in a way or another on all platforms.


\begin{DoxyCode}
const notifier = require('node-notifier');
const path = require('path');

notifier.notify(\{
  title: 'My awesome title',
  message: 'Hello from node, Mr. User!',
  icon: path.join(\_\_dirname, 'coulson.jpg'), // Absolute path (doesn't work on balloons)
  sound: true, // Only Notification Center or Windows Toasters
  wait: true // Wait with callback, until user action is taken against notification
\}, function (err, response) \{
  // Response is response from notification
\});

notifier.on('click', function (notifierObject, options) \{
  // Triggers if `wait: true` and user clicks notification
\});

notifier.on('timeout', function (notifierObject, options) \{
  // Triggers if `wait: true` and notification closes
\});
\end{DoxyCode}


You can also specify what reporter you want to use if you want to customize it or have more specific options per system. See documentation for each reporter below.

Example\+: 
\begin{DoxyCode}
const NotificationCenter = require('node-notifier/notifiers/notificationcenter');
new NotificationCenter(options).notify();

const NotifySend = require('node-notifier/notifiers/notifysend');
new NotifySend(options).notify();

const WindowsToaster = require('node-notifier/notifiers/toaster');
new WindowsToaster(options).notify();

const Growl = require('node-notifier/notifiers/growl');
new Growl(options).notify();

const WindowsBalloon = require('node-notifier/notifiers/balloon');
new WindowsBalloon(options).notify();
\end{DoxyCode}


Or if you are using several (or you are lazy)\+: (note\+: technically, this takes longer to require)


\begin{DoxyCode}
const nn = require('node-notifier');

new nn.NotificationCenter(options).notify();
new nn.NotifySend(options).notify();
new nn.WindowsToaster(options).notify(options);
new nn.WindowsBalloon(options).notify(options);
new nn.Growl(options).notify(options);
\end{DoxyCode}


\subsection*{Contents}


\begin{DoxyItemize}
\item \href{#usage-notificationcenter}{\tt Notification Center documentation}
\item \href{#usage-windowstoaster}{\tt Windows Toaster documentation}
\item \href{#usage-windowsballoon}{\tt Windows Balloon documentation}
\item \href{#usage-growl}{\tt Growl documentation}
\item \href{#usage-notifysend}{\tt Notify-\/send documentation}
\end{DoxyItemize}

\subsubsection*{Usage Notification\+Center}

Same usage and parameter setup as \href{https://github.com/julienXX/terminal-notifier}{\tt terminal-\/notifier}.

Native Notification Center requires mac\+OS version 10.\+8 or higher. If you have an earlier version, Growl will be the fallback. If Growl isn\textquotesingle{}t installed, an error will be returned in the callback.

\paragraph*{Example}

Wrapping around \href{https://github.com/julienXX/terminal-notifier}{\tt terminal-\/notifier}, you can do all terminal-\/notifier can do through properties to the {\ttfamily notify} method. E.\+g. if {\ttfamily terminal-\/notifier} says {\ttfamily -\/message}, you can do `\{message\+: \textquotesingle{}Foo'\}{\ttfamily , or if}terminal-\/notifier{\ttfamily says}-\/list A\+LL{\ttfamily , you can do}\{list\+: \textquotesingle{}A\+LL\textquotesingle{}\}\`{}. Notification is the primary focus for this module, so listing and activating do work, but isn\textquotesingle{}t documented.

\subsubsection*{All notification options with their defaults\+:}


\begin{DoxyCode}
const NotificationCenter = require('node-notifier').NotificationCenter;

var notifier = new NotificationCenter(\{
  withFallback: false, // Use Growl Fallback if <= 10.8
  customPath: void 0 // Relative/Absolute path to binary if you want to use your own fork of
       terminal-notifier
\});

notifier.notify(\{
  'title': void 0,
  'subtitle': void 0,
  'message': void 0,
  'sound': false, // Case Sensitive string for location of sound file, or use one of macOS' native sounds
       (see below)
  'icon': 'Terminal Icon', // Absolute Path to Triggering Icon
  'contentImage': void 0, // Absolute Path to Attached Image (Content Image)
  'open': void 0, // URL to open on Click
  'wait': false, // Wait for User Action against Notification or times out. Same as timeout = 5 seconds

  // New in latest version. See `example/macInput.js` for usage
  timeout: 5, // Takes precedence over wait if both are defined.
  closeLabel: void 0, // String. Label for cancel button
  actions: void 0, // String | Array<String>. Action label or list of labels in case of dropdown
  dropdownLabel: void 0, // String. Label to be used if multiple actions
  reply: false // Boolean. If notification should take input. Value passed as third argument in callback
       and event emitter.
\}, function(error, response, metadata) \{
  console.log(response, metadata);
\});
\end{DoxyCode}


{\bfseries Note\+:} {\ttfamily wait} option is shorthand for {\ttfamily timeout\+: 5} and doesn\textquotesingle{}t make the notification sticky, but sets timeout for 5 seconds. Without {\ttfamily wait} or {\ttfamily timeout} notifications are just fired and forgotten Without given any response. To be able to listen for response (like activation/clicked), you have to define a timeout. This is not true if you have defined {\ttfamily reply}. If using {\ttfamily reply} it\textquotesingle{}s recommended to set a high timeout or no timeout at all.

{\bfseries For mac\+OS notifications, icon and content\+Image, and all forms of reply/actions requires mac\+OS 10.\+9.}

Sound can be one of these\+: {\ttfamily Basso}, {\ttfamily Blow}, {\ttfamily Bottle}, {\ttfamily Frog}, {\ttfamily Funk}, {\ttfamily Glass}, {\ttfamily Hero}, {\ttfamily Morse}, {\ttfamily Ping}, {\ttfamily Pop}, {\ttfamily Purr}, {\ttfamily Sosumi}, {\ttfamily Submarine}, {\ttfamily Tink}. If sound is simply {\ttfamily true}, {\ttfamily Bottle} is used.

See \href{./example/advanced.js}{\tt specific Notification Center example}. Also, \href{./example/macInput.js}{\tt see input example}.

{\bfseries Custom Path clarification}

{\ttfamily custom\+Path} takes a value of a relative or absolute path to the binary of your fork/custom version of terminal-\/notifier.

Example\+: {\ttfamily ./vendor/terminal-\/notifier.app/\+Contents/\+Mac\+O\+S/terminal-\/notifier}

\subsubsection*{Usage Windows\+Toaster}

{\bfseries Note\+:} There are some limitations for images in native Windows 8 notifications\+: The image must be a P\+NG image, and cannot be over 1024x1024 px, or over over 200\+Kb. You also need to specify the image by using an absolute path. These limitations are due to the Toast notification system. A good tip is to use something like {\ttfamily path.\+join} or {\ttfamily path.\+delimiter} to have cross-\/platform pathing.

{\bfseries Windows 10 Note\+:} You might have to activate banner notification for the toast to show.

From \href{https://github.com/mikaelbr/gulp-notify/issues/90#issuecomment-129333034}{\tt mikaelbr/gulp-\/notify\#90 (comment)} \begin{quote}
You can make it work by going to System $>$ Notifications \& Actions. The \textquotesingle{}toast\textquotesingle{} app needs to have Banners enabled. (You can activate banners by clicking on the \textquotesingle{}toast\textquotesingle{} app and setting the \textquotesingle{}Show notification banners\textquotesingle{} to On) \end{quote}


\href{https://github.com/KDE/snoretoast}{\tt Snoretoast} is used to get native Windows Toasts!


\begin{DoxyCode}
const WindowsToaster = require('node-notifier').WindowsToaster;

var notifier = new WindowsToaster(\{
  withFallback: false, // Fallback to Growl or Balloons?
  customPath: void 0 // Relative/Absolute path if you want to use your fork of SnoreToast.exe
\});

notifier.notify(\{
  title: void 0, // String. Required
  message: void 0, // String. Required if remove is not defined
  icon: void 0, // String. Absolute path to Icon
  sound: false, // Bool | String (as defined by
       http://msdn.microsoft.com/en-us/library/windows/apps/hh761492.aspx)
  wait: false, // Bool. Wait for User Action against Notification or times out
  id: void 0, // Number. ID to use for closing notification.
  appID: void 0, // String. App.ID and app Name. Defaults to empty string.
  remove: void 0, // Number. Refer to previously created notification to close.
  install: void 0 // String (path, application, app id).  Creates a shortcut <path> in the start menu which
       point to the executable <application>, appID used for the notifications.
\}, function(error, response) \{
  console.log(response);
\});
\end{DoxyCode}


\subsubsection*{Usage Growl}


\begin{DoxyCode}
const Growl = require('node-notifier').Growl;

var notifier = new Growl(\{
  name: 'Growl Name Used', // Defaults as 'Node'
  host: 'localhost',
  port: 23053
\});

notifier.notify(\{
  title: 'Foo',
  message: 'Hello World',
  icon: fs.readFileSync(\_\_dirname + '/coulson.jpg'),
  wait: false, // Wait for User Action against Notification

  // and other growl options like sticky etc.
  sticky: false,
  label: void 0,
  priority: void 0
\});
\end{DoxyCode}


See more information about using \href{https://github.com/theabraham/growly/}{\tt growly}.

\subsubsection*{Usage Windows\+Balloon}

For earlier Windows versions, the taskbar balloons are used (unless fallback is activated and Growl is running). For balloons, a great project called \href{http://www.paralint.com/projects/notifu/}{\tt notifu} is used.


\begin{DoxyCode}
const WindowsBalloon = require('node-notifier').WindowsBalloon;

var notifier = new WindowsBalloon(\{
  withFallback: false, // Try Windows Toast and Growl first?
  customPath: void 0 // Relative/Absolute path if you want to use your fork of notifu
\});

notifier.notify(\{
  title: void 0,
  message: void 0,
  sound: false, // true | false.
  time: 5000, // How long to show balloon in ms
  wait: false, // Wait for User Action against Notification
  type: 'info' // The notification type : info | warn | error
\}, function(error, response) \{
  console.log(response);
\});
\end{DoxyCode}


See full usage on the \href{http://www.paralint.com/projects/notifu/}{\tt project homepage\+: notifu}.

\subsubsection*{Usage Notify\+Send}

Note\+: notify-\/send doesn\textquotesingle{}t support the wait flag.


\begin{DoxyCode}
const NotifySend = require('node-notifier').NotifySend;

var notifier = new NotifySend();

notifier.notify(\{
  title: 'Foo',
  message: 'Hello World',
  icon: \_\_dirname + '/coulson.jpg',

  // .. and other notify-send flags:
  urgency: void 0,
  time: void 0,
  category: void 0,
  hint: void 0,
\});
\end{DoxyCode}


See flags and options \href{http://manpages.ubuntu.com/manpages/gutsy/man1/notify-send.1.html}{\tt on the man pages}

\subsection*{C\+LI}

C\+LI is moved to separate project\+: \href{https://github.com/mikaelbr/node-notifier-cli}{\tt https\+://github.\+com/mikaelbr/node-\/notifier-\/cli}

\subsection*{Thanks to O\+SS}

{\ttfamily node-\/notifier} is made possible through Open Source Software. A very special thanks to all the modules {\ttfamily node-\/notifier} uses.
\begin{DoxyItemize}
\item \href{https://github.com/julienXX/terminal-notifier}{\tt terminal-\/notifier}
\item \href{https://github.com/KDE/snoretoast}{\tt Snoretoast}
\item \href{http://www.paralint.com/projects/notifu/}{\tt notifu}
\item \href{https://github.com/theabraham/growly/}{\tt growly}
\end{DoxyItemize}

\href{https://npmjs.org/package/node-notifier}{\tt }

\subsection*{Common Issues}

\subsubsection*{Use inside tmux session}

When using node-\/notifier within a tmux session, it can cause a hang in the system. This can be solved by following the steps described in this comment\+: \href{https://github.com/julienXX/terminal-notifier/issues/115#issuecomment-104214742}{\tt https\+://github.\+com/julien\+X\+X/terminal-\/notifier/issues/115\#issuecomment-\/104214742}

See more info here\+: \href{https://github.com/mikaelbr/node-notifier/issues/61#issuecomment-163560801}{\tt https\+://github.\+com/mikaelbr/node-\/notifier/issues/61\#issuecomment-\/163560801}

\subsubsection*{Custom icon without terminal icon on mac\+OS}

Even if you define an icon in the configuration object for {\ttfamily node-\/notifier}, you will see a small Terminal icon in the notification (see the example at the top of this document). This is the way notifications on mac\+OS work, it always show the parent icon of the application initiating the notification. For node-\/notifier, terminal-\/notifier is the initiator and has Terminal icon defined as its icon. To define your custom icon, you need to fork terminal-\/notifier and build your custom version with your icon. See this issue for more info\+: \href{https://github.com/mikaelbr/node-notifier/issues/71}{\tt https\+://github.\+com/mikaelbr/node-\/notifier/issues/71}

\subsubsection*{Within Electron Packaging}

If packaging your Electron app as an {\ttfamily asar}, you will find node-\/notifier will fail to load. Due to the way asar works, you cannot execute a binary from within an asar. As a simple solution, when packaging the app into an asar please make sure you {\ttfamily -\/-\/unpack} the vendor folder of node-\/notifier, so the module still has access to the notification binaries. To do this, you can do so by using the following command\+:


\begin{DoxyCode}
asar pack . app.asar --unpack "./node\_modules/node-notifier/vendor/**"
\end{DoxyCode}


\subsubsection*{Using Webpack}

When using node-\/notifier inside of webpack, you must add the following snippet to your {\ttfamily webpack.\+config.\+js}. The reason this is required, is because node-\/notifier loads the notifiers from a binary, and so a relative file path is needed. When webpack compiles the modules, it supresses file directories, causing node-\/notifier to error on certain platforms. To fix/workaround this, you must tell webpack to keep the relative file directories, by doing so, append the following code to your {\ttfamily webpack.\+config.\+js}


\begin{DoxyCode}
node: \{
  \_\_filename: true,
  \_\_dirname: true
\}
\end{DoxyCode}


\subsection*{License}

\href{http://en.wikipedia.org/wiki/MIT_License}{\tt M\+IT License} 