This example is a copy of example from \href{https://github.com/sockjs/websocket-multiplex/}{\tt websocket-\/multiplex} project\+:


\begin{DoxyItemize}
\item \href{https://github.com/sockjs/websocket-multiplex/}{\tt https\+://github.\+com/sockjs/websocket-\/multiplex/}
\end{DoxyItemize}

To run this example, first install dependencies\+: \begin{DoxyVerb}npm install
\end{DoxyVerb}


And run a server\+: \begin{DoxyVerb}node server.js
\end{DoxyVerb}


That will spawn an http server at \href{http://127.0.0.1:9999/}{\tt http\+://127.\+0.\+0.\+1\+:9999/} which will serve both html (served from the current directory) and also Sock\+JS service (under the \href{http://127.0.0.1:9999/multiplex}{\tt /multiplex} path).

With that set up, Web\+Socket-\/multiplex is able to push three virtual connections over a single Sock\+JS connection. See the code for details. 