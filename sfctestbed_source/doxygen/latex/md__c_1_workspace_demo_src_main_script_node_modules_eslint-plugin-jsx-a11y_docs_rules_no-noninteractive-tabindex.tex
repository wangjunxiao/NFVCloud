Tab key navigation should be limited to elements on the page that can be interacted with. Thus it is not necessary to add a tabindex to items in an unordered list, for example, to make them navigable through assistive technology. These applications already afford page traversal mechanisms based on the H\+T\+ML of the page. Generally, we should try to reduce the size of the page\textquotesingle{}s tab ring rather than increasing it.

\subsection*{How do I resolve this error?}

\subsubsection*{Case\+: I am using an {\ttfamily $<$a$>$} tag. Isn\textquotesingle{}t that interactive?}

The {\ttfamily $<$a$>$} tag is tricky. Consider the following\+:


\begin{DoxyCode}
<a>Edit</a>
<a href="#">Edit</a>
<a role="button">Edit</a>
\end{DoxyCode}


The bare {\ttfamily $<$a$>$} tag is an {\itshape anchor}. It has no semantic AX A\+PI mapping in either A\+R\+IA or the A\+X\+Object model. It\textquotesingle{}s as meaningful as {\ttfamily $<$div$>$}, which is to say it has no meaning. An {\ttfamily $<$a$>$} tag with an {\ttfamily href} attribute has an inherent role of {\ttfamily link}. An {\ttfamily $<$a$>$} tag with an explicit role obtains the designated role. In the example above, this role is {\ttfamily button}.

\subsubsection*{Case\+: I am using \char`\"{}semantic\char`\"{} H\+T\+ML. Isn\textquotesingle{}t that interactive?}

If we take a step back into the field of linguistics for a moment, let\textquotesingle{}s consider what it means for something to be \char`\"{}semantic\char`\"{}. Nothing, in and of itself, has meaning. Meaning is constructed through dialogue. A speaker intends a meaning and a listener/observer interprets a meaning. Each participant constructs their own meaning through dialogue. There is no intrinsic or isolated meaning outside of interaction. Thus, we must ask, given that we have a \char`\"{}speaker\char`\"{} who communicates via \char`\"{}semantic\char`\"{} H\+T\+ML, who is listening/observing?

In our case, the observer is the Accessibility (AX) A\+PI. Browsers interpret H\+T\+ML (inflected at times by A\+R\+IA) to construct a meaning (AX Tree) of the page. Whatever the semantic H\+T\+ML intends has only the force of suggestion to the AX A\+PI. Therefore, we have inconsistencies. For example, there is not yet an A\+R\+IA role for {\ttfamily text} or {\ttfamily label} and thus no way to change a {\ttfamily $<$label$>$} into plain text or a {\ttfamily $<$span$>$} into a label via A\+R\+IA. \textquotesingle{}\textquotesingle{} has an A\+X\+Object correpondant {\ttfamily Div\+Role}, but no such object maps to {\ttfamily $<$span$>$}.

What this lint rule endeavors to do is apply the AX A\+PI understanding of the semantics of an H\+T\+ML document back onto your code. The concept of interactivity boils down to whether a user can do something with the indicated or focused component.

Common interactive roles include\+:


\begin{DoxyEnumerate}
\item {\ttfamily button}
\end{DoxyEnumerate}
\begin{DoxyEnumerate}
\item {\ttfamily link}
\end{DoxyEnumerate}
\begin{DoxyEnumerate}
\item {\ttfamily checkbox}
\end{DoxyEnumerate}
\begin{DoxyEnumerate}
\item {\ttfamily menuitem}
\end{DoxyEnumerate}
\begin{DoxyEnumerate}
\item {\ttfamily menuitemcheckbox}
\end{DoxyEnumerate}
\begin{DoxyEnumerate}
\item {\ttfamily menuitemradio}
\end{DoxyEnumerate}
\begin{DoxyEnumerate}
\item {\ttfamily option}
\end{DoxyEnumerate}
\begin{DoxyEnumerate}
\item {\ttfamily radio}
\end{DoxyEnumerate}
\begin{DoxyEnumerate}
\item {\ttfamily searchbox}
\end{DoxyEnumerate}
\begin{DoxyEnumerate}
\item {\ttfamily switch}
\end{DoxyEnumerate}
\begin{DoxyEnumerate}
\item {\ttfamily textbox}
\end{DoxyEnumerate}

Endeavor to limit tabbable elements to those that a user can act upon.

\subsubsection*{Case\+: Shouldn\textquotesingle{}t I add a tabindex so that users can navigate to this item?}



It is not necessary to put a tabindex on an {\ttfamily $<$article$>$}, for instance or on {\ttfamily $<$li$>$} items; assistive technologies provide affordances to users to find and traverse these containers. Most elements that require a tabindex -- {\ttfamily $<$a href$>$}, {\ttfamily $<$button$>$}, {\ttfamily $<$input$>$}, {\ttfamily $<$textarea$>$} -- have it already.

Your application might require an exception to this rule in the case of an element that captures incoming tab traversal for a composite widget. In that case, turn off this rule on a per instance basis. This is an uncommon case.

\subsubsection*{References}




\begin{DoxyEnumerate}
\item \href{https://www.w3.org/TR/wai-aria-practices-1.1/#kbd_generalnav}{\tt Fundamental Keyboard Navigation Conventions}
\end{DoxyEnumerate}

\subsection*{Rule details}



The recommended options for this rule allow {\ttfamily tab\+Index} on elements with the noninteractive {\ttfamily tabpanel} role. Adding {\ttfamily tab\+Index} to a tabpanel is a recommended practice in some instances.


\begin{DoxyCode}
'jsx-a11y/no-noninteractive-tabindex': [
  'error',
  \{
    tags: [],
    roles: ['tabpanel'],
  \},
]
\end{DoxyCode}


\#\#\# Succeed 
\begin{DoxyCode}
<div />
<MyButton tabIndex=\{0\} />
<button />
<button tabIndex="0" />
<button tabIndex=\{0\} />
<div />
<div tabIndex="-1" />
<div role="button" tabIndex="0" />
<div role="article" tabIndex="-1" />
<article tabIndex="-1" />
\end{DoxyCode}


\subsubsection*{Fail}

 
\begin{DoxyCode}
<div tabIndex="0" />
<div role="article" tabIndex="0" />
<article tabIndex="0" />
<article tabIndex=\{0\} />
\end{DoxyCode}
 