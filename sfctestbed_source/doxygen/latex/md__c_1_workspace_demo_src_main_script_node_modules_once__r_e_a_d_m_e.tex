Only call a function once.

\subsection*{usage}


\begin{DoxyCode}
var once = require('once')

function load (file, cb) \{
  cb = once(cb)
  loader.load('file')
  loader.once('load', cb)
  loader.once('error', cb)
\}
\end{DoxyCode}


Or add to the Function.\+prototype in a responsible way\+:


\begin{DoxyCode}
// only has to be done once
require('once').proto()

function load (file, cb) \{
  cb = cb.once()
  loader.load('file')
  loader.once('load', cb)
  loader.once('error', cb)
\}
\end{DoxyCode}


Ironically, the prototype feature makes this module twice as complicated as necessary.

To check whether you function has been called, use {\ttfamily fn.\+called}. Once the function is called for the first time the return value of the original function is saved in {\ttfamily fn.\+value} and subsequent calls will continue to return this value.


\begin{DoxyCode}
var once = require('once')

function load (cb) \{
  cb = once(cb)
  var stream = createStream()
  stream.once('data', cb)
  stream.once('end', function () \{
    if (!cb.called) cb(new Error('not found'))
  \})
\}
\end{DoxyCode}


\subsection*{{\ttfamily once.\+strict(func)}}

Throw an error if the function is called twice.

Some functions are expected to be called only once. Using {\ttfamily once} for them would potentially hide logical errors.

In the example below, the {\ttfamily greet} function has to call the callback only once\+:


\begin{DoxyCode}
function greet (name, cb) \{
  // return is missing from the if statement
  // when no name is passed, the callback is called twice
  if (!name) cb('Hello anonymous')
  cb('Hello ' + name)
\}

function log (msg) \{
  console.log(msg)
\}

// this will print 'Hello anonymous' but the logical error will be missed
greet(null, once(msg))

// once.strict will print 'Hello anonymous' and throw an error when the callback will be called the second
       time
greet(null, once.strict(msg))
\end{DoxyCode}
 