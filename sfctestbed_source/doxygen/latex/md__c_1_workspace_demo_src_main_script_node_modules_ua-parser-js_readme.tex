 A Java\+Script-\/based User-\/\+Agent string parser. Can be used either in browser (client-\/side) or in node.\+js (server-\/side) environment. Also available as j\+Query/\+Zepto plugin, Bower/\+Meteor package, \& Require\+J\+S/\+A\+MD module. This library aims to identify detailed type of web browser, layout engine, operating system, cpu architecture, and device type/model, entirely from user-\/agent string with a relatively small footprint ($\sim$11\+KB when minified / $\sim$4\+KB gzipped). Written in vanilla Java\+Script, which means it doesn\textquotesingle{}t require any other library and can be used independently. However, it\textquotesingle{}s not recommended to use this library as browser detection since the result may not accurate than using feature detection.

\href{https://travis-ci.org/faisalman/ua-parser-js}{\tt } \href{https://www.npmjs.com/package/ua-parser-js}{\tt } \href{https://www.npmjs.com/package/ua-parser-js}{\tt } \href{https://bower.io/}{\tt } \href{https://cdnjs.com/libraries/UAParser.js}{\tt } \href{https://gratipay.com/UAParser.js}{\tt } \href{http://flattr.com/thing/3867907/faisalmanua-parser-js-on-GitHub}{\tt }


\begin{DoxyItemize}
\item Author \+: Faisal Salman $<$\href{mailto:f@faisalman.com}{\tt f@faisalman.\+com}$>$
\item Demo \+: \href{http://faisalman.github.io/ua-parser-js}{\tt http\+://faisalman.\+github.\+io/ua-\/parser-\/js}
\item Source \+: \href{https://github.com/faisalman/ua-parser-js}{\tt https\+://github.\+com/faisalman/ua-\/parser-\/js}
\end{DoxyItemize}

\section*{Constructor}


\begin{DoxyItemize}
\item {\ttfamily new U\+A\+Parser(\mbox{[}uastring\mbox{]}\mbox{[},extensions\mbox{]})}
\begin{DoxyItemize}
\item returns new instance
\end{DoxyItemize}
\item {\ttfamily U\+A\+Parser(\mbox{[}uastring\mbox{]}\mbox{[},extensions\mbox{]})}
\begin{DoxyItemize}
\item returns result object `\{ ua\+: '\textquotesingle{}, browser\+: \{\}, cpu\+: \{\}, device\+: \{\}, engine\+: \{\}, os\+: \{\} \}\`{}
\end{DoxyItemize}
\end{DoxyItemize}

\section*{Methods}


\begin{DoxyItemize}
\item {\ttfamily get\+Browser()}
\begin{DoxyItemize}
\item returns `\{ name\+: '\textquotesingle{}, version\+: \textquotesingle{}\textquotesingle{} \}\`{}
\end{DoxyItemize}
\end{DoxyItemize}


\begin{DoxyCode}
# Possible 'browser.name':
Amaya, Android Browser, Arora, Avant, Baidu, Blazer, Bolt, Bowser, Camino, Chimera, 
Chrome [WebView], Chromium, Comodo Dragon, Conkeror, Dillo, Dolphin, Doris, Edge, 
Epiphany, Fennec, Firebird, Firefox, Flock, GoBrowser, iCab, ICE Browser, IceApe, 
IceCat, IceDragon, Iceweasel, IE[Mobile], Iron, Jasmine, K-Meleon, Konqueror, Kindle, 
Links, Lunascape, Lynx, Maemo, Maxthon, Midori, Minimo, MIUI Browser, [Mobile] Safari, 
Mosaic, Mozilla, Netfront, Netscape, NetSurf, Nokia, OmniWeb, Opera [Mini/Mobi/Tablet], 
PhantomJS, Phoenix, Polaris, QQBrowser, RockMelt, Silk, Skyfire, SeaMonkey, Sleipnir, 
SlimBrowser, Swiftfox, Tizen, UCBrowser, Vivaldi, w3m, WeChat, Yandex

# 'browser.version' determined dynamically
\end{DoxyCode}



\begin{DoxyItemize}
\item {\ttfamily get\+Device()}
\begin{DoxyItemize}
\item returns `\{ model\+: '\textquotesingle{}, type\+: \textquotesingle{}\textquotesingle{}, vendor\+: \textquotesingle{}\textquotesingle{} \}\`{}
\end{DoxyItemize}
\end{DoxyItemize}


\begin{DoxyCode}
# Possible 'device.type':
console, mobile, tablet, smarttv, wearable, embedded

# Possible 'device.vendor':
Acer, Alcatel, Amazon, Apple, Archos, Asus, BenQ, BlackBerry, Dell, GeeksPhone, 
Google, HP, HTC, Huawei, Jolla, Lenovo, LG, Meizu, Microsoft, Motorola, Nexian, 
Nintendo, Nokia, Nvidia, OnePlus, Ouya, Palm, Panasonic, Pebble, Polytron, RIM, 
Samsung, Sharp, Siemens, Sony[Ericsson], Sprint, Xbox, Xiaomi, ZTE

# 'device.model' determined dynamically
\end{DoxyCode}



\begin{DoxyItemize}
\item {\ttfamily get\+Engine()}
\begin{DoxyItemize}
\item returns `\{ name\+: '\textquotesingle{}, version\+: \textquotesingle{}\textquotesingle{} \}\`{}
\end{DoxyItemize}
\end{DoxyItemize}


\begin{DoxyCode}
# Possible 'engine.name'
Amaya, EdgeHTML, Gecko, iCab, KHTML, Links, Lynx, NetFront, NetSurf, Presto, 
Tasman, Trident, w3m, WebKit

# 'engine.version' determined dynamically
\end{DoxyCode}



\begin{DoxyItemize}
\item {\ttfamily get\+O\+S()}
\begin{DoxyItemize}
\item returns `\{ name\+: '\textquotesingle{}, version\+: \textquotesingle{}\textquotesingle{} \}\`{}
\end{DoxyItemize}
\end{DoxyItemize}


\begin{DoxyCode}
# Possible 'os.name'
AIX, Amiga OS, Android, Arch, Bada, BeOS, BlackBerry, CentOS, Chromium OS, Contiki,
Fedora, Firefox OS, FreeBSD, Debian, DragonFly, Gentoo, GNU, Haiku, Hurd, iOS, 
Joli, Linpus, Linux, Mac OS, Mageia, Mandriva, MeeGo, Minix, Mint, Morph OS, NetBSD, 
Nintendo, OpenBSD, OpenVMS, OS/2, Palm, PC-BSD, PCLinuxOS, Plan9, Playstation, QNX, RedHat, 
RIM Tablet OS, RISC OS, Sailfish, Series40, Slackware, Solaris, SUSE, Symbian, Tizen, 
Ubuntu, UNIX, VectorLinux, WebOS, Windows [Phone/Mobile], Zenwalk

# 'os.version' determined dynamically
\end{DoxyCode}



\begin{DoxyItemize}
\item {\ttfamily get\+C\+P\+U()}
\begin{DoxyItemize}
\item returns `\{ architecture\+: '\textquotesingle{} \}\`{}
\end{DoxyItemize}
\end{DoxyItemize}


\begin{DoxyCode}
# Possible 'cpu.architecture'
68k, amd64, arm[64], avr, ia[32/64], irix[64], mips[64], pa-risc, ppc, sparc[64]
\end{DoxyCode}



\begin{DoxyItemize}
\item {\ttfamily get\+Result()}
\begin{DoxyItemize}
\item returns `\{ ua\+: '\textquotesingle{}, browser\+: \{\}, cpu\+: \{\}, device\+: \{\}, engine\+: \{\}, os\+: \{\} \}\`{}
\end{DoxyItemize}
\item {\ttfamily get\+U\+A()}
\begin{DoxyItemize}
\item returns UA string of current instance
\end{DoxyItemize}
\item {\ttfamily set\+U\+A(uastring)}
\begin{DoxyItemize}
\item set UA string to parse
\item returns current instance
\end{DoxyItemize}
\end{DoxyItemize}

\section*{Example}


\begin{DoxyCode}
<!doctype html>
<html>
<head>
<script type="text/javascript" src="ua-parser.min.js"></script>
<script type="text/javascript">

    var parser = new UAParser();

    // by default it takes ua string from current browser's window.navigator.userAgent
    console.log(parser.getResult());
    /*
        /// this will print an object structured like this:
        \{
            ua: "",
            browser: \{
                name: "",
                version: ""
            \},
            engine: \{
                name: "",
                version: ""
            \},
            os: \{
                name: "",
                version: ""
            \},
            device: \{
                model: "",
                type: "",
                vendor: ""
            \},
            cpu: \{
                architecture: ""
            \}
        \}
    */

    // let's test a custom user-agent string as an example
    var uastring = "Mozilla/5.0 (X11; Linux x86\_64) AppleWebKit/535.2 (KHTML, like Gecko) Ubuntu/11.10
       Chromium/15.0.874.106 Chrome/15.0.874.106 Safari/535.2";
    parser.setUA(uastring);

    var result = parser.getResult();
    // this will also produce the same result (without instantiation):
    // var result = UAParser(uastring);

    console.log(result.browser);        // \{name: "Chromium", version: "15.0.874.106"\}
    console.log(result.device);         // \{model: undefined, type: undefined, vendor: undefined\}
    console.log(result.os);             // \{name: "Ubuntu", version: "11.10"\}
    console.log(result.os.version);     // "11.10"
    console.log(result.engine.name);    // "WebKit"
    console.log(result.cpu.architecture);   // "amd64"

    // do some other tests
    var uastring2 = "Mozilla/5.0 (compatible; Konqueror/4.1; OpenBSD) KHTML/4.1.4 (like Gecko)";
    console.log(parser.setUA(uastring2).getBrowser().name); // "Konqueror"
    console.log(parser.getOS());                            // \{name: "OpenBSD", version: undefined\}
    console.log(parser.getEngine());                        // \{name: "KHTML", version: "4.1.4"\}

    var uastring3 = 'Mozilla/5.0 (PlayBook; U; RIM Tablet OS 1.0.0; en-US) AppleWebKit/534.11 (KHTML, like
       Gecko) Version/7.1.0.7 Safari/534.11';
    console.log(parser.setUA(uastring3).getDevice().model); // "PlayBook"
    console.log(parser.getOS())                             // \{name: "RIM Tablet OS", version: "1.0.0"\}
    console.log(parser.getBrowser().name);                  // "Safari"

</script>
</head>
<body>
</body>
</html>
\end{DoxyCode}


\subsection*{Using node.\+js}


\begin{DoxyCode}
$ npm install ua-parser-js
\end{DoxyCode}



\begin{DoxyCode}
var http = require('http');
var parser = require('ua-parser-js');

http.createServer(function (req, res) \{
    // get user-agent header
    var ua = parser(req.headers['user-agent']);
    // write the result as response
    res.end(JSON.stringify(ua, null, '  '));
\})
.listen(1337, '127.0.0.1');

console.log('Server running at http://127.0.0.1:1337/');
\end{DoxyCode}


\subsection*{Using requirejs}


\begin{DoxyCode}
requirejs.config(\{
    baseUrl : 'js/lib', // path to your script directory
    paths   : \{
        'ua-parser-js' : 'ua-parser.min'
    \}
\});

requirejs(['ua-parser-js'], function(UAParser) \{
    var parser = new UAParser();
    console.log(parser.getResult());
\});
\end{DoxyCode}


\subsection*{Using bower}


\begin{DoxyCode}
$ bower install ua-parser-js
\end{DoxyCode}


\subsection*{Using meteor}


\begin{DoxyCode}
$ meteor add faisalman:ua-parser-js
\end{DoxyCode}


\subsection*{Using j\+Query/\+Zepto (\$.ua)}

Although written in vanilla js (which means it doesn\textquotesingle{}t depends on j\+Query), this library will automatically detect if j\+Query/\+Zepto is present and create {\ttfamily \$.ua} object based on browser\textquotesingle{}s user-\/agent (although in case you need, {\ttfamily window.\+U\+A\+Parser} constructor is still present). To get/set user-\/agent you can use\+: {\ttfamily \$.ua.\+get()} / {\ttfamily \$.ua.\+set(uastring)}.


\begin{DoxyCode}
// In browser with default user-agent: 'Mozilla/5.0 (Linux; U; Android 2.3.4; en-us; Sprint APA7373KT
       Build/GRJ22) AppleWebKit/533.1 (KHTML, like Gecko) Version/4.0':

// Do some tests
console.log($.ua.device);           // \{vendor: "HTC", model: "Evo Shift 4G", type: "mobile"\}
console.log($.ua.os);               // \{name: "Android", version: "2.3.4"\}
console.log($.ua.os.name);          // "Android"
console.log($.ua.get());            // "Mozilla/5.0 (Linux; U; Android 2.3.4; en-us; Sprint APA7373KT
       Build/GRJ22) AppleWebKit/533.1 (KHTML, like Gecko) Version/4.0"

// reset to custom user-agent
$.ua.set('Mozilla/5.0 (Linux; U; Android 3.0.1; en-us; Xoom Build/HWI69) AppleWebKit/534.13 (KHTML, like
       Gecko) Version/4.0 Safari/534.13');

// Test again
console.log($.ua.browser.name);     // "Safari"
console.log($.ua.engine.name);      // "Webkit"
console.log($.ua.device);           // \{vendor: "Motorola", model: "Xoom", type: "tablet"\}
console.log(parseInt($.ua.browser.version.split('.')[0], 10));  // 4

// Add class to <body> tag
// <body class="ua-browser-safari ua-devicetype-tablet">
$('body').addClass('ua-browser-' + $.ua.browser.name + ' ua-devicetype-' + $.ua.device.type);
\end{DoxyCode}


\subsection*{Extending regex patterns}


\begin{DoxyItemize}
\item {\ttfamily U\+A\+Parser(uastring\mbox{[}, extensions\mbox{]})}
\end{DoxyItemize}

Pass your own regexes to extend the limited matching rules.


\begin{DoxyCode}
// Example:
var uaString = 'Mozilla/5.0 MyOwnBrowser/1.3';
var myOwnRegex = [[/(myownbrowser)\(\backslash\)/([\(\backslash\)w\(\backslash\).]+)/i], [UAParser.BROWSER.NAME, UAParser.BROWSER.VERSION]];
var parser = new UAParser(uaString, \{ browser: myOwnRegex \});
console.log(parser.getBrowser());   // \{name: "MyOwnBrowser", version: "1.3"\}
\end{DoxyCode}


\section*{Development}

\subsection*{Contribute}


\begin{DoxyItemize}
\item Fork and clone this repository
\item Make some changes as required
\item Write a unit test to showcase your feature
\item Run the test suites to make sure the changes you made didn\textquotesingle{}t break anything {\ttfamily \$ npm run test}
\item Commit and push to your own repository
\item Submit a pull request to this repository under {\ttfamily develop} branch
\item Profit? \$\$\$
\end{DoxyItemize}

\subsection*{Build}

Build a minified \& packed script


\begin{DoxyCode}
$ npm run build
\end{DoxyCode}


\section*{License}

Dual licensed under G\+P\+Lv2 \& M\+IT

Copyright © 2012-\/2016 Faisal Salman $<$\href{mailto:fyzlman@gmail.com}{\tt fyzlman@gmail.\+com}$>$

Permission is hereby granted, free of charge, to any person obtaining a copy of this software and associated documentation files (the \char`\"{}\+Software\char`\"{}), to deal in the Software without restriction, including without limitation the rights to use, copy, modify, merge, publish, distribute, sublicense, and/or sell copies of the Software, and to permit persons to whom the Software is furnished to do so, subject to the following conditions\+:

The above copyright notice and this permission notice shall be included in all copies or substantial portions of the Software. 