\href{https://www.npmjs.com/package/eslint}{\tt } \href{https://travis-ci.org/eslint/eslint}{\tt } \href{https://ci.appveyor.com/project/nzakas/eslint/branch/master}{\tt } \href{https://coveralls.io/r/eslint/eslint?branch=master}{\tt } \href{https://www.npmjs.com/package/eslint}{\tt } \href{https://www.bountysource.com/trackers/282608-eslint?utm_source=282608&utm_medium=shield&utm_campaign=TRACKER_BADGE}{\tt } \href{https://gitter.im/eslint/eslint?utm_source=badge&utm_medium=badge&utm_campaign=pr-badge&utm_content=badge}{\tt }

\section*{E\+S\+Lint}

\href{http://eslint.org}{\tt Website} $\vert$ \href{http://eslint.org/docs/user-guide/configuring}{\tt Configuring} $\vert$ \href{http://eslint.org/docs/rules/}{\tt Rules} $\vert$ \href{http://eslint.org/docs/developer-guide/contributing}{\tt Contributing} $\vert$ \href{http://eslint.org/docs/developer-guide/contributing/reporting-bugs}{\tt Reporting Bugs} $\vert$ \href{https://js.foundation/conduct/}{\tt Code of Conduct} $\vert$ \href{https://twitter.com/geteslint}{\tt Twitter} $\vert$ \href{https://groups.google.com/group/eslint}{\tt Mailing List} $\vert$ \href{https://gitter.im/eslint/eslint}{\tt Chat Room}

E\+S\+Lint is a tool for identifying and reporting on patterns found in E\+C\+M\+A\+Script/\+Java\+Script code. In many ways, it is similar to J\+S\+Lint and J\+S\+Hint with a few exceptions\+:


\begin{DoxyItemize}
\item E\+S\+Lint uses \href{https://github.com/eslint/espree}{\tt Espree} for Java\+Script parsing.
\item E\+S\+Lint uses an A\+ST to evaluate patterns in code.
\item E\+S\+Lint is completely pluggable, every single rule is a plugin and you can add more at runtime.
\end{DoxyItemize}

\subsection*{Installation and Usage}

There are two ways to install E\+S\+Lint\+: globally and locally.

\subsubsection*{Local Installation and Usage}

If you want to include E\+S\+Lint as part of your project\textquotesingle{}s build system, we recommend installing it locally. You can do so using npm\+:


\begin{DoxyCode}
$ npm install eslint --save-dev
\end{DoxyCode}


You should then setup a configuration file\+:


\begin{DoxyCode}
$ ./node\_modules/.bin/eslint --init
\end{DoxyCode}


After that, you can run E\+S\+Lint on any file or directory like this\+:


\begin{DoxyCode}
$ ./node\_modules/.bin/eslint yourfile.js
\end{DoxyCode}


Any plugins or shareable configs that you use must also be installed locally to work with a locally-\/installed E\+S\+Lint.

\subsubsection*{Global Installation and Usage}

If you want to make E\+S\+Lint available to tools that run across all of your projects, we recommend installing E\+S\+Lint globally. You can do so using npm\+:


\begin{DoxyCode}
$ npm install -g eslint
\end{DoxyCode}


You should then setup a configuration file\+:


\begin{DoxyCode}
$ eslint --init
\end{DoxyCode}


After that, you can run E\+S\+Lint on any file or directory like this\+:


\begin{DoxyCode}
$ eslint yourfile.js
\end{DoxyCode}


Any plugins or shareable configs that you use must also be installed globally to work with a globally-\/installed E\+S\+Lint.

{\bfseries Note\+:} {\ttfamily eslint -\/-\/init} is intended for setting up and configuring E\+S\+Lint on a per-\/project basis and will perform a local installation of E\+S\+Lint and its plugins in the directory in which it is run. If you prefer using a global installation of E\+S\+Lint, any plugins used in your configuration must also be installed globally.

\subsection*{Configuration}

After running {\ttfamily eslint -\/-\/init}, you\textquotesingle{}ll have a {\ttfamily .eslintrc} file in your directory. In it, you\textquotesingle{}ll see some rules configured like this\+:


\begin{DoxyCode}
\{
    "rules": \{
        "semi": ["error", "always"],
        "quotes": ["error", "double"]
    \}
\}
\end{DoxyCode}


The names {\ttfamily \char`\"{}semi\char`\"{}} and {\ttfamily \char`\"{}quotes\char`\"{}} are the names of \href{http://eslint.org/docs/rules}{\tt rules} in E\+S\+Lint. The first value is the error level of the rule and can be one of these values\+:


\begin{DoxyItemize}
\item {\ttfamily \char`\"{}off\char`\"{}} or {\ttfamily 0} -\/ turn the rule off
\item {\ttfamily \char`\"{}warn\char`\"{}} or {\ttfamily 1} -\/ turn the rule on as a warning (doesn\textquotesingle{}t affect exit code)
\item {\ttfamily \char`\"{}error\char`\"{}} or {\ttfamily 2} -\/ turn the rule on as an error (exit code will be 1)
\end{DoxyItemize}

The three error levels allow you fine-\/grained control over how E\+S\+Lint applies rules (for more configuration options and details, see the \href{http://eslint.org/docs/user-guide/configuring}{\tt configuration docs}).

\subsection*{Sponsors}


\begin{DoxyItemize}
\item Site search (\href{http://eslint.org}{\tt eslint.\+org}) is sponsored by \href{https://www.algolia.com}{\tt Algolia}
\end{DoxyItemize}

\subsection*{Team}

These folks keep the project moving and are resources for help.

\subsubsection*{Technical Steering Committee (T\+SC)}


\begin{DoxyItemize}
\item Nicholas C. Zakas (\href{https://github.com/nzakas}{\tt })
\item Ilya Volodin (\href{https://github.com/ilyavolodin}{\tt })
\item Brandon Mills (\href{https://github.com/btmills}{\tt })
\item Gyandeep Singh (\href{https://github.com/gyandeeps}{\tt })
\item Toru Nagashima (\href{https://github.com/mysticatea}{\tt })
\item Alberto Rodríguez (\href{https://github.com/alberto}{\tt })
\item Kai Cataldo (\href{https://github.com/kaicataldo}{\tt })
\item Teddy Katz (\href{https://github.com/not-an-aardvark}{\tt -\/an-\/aardvark})
\end{DoxyItemize}

\subsubsection*{Development Team}


\begin{DoxyItemize}
\item Mathias Schreck (\href{https://github.com/lo1tuma}{\tt })
\item Jamund Ferguson (\href{https://github.com/xjamundx}{\tt })
\item Ian Van\+Schooten (\href{https://github.com/ianvs}{\tt })
\item Burak Yiğit Kaya (\href{https://github.com/byk}{\tt })
\item Michael Ficarra (\href{https://github.com/michaelficarra}{\tt })
\item Mark Pedrotti (\href{https://github.com/pedrottimark}{\tt })
\item Oleg Gaidarenko (\href{https://github.com/markelog}{\tt })
\item Mike Sherov \href{https://github.com/mikesherov}{\tt })
\item Henry Zhu (\href{https://github.com/hzoo}{\tt })
\item Marat Dulin (\href{https://github.com/mdevils}{\tt })
\item Alexej Yaroshevich (\href{https://github.com/zxqfox}{\tt })
\item Kevin Partington (\href{https://github.com/platinumazure}{\tt })
\item Vitor Balocco (\href{https://github.com/vitorbal}{\tt })
\item James Henry (\href{https://github.com/JamesHenry}{\tt })
\item Reyad Attiyat (\href{https://github.com/soda0289}{\tt })
\end{DoxyItemize}

\subsection*{Releases}

We have scheduled releases every two weeks on Friday or Saturday.

\subsection*{Filing Issues}

Before filing an issue, please be sure to read the guidelines for what you\textquotesingle{}re reporting\+:


\begin{DoxyItemize}
\item \href{http://eslint.org/docs/developer-guide/contributing/reporting-bugs}{\tt Bug Report}
\item \href{http://eslint.org/docs/developer-guide/contributing/new-rules}{\tt Propose a New Rule}
\item \href{http://eslint.org/docs/developer-guide/contributing/rule-changes}{\tt Proposing a Rule Change}
\item \href{http://eslint.org/docs/developer-guide/contributing/changes}{\tt Request a Change}
\end{DoxyItemize}

\subsection*{Semantic Versioning Policy}

E\+S\+Lint follows \href{http://semver.org}{\tt semantic versioning}. However, due to the nature of E\+S\+Lint as a code quality tool, it\textquotesingle{}s not always clear when a minor or major version bump occurs. To help clarify this for everyone, we\textquotesingle{}ve defined the following semantic versioning policy for E\+S\+Lint\+:


\begin{DoxyItemize}
\item Patch release (intended to not break your lint build)
\begin{DoxyItemize}
\item A bug fix in a rule that results in E\+S\+Lint reporting fewer errors.
\item A bug fix to the C\+LI or core (including formatters).
\item Improvements to documentation.
\item Non-\/user-\/facing changes such as refactoring code, adding, deleting, or modifying tests, and increasing test coverage.
\item Re-\/releasing after a failed release (i.\+e., publishing a release that doesn\textquotesingle{}t work for anyone).
\end{DoxyItemize}
\item Minor release (might break your lint build)
\begin{DoxyItemize}
\item A bug fix in a rule that results in E\+S\+Lint reporting more errors.
\item A new rule is created.
\item A new option to an existing rule that does not result in E\+S\+Lint reporting more errors by default.
\item An existing rule is deprecated.
\item A new C\+LI capability is created.
\item New capabilities to the public A\+PI are added (new classes, new methods, new arguments to existing methods, etc.).
\item A new formatter is created.
\end{DoxyItemize}
\item Major release (likely to break your lint build)
\begin{DoxyItemize}
\item {\ttfamily eslint\+:recommended} is updated.
\item A new option to an existing rule that results in E\+S\+Lint reporting more errors by default.
\item An existing rule is removed.
\item An existing formatter is removed.
\item Part of the public A\+PI is removed or changed in an incompatible way.
\end{DoxyItemize}
\end{DoxyItemize}

According to our policy, any minor update may report more errors than the previous release (ex\+: from a bug fix). As such, we recommend using the tilde ({\ttfamily $\sim$}) in {\ttfamily package.\+json} e.\+g. {\ttfamily \char`\"{}eslint\char`\"{}\+: \char`\"{}$\sim$3.\+1.\+0\char`\"{}} to guarantee the results of your builds.

\subsection*{Frequently Asked Questions}

\subsubsection*{How is E\+S\+Lint different from J\+S\+Hint?}

The most significant difference is that E\+Slint has pluggable linting rules. That means you can use the rules it comes with, or you can extend it with rules created by others or by yourself!

\subsubsection*{How does E\+S\+Lint performance compare to J\+S\+Hint?}

E\+S\+Lint is slower than J\+S\+Hint, usually 2-\/3x slower on a single file. This is because E\+S\+Lint uses Espree to construct an A\+ST before it can evaluate your code whereas J\+S\+Hint evaluates your code as it\textquotesingle{}s being parsed. The speed is also based on the number of rules you enable; the more rules you enable, the slower the process.

Despite being slower, we believe that E\+S\+Lint is fast enough to replace J\+S\+Hint without causing significant pain.

\subsubsection*{I heard E\+S\+Lint is going to replace J\+S\+CS?}

Yes. Since we are solving the same problems, E\+S\+Lint and J\+S\+CS teams have decided to join forces and work together in the development of E\+S\+Lint instead of competing with each other. You can read more about this in both \href{http://eslint.org/blog/2016/04/welcoming-jscs-to-eslint}{\tt E\+S\+Lint} and \href{https://medium.com/@markelog/jscs-end-of-the-line-bc9bf0b3fdb2#.u76sx334n}{\tt J\+S\+CS} announcements.

\subsubsection*{So, should I stop using J\+S\+CS and start using E\+S\+Lint?}

Maybe, depending on how much you need it. \href{http://eslint.org/blog/2016/07/jscs-end-of-life}{\tt J\+S\+CS has reached end of life}, but if it is working for you then there is no reason to move yet. We are still working to smooth the transition. You can see our progress \href{https://github.com/eslint/eslint/milestones/JSCS%20Compatibility}{\tt here}. We’ll announce when all of the changes necessary to support J\+S\+CS users in E\+S\+Lint are complete and will start encouraging J\+S\+CS users to switch to E\+S\+Lint at that time.

If you are having issues with J\+S\+CS, you can try to move to E\+S\+Lint. We are focusing our time and energy on J\+S\+CS compatibility issues.

\subsubsection*{Is E\+S\+Lint just linting or does it also check style?}

E\+S\+Lint does both traditional linting (looking for problematic patterns) and style checking (enforcement of conventions). You can use it for both.

\subsubsection*{Does E\+S\+Lint support J\+SX?}

Yes, E\+S\+Lint natively supports parsing J\+SX syntax (this must be enabled in \href{http://eslint.org/docs/user-guide/configuring}{\tt configuration}.). Please note that supporting J\+SX syntax {\itshape is not} the same as supporting React. React applies specific semantics to J\+SX syntax that E\+S\+Lint doesn\textquotesingle{}t recognize. We recommend using \href{https://www.npmjs.com/package/eslint-plugin-react}{\tt eslint-\/plugin-\/react} if you are using React and want React semantics.

\subsubsection*{What about E\+C\+M\+A\+Script 6 support?}

E\+S\+Lint has full support for E\+C\+M\+A\+Script 6. By default, this support is off. You can enable E\+C\+M\+A\+Script 6 support through \href{http://eslint.org/docs/user-guide/configuring}{\tt configuration}.

\subsubsection*{What about experimental features?}

E\+S\+Lint doesn\textquotesingle{}t natively support experimental E\+C\+M\+A\+Script language features. You can use \href{https://github.com/babel/babel-eslint}{\tt babel-\/eslint} to use any option available in Babel.

Once a language feature has been adopted into the E\+C\+M\+A\+Script standard (stage 4 according to the \href{https://tc39.github.io/process-document/}{\tt T\+C39 process}), we will accept issues and pull requests related to the new feature, subject to our \href{http://eslint.org/docs/developer-guide/contributing}{\tt contributing guidelines}. Until then, please use the appropriate parser and plugin(s) for your experimental feature.

\subsubsection*{Where to ask for help?}

Join our \href{https://groups.google.com/group/eslint}{\tt Mailing List} or \href{https://gitter.im/eslint/eslint}{\tt Chatroom} 