\href{https://travis-ci.org/GoogleChrome/sw-toolbox}{\tt } \href{https://david-dm.org/googlechrome/sw-toolbox}{\tt } \href{https://david-dm.org/googlechrome/sw-toolbox?type=dev}{\tt }

\begin{quote}
A collection of tools for \href{https://w3c.github.io/ServiceWorker/}{\tt service workers} \end{quote}


Service Worker Toolbox provides some simple helpers for use in creating your own service workers. Specifically, it provides common caching strategies for dynamic content, such as A\+PI calls, third-\/party resources, and large or infrequently used local resources that you don\textquotesingle{}t want precached.

Service Worker Toolbox provides an \href{https://googlechrome.github.io/sw-toolbox/usage.html#express-style-routes}{\tt expressive approach} to using those strategies for runtime requests. If you\textquotesingle{}re not sure what service workers are or what they are for, start with https\+://github.com/slightlyoff/\+Service\+Worker/blob/master/explainer.\+md \char`\"{}the explainer doc\char`\"{}.

\subsection*{What if I need precaching as well?}

Then you should go check out \href{https://github.com/GoogleChrome/sw-precache}{\tt {\ttfamily sw-\/precache}} before doing anything else. In addition to precaching static resources, {\ttfamily sw-\/precache} supports optional \href{https://github.com/GoogleChrome/sw-precache#runtime-caching}{\tt runtime caching} through a simple, declarative configuration that incorporates Service Worker Toolbox under the hood.

\subsection*{Install}

Service Worker Toolbox is available through Bower, npm or direct from Git\+Hub\+:

{\ttfamily bower install -\/-\/save sw-\/toolbox}

{\ttfamily npm install -\/-\/save sw-\/toolbox}

{\ttfamily git clone \href{https://github.com/GoogleChrome/sw-toolbox.git}{\tt https\+://github.\+com/\+Google\+Chrome/sw-\/toolbox.\+git}}

\subsubsection*{Register your service worker}

From your registering page, register your service worker in the normal way. For example\+:


\begin{DoxyCode}
navigator.serviceWorker.register('my-service-worker.js');
\end{DoxyCode}


As implemented in Chrome 40 or later, a service worker must exist at the root of the scope that you intend it to control, or higher. So if you want all of the pages under {\ttfamily /myapp/} to be controlled by the worker, the worker script itself must be served from either {\ttfamily /} or {\ttfamily /myapp/}. The default scope is the containing path of the service worker script.

For even lower friction, you can instead include the Service Worker Toolbox companion script in your H\+T\+ML as shown below. Be aware that this is not customizable. If you need to do anything fancier than register with a default scope, you\textquotesingle{}ll need to use the standard registration.


\begin{DoxyCode}
<script src="/path/to/sw-toolbox/companion.js" data-service-worker="my-service-worker.js"></script>
\end{DoxyCode}


\subsubsection*{Add Service Worker Toolbox to your service worker script}

In your service worker you just need to use {\ttfamily import\+Scripts} to load Service Worker Toolbox\+:


\begin{DoxyCode}
importScripts('bower\_components/sw-toolbox/sw-toolbox.js');  // Update path to match your own setup.
\end{DoxyCode}


\subsubsection*{Use the toolbox}

To understand how to use the toolbox read the \href{https://googlechrome.github.io/sw-toolbox/usage.html#main}{\tt Usage} and \href{https://googlechrome.github.io/sw-toolbox/api.html#main}{\tt A\+PI} documentation.

\subsection*{Support}

If you’ve found an error in this library, please file an issue at \href{https://github.com/GoogleChrome/sw-toolbox/issues}{\tt https\+://github.\+com/\+Google\+Chrome/sw-\/toolbox/issues}.

Patches are encouraged, and may be submitted by forking this project and submitting a \href{https://github.com/GoogleChrome/sw-toolbox/pulls}{\tt pull request through this Git\+Hub repo}.

\subsection*{License}

Copyright 2015-\/2016 Google, Inc.

Licensed under the \mbox{[}Apache License, Version 2.\+0\mbox{]}(L\+I\+C\+E\+N\+SE) (the \char`\"{}\+License\char`\"{}); you may not use this file except in compliance with the License. You may obtain a copy of the License at

\href{http://www.apache.org/licenses/LICENSE-2.0}{\tt http\+://www.\+apache.\+org/licenses/\+L\+I\+C\+E\+N\+S\+E-\/2.\+0}

Unless required by applicable law or agreed to in writing, software distributed under the License is distributed on an \char`\"{}\+A\+S I\+S\char`\"{} B\+A\+S\+IS, W\+I\+T\+H\+O\+UT W\+A\+R\+R\+A\+N\+T\+I\+ES OR C\+O\+N\+D\+I\+T\+I\+O\+NS OF A\+NY K\+I\+ND, either express or implied. See the License for the specific language governing permissions and limitations under the License. 