\href{http://badge.fury.io/js/html-webpack-plugin}{\tt } \href{https://david-dm.org/jantimon/html-webpack-plugin}{\tt } \href{https://travis-ci.org/jantimon/html-webpack-plugin}{\tt } \href{https://ci.appveyor.com/project/jantimon/html-webpack-plugin}{\tt } \href{https://github.com/Flet/semistandard}{\tt } \href{https://www.bithound.io/github/jantimon/html-webpack-plugin/master/dependencies/npm}{\tt } \mbox{[}\mbox{]}()

\href{https://nodei.co/npm/html-webpack-plugin/}{\tt }

This is a \href{http://webpack.github.io/}{\tt webpack} plugin that simplifies creation of H\+T\+ML files to serve your webpack bundles. This is especially useful for webpack bundles that include a hash in the filename which changes every compilation. You can either let the plugin generate an H\+T\+ML file for you, supply your own template using lodash templates or use your own loader.

Maintainer\+: Jan Nicklas \href{https://twitter.com/jantimon}{\tt }

\subsection*{Installation }

Install the plugin with npm\+: 
\begin{DoxyCode}
$ npm install html-webpack-plugin --save-dev
\end{DoxyCode}


\subsection*{Third party addons\+: }

The html-\/webpack-\/plugin provides \href{https://github.com/jantimon/html-webpack-plugin#events}{\tt hooks} to extend it to your needs. There are already some really powerful plugins which can be integrated with zero configuration\+:


\begin{DoxyItemize}
\item \href{https://www.npmjs.com/package/webpack-subresource-integrity}{\tt webpack-\/subresource-\/integrity} for enhanced asset security
\item \href{https://github.com/lettertwo/appcache-webpack-plugin}{\tt appcache-\/webpack-\/plugin} for i\+OS and Android offline usage
\item \href{https://github.com/jantimon/favicons-webpack-plugin}{\tt favicons-\/webpack-\/plugin} which generates favicons and icons for i\+OS, Android and desktop browsers
\item \href{https://github.com/jantimon/html-webpack-harddisk-plugin}{\tt html-\/webpack-\/harddisk-\/plugin} can be used to always write to disk the html file, useful when webpack-\/dev-\/server / H\+MR are being used
\item \href{https://github.com/DustinJackson/html-webpack-inline-source-plugin}{\tt html-\/webpack-\/inline-\/source-\/plugin} to inline your assets in the resulting H\+T\+ML file
\item \href{https://github.com/jamesjieye/html-webpack-exclude-assets-plugin}{\tt html-\/webpack-\/exclude-\/assets-\/plugin} for excluding assets using regular expressions
\item \href{https://github.com/jharris4/html-webpack-include-assets-plugin}{\tt html-\/webpack-\/include-\/assets-\/plugin} for including lists of js or css file paths (such as those copied by the copy-\/webpack-\/plugin).
\item \href{https://github.com/numical/script-ext-html-webpack-plugin}{\tt script-\/ext-\/html-\/webpack-\/plugin} to add {\ttfamily async}, {\ttfamily defer} or {\ttfamily module} attributes to your{\ttfamily $<$script$>$} elements, or even in-\/line them
\item \href{https://github.com/numical/style-ext-html-webpack-plugin}{\tt style-\/ext-\/html-\/webpack-\/plugin} to convert your {\ttfamily $<$link$>$}s to external stylesheets into {\ttfamily $<$style$>$} elements containing internal C\+SS
\item \href{https://github.com/jantimon/resource-hints-webpack-plugin}{\tt resource-\/hints-\/webpack-\/plugin} to add resource hints for faster initial page loads using `$<$link rel=\textquotesingle{}preload'$>${\ttfamily and}$<$link rel=\char`\"{}prefetch\char`\"{}$>${\ttfamily }
\item {\ttfamily \mbox{[}preload-\/webpack-\/plugin\mbox{]}(\href{https://github.com/GoogleChrome/preload-webpack-plugin}{\tt https\+://github.\+com/\+Google\+Chrome/preload-\/webpack-\/plugin}) for automatically wiring up asynchronous (and other types) of Java\+Script chunks using}$<$link rel=\char`\"{}preload\char`\"{}$>${\ttfamily helping with lazy-\/loading}
\item {\ttfamily \mbox{[}link-\/media-\/html-\/webpack-\/plugin\mbox{]}(\href{https://github.com/yaycmyk/link-media-html-webpack-plugin}{\tt https\+://github.\+com/yaycmyk/link-\/media-\/html-\/webpack-\/plugin}) allows for injected stylesheet}$<$link$>${\ttfamily tags to have their media attribute set automatically; useful for providing specific desktop/mobile/print etc. stylesheets that the browser will conditionally download}
\item {\ttfamily \mbox{[}inline-\/chunk-\/manifest-\/html-\/webpack-\/plugin\mbox{]}(\href{https://github.com/jouni-kantola/inline-chunk-manifest-html-webpack-plugin}{\tt https\+://github.\+com/jouni-\/kantola/inline-\/chunk-\/manifest-\/html-\/webpack-\/plugin}) for inlining webpack\textquotesingle{}s chunk manifest. Default extracts manifest and inlines in}$<$head$>$\`{}.
\end{DoxyItemize}

\subsection*{Basic Usage }

The plugin will generate an H\+T\+M\+L5 file for you that includes all your webpack bundles in the body using {\ttfamily script} tags. Just add the plugin to your webpack config as follows\+:


\begin{DoxyCode}
var HtmlWebpackPlugin = require('html-webpack-plugin');
var webpackConfig = \{
  entry: 'index.js',
  output: \{
    path: \_\_dirname + '/dist',
    filename: 'index\_bundle.js'
  \},
  plugins: [new HtmlWebpackPlugin()]
\};
\end{DoxyCode}


This will generate a file {\ttfamily dist/index.\+html} containing the following\+: 
\begin{DoxyCode}
<!DOCTYPE html>
<html>
  <head>
    <meta charset="UTF-8">
    <title>Webpack App</title>
  </head>
  <body>
    <script src="index\_bundle.js"></script>
  </body>
</html>
\end{DoxyCode}


If you have multiple webpack entry points, they will all be included with {\ttfamily script} tags in the generated H\+T\+ML.

If you have any C\+SS assets in webpack\textquotesingle{}s output (for example, C\+SS extracted with the \href{https://github.com/webpack/extract-text-webpack-plugin}{\tt Extract\+Text\+Plugin}) then these will be included with {\ttfamily $<$link$>$} tags in the H\+T\+ML head.

\subsection*{Configuration }

You can pass a hash of configuration options to {\ttfamily Html\+Webpack\+Plugin}. Allowed values are as follows\+:


\begin{DoxyItemize}
\item {\ttfamily title}\+: The title to use for the generated H\+T\+ML document.
\item {\ttfamily filename}\+: The file to write the H\+T\+ML to. Defaults to {\ttfamily index.\+html}. You can specify a subdirectory here too (eg\+: {\ttfamily assets/admin.\+html}).
\item {\ttfamily template}\+: Webpack require path to the template. Please see the https\+://github.com/jantimon/html-\/webpack-\/plugin/blob/master/docs/template-\/option.\+md \char`\"{}docs\char`\"{} for details.
\item {\ttfamily inject}\+: `true $\vert$ \textquotesingle{}head' $\vert$ \textquotesingle{}body\textquotesingle{} $\vert$ false{\ttfamily Inject all assets into the given}template{\ttfamily or}template\+Content{\ttfamily -\/ When passing}true{\ttfamily or}\textquotesingle{}body\textquotesingle{}{\ttfamily all javascript resources will be placed at the bottom of the body element.}\textquotesingle{}head\textquotesingle{}{\ttfamily will place the scripts in the head element. -\/}favicon{\ttfamily \+: Adds the given favicon path to the output html. -\/}minify{\ttfamily \+:}\{...\} $\vert$ false{\ttfamily Pass \mbox{[}html-\/minifier\mbox{]}(\href{https://github.com/kangax/html-minifier#options-quick-reference}{\tt https\+://github.\+com/kangax/html-\/minifier\#options-\/quick-\/reference})\textquotesingle{}s options as object to minify the output. -\/}hash{\ttfamily \+:}true $\vert$ false{\ttfamily if}true\`{} then append a unique webpack compilation hash to all included scripts and C\+SS files. This is useful for cache busting.
\item {\ttfamily cache}\+: {\ttfamily true $\vert$ false} if {\ttfamily true} (default) try to emit the file only if it was changed.
\item {\ttfamily show\+Errors}\+: {\ttfamily true $\vert$ false} if {\ttfamily true} (default) errors details will be written into the H\+T\+ML page.
\item {\ttfamily chunks}\+: Allows you to add only some chunks (e.\+g. only the unit-\/test chunk)
\item {\ttfamily chunks\+Sort\+Mode}\+: Allows to control how chunks should be sorted before they are included to the html. Allowed values\+: \textquotesingle{}none\textquotesingle{} $\vert$ \textquotesingle{}auto\textquotesingle{} $\vert$ \textquotesingle{}dependency\textquotesingle{} $\vert$ \{function\} -\/ default\+: \textquotesingle{}auto\textquotesingle{}
\item {\ttfamily exclude\+Chunks}\+: Allows you to skip some chunks (e.\+g. don\textquotesingle{}t add the unit-\/test chunk)
\item {\ttfamily xhtml}\+: {\ttfamily true $\vert$ false} If {\ttfamily true} render the {\ttfamily link} tags as self-\/closing, X\+H\+T\+ML compliant. Default is {\ttfamily false}
\end{DoxyItemize}

Here\textquotesingle{}s an example webpack config illustrating how to use these options\+: 
\begin{DoxyCode}
\{
  entry: 'index.js',
  output: \{
    path: \_\_dirname + '/dist',
    filename: 'index\_bundle.js'
  \},
  plugins: [
    new HtmlWebpackPlugin(\{
      title: 'My App',
      filename: 'assets/admin.html'
    \})
  ]
\}
\end{DoxyCode}


\subsection*{F\+AQ }


\begin{DoxyItemize}
\item https\+://github.com/jantimon/html-\/webpack-\/plugin/blob/master/docs/template-\/option.\+md \char`\"{}\+Why is my H\+T\+M\+L minified?\char`\"{}
\item https\+://github.com/jantimon/html-\/webpack-\/plugin/blob/master/docs/template-\/option.\+md \char`\"{}\+Why is my \`{}$<$\% ... \%$>$\`{} template not working?\char`\"{}
\item https\+://github.com/jantimon/html-\/webpack-\/plugin/blob/master/docs/template-\/option.\+md \char`\"{}\+How can I use handlebars/pug/ejs as template engine\char`\"{}
\end{DoxyItemize}

\subsection*{Generating Multiple H\+T\+ML Files }

To generate more than one H\+T\+ML file, declare the plugin more than once in your plugins array\+: 
\begin{DoxyCode}
\{
  entry: 'index.js',
  output: \{
    path: \_\_dirname + '/dist',
    filename: 'index\_bundle.js'
  \},
  plugins: [
    new HtmlWebpackPlugin(), // Generates default index.html
    new HtmlWebpackPlugin(\{  // Also generate a test.html
      filename: 'test.html',
      template: 'src/assets/test.html'
    \})
  ]
\}
\end{DoxyCode}


\subsection*{Writing Your Own Templates }

If the default generated H\+T\+ML doesn\textquotesingle{}t meet your needs you can supply your own template. The easiest way is to use the {\ttfamily template} option and pass a custom H\+T\+ML file. The html-\/webpack-\/plugin will automatically inject all necessary C\+SS, JS, manifest and favicon files into the markup.


\begin{DoxyCode}
plugins: [
  new HtmlWebpackPlugin(\{
    title: 'Custom template',
    template: 'my-index.ejs', // Load a custom template (ejs by default see the FAQ for details)
  \})
]
\end{DoxyCode}


{\ttfamily my-\/index.\+ejs}\+:


\begin{DoxyCode}
<!DOCTYPE html>
<html>
  <head>
    <meta http-equiv="Content-type" content="text/html; charset=utf-8"/>
    <title><%= htmlWebpackPlugin.options.title %></title>
  </head>
  <body>
  </body>
</html>
\end{DoxyCode}


If you already have a template loader, you can use it to parse the template. Please note that this will also happen if you specifiy the html-\/loader and use {\ttfamily .html} file as template.


\begin{DoxyCode}
module: \{
  loaders: [
    \{ test: /\(\backslash\).hbs$/, loader: "handlebars" \}
  ]
\},
plugins: [
  new HtmlWebpackPlugin(\{
    title: 'Custom template using Handlebars',
    template: 'my-index.hbs'
  \})
]
\end{DoxyCode}


You can use the lodash syntax out of the box. If the {\ttfamily inject} feature doesn\textquotesingle{}t fit your needs and you want full control over the asset placement use the \href{https://github.com/jaketrent/html-webpack-template/blob/86f285d5c790a6c15263f5cc50fd666d51f974fd/index.html}{\tt default template} of the \href{https://github.com/jaketrent/html-webpack-template}{\tt html-\/webpack-\/template project} as a starting point for writing your own.

The following variables are available in the template\+:
\begin{DoxyItemize}
\item {\ttfamily html\+Webpack\+Plugin}\+: data specific to this plugin
\begin{DoxyItemize}
\item {\ttfamily html\+Webpack\+Plugin.\+files}\+: a massaged representation of the {\ttfamily assets\+By\+Chunk\+Name} attribute of webpack\textquotesingle{}s \href{https://github.com/webpack/docs/wiki/node.js-api#stats}{\tt stats} object. It contains a mapping from entry point name to the bundle filename, eg\+: \`{}\`{}\`{}json \char`\"{}html\+Webpack\+Plugin\char`\"{}\+: \{ \char`\"{}files\char`\"{}\+: \{ \char`\"{}css\char`\"{}\+: \mbox{[} \char`\"{}main.\+css\char`\"{} \mbox{]}, \char`\"{}js\char`\"{}\+: \mbox{[} \char`\"{}assets/head\+\_\+bundle.\+js\char`\"{}, \char`\"{}assets/main\+\_\+bundle.\+js\char`\"{}\mbox{]}, \char`\"{}chunks\char`\"{}\+: \{ \char`\"{}head\char`\"{}\+: \{ \char`\"{}entry\char`\"{}\+: \char`\"{}assets/head\+\_\+bundle.\+js\char`\"{}, \char`\"{}css\char`\"{}\+: \mbox{[} \char`\"{}main.\+css\char`\"{} \mbox{]} \}, \char`\"{}main\char`\"{}\+: \{ \char`\"{}entry\char`\"{}\+: \char`\"{}assets/main\+\_\+bundle.\+js\char`\"{}, \char`\"{}css\char`\"{}\+: \mbox{[}\mbox{]} \}, \} \} \} \`{}\`{}\`{} If you\textquotesingle{}ve set a public\+Path in your webpack config this will be reflected correctly in this assets hash.
\item {\ttfamily html\+Webpack\+Plugin.\+options}\+: the options hash that was passed to the plugin. In addition to the options actually used by this plugin, you can use this hash to pass arbitrary data through to your template.
\end{DoxyItemize}
\item {\ttfamily webpack}\+: the webpack \href{https://github.com/webpack/docs/wiki/node.js-api#stats}{\tt stats} object. Note that this is the stats object as it was at the time the H\+T\+ML template was emitted and as such may not have the full set of stats that are available after the webpack run is complete.
\item {\ttfamily webpack\+Config}\+: the webpack configuration that was used for this compilation. This can be used, for example, to get the {\ttfamily public\+Path} ({\ttfamily webpack\+Config.\+output.\+public\+Path}).
\end{DoxyItemize}

\subsection*{Filtering chunks }

To include only certain chunks you can limit the chunks being used\+:


\begin{DoxyCode}
plugins: [
  new HtmlWebpackPlugin(\{
    chunks: ['app']
  \})
]
\end{DoxyCode}


It is also possible to exclude certain chunks by setting the {\ttfamily exclude\+Chunks} option\+:


\begin{DoxyCode}
plugins: [
  new HtmlWebpackPlugin(\{
    excludeChunks: ['dev-helper']
  \})
]
\end{DoxyCode}


\subsection*{Events }

To allow other \href{https://github.com/webpack/docs/wiki/plugins}{\tt plugins} to alter the H\+T\+ML this plugin executes the following events\+:

Async\+:


\begin{DoxyItemize}
\item {\ttfamily html-\/webpack-\/plugin-\/before-\/html-\/generation}
\item {\ttfamily html-\/webpack-\/plugin-\/before-\/html-\/processing}
\item {\ttfamily html-\/webpack-\/plugin-\/alter-\/asset-\/tags}
\item {\ttfamily html-\/webpack-\/plugin-\/after-\/html-\/processing}
\item {\ttfamily html-\/webpack-\/plugin-\/after-\/emit}
\end{DoxyItemize}

Sync\+:


\begin{DoxyItemize}
\item {\ttfamily html-\/webpack-\/plugin-\/alter-\/chunks}
\end{DoxyItemize}

Example implementation\+: \href{https://github.com/jantimon/html-webpack-harddisk-plugin}{\tt html-\/webpack-\/harddisk-\/plugin}

Usage\+:


\begin{DoxyCode}
// MyPlugin.js

function MyPlugin(options) \{
  // Configure your plugin with options...
\}

MyPlugin.prototype.apply = function(compiler) \{
  // ...
  compiler.plugin('compilation', function(compilation) \{
    console.log('The compiler is starting a new compilation...');

    compilation.plugin('html-webpack-plugin-before-html-processing', function(htmlPluginData, callback) \{
      htmlPluginData.html += 'The magic footer';
      callback(null, htmlPluginData);
    \});
  \});

\};

module.exports = MyPlugin;
\end{DoxyCode}
 Then in {\ttfamily webpack.\+config.\+js}


\begin{DoxyCode}
plugins: [
  new MyPlugin(\{options: ''\})
]
\end{DoxyCode}


Note that the callback must be passed the html\+Plugin\+Data in order to pass this onto any other plugins listening on the same {\ttfamily html-\/webpack-\/plugin-\/before-\/html-\/processing} event.

\section*{Contribution}

You\textquotesingle{}re free to contribute to this project by submitting \href{https://github.com/jantimon/html-webpack-plugin/issues}{\tt issues} and/or \href{https://github.com/jantimon/html-webpack-plugin/pulls}{\tt pull requests}. This project is test-\/driven, so keep in mind that every change and new feature should be covered by tests. This project uses the \href{https://github.com/Flet/semistandard}{\tt semistandard code style}.

Before running the tests, make sure to execute {\ttfamily yarn link} and {\ttfamily yarn link html-\/webpack-\/plugin} (or the npm variant of this).

\section*{License}

This project is licensed under \href{https://github.com/jantimon/html-webpack-plugin/blob/master/LICENSE}{\tt M\+IT}. 