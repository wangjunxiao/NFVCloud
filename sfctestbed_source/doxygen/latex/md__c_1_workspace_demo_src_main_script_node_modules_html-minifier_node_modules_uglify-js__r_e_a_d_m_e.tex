Uglify\+JS is a Java\+Script parser, minifier, compressor and beautifier toolkit.

\paragraph*{Note\+:}


\begin{DoxyItemize}
\item $\ast$$\ast${\ttfamily uglify-\/js@3} has a simplified \href{#api-reference}{\tt A\+PI} and \href{#command-line-usage}{\tt C\+LI} that is not backwards compatible with \href{https://github.com/mishoo/UglifyJS2/tree/v2.x}{\tt {\ttfamily uglify-\/js@2}}$\ast$$\ast$.
\item {\bfseries Documentation for Uglify\+JS {\ttfamily 2.\+x} releases can be found \href{https://github.com/mishoo/UglifyJS2/tree/v2.x}{\tt here}}.
\item {\ttfamily uglify-\/js} only supports E\+C\+M\+A\+Script 5 (E\+S5).
\item Those wishing to minify E\+S2015+ (E\+S6+) should use the {\ttfamily npm} package \href{https://github.com/mishoo/UglifyJS2/tree/harmony}{\tt {\bfseries uglify-\/es}}.
\end{DoxyItemize}

\subsection*{Install }

First make sure you have installed the latest version of \href{http://nodejs.org/}{\tt node.\+js} (You may need to restart your computer after this step).

From N\+PM for use as a command line app\+: \begin{DoxyVerb}npm install uglify-js -g
\end{DoxyVerb}


From N\+PM for programmatic use\+: \begin{DoxyVerb}npm install uglify-js
\end{DoxyVerb}


\section*{Command line usage}

\begin{DoxyVerb}uglifyjs [input files] [options]
\end{DoxyVerb}


Uglify\+JS can take multiple input files. It\textquotesingle{}s recommended that you pass the input files first, then pass the options. Uglify\+JS will parse input files in sequence and apply any compression options. The files are parsed in the same global scope, that is, a reference from a file to some variable/function declared in another file will be matched properly.

If no input file is specified, Uglify\+JS will read from S\+T\+D\+IN.

If you wish to pass your options before the input files, separate the two with a double dash to prevent input files being used as option arguments\+: \begin{DoxyVerb}uglifyjs --compress --mangle -- input.js
\end{DoxyVerb}


\subsubsection*{Command line options}

\`{}\`{}\`{} -\/h, --help Print usage information. {\ttfamily -\/-\/help options} for details on available options. -\/V, --version Print version number. -\/p, --parse $<$options$>$ Specify parser options\+: {\ttfamily acorn} Use Acorn for parsing. {\ttfamily bare\+\_\+returns} Allow return outside of functions. Useful when minifying Common\+JS modules and Userscripts that may be anonymous function wrapped (I\+I\+FE) by the .user.\+js engine {\ttfamily caller}. {\ttfamily expression} Parse a single expression, rather than a program (for parsing J\+S\+ON). {\ttfamily spidermonkey} Assume input files are Spider\+Monkey A\+ST format (as J\+S\+ON). -\/c, --compress \mbox{[}options\mbox{]} Enable compressor/specify compressor options\+: {\ttfamily pure\+\_\+funcs} List of functions that can be safely removed when their return values are not used. -\/m, --mangle \mbox{[}options\mbox{]} Mangle names/specify mangler options\+: {\ttfamily reserved} List of names that should not be mangled. --mangle-\/props \mbox{[}options\mbox{]} Mangle properties/specify mangler options\+: {\ttfamily builtins} Mangle property names that overlaps with standard Java\+Script globals. {\ttfamily debug} Add debug prefix and suffix. {\ttfamily domprops} Mangle property names that overlaps with D\+OM properties. {\ttfamily keep\+\_\+quoted} Only mangle unquoted properies. {\ttfamily regex} Only mangle matched property names. {\ttfamily reserved} List of names that should not be mangled. -\/b, --beautify \mbox{[}options\mbox{]} Beautify output/specify output options\+: {\ttfamily beautify} Enabled with {\ttfamily -\/-\/beautify} by default. {\ttfamily preamble} Preamble to prepend to the output. You can use this to insert a comment, for example for licensing information. This will not be parsed, but the source map will adjust for its presence. {\ttfamily quote\+\_\+style} Quote style\+: 0 -\/ auto 1 -\/ single 2 -\/ double 3 -\/ original {\ttfamily wrap\+\_\+iife} Wrap I\+I\+F\+Es in parenthesis. Note\+: you may want to disable {\ttfamily negate\+\_\+iife} under compressor options. -\/o, --output $<$file$>$ Output file path (default S\+T\+D\+O\+UT). Specify {\ttfamily ast} or {\ttfamily spidermonkey} to write Uglify\+JS or Spider\+Monkey A\+ST as J\+S\+ON to S\+T\+D\+O\+UT respectively. --comments \mbox{[}filter\mbox{]} Preserve copyright comments in the output. By default this works like Google Closure, keeping J\+S\+Doc-\/style comments that contain \char`\"{}@license\char`\"{} or \char`\"{}@preserve\char`\"{}. You can optionally pass one of the following arguments to this flag\+:
\begin{DoxyItemize}
\item \char`\"{}all\char`\"{} to keep all comments
\item a valid JS Reg\+Exp like {\ttfamily /foo/} or {\ttfamily /$^\wedge$!/} to keep only matching comments. Note that currently not {\itshape all} comments can be kept when compression is on, because of dead code removal or cascading statements into sequences. --config-\/file $<$file$>$ Read {\ttfamily minify()} options from J\+S\+ON file. -\/d, --define $<$expr$>$\mbox{[}=value\mbox{]} Global definitions. --ie8 Support non-\/standard Internet Explorer 8. Equivalent to setting {\ttfamily ie8\+: true} in {\ttfamily minify()} for {\ttfamily compress}, {\ttfamily mangle} and {\ttfamily output} options. By default Uglify\+JS will not try to be I\+E-\/proof. --keep-\/fnames Do not mangle/drop function names. Useful for code relying on Function.\+prototype.\+name. --name-\/cache $<$file$>$ File to hold mangled name mappings. --self Build Uglify\+JS as a library (implies --wrap Uglify\+JS) --source-\/map \mbox{[}options\mbox{]} Enable source map/specify source map options\+: {\ttfamily base} Path to compute relative paths from input files. {\ttfamily content} Input source map, useful if you\textquotesingle{}re compressing JS that was generated from some other original code. Specify \char`\"{}inline\char`\"{} if the source map is included within the sources. {\ttfamily filename} Name and/or location of the output source. {\ttfamily include\+Sources} Pass this flag if you want to include the content of source files in the source map as sources\+Content property. {\ttfamily root} Path to the original source to be included in the source map. {\ttfamily url} If specified, path to the source map to append in {\ttfamily //\# source\+Mapping\+U\+RL}. --timings Display operations run time on S\+T\+D\+E\+RR. --toplevel Compress and/or mangle variables in top level scope. --verbose Print diagnostic messages. --warn Print warning messages. --wrap $<$name$>$ Embed everything in a big function, making the “exports” and “global” variables available. You need to pass an argument to this option to specify the name that your module will take when included in, say, a browser. \`{}\`{}\`{}
\end{DoxyItemize}

Specify {\ttfamily -\/-\/output} ({\ttfamily -\/o}) to declare the output file. Otherwise the output goes to S\+T\+D\+O\+UT.

\subsection*{C\+LI source map options}

Uglify\+JS can generate a source map file, which is highly useful for debugging your compressed Java\+Script. To get a source map, pass {\ttfamily -\/-\/source-\/map -\/-\/output output.\+js} (source map will be written out to {\ttfamily output.\+js.\+map}).

Additional options\+:


\begin{DoxyItemize}
\item {\ttfamily -\/-\/source-\/map filename=$<$N\+A\+ME$>$} to specify the name of the source map.
\item {\ttfamily -\/-\/source-\/map root=$<$\mbox{\hyperlink{namespace_u_r_l}{U\+RL}}$>$} to pass the \mbox{\hyperlink{namespace_u_r_l}{U\+RL}} where the original files can be found. Otherwise Uglify\+JS assumes H\+T\+TP {\ttfamily X-\/\+Source\+Map} is being used and will omit the {\ttfamily //\# source\+Mapping\+U\+RL=} directive.
\item {\ttfamily -\/-\/source-\/map url=$<$\mbox{\hyperlink{namespace_u_r_l}{U\+RL}}$>$} to specify the \mbox{\hyperlink{namespace_u_r_l}{U\+RL}} where the source map can be found.
\end{DoxyItemize}

For example\+: \begin{DoxyVerb}uglifyjs js/file1.js js/file2.js \
         -o foo.min.js -c -m \
         --source-map root="http://foo.com/src",url=foo.min.js.map
\end{DoxyVerb}


The above will compress and mangle {\ttfamily file1.\+js} and {\ttfamily file2.\+js}, will drop the output in {\ttfamily foo.\+min.\+js} and the source map in {\ttfamily foo.\+min.\+js.\+map}. The source mapping will refer to {\ttfamily \href{http://foo.com/src/js/file1.js}{\tt http\+://foo.\+com/src/js/file1.\+js}} and {\ttfamily \href{http://foo.com/src/js/file2.js}{\tt http\+://foo.\+com/src/js/file2.\+js}} (in fact it will list {\ttfamily \href{http://foo.com/src}{\tt http\+://foo.\+com/src}} as the source map root, and the original files as {\ttfamily js/file1.\+js} and {\ttfamily js/file2.\+js}).

\subsubsection*{Composed source map}

When you\textquotesingle{}re compressing JS code that was output by a compiler such as Coffee\+Script, mapping to the JS code won\textquotesingle{}t be too helpful. Instead, you\textquotesingle{}d like to map back to the original code (i.\+e. Coffee\+Script). Uglify\+JS has an option to take an input source map. Assuming you have a mapping from Coffee\+Script → compiled JS, Uglify\+JS can generate a map from Coffee\+Script → compressed JS by mapping every token in the compiled JS to its original location.

To use this feature pass {\ttfamily -\/-\/source-\/map content=\char`\"{}/path/to/input/source.\+map\char`\"{}} or {\ttfamily -\/-\/source-\/map content=inline} if the source map is included inline with the sources.

\subsection*{C\+LI compress options}

You need to pass {\ttfamily -\/-\/compress} ({\ttfamily -\/c}) to enable the compressor. Optionally you can pass a comma-\/separated list of \href{#compress-options}{\tt compress options}.

Options are in the form {\ttfamily foo=bar}, or just {\ttfamily foo} (the latter implies a boolean option that you want to set {\ttfamily true}; it\textquotesingle{}s effectively a shortcut for {\ttfamily foo=true}).

Example\+: \begin{DoxyVerb}uglifyjs file.js -c toplevel,sequences=false
\end{DoxyVerb}


\subsection*{C\+LI mangle options}

To enable the mangler you need to pass {\ttfamily -\/-\/mangle} ({\ttfamily -\/m}). The following (comma-\/separated) options are supported\+:


\begin{DoxyItemize}
\item {\ttfamily toplevel} — mangle names declared in the top level scope (disabled by default).
\item {\ttfamily eval} — mangle names visible in scopes where {\ttfamily eval} or {\ttfamily with} are used (disabled by default).
\end{DoxyItemize}

When mangling is enabled but you want to prevent certain names from being mangled, you can declare those names with {\ttfamily -\/-\/mangle reserved} — pass a comma-\/separated list of names. For example\+: \begin{DoxyVerb}uglifyjs ... -m reserved=[$,require,exports]
\end{DoxyVerb}


to prevent the {\ttfamily require}, {\ttfamily exports} and {\ttfamily \$} names from being changed.

\subsubsection*{C\+LI mangling property names ({\ttfamily -\/-\/mangle-\/props})}

{\bfseries Note\+:} T\+H\+IS W\+I\+LL P\+R\+O\+B\+A\+B\+LY B\+R\+E\+AK Y\+O\+UR C\+O\+DE. Mangling property names is a separate step, different from variable name mangling. Pass {\ttfamily -\/-\/mangle-\/props} to enable it. It will mangle all properties in the input code with the exception of built in D\+OM properties and properties in core javascript classes. For example\+:


\begin{DoxyCode}
// example.js
var x = \{
    baz\_: 0,
    foo\_: 1,
    calc: function() \{
        return this.foo\_ + this.baz\_;
    \}
\};
x.bar\_ = 2;
x["baz\_"] = 3;
console.log(x.calc());
\end{DoxyCode}
 Mangle all properties (except for javascript {\ttfamily builtins})\+: 
\begin{DoxyCode}
$ uglifyjs example.js -c -m --mangle-props
\end{DoxyCode}
 
\begin{DoxyCode}
var x=\{o:0,\_:1,l:function()\{return this.\_+this.o\}\};x.t=2,x.o=3,console.log(x.l());
\end{DoxyCode}
 Mangle all properties except for {\ttfamily reserved} properties\+: 
\begin{DoxyCode}
$ uglifyjs example.js -c -m --mangle-props reserved=[foo\_,bar\_]
\end{DoxyCode}
 
\begin{DoxyCode}
var x=\{o:0,foo\_:1,\_:function()\{return this.foo\_+this.o\}\};x.bar\_=2,x.o=3,console.log(x.\_());
\end{DoxyCode}
 Mangle all properties matching a {\ttfamily regex}\+: 
\begin{DoxyCode}
$ uglifyjs example.js -c -m --mangle-props regex=/\_$/
\end{DoxyCode}
 
\begin{DoxyCode}
var x=\{o:0,\_:1,calc:function()\{return this.\_+this.o\}\};x.l=2,x.o=3,console.log(x.calc());
\end{DoxyCode}


Combining mangle properties options\+: 
\begin{DoxyCode}
$ uglifyjs example.js -c -m --mangle-props regex=/\_$/,reserved=[bar\_]
\end{DoxyCode}
 
\begin{DoxyCode}
var x=\{o:0,\_:1,calc:function()\{return this.\_+this.o\}\};x.bar\_=2,x.o=3,console.log(x.calc());
\end{DoxyCode}


In order for this to be of any use, we avoid mangling standard JS names by default ({\ttfamily -\/-\/mangle-\/props builtins} to override).

A default exclusion file is provided in {\ttfamily tools/domprops.\+json} which should cover most standard JS and D\+OM properties defined in various browsers. Pass {\ttfamily -\/-\/mangle-\/props domprops} to disable this feature.

A regular expression can be used to define which property names should be mangled. For example, {\ttfamily -\/-\/mangle-\/props regex=/$^\wedge$\+\_\+/} will only mangle property names that start with an underscore.

When you compress multiple files using this option, in order for them to work together in the end we need to ensure somehow that one property gets mangled to the same name in all of them. For this, pass {\ttfamily -\/-\/name-\/cache filename.\+json} and Uglify\+JS will maintain these mappings in a file which can then be reused. It should be initially empty. Example\+:


\begin{DoxyCode}
$ rm -f /tmp/cache.json  # start fresh
$ uglifyjs file1.js file2.js --mangle-props --name-cache /tmp/cache.json -o part1.js
$ uglifyjs file3.js file4.js --mangle-props --name-cache /tmp/cache.json -o part2.js
\end{DoxyCode}


Now, {\ttfamily part1.\+js} and {\ttfamily part2.\+js} will be consistent with each other in terms of mangled property names.

Using the name cache is not necessary if you compress all your files in a single call to Uglify\+JS.

\subsubsection*{Mangling unquoted names ({\ttfamily -\/-\/mangle-\/props keep\+\_\+quoted})}

Using quoted property name ({\ttfamily o\mbox{[}\char`\"{}foo\char`\"{}\mbox{]}}) reserves the property name ({\ttfamily foo}) so that it is not mangled throughout the entire script even when used in an unquoted style ({\ttfamily o.\+foo}). Example\+:


\begin{DoxyCode}
// stuff.js
var o = \{
    "foo": 1,
    bar: 3
\};
o.foo += o.bar;
console.log(o.foo);
\end{DoxyCode}
 
\begin{DoxyCode}
$ uglifyjs stuff.js --mangle-props keep\_quoted -c -m
\end{DoxyCode}
 
\begin{DoxyCode}
var o=\{foo:1,o:3\};o.foo+=o.o,console.log(o.foo);
\end{DoxyCode}


\subsubsection*{Debugging property name mangling}

You can also pass {\ttfamily -\/-\/mangle-\/props debug} in order to mangle property names without completely obscuring them. For example the property {\ttfamily o.\+foo} would mangle to {\ttfamily o.\+\_\+\$foo\$\+\_\+} with this option. This allows property mangling of a large codebase while still being able to debug the code and identify where mangling is breaking things.


\begin{DoxyCode}
$ uglifyjs stuff.js --mangle-props debug -c -m
\end{DoxyCode}
 
\begin{DoxyCode}
var o=\{\_$foo$\_:1,\_$bar$\_:3\};o.\_$foo$\_+=o.\_$bar$\_,console.log(o.\_$foo$\_);
\end{DoxyCode}


You can also pass a custom suffix using {\ttfamily -\/-\/mangle-\/props debug=X\+YZ}. This would then mangle {\ttfamily o.\+foo} to {\ttfamily o.\+\_\+\$foo\$\+X\+Y\+Z\+\_\+}. You can change this each time you compile a script to identify how a property got mangled. One technique is to pass a random number on every compile to simulate mangling changing with different inputs (e.\+g. as you update the input script with new properties), and to help identify mistakes like writing mangled keys to storage.

\section*{A\+PI Reference}

Assuming installation via N\+PM, you can load Uglify\+JS in your application like this\+: 
\begin{DoxyCode}
var UglifyJS = require("uglify-js");
\end{DoxyCode}


There is a single high level function, $\ast$$\ast${\ttfamily minify(code, options)}$\ast$$\ast$, which will perform all minification \href{#minify-options}{\tt phases} in a configurable manner. By default {\ttfamily minify()} will enable the options \href{#compress-options}{\tt {\ttfamily compress}} and \href{#mangle-options}{\tt {\ttfamily mangle}}. Example\+: 
\begin{DoxyCode}
var code = "function add(first, second) \{ return first + second; \}";
var result = UglifyJS.minify(code);
console.log(result.error); // runtime error, or `undefined` if no error
console.log(result.code);  // minified output: function add(n,d)\{return n+d\}
\end{DoxyCode}


You can {\ttfamily minify} more than one Java\+Script file at a time by using an object for the first argument where the keys are file names and the values are source code\+: 
\begin{DoxyCode}
var code = \{
    "file1.js": "function add(first, second) \{ return first + second; \}",
    "file2.js": "console.log(add(1 + 2, 3 + 4));"
\};
var result = UglifyJS.minify(code);
console.log(result.code);
// function add(d,n)\{return d+n\}console.log(add(3,7));
\end{DoxyCode}


The {\ttfamily toplevel} option\+: 
\begin{DoxyCode}
var code = \{
    "file1.js": "function add(first, second) \{ return first + second; \}",
    "file2.js": "console.log(add(1 + 2, 3 + 4));"
\};
var options = \{ toplevel: true \};
var result = UglifyJS.minify(code, options);
console.log(result.code);
// console.log(3+7);
\end{DoxyCode}


The {\ttfamily name\+Cache} option\+: 
\begin{DoxyCode}
var options = \{
    mangle: \{
        toplevel: true,
    \},
    nameCache: \{\}
\};
var result1 = UglifyJS.minify(\{
    "file1.js": "function add(first, second) \{ return first + second; \}"
\}, options);
var result2 = UglifyJS.minify(\{
    "file2.js": "console.log(add(1 + 2, 3 + 4));"
\}, options);
console.log(result1.code);
// function n(n,r)\{return n+r\}
console.log(result2.code);
// console.log(n(3,7));
\end{DoxyCode}


You may persist the name cache to the file system in the following way\+: 
\begin{DoxyCode}
var cacheFileName = "/tmp/cache.json";
var options = \{
    mangle: \{
        properties: true,
    \},
    nameCache: JSON.parse(fs.readFileSync(cacheFileName, "utf8"))
\};
fs.writeFileSync("part1.js", UglifyJS.minify(\{
    "file1.js": fs.readFileSync("file1.js", "utf8"),
    "file2.js": fs.readFileSync("file2.js", "utf8")
\}, options).code, "utf8");
fs.writeFileSync("part2.js", UglifyJS.minify(\{
    "file3.js": fs.readFileSync("file3.js", "utf8"),
    "file4.js": fs.readFileSync("file4.js", "utf8")
\}, options).code, "utf8");
fs.writeFileSync(cacheFileName, JSON.stringify(options.nameCache), "utf8");
\end{DoxyCode}


An example of a combination of {\ttfamily minify()} options\+: 
\begin{DoxyCode}
var code = \{
    "file1.js": "function add(first, second) \{ return first + second; \}",
    "file2.js": "console.log(add(1 + 2, 3 + 4));"
\};
var options = \{
    toplevel: true,
    compress: \{
        global\_defs: \{
            "@console.log": "alert"
        \},
        passes: 2
    \},
    output: \{
        beautify: false,
        preamble: "/* uglified */"
    \}
\};
var result = UglifyJS.minify(code, options);
console.log(result.code);
// /* uglified */
// alert(10);"
\end{DoxyCode}


To produce warnings\+: 
\begin{DoxyCode}
var code = "function f()\{ var u; return 2 + 3; \}";
var options = \{ warnings: true \};
var result = UglifyJS.minify(code, options);
console.log(result.error);    // runtime error, `undefined` in this case
console.log(result.warnings); // [ 'Dropping unused variable u [0:1,18]' ]
console.log(result.code);     // function f()\{return 5\}
\end{DoxyCode}


An error example\+: 
\begin{DoxyCode}
var result = UglifyJS.minify(\{"foo.js" : "if (0) else console.log(1);"\});
console.log(JSON.stringify(result.error));
// \{"message":"Unexpected token: keyword (else)","filename":"foo.js","line":1,"col":7,"pos":7\}
\end{DoxyCode}
 Note\+: unlike {\ttfamily uglify-\/js@2.\+x}, the {\ttfamily 3.\+x} A\+PI does not throw errors. To achieve a similar effect one could do the following\+: 
\begin{DoxyCode}
var result = UglifyJS.minify(code, options);
if (result.error) throw result.error;
\end{DoxyCode}


\subsection*{Minify options}


\begin{DoxyItemize}
\item {\ttfamily warnings} (default {\ttfamily false}) — pass {\ttfamily true} to return compressor warnings in {\ttfamily result.\+warnings}. Use the value {\ttfamily \char`\"{}verbose\char`\"{}} for more detailed warnings.
\item {\ttfamily parse} (default {\ttfamily \{\}}) — pass an object if you wish to specify some additional \href{#parse-options}{\tt parse options}.
\item {\ttfamily compress} (default {\ttfamily \{\}}) — pass {\ttfamily false} to skip compressing entirely. Pass an object to specify custom \href{#compress-options}{\tt compress options}.
\item {\ttfamily mangle} (default {\ttfamily true}) — pass {\ttfamily false} to skip mangling names, or pass an object to specify \href{#mangle-options}{\tt mangle options} (see below).
\begin{DoxyItemize}
\item {\ttfamily mangle.\+properties} (default {\ttfamily false}) — a subcategory of the mangle option. Pass an object to specify custom \href{#mangle-properties-options}{\tt mangle property options}.
\end{DoxyItemize}
\item {\ttfamily output} (default {\ttfamily null}) — pass an object if you wish to specify additional \href{#output-options}{\tt output options}. The defaults are optimized for best compression.
\item {\ttfamily source\+Map} (default {\ttfamily false}) -\/ pass an object if you wish to specify \href{#source-map-options}{\tt source map options}.
\item {\ttfamily toplevel} (default {\ttfamily false}) -\/ set to {\ttfamily true} if you wish to enable top level variable and function name mangling and to drop unused variables and functions.
\item {\ttfamily name\+Cache} (default {\ttfamily null}) -\/ pass an empty object {\ttfamily \{\}} or a previously used {\ttfamily name\+Cache} object if you wish to cache mangled variable and property names across multiple invocations of {\ttfamily minify()}. Note\+: this is a read/write property. {\ttfamily minify()} will read the name cache state of this object and update it during minification so that it may be reused or externally persisted by the user.
\item {\ttfamily ie8} (default {\ttfamily false}) -\/ set to {\ttfamily true} to support I\+E8.
\end{DoxyItemize}

\subsection*{Minify options structure}


\begin{DoxyCode}
\{
    warnings: false,
    parse: \{
        // parse options
    \},
    compress: \{
        // compress options
    \},
    mangle: \{
        // mangle options

        properties: \{
            // mangle property options
        \}
    \},
    output: \{
        // output options
    \},
    sourceMap: \{
        // source map options
    \},
    nameCache: null, // or specify a name cache object
    toplevel: false,
    ie8: false,
\}
\end{DoxyCode}


\subsubsection*{Source map options}

To generate a source map\+: 
\begin{DoxyCode}
var result = UglifyJS.minify(\{"file1.js": "var a = function() \{\};"\}, \{
    sourceMap: \{
        filename: "out.js",
        url: "out.js.map"
    \}
\});
console.log(result.code); // minified output
console.log(result.map);  // source map
\end{DoxyCode}


Note that the source map is not saved in a file, it\textquotesingle{}s just returned in {\ttfamily result.\+map}. The value passed for {\ttfamily source\+Map.\+url} is only used to set {\ttfamily //\# source\+Mapping\+U\+RL=out.\+js.\+map} in {\ttfamily result.\+code}. The value of {\ttfamily filename} is only used to set {\ttfamily file} attribute (see \href{https://docs.google.com/document/d/1U1RGAehQwRypUTovF1KRlpiOFze0b-_2gc6fAH0KY0k}{\tt the spec}) in source map file.

You can set option {\ttfamily source\+Map.\+url} to be {\ttfamily \char`\"{}inline\char`\"{}} and source map will be appended to code.

You can also specify source\+Root property to be included in source map\+: 
\begin{DoxyCode}
var result = UglifyJS.minify(\{"file1.js": "var a = function() \{\};"\}, \{
    sourceMap: \{
        root: "http://example.com/src",
        url: "out.js.map"
    \}
\});
\end{DoxyCode}


If you\textquotesingle{}re compressing compiled Java\+Script and have a source map for it, you can use {\ttfamily source\+Map.\+content}\+: \`{}\`{}\`{}javascript var result = Uglify\+J\+S.\+minify(\{\char`\"{}compiled.\+js\char`\"{}\+: \char`\"{}compiled code\char`\"{}\}, \{ source\+Map\+: \{ content\+: \char`\"{}content from compiled.\+js.\+map\char`\"{}, url\+: \char`\"{}minified.\+js.\+map\char`\"{} \} \}); // same as before, it returns {\ttfamily code} and {\ttfamily map} \`{}\`{}\`{}

If you\textquotesingle{}re using the {\ttfamily X-\/\+Source\+Map} header instead, you can just omit {\ttfamily source\+Map.\+url}.

\subsection*{Parse options}


\begin{DoxyItemize}
\item {\ttfamily bare\+\_\+returns} (default {\ttfamily false}) -- support top level {\ttfamily return} statements
\item {\ttfamily html5\+\_\+comments} (default {\ttfamily true})
\item {\ttfamily shebang} (default {\ttfamily true}) -- support {\ttfamily \#!command} as the first line
\end{DoxyItemize}

\subsection*{Compress options}


\begin{DoxyItemize}
\item {\ttfamily sequences} (default\+: true) -- join consecutive simple statements using the comma operator. May be set to a positive integer to specify the maximum number of consecutive comma sequences that will be generated. If this option is set to {\ttfamily true} then the default {\ttfamily sequences} limit is {\ttfamily 200}. Set option to {\ttfamily false} or {\ttfamily 0} to disable. The smallest {\ttfamily sequences} length is {\ttfamily 2}. A {\ttfamily sequences} value of {\ttfamily 1} is grandfathered to be equivalent to {\ttfamily true} and as such means {\ttfamily 200}. On rare occasions the default sequences limit leads to very slow compress times in which case a value of {\ttfamily 20} or less is recommended.
\item {\ttfamily properties} -- rewrite property access using the dot notation, for example {\ttfamily foo\mbox{[}\char`\"{}bar\char`\"{}\mbox{]} → foo.\+bar}
\item {\ttfamily dead\+\_\+code} -- remove unreachable code
\item {\ttfamily drop\+\_\+debugger} -- remove {\ttfamily debugger;} statements
\item {\ttfamily unsafe} (default\+: false) -- apply \char`\"{}unsafe\char`\"{} transformations (discussion below)
\item {\ttfamily unsafe\+\_\+comps} (default\+: false) -- Reverse {\ttfamily $<$} and {\ttfamily $<$=} to {\ttfamily $>$} and {\ttfamily $>$=} to allow improved compression. This might be unsafe when an at least one of two operands is an object with computed values due the use of methods like {\ttfamily get}, or {\ttfamily value\+Of}. This could cause change in execution order after operands in the comparison are switching. Compression only works if both {\ttfamily comparisons} and {\ttfamily unsafe\+\_\+comps} are both set to true.
\item {\ttfamily unsafe\+\_\+\+Func} (default\+: false) -- compress and mangle {\ttfamily Function(args, code)} when both {\ttfamily args} and {\ttfamily code} are string literals.
\item {\ttfamily unsafe\+\_\+math} (default\+: false) -- optimize numerical expressions like {\ttfamily 2 $\ast$ x $\ast$ 3} into {\ttfamily 6 $\ast$ x}, which may give imprecise floating point results.
\item {\ttfamily unsafe\+\_\+proto} (default\+: false) -- optimize expressions like {\ttfamily Array.\+prototype.\+slice.\+call(a)} into {\ttfamily \mbox{[}\mbox{]}.slice.\+call(a)}
\item {\ttfamily unsafe\+\_\+regexp} (default\+: false) -- enable substitutions of variables with {\ttfamily Reg\+Exp} values the same way as if they are constants.
\item {\ttfamily conditionals} -- apply optimizations for {\ttfamily if}-\/s and conditional expressions
\item {\ttfamily comparisons} -- apply certain optimizations to binary nodes, for example\+: {\ttfamily !(a $<$= b) → a $>$ b} (only when {\ttfamily unsafe\+\_\+comps}), attempts to negate binary nodes, e.\+g. {\ttfamily a = !b \&\& !c \&\& !d \&\& !e → a=!(b$\vert$$\vert$c$\vert$$\vert$d$\vert$$\vert$e)} etc.
\item {\ttfamily evaluate} -- attempt to evaluate constant expressions
\item {\ttfamily booleans} -- various optimizations for boolean context, for example {\ttfamily !!a ? b \+: c → a ? b \+: c}
\item {\ttfamily loops} -- optimizations for {\ttfamily do}, {\ttfamily while} and {\ttfamily for} loops when we can statically determine the condition
\item {\ttfamily unused} -- drop unreferenced functions and variables (simple direct variable assignments do not count as references unless set to {\ttfamily \char`\"{}keep\+\_\+assign\char`\"{}})
\item {\ttfamily toplevel} -- drop unreferenced functions ({\ttfamily \char`\"{}funcs\char`\"{}}) and/or variables ({\ttfamily \char`\"{}vars\char`\"{}}) in the top level scope ({\ttfamily false} by default, {\ttfamily true} to drop both unreferenced functions and variables)
\item {\ttfamily top\+\_\+retain} -- prevent specific toplevel functions and variables from {\ttfamily unused} removal (can be array, comma-\/separated, Reg\+Exp or function. Implies {\ttfamily toplevel})
\item {\ttfamily hoist\+\_\+funs} -- hoist function declarations
\item {\ttfamily hoist\+\_\+vars} (default\+: false) -- hoist {\ttfamily var} declarations (this is {\ttfamily false} by default because it seems to increase the size of the output in general)
\item {\ttfamily if\+\_\+return} -- optimizations for if/return and if/continue
\item {\ttfamily inline} -- embed simple functions
\item {\ttfamily join\+\_\+vars} -- join consecutive {\ttfamily var} statements
\item {\ttfamily cascade} -- small optimization for sequences, transform {\ttfamily x, x} into {\ttfamily x} and {\ttfamily x = something(), x} into {\ttfamily x = something()}
\item {\ttfamily collapse\+\_\+vars} -- Collapse single-\/use non-\/constant variables -\/ side effects permitting.
\item {\ttfamily reduce\+\_\+vars} -- Improve optimization on variables assigned with and used as constant values.
\item {\ttfamily warnings} -- display warnings when dropping unreachable code or unused declarations etc.
\item {\ttfamily negate\+\_\+iife} -- negate \char`\"{}\+Immediately-\/\+Called Function Expressions\char`\"{} where the return value is discarded, to avoid the parens that the code generator would insert.
\item {\ttfamily pure\+\_\+getters} -- the default is {\ttfamily false}. If you pass {\ttfamily true} for this, Uglify\+JS will assume that object property access (e.\+g. {\ttfamily foo.\+bar} or {\ttfamily foo\mbox{[}\char`\"{}bar\char`\"{}\mbox{]}}) doesn\textquotesingle{}t have any side effects. Specify {\ttfamily \char`\"{}strict\char`\"{}} to treat {\ttfamily foo.\+bar} as side-\/effect-\/free only when {\ttfamily foo} is certain to not throw, i.\+e. not {\ttfamily null} or {\ttfamily undefined}.
\item {\ttfamily pure\+\_\+funcs} -- default {\ttfamily null}. You can pass an array of names and Uglify\+JS will assume that those functions do not produce side effects. D\+A\+N\+G\+ER\+: will not check if the name is redefined in scope. An example case here, for instance {\ttfamily var q = Math.\+floor(a/b)}. If variable {\ttfamily q} is not used elsewhere, Uglify\+JS will drop it, but will still keep the {\ttfamily Math.\+floor(a/b)}, not knowing what it does. You can pass `pure\+\_\+funcs\+: \mbox{[} \textquotesingle{}Math.\+floor' \mbox{]}\`{} to let it know that this function won\textquotesingle{}t produce any side effect, in which case the whole statement would get discarded. The current implementation adds some overhead (compression will be slower).
\item {\ttfamily drop\+\_\+console} -- default {\ttfamily false}. Pass {\ttfamily true} to discard calls to {\ttfamily console.$\ast$} functions. If you wish to drop a specific function call such as {\ttfamily console.\+info} and/or retain side effects from function arguments after dropping the function call then use {\ttfamily pure\+\_\+funcs} instead.
\item {\ttfamily expression} -- default {\ttfamily false}. Pass {\ttfamily true} to preserve completion values from terminal statements without {\ttfamily return}, e.\+g. in bookmarklets.
\item {\ttfamily keep\+\_\+fargs} -- default {\ttfamily true}. Prevents the compressor from discarding unused function arguments. You need this for code which relies on {\ttfamily Function.\+length}.
\item {\ttfamily keep\+\_\+fnames} -- default {\ttfamily false}. Pass {\ttfamily true} to prevent the compressor from discarding function names. Useful for code relying on {\ttfamily Function.\+prototype.\+name}. See also\+: the {\ttfamily keep\+\_\+fnames} \href{#mangle}{\tt mangle option}.
\item {\ttfamily passes} -- default {\ttfamily 1}. Number of times to run compress with a maximum of 3. In some cases more than one pass leads to further compressed code. Keep in mind more passes will take more time.
\item {\ttfamily keep\+\_\+infinity} -- default {\ttfamily false}. Pass {\ttfamily true} to prevent {\ttfamily Infinity} from being compressed into {\ttfamily 1/0}, which may cause performance issues on Chrome.
\item {\ttfamily side\+\_\+effects} -- default {\ttfamily true}. Pass {\ttfamily false} to disable potentially dropping functions marked as \char`\"{}pure\char`\"{}. A function call is marked as \char`\"{}pure\char`\"{} if a comment annotation {\ttfamily /$\ast$@\+\_\+\+\_\+\+P\+U\+R\+E\+\_\+\+\_\+$\ast$/} or {\ttfamily /$\ast$\#\+\_\+\+\_\+\+P\+U\+R\+E\+\_\+\+\_\+$\ast$/} immediately precedes the call. For example\+: {\ttfamily /$\ast$@\+\_\+\+\_\+\+P\+U\+R\+E\+\_\+\+\_\+$\ast$/foo();}
\end{DoxyItemize}

\subsection*{Mangle options}


\begin{DoxyItemize}
\item {\ttfamily reserved} (default {\ttfamily \mbox{[}\mbox{]}}). Pass an array of identifiers that should be excluded from mangling. Example\+: {\ttfamily \mbox{[}\char`\"{}foo\char`\"{}, \char`\"{}bar\char`\"{}\mbox{]}}.
\item {\ttfamily toplevel} (default {\ttfamily false}). Pass {\ttfamily true} to mangle names declared in the top level scope.
\item {\ttfamily keep\+\_\+fnames} (default {\ttfamily false}). Pass {\ttfamily true} to not mangle function names. Useful for code relying on {\ttfamily Function.\+prototype.\+name}. See also\+: the {\ttfamily keep\+\_\+fnames} \href{#compress-options}{\tt compress option}.
\item {\ttfamily eval} (default {\ttfamily false}). Pass {\ttfamily true} to mangle names visible in scopes where {\ttfamily eval} or {\ttfamily with} are used.
\end{DoxyItemize}

Examples\+:


\begin{DoxyCode}
// test.js
var globalVar;
function funcName(firstLongName, anotherLongName) \{
    var myVariable = firstLongName +  anotherLongName;
\}
\end{DoxyCode}
 
\begin{DoxyCode}
var code = fs.readFileSync("test.js", "utf8");

UglifyJS.minify(code).code;
// 'function funcName(a,n)\{\}var globalVar;'

UglifyJS.minify(code, \{ mangle: \{ reserved: ['firstLongName'] \} \}).code;
// 'function funcName(firstLongName,a)\{\}var globalVar;'

UglifyJS.minify(code, \{ mangle: \{ toplevel: true \} \}).code;
// 'function n(n,a)\{\}var a;'
\end{DoxyCode}


\subsubsection*{Mangle properties options}


\begin{DoxyItemize}
\item {\ttfamily reserved} (default\+: {\ttfamily \mbox{[}\mbox{]}}) -- Do not mangle property names listed in the {\ttfamily reserved} array.
\item {\ttfamily regex} (default\+: {\ttfamily null}) -\/— Pass a Reg\+Exp literal to only mangle property names matching the regular expression.
\item {\ttfamily keep\+\_\+quoted} (default\+: {\ttfamily false}) -\/— Only mangle unquoted property names.
\item {\ttfamily debug} (default\+: {\ttfamily false}) -\/— Mangle names with the original name still present. Pass an empty string {\ttfamily \char`\"{}\char`\"{}} to enable, or a non-\/empty string to set the debug suffix.
\item {\ttfamily builtins} (default\+: {\ttfamily false}) -- Use {\ttfamily true} to allow the mangling of builtin D\+OM properties. Not recommended to override this setting.
\end{DoxyItemize}

\subsection*{Output options}

The code generator tries to output shortest code possible by default. In case you want beautified output, pass {\ttfamily -\/-\/beautify} ({\ttfamily -\/b}). Optionally you can pass additional arguments that control the code output\+:


\begin{DoxyItemize}
\item {\ttfamily ascii\+\_\+only} (default {\ttfamily false}) -- escape Unicode characters in strings and regexps (affects directives with non-\/ascii characters becoming invalid)
\item {\ttfamily beautify} (default {\ttfamily true}) -- whether to actually beautify the output. Passing {\ttfamily -\/b} will set this to true, but you might need to pass {\ttfamily -\/b} even when you want to generate minified code, in order to specify additional arguments, so you can use {\ttfamily -\/b beautify=false} to override it.
\item {\ttfamily bracketize} (default {\ttfamily false}) -- always insert brackets in {\ttfamily if}, {\ttfamily for}, {\ttfamily do}, {\ttfamily while} or {\ttfamily with} statements, even if their body is a single statement.
\item {\ttfamily comments} (default {\ttfamily false}) -- pass {\ttfamily true} or {\ttfamily \char`\"{}all\char`\"{}} to preserve all comments, {\ttfamily \char`\"{}some\char`\"{}} to preserve some comments, a regular expression string (e.\+g. {\ttfamily /$^\wedge$!/}) or a function.
\item {\ttfamily indent\+\_\+level} (default 4)
\item {\ttfamily indent\+\_\+start} (default 0) -- prefix all lines by that many spaces
\item {\ttfamily inline\+\_\+script} (default {\ttfamily false}) -- escape the slash in occurrences of {\ttfamily $<$/script} in strings
\item {\ttfamily keep\+\_\+quoted\+\_\+props} (default {\ttfamily false}) -- when turned on, prevents stripping quotes from property names in object literals.
\item {\ttfamily max\+\_\+line\+\_\+len} (default {\ttfamily false}) -- maximum line length (for uglified code)
\item {\ttfamily preamble} (default {\ttfamily null}) -- when passed it must be a string and it will be prepended to the output literally. The source map will adjust for this text. Can be used to insert a comment containing licensing information, for example.
\item {\ttfamily preserve\+\_\+line} (default {\ttfamily false}) -- pass {\ttfamily true} to preserve lines, but it only works if {\ttfamily beautify} is set to {\ttfamily false}.
\item {\ttfamily quote\+\_\+keys} (default {\ttfamily false}) -- pass {\ttfamily true} to quote all keys in literal objects
\item {\ttfamily quote\+\_\+style} (default {\ttfamily 0}) -- preferred quote style for strings (affects quoted property names and directives as well)\+:
\begin{DoxyItemize}
\item {\ttfamily 0} -- prefers double quotes, switches to single quotes when there are more double quotes in the string itself. {\ttfamily 0} is best for gzip size.
\item {\ttfamily 1} -- always use single quotes
\item {\ttfamily 2} -- always use double quotes
\item {\ttfamily 3} -- always use the original quotes
\end{DoxyItemize}
\item {\ttfamily semicolons} (default {\ttfamily true}) -- separate statements with semicolons. If you pass {\ttfamily false} then whenever possible we will use a newline instead of a semicolon, leading to more readable output of uglified code (size before gzip could be smaller; size after gzip insignificantly larger).
\item {\ttfamily shebang} (default {\ttfamily true}) -- preserve shebang {\ttfamily \#!} in preamble (bash scripts)
\item {\ttfamily width} (default 80) -- only takes effect when beautification is on, this specifies an (orientative) line width that the beautifier will try to obey. It refers to the width of the line text (excluding indentation). It doesn\textquotesingle{}t work very well currently, but it does make the code generated by Uglify\+JS more readable.
\item {\ttfamily wrap\+\_\+iife} (default {\ttfamily false}) -- pass {\ttfamily true} to wrap immediately invoked function expressions. See \href{https://github.com/mishoo/UglifyJS2/issues/640}{\tt \#640} for more details.
\end{DoxyItemize}

\section*{Miscellaneous}

\subsubsection*{Keeping copyright notices or other comments}

You can pass {\ttfamily -\/-\/comments} to retain certain comments in the output. By default it will keep J\+S\+Doc-\/style comments that contain \char`\"{}@preserve\char`\"{}, \char`\"{}@license\char`\"{} or \char`\"{}@cc\+\_\+on\char`\"{} (conditional compilation for IE). You can pass {\ttfamily -\/-\/comments all} to keep all the comments, or a valid Java\+Script regexp to keep only comments that match this regexp. For example {\ttfamily -\/-\/comments /$^\wedge$!/} will keep comments like {\ttfamily /$\ast$! Copyright Notice $\ast$/}.

Note, however, that there might be situations where comments are lost. For example\+: 
\begin{DoxyCode}
function f() \{
    /** @preserve Foo Bar */
    function g() \{
        // this function is never called
    \}
    return something();
\}
\end{DoxyCode}


Even though it has \char`\"{}@preserve\char`\"{}, the comment will be lost because the inner function {\ttfamily g} (which is the A\+ST node to which the comment is attached to) is discarded by the compressor as not referenced.

The safest comments where to place copyright information (or other info that needs to be kept in the output) are comments attached to toplevel nodes.

\subsubsection*{The {\ttfamily unsafe} {\ttfamily compress} option}

It enables some transformations that {\itshape might} break code logic in certain contrived cases, but should be fine for most code. You might want to try it on your own code, it should reduce the minified size. Here\textquotesingle{}s what happens when this flag is on\+:


\begin{DoxyItemize}
\item {\ttfamily new Array(1, 2, 3)} or {\ttfamily Array(1, 2, 3)} → {\ttfamily \mbox{[} 1, 2, 3 \mbox{]}}
\item {\ttfamily new Object()} → {\ttfamily \{\}}
\item {\ttfamily String(exp)} or {\ttfamily exp.\+to\+String()} → {\ttfamily \char`\"{}\char`\"{} + exp}
\item {\ttfamily new Object/\+Reg\+Exp/\+Function/\+Error/\+Array (...)} → we discard the {\ttfamily new}
\item {\ttfamily typeof foo == \char`\"{}undefined\char`\"{}} → {\ttfamily foo === void 0}
\item {\ttfamily void 0} → {\ttfamily undefined} (if there is a variable named \char`\"{}undefined\char`\"{} in scope; we do it because the variable name will be mangled, typically reduced to a single character)
\end{DoxyItemize}

\subsubsection*{Conditional compilation}

You can use the {\ttfamily -\/-\/define} ({\ttfamily -\/d}) switch in order to declare global variables that Uglify\+JS will assume to be constants (unless defined in scope). For example if you pass {\ttfamily -\/-\/define D\+E\+B\+UG=false} then, coupled with dead code removal Uglify\+JS will discard the following from the output\+: 
\begin{DoxyCode}
if (DEBUG) \{
    console.log("debug stuff");
\}
\end{DoxyCode}


You can specify nested constants in the form of {\ttfamily -\/-\/define env.\+D\+E\+B\+UG=false}.

Uglify\+JS will warn about the condition being always false and about dropping unreachable code; for now there is no option to turn off only this specific warning, you can pass {\ttfamily warnings=false} to turn off {\itshape all} warnings.

Another way of doing that is to declare your globals as constants in a separate file and include it into the build. For example you can have a {\ttfamily build/defines.\+js} file with the following\+: 
\begin{DoxyCode}
var DEBUG = false;
var PRODUCTION = true;
// etc.
\end{DoxyCode}


and build your code like this\+: \begin{DoxyVerb}uglifyjs build/defines.js js/foo.js js/bar.js... -c
\end{DoxyVerb}


Uglify\+JS will notice the constants and, since they cannot be altered, it will evaluate references to them to the value itself and drop unreachable code as usual. The build will contain the {\ttfamily const} declarations if you use them. If you are targeting $<$ E\+S6 environments which does not support {\ttfamily const}, using {\ttfamily var} with {\ttfamily reduce\+\_\+vars} (enabled by default) should suffice.

\subsubsection*{Conditional compilation A\+PI}

You can also use conditional compilation via the programmatic A\+PI. With the difference that the property name is {\ttfamily global\+\_\+defs} and is a compressor property\+:


\begin{DoxyCode}
var result = UglifyJS.minify(fs.readFileSync("input.js", "utf8"), \{
    compress: \{
        dead\_code: true,
        global\_defs: \{
            DEBUG: false
        \}
    \}
\});
\end{DoxyCode}


To replace an identifier with an arbitrary non-\/constant expression it is necessary to prefix the {\ttfamily global\+\_\+defs} key with {\ttfamily \char`\"{}@\char`\"{}} to instruct Uglify\+JS to parse the value as an expression\+: 
\begin{DoxyCode}
UglifyJS.minify("alert('hello');", \{
    compress: \{
        global\_defs: \{
            "@alert": "console.log"
        \}
    \}
\}).code;
// returns: 'console.log("hello");'
\end{DoxyCode}


Otherwise it would be replaced as string literal\+: 
\begin{DoxyCode}
UglifyJS.minify("alert('hello');", \{
    compress: \{
        global\_defs: \{
            "alert": "console.log"
        \}
    \}
\}).code;
// returns: '"console.log"("hello");'
\end{DoxyCode}


\#\#\# Using native Uglify A\+ST with {\ttfamily minify()} 
\begin{DoxyCode}
// example: parse only, produce native Uglify AST

var result = UglifyJS.minify(code, \{
    parse: \{\},
    compress: false,
    mangle: false,
    output: \{
        ast: true,
        code: false  // optional - faster if false
    \}
\});

// result.ast contains native Uglify AST
\end{DoxyCode}
 
\begin{DoxyCode}
// example: accept native Uglify AST input and then compress and mangle
//          to produce both code and native AST.

var result = UglifyJS.minify(ast, \{
    compress: \{\},
    mangle: \{\},
    output: \{
        ast: true,
        code: true  // optional - faster if false
    \}
\});

// result.ast contains native Uglify AST
// result.code contains the minified code in string form.
\end{DoxyCode}


\subsubsection*{Working with Uglify A\+ST}

Transversal and transformation of the native A\+ST can be performed through \href{http://lisperator.net/uglifyjs/walk}{\tt {\ttfamily Tree\+Walker}} and \href{http://lisperator.net/uglifyjs/transform}{\tt {\ttfamily Tree\+Transformer}} respectively.

\subsubsection*{E\+S\+Tree / Spider\+Monkey A\+ST}

Uglify\+JS has its own abstract syntax tree format; for \href{http://lisperator.net/blog/uglifyjs-why-not-switching-to-spidermonkey-ast/}{\tt practical reasons} we can\textquotesingle{}t easily change to using the Spider\+Monkey A\+ST internally. However, Uglify\+JS now has a converter which can import a Spider\+Monkey A\+ST.

For example \href{https://github.com/ternjs/acorn}{\tt Acorn} is a super-\/fast parser that produces a Spider\+Monkey A\+ST. It has a small C\+LI utility that parses one file and dumps the A\+ST in J\+S\+ON on the standard output. To use Uglify\+JS to mangle and compress that\+: \begin{DoxyVerb}acorn file.js | uglifyjs -p spidermonkey -m -c
\end{DoxyVerb}


The {\ttfamily -\/p spidermonkey} option tells Uglify\+JS that all input files are not Java\+Script, but JS code described in Spider\+Monkey A\+ST in J\+S\+ON. Therefore we don\textquotesingle{}t use our own parser in this case, but just transform that A\+ST into our internal A\+ST.

\subsubsection*{Use Acorn for parsing}

More for fun, I added the {\ttfamily -\/p acorn} option which will use Acorn to do all the parsing. If you pass this option, Uglify\+JS will {\ttfamily require(\char`\"{}acorn\char`\"{})}.

Acorn is really fast (e.\+g. 250ms instead of 380ms on some 650K code), but converting the Spider\+Monkey tree that Acorn produces takes another 150ms so in total it\textquotesingle{}s a bit more than just using Uglify\+JS\textquotesingle{}s own parser. 