\href{https://www.npmjs.com/package/sockjs-client}{\tt }\href{https://travis-ci.org/sockjs/sockjs-client}{\tt }\href{https://david-dm.org/sockjs/sockjs-client}{\tt }\href{https://gitter.im/sockjs/sockjs-client}{\tt } \href{https://saucelabs.com/u/brycekahle}{\tt }

Sock\+JS is a browser Java\+Script library that provides a Web\+Socket-\/like object. Sock\+JS gives you a coherent, cross-\/browser, Javascript A\+PI which creates a low latency, full duplex, cross-\/domain communication channel between the browser and the web server.

Under the hood Sock\+JS tries to use native Web\+Sockets first. If that fails it can use a variety of browser-\/specific transport protocols and presents them through Web\+Socket-\/like abstractions.

Sock\+JS is intended to work for all modern browsers and in environments which don\textquotesingle{}t support the Web\+Socket protocol -- for example, behind restrictive corporate proxies.

Sock\+J\+S-\/client does require a server counterpart\+:


\begin{DoxyItemize}
\item \href{https://github.com/sockjs/sockjs-node}{\tt Sock\+J\+S-\/node} is a Sock\+JS server for Node.\+js.
\end{DoxyItemize}

Philosophy\+:


\begin{DoxyItemize}
\item The A\+PI should follow \href{https://www.w3.org/TR/websockets/}{\tt H\+T\+M\+L5 Websockets A\+PI} as closely as possible.
\item All the transports must support cross domain connections out of the box. It\textquotesingle{}s possible and recommended to host a Sock\+JS server on a different server than your main web site.
\item There is support for at least one streaming protocol for every major browser.
\item Streaming transports should work cross-\/domain and should support cookies (for cookie-\/based sticky sessions).
\item Polling transports are used as a fallback for old browsers and hosts behind restrictive proxies.
\item Connection establishment should be fast and lightweight.
\item No Flash inside (no need to open port 843 -\/ which doesn\textquotesingle{}t work through proxies, no need to host \textquotesingle{}crossdomain.\+xml\textquotesingle{}, no need \href{https://github.com/gimite/web-socket-js/issues/49}{\tt to wait for 3 seconds} in order to detect problems)
\end{DoxyItemize}

Subscribe to \href{https://groups.google.com/forum/#!forum/sockjs}{\tt Sock\+JS mailing list} for discussions and support.

Sock\+JS family\+:


\begin{DoxyItemize}
\item \href{https://github.com/sockjs/sockjs-client}{\tt Sock\+J\+S-\/client} Java\+Script client library
\item \href{https://github.com/sockjs/sockjs-node}{\tt Sock\+J\+S-\/node} Node.\+js server
\item \href{https://github.com/sockjs/sockjs-erlang}{\tt Sock\+J\+S-\/erlang} Erlang server
\item \href{https://github.com/flaviogrossi/sockjs-cyclone}{\tt Sock\+J\+S-\/cyclone} Python/\+Cyclone/\+Twisted server
\item \href{https://github.com/MrJoes/sockjs-tornado}{\tt Sock\+J\+S-\/tornado} Python/\+Tornado server
\item \href{https://github.com/DesertBus/sockjs-twisted/}{\tt Sock\+J\+S-\/twisted} Python/\+Twisted server
\item \href{https://github.com/aio-libs/sockjs/}{\tt Sock\+J\+S-\/aiohttp} Python/\+Aiohttp server
\item \href{https://projects.spring.io/spring-framework}{\tt Spring Framework} Java \href{https://docs.spring.io/spring-framework/docs/current/spring-framework-reference/html/websocket.html#websocket-fallback-sockjs-client}{\tt client} \& server
\item \href{https://github.com/vert-x/vert.x}{\tt vert.\+x} Java/vert.\+x server
\item \href{https://xitrum-framework.github.io/}{\tt Xitrum} Scala server
\item \href{https://github.com/Atmosphere/atmosphere}{\tt Atmosphere Framework} Java\+EE Server, Play Framework, Netty, Vert.\+x
\end{DoxyItemize}

Work in progress\+:


\begin{DoxyItemize}
\item \href{https://github.com/nyarly/sockjs-ruby}{\tt Sock\+J\+S-\/ruby}
\item \href{https://github.com/cgbystrom/sockjs-netty}{\tt Sock\+J\+S-\/netty}
\item \href{https://github.com/ksava/sockjs-gevent}{\tt Sock\+J\+S-\/gevent} (\href{https://github.com/njoyce/sockjs-gevent}{\tt Sock\+J\+S-\/gevent fork})
\item \href{https://github.com/fafhrd91/pyramid_sockjs}{\tt pyramid-\/\+Sock\+JS}
\item \href{https://github.com/wildcloud/wildcloud-websockets}{\tt wildcloud-\/websockets}
\item \href{https://github.com/Palmik/wai-sockjs}{\tt wai-\/\+Sock\+JS}
\item \href{https://github.com/vti/sockjs-perl}{\tt Sock\+J\+S-\/perl}
\item \href{https://github.com/igm/sockjs-go/}{\tt Sock\+J\+S-\/go}
\end{DoxyItemize}

\subsection*{Getting Started }

Sock\+JS mimics the \href{https://www.w3.org/TR/websockets/}{\tt Web\+Sockets A\+PI}, but instead of {\ttfamily Web\+Socket} there is a {\ttfamily Sock\+JS} Javascript object.

First, you need to load the Sock\+JS Java\+Script library. For example, you can put that in your H\+T\+ML head\+:


\begin{DoxyCode}
<script src="https://cdn.jsdelivr.net/sockjs/1/sockjs.min.js"></script>
\end{DoxyCode}


After the script is loaded you can establish a connection with the Sock\+JS server. Here\textquotesingle{}s a simple example\+:


\begin{DoxyCode}
var sock = new SockJS('https://mydomain.com/my\_prefix');
sock.onopen = function() \{
    console.log('open');
\};
sock.onmessage = function(e) \{
    console.log('message', e.data);
\};
sock.onclose = function() \{
    console.log('close');
\};

sock.send('test');
sock.close();
\end{DoxyCode}


\subsection*{Sock\+J\+S-\/client A\+PI }

\subsubsection*{Sock\+JS class}

Similar to the \textquotesingle{}Web\+Socket\textquotesingle{} A\+PI, the \textquotesingle{}Sock\+JS\textquotesingle{} constructor takes one, or more arguments\+:


\begin{DoxyCode}
var sockjs = new SockJS(url, \_reserved, options);
\end{DoxyCode}


{\ttfamily url} may contain a query string, if one is desired.

Where {\ttfamily options} is a hash which can contain\+:


\begin{DoxyItemize}
\item {\bfseries server (string)}

String to append to url for actual data connection. Defaults to a random 4 digit number.
\item {\bfseries transports (string OR array of strings)}

Sometimes it is useful to disable some fallback transports. This option allows you to supply a list transports that may be used by Sock\+JS. By default all available transports will be used.
\item {\bfseries session\+Id (number OR function)}

Both client and server use session identifiers to distinguish connections. If you specify this option as a number, Sock\+JS will use its random string generator function to generate session ids that are N-\/character long (where N corresponds to the number specified by {\bfseries session\+Id}). When you specify this option as a function, the function must return a randomly generated string. Every time Sock\+JS needs to generate a session id it will call this function and use the returned string directly. If you don\textquotesingle{}t specify this option, the default is to use the default random string generator to generate 8-\/character long session ids.
\end{DoxyItemize}

Although the \textquotesingle{}Sock\+JS\textquotesingle{} object tries to emulate the \textquotesingle{}Web\+Socket\textquotesingle{} behaviour, it\textquotesingle{}s impossible to support all of its features. An important Sock\+JS limitation is the fact that you\textquotesingle{}re not allowed to open more than one Sock\+JS connection to a single domain at a time. This limitation is caused by an in-\/browser limit of outgoing connections -\/ usually \href{https://stackoverflow.com/questions/985431/max-parallel-http-connections-in-a-browser}{\tt browsers don\textquotesingle{}t allow opening more than two outgoing connections to a single domain}. A single Sock\+JS session requires those two connections -\/ one for downloading data, the other for sending messages. Opening a second Sock\+JS session at the same time would most likely block, and can result in both sessions timing out.

Opening more than one Sock\+JS connection at a time is generally a bad practice. If you absolutely must do it, you can use multiple subdomains, using a different subdomain for every Sock\+JS connection.

\subsection*{Supported transports, by browser (html served from \href{http://}{\tt http\+://} or \href{https://}{\tt https\+://}) }

\tabulinesep=1mm
\begin{longtabu} spread 0pt [c]{*{4}{|X[-1]}|}
\hline
\rowcolor{\tableheadbgcolor}\textbf{ {\itshape Browser}  }&\textbf{ {\itshape Websockets}  }&\textbf{ {\itshape Streaming}  }&\textbf{ {\itshape Polling} -\/-\/-\/-\/-\/---   }\\\cline{1-4}
\endfirsthead
\hline
\endfoot
\hline
\rowcolor{\tableheadbgcolor}\textbf{ {\itshape Browser}  }&\textbf{ {\itshape Websockets}  }&\textbf{ {\itshape Streaming}  }&\textbf{ {\itshape Polling} -\/-\/-\/-\/-\/---   }\\\cline{1-4}
\endhead
IE 6, 7  &no  &no  &jsonp-\/polling   \\\cline{1-4}
IE 8, 9 (cookies=no)  &no  &xdr-\/streaming {$\dagger$}  &xdr-\/polling {$\dagger$}   \\\cline{1-4}
IE 8, 9 (cookies=yes)  &no  &iframe-\/htmlfile  &iframe-\/xhr-\/polling   \\\cline{1-4}
IE 10  &rfc6455  &xhr-\/streaming  &xhr-\/polling   \\\cline{1-4}
Chrome 6-\/13  &hixie-\/76  &xhr-\/streaming  &xhr-\/polling   \\\cline{1-4}
Chrome 14+  &hybi-\/10 / rfc6455  &xhr-\/streaming  &xhr-\/polling   \\\cline{1-4}
Firefox $<$10  &no {$\ddagger$}  &xhr-\/streaming  &xhr-\/polling   \\\cline{1-4}
Firefox 10+  &hybi-\/10 / rfc6455  &xhr-\/streaming  &xhr-\/polling   \\\cline{1-4}
Safari 5.\+x  &hixie-\/76  &xhr-\/streaming  &xhr-\/polling   \\\cline{1-4}
Safari 6+  &rfc6455  &xhr-\/streaming  &xhr-\/polling   \\\cline{1-4}
Opera 10.\+70+  &no {$\ddagger$}  &iframe-\/eventsource  &iframe-\/xhr-\/polling   \\\cline{1-4}
Opera 12.\+10+  &rfc6455  &xhr-\/streaming  &xhr-\/polling   \\\cline{1-4}
Konqueror  &no  &no  &jsonp-\/polling   \\\cline{1-4}
\end{longtabu}



\begin{DoxyItemize}
\item {\bfseries {$\dagger$}}\+: IE 8+ supports \href{https://blogs.msdn.microsoft.com/ieinternals/2010/05/13/xdomainrequest-restrictions-limitations-and-workarounds/}{\tt X\+Domain\+Request}, which is essentially a modified A\+J\+A\+X/\+X\+HR that can do requests across domains. But unfortunately it doesn\textquotesingle{}t send any cookies, which makes it inappropriate for deployments when the load balancer uses J\+S\+E\+S\+S\+I\+O\+N\+ID cookie to do sticky sessions.
\item {\bfseries {$\ddagger$}}\+: Firefox 4.\+0 and Opera 11.\+00 and shipped with disabled Websockets \char`\"{}hixie-\/76\char`\"{}. They can still be enabled by manually changing a browser setting.
\end{DoxyItemize}

\subsection*{Supported transports, by browser (html served from \href{file://}{\tt file\+://}) }

Sometimes you may want to serve your html from \char`\"{}file\+://\char`\"{} address -\/ for development or if you\textquotesingle{}re using Phone\+Gap or similar technologies. But due to the Cross Origin Policy files served from \char`\"{}file\+://\char`\"{} have no Origin, and that means some of Sock\+JS transports won\textquotesingle{}t work. For this reason the Sock\+JS transport table is different than usually, major differences are\+:

\tabulinesep=1mm
\begin{longtabu} spread 0pt [c]{*{4}{|X[-1]}|}
\hline
\rowcolor{\tableheadbgcolor}\textbf{ {\itshape Browser}  }&\textbf{ {\itshape Websockets}  }&\textbf{ {\itshape Streaming}  }&\textbf{ {\itshape Polling} -\/-\/-\/-\/-\/---   }\\\cline{1-4}
\endfirsthead
\hline
\endfoot
\hline
\rowcolor{\tableheadbgcolor}\textbf{ {\itshape Browser}  }&\textbf{ {\itshape Websockets}  }&\textbf{ {\itshape Streaming}  }&\textbf{ {\itshape Polling} -\/-\/-\/-\/-\/---   }\\\cline{1-4}
\endhead
IE 8, 9  &same as above  &iframe-\/htmlfile  &iframe-\/xhr-\/polling   \\\cline{1-4}
Other  &same as above  &iframe-\/eventsource  &iframe-\/xhr-\/polling   \\\cline{1-4}
\end{longtabu}


\subsection*{Supported transports, by name }

\tabulinesep=1mm
\begin{longtabu} spread 0pt [c]{*{2}{|X[-1]}|}
\hline
\rowcolor{\tableheadbgcolor}\textbf{ {\itshape Transport}  }&\textbf{ {\itshape References}
\begin{DoxyItemize}
\item 
\end{DoxyItemize}}\\\cline{1-2}
\endfirsthead
\hline
\endfoot
\hline
\rowcolor{\tableheadbgcolor}\textbf{ {\itshape Transport}  }&\textbf{ {\itshape References}
\begin{DoxyItemize}
\item 
\end{DoxyItemize}}\\\cline{1-2}
\endhead
websocket (rfc6455)  &\href{https://www.rfc-editor.org/rfc/rfc6455.txt}{\tt rfc 6455}   \\\cline{1-2}
websocket (hixie-\/76)  &\href{https://tools.ietf.org/html/draft-hixie-thewebsocketprotocol-76}{\tt draft-\/hixie-\/thewebsocketprotocol-\/76}   \\\cline{1-2}
websocket (hybi-\/10)  &\href{https://tools.ietf.org/html/draft-ietf-hybi-thewebsocketprotocol-10}{\tt draft-\/ietf-\/hybi-\/thewebsocketprotocol-\/10}   \\\cline{1-2}
xhr-\/streaming  &Transport using \href{https://secure.wikimedia.org/wikipedia/en/wiki/XMLHttpRequest#Cross-domain_requests}{\tt Cross domain X\+HR} \href{http://www.debugtheweb.com/test/teststreaming.aspx}{\tt streaming} capability (ready\+State=3).   \\\cline{1-2}
xdr-\/streaming  &Transport using \href{https://blogs.msdn.microsoft.com/ieinternals/2010/05/13/xdomainrequest-restrictions-limitations-and-workarounds/}{\tt X\+Domain\+Request} \href{http://www.debugtheweb.com/test/teststreaming.aspx}{\tt streaming} capability (ready\+State=3).   \\\cline{1-2}
eventsource  &\href{https://html.spec.whatwg.org/multipage/comms.html#server-sent-events}{\tt Event\+Source/\+Server-\/sent events}.   \\\cline{1-2}
iframe-\/eventsource  &\href{https://html.spec.whatwg.org/multipage/comms.html#server-sent-events}{\tt Event\+Source/\+Server-\/sent events} used from an \href{https://developer.mozilla.org/en/DOM/window.postMessage}{\tt iframe via post\+Message}.   \\\cline{1-2}
htmlfile  &\href{http://cometdaily.com/2007/11/18/ie-activexhtmlfile-transport-part-ii/}{\tt Html\+File}.   \\\cline{1-2}
iframe-\/htmlfile  &\href{http://cometdaily.com/2007/11/18/ie-activexhtmlfile-transport-part-ii/}{\tt Html\+File} used from an \href{https://developer.mozilla.org/en/DOM/window.postMessage}{\tt iframe via post\+Message}.   \\\cline{1-2}
xhr-\/polling  &Long-\/polling using \href{https://secure.wikimedia.org/wikipedia/en/wiki/XMLHttpRequest#Cross-domain_requests}{\tt cross domain X\+HR}.   \\\cline{1-2}
xdr-\/polling  &Long-\/polling using \href{https://blogs.msdn.microsoft.com/ieinternals/2010/05/13/xdomainrequest-restrictions-limitations-and-workarounds/}{\tt X\+Domain\+Request}.   \\\cline{1-2}
iframe-\/xhr-\/polling  &Long-\/polling using normal A\+J\+AX from an \href{https://developer.mozilla.org/en/DOM/window.postMessage}{\tt iframe via post\+Message}.   \\\cline{1-2}
jsonp-\/polling  &Slow and old fashioned \href{https://secure.wikimedia.org/wikipedia/en/wiki/JSONP}{\tt J\+S\+O\+NP polling}. This transport will show \char`\"{}busy indicator\char`\"{} (aka\+: \char`\"{}spinning wheel\char`\"{}) when sending data.   \\\cline{1-2}
\end{longtabu}


\subsection*{Connecting to Sock\+JS without the client }

Although the main point of Sock\+JS is to enable browser-\/to-\/server connectivity, it is possible to connect to Sock\+JS from an external application. Any Sock\+JS server complying with 0.\+3 protocol does support a raw Web\+Socket url. The raw Web\+Socket url for the test server looks like\+:


\begin{DoxyItemize}
\item ws\+://localhost\+:8081/echo/websocket
\end{DoxyItemize}

You can connect any Web\+Socket R\+FC 6455 compliant Web\+Socket client to this url. This can be a command line client, external application, third party code or even a browser (though I don\textquotesingle{}t know why you would want to do so).

\subsection*{Deployment }

You should use a version of sockjs-\/client that supports the protocol used by your server. For example\+:


\begin{DoxyCode}
<script src="https://cdn.jsdelivr.net/sockjs/1/sockjs.min.js"></script>
\end{DoxyCode}


For server-\/side deployment tricks, especially about load balancing and session stickiness, take a look at the \href{https://github.com/sockjs/sockjs-node#readme}{\tt Sock\+J\+S-\/node readme}.

\subsection*{Development and testing }

Sock\+J\+S-\/client needs \href{https://nodejs.org/}{\tt node.\+js} for running a test server and Java\+Script minification. If you want to work on Sock\+J\+S-\/client source code, checkout the git repo and follow these steps\+: \begin{DoxyVerb}cd sockjs-client
npm install
\end{DoxyVerb}


To generate Java\+Script, run\+: \begin{DoxyVerb}gulp browserify
\end{DoxyVerb}


To generate minified Java\+Script, run\+: \begin{DoxyVerb}gulp browserify:min
\end{DoxyVerb}


Both commands output into the {\ttfamily build} directory.

\subsubsection*{Testing}

Once you\textquotesingle{}ve compiled the Sock\+J\+S-\/client you may want to check if your changes pass all the tests. \begin{DoxyVerb}npm run test:browser_local
\end{DoxyVerb}


This will start \href{https://github.com/defunctzombie/zuul}{\tt zuul} and a test support server. Open the browser to \href{http://localhost:9090/_zuul}{\tt http\+://localhost\+:9090/\+\_\+zuul} and watch the tests run.

\subsection*{Browser Quirks }

There are various browser quirks which we don\textquotesingle{}t intend to address\+:


\begin{DoxyItemize}
\item Pressing E\+SC in Firefox, before Firefox 20, closes the Sock\+JS connection. For a workaround and discussion see \href{https://github.com/sockjs/sockjs-client/issues/18}{\tt \#18}.
\item {\ttfamily jsonp-\/polling} transport will show a \char`\"{}spinning wheel\char`\"{} (aka. \char`\"{}busy indicator\char`\"{}) when sending data.
\item You can\textquotesingle{}t open more than one Sock\+JS connection to one domain at the same time due to \href{https://stackoverflow.com/questions/985431/max-parallel-http-connections-in-a-browser}{\tt the browser\textquotesingle{}s limit of concurrent connections} (this limit is not counting native Web\+Socket connections).
\item Although Sock\+JS is trying to escape any strange Unicode characters (even invalid ones -\/ \href{https://en.wikipedia.org/wiki/Mapping_of_Unicode_characters#Surrogates}{\tt like surrogates -\/} or \href{https://en.wikipedia.org/wiki/Unicode#Character_General_Category}{\tt and }) it\textquotesingle{}s advisable to use only valid characters. Using invalid characters is a bit slower, and may not work with Sock\+JS servers that have proper Unicode support.
\item Having a global function called {\ttfamily onmessage} or such is probably a bad idea, as it could be called by the built-\/in {\ttfamily post\+Message} A\+PI.
\item From Sock\+JS\textquotesingle{} point of view there is nothing special about S\+S\+L/\+H\+T\+T\+PS. Connecting between unencrypted and encrypted sites should work just fine.
\item Although Sock\+JS does its best to support both prefix and cookie based sticky sessions, the latter may not work well cross-\/domain with browsers that don\textquotesingle{}t accept third-\/party cookies by default (Safari). In order to get around this make sure you\textquotesingle{}re connecting to Sock\+JS from the same parent domain as the main site. For example \textquotesingle{}sockjs.\+a.\+com\textquotesingle{} is able to set cookies if you\textquotesingle{}re connecting from \textquotesingle{}www.\+a.\+com\textquotesingle{} or \textquotesingle{}a.\+com\textquotesingle{}.
\item Trying to connect from secure \char`\"{}https\+://\char`\"{} to insecure \char`\"{}http\+://\char`\"{} is not a good idea. The other way around should be fine.
\item Long polling is known to cause problems on Heroku, but a \href{https://github.com/sockjs/sockjs-node/issues/57#issuecomment-5242187}{\tt workaround for Sock\+JS is available}.
\item Sock\+JS \href{https://github.com/sockjs/sockjs-client/issues/94}{\tt websocket transport is more stable over S\+SL}. If you\textquotesingle{}re a serious Sock\+JS user then consider using S\+SL (\href{https://www.ietf.org/mail-archive/web/hybi/current/msg01605.html}{\tt more info}). 
\end{DoxyItemize}