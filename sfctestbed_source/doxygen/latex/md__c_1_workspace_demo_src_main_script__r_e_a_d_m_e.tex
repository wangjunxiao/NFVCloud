This project was bootstrapped with \href{https://github.com/facebookincubator/create-react-app}{\tt Create React App}.

Below you will find some information on how to perform common tasks.~\newline
 You can find the most recent version of this guide https\+://github.com/facebookincubator/create-\/react-\/app/blob/master/packages/react-\/scripts/template/\+R\+E\+A\+D\+M\+E.\+md \char`\"{}here\char`\"{}.

\subsection*{Table of Contents}


\begin{DoxyItemize}
\item \href{#updating-to-new-releases}{\tt Updating to New Releases}
\item \href{#sending-feedback}{\tt Sending Feedback}
\item \href{#folder-structure}{\tt Folder Structure}
\item \href{#available-scripts}{\tt Available Scripts}
\begin{DoxyItemize}
\item \href{#npm-start}{\tt npm start}
\item \href{#npm-test}{\tt npm test}
\item \href{#npm-run-build}{\tt npm run build}
\item \href{#npm-run-eject}{\tt npm run eject}
\end{DoxyItemize}
\item \href{#supported-language-features-and-polyfills}{\tt Supported Language Features and Polyfills}
\item \href{#syntax-highlighting-in-the-editor}{\tt Syntax Highlighting in the Editor}
\item \href{#displaying-lint-output-in-the-editor}{\tt Displaying Lint Output in the Editor}
\item \href{#debugging-in-the-editor}{\tt Debugging in the Editor}
\item \href{#formatting-code-automatically}{\tt Formatting Code Automatically}
\item \href{#changing-the-page-title}{\tt Changing the Page {\ttfamily $<$title$>$}}
\item \href{#installing-a-dependency}{\tt Installing a Dependency}
\item \href{#importing-a-component}{\tt Importing a Component}
\item \href{#code-splitting}{\tt Code Splitting}
\item \href{#adding-a-stylesheet}{\tt Adding a Stylesheet}
\item \href{#post-processing-css}{\tt Post-\/\+Processing C\+SS}
\item \href{#adding-a-css-preprocessor-sass-less-etc}{\tt Adding a C\+SS Preprocessor (Sass, Less etc.)}
\item \href{#adding-images-fonts-and-files}{\tt Adding Images, Fonts, and Files}
\item \href{#using-the-public-folder}{\tt Using the {\ttfamily public} Folder}
\begin{DoxyItemize}
\item \href{#changing-the-html}{\tt Changing the H\+T\+ML}
\item \href{#adding-assets-outside-of-the-module-system}{\tt Adding Assets Outside of the Module System}
\item \href{#when-to-use-the-public-folder}{\tt When to Use the {\ttfamily public} Folder}
\end{DoxyItemize}
\item \href{#using-global-variables}{\tt Using Global Variables}
\item \href{#adding-bootstrap}{\tt Adding Bootstrap}
\begin{DoxyItemize}
\item \href{#using-a-custom-theme}{\tt Using a Custom Theme}
\end{DoxyItemize}
\item \href{#adding-flow}{\tt Adding Flow}
\item \href{#adding-custom-environment-variables}{\tt Adding Custom Environment Variables}
\begin{DoxyItemize}
\item \href{#referencing-environment-variables-in-the-html}{\tt Referencing Environment Variables in the H\+T\+ML}
\item \href{#adding-temporary-environment-variables-in-your-shell}{\tt Adding Temporary Environment Variables In Your Shell}
\item \href{#adding-development-environment-variables-in-env}{\tt Adding Development Environment Variables In {\ttfamily .env}}
\end{DoxyItemize}
\item \href{#can-i-use-decorators}{\tt Can I Use Decorators?}
\item \href{#integrating-with-an-api-backend}{\tt Integrating with an A\+PI Backend}
\begin{DoxyItemize}
\item \href{#node}{\tt Node}
\item \href{#ruby-on-rails}{\tt Ruby on Rails}
\end{DoxyItemize}
\item \href{#proxying-api-requests-in-development}{\tt Proxying A\+PI Requests in Development}
\begin{DoxyItemize}
\item \href{#invalid-host-header-errors-after-configuring-proxy}{\tt \char`\"{}\+Invalid Host Header\char`\"{} Errors After Configuring Proxy}
\item \href{#configuring-the-proxy-manually}{\tt Configuring the Proxy Manually}
\item \href{#configuring-a-websocket-proxy}{\tt Configuring a Web\+Socket Proxy}
\end{DoxyItemize}
\item \href{#using-https-in-development}{\tt Using H\+T\+T\+PS in Development}
\item \href{#generating-dynamic-meta-tags-on-the-server}{\tt Generating Dynamic {\ttfamily $<$meta$>$} Tags on the Server}
\item \href{#pre-rendering-into-static-html-files}{\tt Pre-\/\+Rendering into Static H\+T\+ML Files}
\item \href{#injecting-data-from-the-server-into-the-page}{\tt Injecting Data from the Server into the Page}
\item \href{#running-tests}{\tt Running Tests}
\begin{DoxyItemize}
\item \href{#filename-conventions}{\tt Filename Conventions}
\item \href{#command-line-interface}{\tt Command Line Interface}
\item \href{#version-control-integration}{\tt Version Control Integration}
\item \href{#writing-tests}{\tt Writing Tests}
\item \href{#testing-components}{\tt Testing Components}
\item \href{#using-third-party-assertion-libraries}{\tt Using Third Party Assertion Libraries}
\item \href{#initializing-test-environment}{\tt Initializing Test Environment}
\item \href{#focusing-and-excluding-tests}{\tt Focusing and Excluding Tests}
\item \href{#coverage-reporting}{\tt Coverage Reporting}
\item \href{#continuous-integration}{\tt Continuous Integration}
\item \href{#disabling-jsdom}{\tt Disabling jsdom}
\item \href{#snapshot-testing}{\tt Snapshot Testing}
\item \href{#editor-integration}{\tt Editor Integration}
\end{DoxyItemize}
\item \href{#developing-components-in-isolation}{\tt Developing Components in Isolation}
\begin{DoxyItemize}
\item \href{#getting-started-with-storybook}{\tt Getting Started with Storybook}
\item \href{#getting-started-with-styleguidist}{\tt Getting Started with Styleguidist}
\end{DoxyItemize}
\item \href{#making-a-progressive-web-app}{\tt Making a Progressive Web App}
\begin{DoxyItemize}
\item \href{#offline-first-considerations}{\tt Offline-\/\+First Considerations}
\item \href{#progressive-web-app-metadata}{\tt Progressive Web App Metadata}
\end{DoxyItemize}
\item \href{#analyzing-the-bundle-size}{\tt Analyzing the Bundle Size}
\item \href{#deployment}{\tt Deployment}
\begin{DoxyItemize}
\item \href{#static-server}{\tt Static Server}
\item \href{#other-solutions}{\tt Other Solutions}
\item \href{#serving-apps-with-client-side-routing}{\tt Serving Apps with Client-\/\+Side Routing}
\item \href{#building-for-relative-paths}{\tt Building for Relative Paths}
\item \href{#azure}{\tt Azure}
\item \href{#firebase}{\tt Firebase}
\item \href{#github-pages}{\tt Git\+Hub Pages}
\item \href{#heroku}{\tt Heroku}
\item \href{#modulus}{\tt Modulus}
\item \href{#netlify}{\tt Netlify}
\item \href{#now}{\tt Now}
\item \href{#s3-and-cloudfront}{\tt S3 and Cloud\+Front}
\item \href{#surge}{\tt Surge}
\end{DoxyItemize}
\item \href{#advanced-configuration}{\tt Advanced Configuration}
\item \href{#troubleshooting}{\tt Troubleshooting}
\begin{DoxyItemize}
\item \href{#npm-start-doesnt-detect-changes}{\tt {\ttfamily npm start} doesn’t detect changes}
\item \href{#npm-test-hangs-on-macos-sierra}{\tt {\ttfamily npm test} hangs on mac\+OS Sierra}
\item \href{#npm-run-build-exits-too-early}{\tt {\ttfamily npm run build} exits too early}
\item \href{#npm-run-build-fails-on-heroku}{\tt {\ttfamily npm run build} fails on Heroku}
\item \href{#momentjs-locales-are-missing}{\tt Moment.\+js locales are missing}
\end{DoxyItemize}
\item \href{#something-missing}{\tt Something Missing?}
\end{DoxyItemize}

\subsection*{Updating to New Releases}

Create React App is divided into two packages\+:


\begin{DoxyItemize}
\item {\ttfamily create-\/react-\/app} is a global command-\/line utility that you use to create new projects.
\item {\ttfamily react-\/scripts} is a development dependency in the generated projects (including this one).
\end{DoxyItemize}

You almost never need to update {\ttfamily create-\/react-\/app} itself\+: it delegates all the setup to {\ttfamily react-\/scripts}.

When you run {\ttfamily create-\/react-\/app}, it always creates the project with the latest version of {\ttfamily react-\/scripts} so you’ll get all the new features and improvements in newly created apps automatically.

To update an existing project to a new version of {\ttfamily react-\/scripts}, https\+://github.com/facebookincubator/create-\/react-\/app/blob/master/\+C\+H\+A\+N\+G\+E\+L\+O\+G.\+md \char`\"{}open the changelog\char`\"{}, find the version you’re currently on (check {\ttfamily package.\+json} in this folder if you’re not sure), and apply the migration instructions for the newer versions.

In most cases bumping the {\ttfamily react-\/scripts} version in {\ttfamily package.\+json} and running {\ttfamily npm install} in this folder should be enough, but it’s good to consult the https\+://github.com/facebookincubator/create-\/react-\/app/blob/master/\+C\+H\+A\+N\+G\+E\+L\+O\+G.\+md \char`\"{}changelog\char`\"{} for potential breaking changes.

We commit to keeping the breaking changes minimal so you can upgrade {\ttfamily react-\/scripts} painlessly.

\subsection*{Sending Feedback}

We are always open to \href{https://github.com/facebookincubator/create-react-app/issues}{\tt your feedback}.

\subsection*{Folder Structure}

After creation, your project should look like this\+:


\begin{DoxyCode}
my-app/
  README.md
  node\_modules/
  package.json
  public/
    index.html
    favicon.ico
  src/
    App.css
    App.js
    App.test.js
    index.css
    index.js
    logo.svg
\end{DoxyCode}


For the project to build, {\bfseries these files must exist with exact filenames}\+:


\begin{DoxyItemize}
\item {\ttfamily public/index.\+html} is the page template;
\item {\ttfamily src/index.\+js} is the Java\+Script entry point.
\end{DoxyItemize}

You can delete or rename the other files.

You may create subdirectories inside {\ttfamily src}. For faster rebuilds, only files inside {\ttfamily src} are processed by Webpack.~\newline
 You need to {\bfseries put any JS and C\+SS files inside {\ttfamily src}}, otherwise Webpack won’t see them.

Only files inside {\ttfamily public} can be used from {\ttfamily public/index.\+html}.~\newline
 Read instructions below for using assets from Java\+Script and H\+T\+ML.

You can, however, create more top-\/level directories.~\newline
 They will not be included in the production build so you can use them for things like documentation.

\subsection*{Available Scripts}

In the project directory, you can run\+:

\subsubsection*{{\ttfamily npm start}}

Runs the app in the development mode.~\newline
 Open \href{http://localhost:3000}{\tt http\+://localhost\+:3000} to view it in the browser.

The page will reload if you make edits.~\newline
 You will also see any lint errors in the console.

\subsubsection*{{\ttfamily npm test}}

Launches the test runner in the interactive watch mode.~\newline
 See the section about \href{#running-tests}{\tt running tests} for more information.

\subsubsection*{{\ttfamily npm run build}}

Builds the app for production to the {\ttfamily build} folder.~\newline
 It correctly bundles React in production mode and optimizes the build for the best performance.

The build is minified and the filenames include the hashes.~\newline
 Your app is ready to be deployed!

See the section about \href{#deployment}{\tt deployment} for more information.

\subsubsection*{{\ttfamily npm run eject}}

{\bfseries Note\+: this is a one-\/way operation. Once you {\ttfamily eject}, you can’t go back!}

If you aren’t satisfied with the build tool and configuration choices, you can {\ttfamily eject} at any time. This command will remove the single build dependency from your project.

Instead, it will copy all the configuration files and the transitive dependencies (Webpack, Babel, E\+S\+Lint, etc) right into your project so you have full control over them. All of the commands except {\ttfamily eject} will still work, but they will point to the copied scripts so you can tweak them. At this point you’re on your own.

You don’t have to ever use {\ttfamily eject}. The curated feature set is suitable for small and middle deployments, and you shouldn’t feel obligated to use this feature. However we understand that this tool wouldn’t be useful if you couldn’t customize it when you are ready for it.

\subsection*{Supported Language Features and Polyfills}

This project supports a superset of the latest Java\+Script standard.~\newline
 In addition to \href{https://github.com/lukehoban/es6features}{\tt E\+S6} syntax features, it also supports\+:


\begin{DoxyItemize}
\item \href{https://github.com/rwaldron/exponentiation-operator}{\tt Exponentiation Operator} (E\+S2016).
\item \href{https://github.com/tc39/ecmascript-asyncawait}{\tt Async/await} (E\+S2017).
\item \href{https://github.com/sebmarkbage/ecmascript-rest-spread}{\tt Object Rest/\+Spread Properties} (stage 3 proposal).
\item \href{https://github.com/tc39/proposal-dynamic-import}{\tt Dynamic import()} (stage 3 proposal)
\item \href{https://github.com/tc39/proposal-class-public-fields}{\tt Class Fields and Static Properties} (stage 2 proposal).
\item \href{https://facebook.github.io/react/docs/introducing-jsx.html}{\tt J\+SX} and \href{https://flowtype.org/}{\tt Flow} syntax.
\end{DoxyItemize}

Learn more about \href{https://babeljs.io/docs/plugins/#presets-stage-x-experimental-presets-}{\tt different proposal stages}.

While we recommend to use experimental proposals with some caution, Facebook heavily uses these features in the product code, so we intend to provide \href{https://medium.com/@cpojer/effective-javascript-codemods-5a6686bb46fb}{\tt codemods} if any of these proposals change in the future.

Note that {\bfseries the project only includes a few E\+S6 \href{https://en.wikipedia.org/wiki/Polyfill}{\tt polyfills}}\+:


\begin{DoxyItemize}
\item \href{https://developer.mozilla.org/en/docs/Web/JavaScript/Reference/Global_Objects/Object/assign}{\tt {\ttfamily Object.\+assign()}} via \href{https://github.com/sindresorhus/object-assign}{\tt {\ttfamily object-\/assign}}.
\item \href{https://developer.mozilla.org/en-US/docs/Web/JavaScript/Reference/Global_Objects/Promise}{\tt {\ttfamily Promise}} via \href{https://github.com/then/promise}{\tt {\ttfamily promise}}.
\item \href{https://developer.mozilla.org/en/docs/Web/API/Fetch_API}{\tt {\ttfamily fetch()}} via \href{https://github.com/github/fetch}{\tt {\ttfamily whatwg-\/fetch}}.
\end{DoxyItemize}

If you use any other E\+S6+ features that need {\bfseries runtime support} (such as {\ttfamily Array.\+from()} or {\ttfamily Symbol}), make sure you are including the appropriate polyfills manually, or that the browsers you are targeting already support them.

\subsection*{Syntax Highlighting in the Editor}

To configure the syntax highlighting in your favorite text editor, head to the \href{https://babeljs.io/docs/editors}{\tt relevant Babel documentation page} and follow the instructions. Some of the most popular editors are covered.

\subsection*{Displaying Lint Output in the Editor}

$>$Note\+: this feature is available with {\ttfamily react-\/scripts@0.\+2.\+0} and higher.~\newline
 $>$It also only works with npm 3 or higher.

Some editors, including Sublime Text, Atom, and Visual Studio Code, provide plugins for E\+S\+Lint.

They are not required for linting. You should see the linter output right in your terminal as well as the browser console. However, if you prefer the lint results to appear right in your editor, there are some extra steps you can do.

You would need to install an E\+S\+Lint plugin for your editor first. Then, add a file called {\ttfamily .eslintrc} to the project root\+:


\begin{DoxyCode}
\{
  "extends": "react-app"
\}
\end{DoxyCode}


Now your editor should report the linting warnings.

Note that even if you edit your {\ttfamily .eslintrc} file further, these changes will {\bfseries only affect the editor integration}. They won’t affect the terminal and in-\/browser lint output. This is because Create React App intentionally provides a minimal set of rules that find common mistakes.

If you want to enforce a coding style for your project, consider using \href{https://github.com/jlongster/prettier}{\tt Prettier} instead of E\+S\+Lint style rules.

\subsection*{Debugging in the Editor}

{\bfseries This feature is currently only supported by \href{https://code.visualstudio.com}{\tt Visual Studio Code} editor.}

Visual Studio Code supports debugging out of the box with Create React App. This enables you as a developer to write and debug your React code without leaving the editor, and most importantly it enables you to have a continuous development workflow, where context switching is minimal, as you don’t have to switch between tools.

You would need to have the latest version of \href{https://code.visualstudio.com}{\tt VS Code} and VS Code \href{https://marketplace.visualstudio.com/items?itemName=msjsdiag.debugger-for-chrome}{\tt Chrome Debugger Extension} installed.

Then add the block below to your {\ttfamily launch.\+json} file and put it inside the {\ttfamily .vscode} folder in your app’s root directory.


\begin{DoxyCode}
\{
  "version": "0.2.0",
  "configurations": [\{
    "name": "Chrome",
    "type": "chrome",
    "request": "launch",
    "url": "http://localhost:3000",
    "webRoot": "$\{workspaceRoot\}/src",
    "userDataDir": "$\{workspaceRoot\}/.vscode/chrome",
    "sourceMapPathOverrides": \{
      "webpack:///src/*": "$\{webRoot\}/*"
    \}
  \}]
\}
\end{DoxyCode}


Start your app by running {\ttfamily npm start}, and start debugging in VS Code by pressing {\ttfamily F5} or by clicking the green debug icon. You can now write code, set breakpoints, make changes to the code, and debug your newly modified code—all from your editor.

\subsection*{Formatting Code Automatically}

Prettier is an opinionated code formatter with support for Java\+Script, C\+SS and J\+S\+ON. With Prettier you can format the code you write automatically to ensure a code style within your project. See the \href{https://github.com/prettier/prettier}{\tt Prettier\textquotesingle{}s Git\+Hub page} for more information, and look at this \href{https://prettier.github.io/prettier/}{\tt page to see it in action}.

To format our code whenever we make a commit in git, we need to install the following dependencies\+:


\begin{DoxyCode}
npm install --save husky lint-staged prettier
\end{DoxyCode}


Alternatively you may use {\ttfamily yarn}\+:


\begin{DoxyCode}
yarn add husky lint-staged prettier
\end{DoxyCode}



\begin{DoxyItemize}
\item {\ttfamily husky} makes it easy to use githooks as if they are npm scripts.
\item {\ttfamily lint-\/staged} allows us to run scripts on staged files in git. See this \href{https://medium.com/@okonetchnikov/make-linting-great-again-f3890e1ad6b8}{\tt blog post about lint-\/staged to learn more about it}.
\item {\ttfamily prettier} is the Java\+Script formatter we will run before commits.
\end{DoxyItemize}

Now we can make sure every file is formatted correctly by adding a few lines to the {\ttfamily package.\+json} in the project root.

Add the following line to {\ttfamily scripts} section\+:


\begin{DoxyCode}
  "scripts": \{
+   "precommit": "lint-staged",
    "start": "react-scripts start",
    "build": "react-scripts build",
\end{DoxyCode}


Next we add a \textquotesingle{}lint-\/staged\textquotesingle{} field to the {\ttfamily package.\+json}, for example\+:


\begin{DoxyCode}
  "dependencies": \{
    // ...
  \},
+ "lint-staged": \{
+   "src/**/*.\{js,jsx,json,css\}": [
+     "prettier --single-quote --write",
+     "git add"
+   ]
+ \},
  "scripts": \{
\end{DoxyCode}


Now, whenever you make a commit, Prettier will format the changed files automatically. You can also run {\ttfamily ./node\+\_\+modules/.bin/prettier -\/-\/single-\/quote -\/-\/write \char`\"{}src/$\ast$$\ast$/$\ast$.\{js,jsx\}\char`\"{}} to format your entire project for the first time.

Next you might want to integrate Prettier in your favorite editor. Read the section on \href{https://github.com/prettier/prettier#editor-integration}{\tt Editor Integration} on the Prettier Git\+Hub page.

\subsection*{Changing the Page {\ttfamily $<$title$>$}}

You can find the source H\+T\+ML file in the {\ttfamily public} folder of the generated project. You may edit the {\ttfamily $<$title$>$} tag in it to change the title from “\+React App” to anything else.

Note that normally you wouldn’t edit files in the {\ttfamily public} folder very often. For example, \href{#adding-a-stylesheet}{\tt adding a stylesheet} is done without touching the H\+T\+ML.

If you need to dynamically update the page title based on the content, you can use the browser \href{https://developer.mozilla.org/en-US/docs/Web/API/Document/title}{\tt {\ttfamily document.\+title}} A\+PI. For more complex scenarios when you want to change the title from React components, you can use \href{https://github.com/nfl/react-helmet}{\tt React Helmet}, a third party library.

If you use a custom server for your app in production and want to modify the title before it gets sent to the browser, you can follow advice in \href{#generating-dynamic-meta-tags-on-the-server}{\tt this section}. Alternatively, you can pre-\/build each page as a static H\+T\+ML file which then loads the Java\+Script bundle, which is covered \href{#pre-rendering-into-static-html-files}{\tt here}.

\subsection*{Installing a Dependency}

The generated project includes React and React\+D\+OM as dependencies. It also includes a set of scripts used by Create React App as a development dependency. You may install other dependencies (for example, React Router) with {\ttfamily npm}\+:


\begin{DoxyCode}
npm install --save react-router
\end{DoxyCode}


Alternatively you may use {\ttfamily yarn}\+:


\begin{DoxyCode}
yarn add react-router
\end{DoxyCode}


This works for any library, not just {\ttfamily react-\/router}.

\subsection*{Importing a Component}

This project setup supports E\+S6 modules thanks to Babel.~\newline
 While you can still use {\ttfamily require()} and {\ttfamily module.\+exports}, we encourage you to use \href{http://exploringjs.com/es6/ch_modules.html}{\tt {\ttfamily import} and {\ttfamily export}} instead.

For example\+:

\subsubsection*{{\ttfamily Button.\+js}}


\begin{DoxyCode}
import React, \{ Component \} from 'react';

class Button extends Component \{
  render() \{
    // ...
  \}
\}

export default Button; // Don’t forget to use export default!
\end{DoxyCode}


\subsubsection*{{\ttfamily Danger\+Button.\+js}}


\begin{DoxyCode}
import React, \{ Component \} from 'react';
import Button from './Button'; // Import a component from another file

class DangerButton extends Component \{
  render() \{
    return <Button color="red" />;
  \}
\}

export default DangerButton;
\end{DoxyCode}


Be aware of the \href{http://stackoverflow.com/questions/36795819/react-native-es-6-when-should-i-use-curly-braces-for-import/36796281#36796281}{\tt difference between default and named exports}. It is a common source of mistakes.

We suggest that you stick to using default imports and exports when a module only exports a single thing (for example, a component). That’s what you get when you use {\ttfamily export default Button} and `import Button from './\+Button\textquotesingle{}\`{}.

Named exports are useful for utility modules that export several functions. A module may have at most one default export and as many named exports as you like.

Learn more about E\+S6 modules\+:


\begin{DoxyItemize}
\item \href{http://stackoverflow.com/questions/36795819/react-native-es-6-when-should-i-use-curly-braces-for-import/36796281#36796281}{\tt When to use the curly braces?}
\item \href{http://exploringjs.com/es6/ch_modules.html}{\tt Exploring E\+S6\+: Modules}
\item \href{https://leanpub.com/understandinges6/read#leanpub-auto-encapsulating-code-with-modules}{\tt Understanding E\+S6\+: Modules}
\end{DoxyItemize}

\subsection*{Code Splitting}

Instead of downloading the entire app before users can use it, code splitting allows you to split your code into small chunks which you can then load on demand.

This project setup supports code splitting via \href{http://2ality.com/2017/01/import-operator.html#loading-code-on-demand}{\tt dynamic {\ttfamily import()}}. Its \href{https://github.com/tc39/proposal-dynamic-import}{\tt proposal} is in stage 3. The {\ttfamily import()} function-\/like form takes the module name as an argument and returns a \href{https://developer.mozilla.org/en-US/docs/Web/JavaScript/Reference/Global_Objects/Promise}{\tt {\ttfamily Promise}} which always resolves to the namespace object of the module.

Here is an example\+:

\subsubsection*{{\ttfamily module\+A.\+js}}


\begin{DoxyCode}
const moduleA = 'Hello';

export \{ moduleA \};
\end{DoxyCode}
 \subsubsection*{{\ttfamily App.\+js}}


\begin{DoxyCode}
import React, \{ Component \} from 'react';

class App extends Component \{
  handleClick = () => \{
    import('./moduleA')
      .then((\{ moduleA \}) => \{
        // Use moduleA
      \})
      .catch(err => \{
        // Handle failure
      \});
  \};

  render() \{
    return (
      <div>
        <button onClick=\{this.handleClick\}>Load</button>
      </div>
    );
  \}
\}

export default App;
\end{DoxyCode}


This will make {\ttfamily module\+A.\+js} and all its unique dependencies as a separate chunk that only loads after the user clicks the \textquotesingle{}Load\textquotesingle{} button.

You can also use it with {\ttfamily async} / {\ttfamily await} syntax if you prefer it.

\subsubsection*{With React Router}

If you are using React Router check out \href{http://serverless-stack.com/chapters/code-splitting-in-create-react-app.html}{\tt this tutorial} on how to use code splitting with it. You can find the companion Git\+Hub repository \href{https://github.com/AnomalyInnovations/serverless-stack-demo-client/tree/code-splitting-in-create-react-app}{\tt here}.

\subsection*{Adding a Stylesheet}

This project setup uses \href{https://webpack.js.org/}{\tt Webpack} for handling all assets. Webpack offers a custom way of “extending” the concept of {\ttfamily import} beyond Java\+Script. To express that a Java\+Script file depends on a C\+SS file, you need to {\bfseries import the C\+SS from the Java\+Script file}\+:

\subsubsection*{{\ttfamily Button.\+css}}


\begin{DoxyCode}
.Button \{
  padding: 20px;
\}
\end{DoxyCode}


\subsubsection*{{\ttfamily Button.\+js}}


\begin{DoxyCode}
import React, \{ Component \} from 'react';
import './Button.css'; // Tell Webpack that Button.js uses these styles

class Button extends Component \{
  render() \{
    // You can use them as regular CSS styles
    return <div className="Button" />;
  \}
\}
\end{DoxyCode}


{\bfseries This is not required for React} but many people find this feature convenient. You can read about the benefits of this approach \href{https://medium.com/seek-ui-engineering/block-element-modifying-your-javascript-components-d7f99fcab52b}{\tt here}. However you should be aware that this makes your code less portable to other build tools and environments than Webpack.

In development, expressing dependencies this way allows your styles to be reloaded on the fly as you edit them. In production, all C\+SS files will be concatenated into a single minified {\ttfamily .css} file in the build output.

If you are concerned about using Webpack-\/specific semantics, you can put all your C\+SS right into {\ttfamily src/index.\+css}. It would still be imported from {\ttfamily src/index.\+js}, but you could always remove that import if you later migrate to a different build tool.

\subsection*{Post-\/\+Processing C\+SS}

This project setup minifies your C\+SS and adds vendor prefixes to it automatically through \href{https://github.com/postcss/autoprefixer}{\tt Autoprefixer} so you don’t need to worry about it.

For example, this\+:


\begin{DoxyCode}
.App \{
  display: flex;
  flex-direction: row;
  align-items: center;
\}
\end{DoxyCode}


becomes this\+:


\begin{DoxyCode}
.App \{
  display: -webkit-box;
  display: -ms-flexbox;
  display: flex;
  -webkit-box-orient: horizontal;
  -webkit-box-direction: normal;
      -ms-flex-direction: row;
          flex-direction: row;
  -webkit-box-align: center;
      -ms-flex-align: center;
          align-items: center;
\}
\end{DoxyCode}


If you need to disable autoprefixing for some reason, \href{https://github.com/postcss/autoprefixer#disabling}{\tt follow this section}.

\subsection*{Adding a C\+SS Preprocessor (Sass, Less etc.)}

Generally, we recommend that you don’t reuse the same C\+SS classes across different components. For example, instead of using a {\ttfamily .Button} C\+SS class in {\ttfamily $<$Accept\+Button$>$} and {\ttfamily $<$Reject\+Button$>$} components, we recommend creating a {\ttfamily $<$Button$>$} component with its own {\ttfamily .Button} styles, that both {\ttfamily $<$Accept\+Button$>$} and {\ttfamily $<$Reject\+Button$>$} can render (but \href{https://facebook.github.io/react/docs/composition-vs-inheritance.html}{\tt not inherit}).

Following this rule often makes C\+SS preprocessors less useful, as features like mixins and nesting are replaced by component composition. You can, however, integrate a C\+SS preprocessor if you find it valuable. In this walkthrough, we will be using Sass, but you can also use Less, or another alternative.

First, let’s install the command-\/line interface for Sass\+:


\begin{DoxyCode}
npm install --save node-sass-chokidar
\end{DoxyCode}


Alternatively you may use {\ttfamily yarn}\+:


\begin{DoxyCode}
yarn add node-sass-chokidar
\end{DoxyCode}


Then in {\ttfamily package.\+json}, add the following lines to {\ttfamily scripts}\+:


\begin{DoxyCode}
   "scripts": \{
+    "build-css": "node-sass-chokidar src/ -o src/",
+    "watch-css": "npm run build-css && node-sass-chokidar src/ -o src/ --watch --recursive",
     "start": "react-scripts start",
     "build": "react-scripts build",
     "test": "react-scripts test --env=jsdom",
\end{DoxyCode}


$>$Note\+: To use a different preprocessor, replace {\ttfamily build-\/css} and {\ttfamily watch-\/css} commands according to your preprocessor’s documentation.

Now you can rename {\ttfamily src/\+App.\+css} to {\ttfamily src/\+App.\+scss} and run {\ttfamily npm run watch-\/css}. The watcher will find every Sass file in {\ttfamily src} subdirectories, and create a corresponding C\+SS file next to it, in our case overwriting {\ttfamily src/\+App.\+css}. Since {\ttfamily src/\+App.\+js} still imports {\ttfamily src/\+App.\+css}, the styles become a part of your application. You can now edit {\ttfamily src/\+App.\+scss}, and {\ttfamily src/\+App.\+css} will be regenerated.

To share variables between Sass files, you can use Sass imports. For example, {\ttfamily src/\+App.\+scss} and other component style files could include {\ttfamily @import \char`\"{}./shared.\+scss\char`\"{};} with variable definitions.

To enable importing files without using relative paths, you can add the {\ttfamily -\/-\/include-\/path} option to the command in {\ttfamily package.\+json}.


\begin{DoxyCode}
"build-css": "node-sass-chokidar --include-path ./src --include-path ./node\_modules src/ -o src/",
"watch-css": "npm run build-css && node-sass-chokidar --include-path ./src --include-path ./node\_modules
       src/ -o src/ --watch --recursive",
\end{DoxyCode}


This will allow you to do imports like


\begin{DoxyCode}
@import 'styles/\_colors.scss'; // assuming a styles directory under src/
@import 'nprogress/nprogress'; // importing a css file from the nprogress node module
\end{DoxyCode}


At this point you might want to remove all C\+SS files from the source control, and add {\ttfamily src/$\ast$$\ast$/$\ast$.css} to your {\ttfamily .gitignore} file. It is generally a good practice to keep the build products outside of the source control.

As a final step, you may find it convenient to run {\ttfamily watch-\/css} automatically with {\ttfamily npm start}, and run {\ttfamily build-\/css} as a part of {\ttfamily npm run build}. You can use the {\ttfamily \&\&} operator to execute two scripts sequentially. However, there is no cross-\/platform way to run two scripts in parallel, so we will install a package for this\+:


\begin{DoxyCode}
npm install --save npm-run-all
\end{DoxyCode}


Alternatively you may use {\ttfamily yarn}\+:


\begin{DoxyCode}
yarn add npm-run-all
\end{DoxyCode}


Then we can change {\ttfamily start} and {\ttfamily build} scripts to include the C\+SS preprocessor commands\+:


\begin{DoxyCode}
   "scripts": \{
     "build-css": "node-sass-chokidar src/ -o src/",
     "watch-css": "npm run build-css && node-sass-chokidar src/ -o src/ --watch --recursive",
-    "start": "react-scripts start",
-    "build": "react-scripts build",
+    "start-js": "react-scripts start",
+    "start": "npm-run-all -p watch-css start-js",
+    "build": "npm run build-css && react-scripts build",
     "test": "react-scripts test --env=jsdom",
     "eject": "react-scripts eject"
   \}
\end{DoxyCode}


Now running {\ttfamily npm start} and {\ttfamily npm run build} also builds Sass files.

{\bfseries Why {\ttfamily node-\/sass-\/chokidar}?}

{\ttfamily node-\/sass} has been reported as having the following issues\+:


\begin{DoxyItemize}
\item {\ttfamily node-\/sass -\/-\/watch} has been reported to have {\itshape performance issues} in certain conditions when used in a virtual machine or with docker.
\item Infinite styles compiling \href{https://github.com/facebookincubator/create-react-app/issues/1939}{\tt \#1939}
\item {\ttfamily node-\/sass} has been reported as having issues with detecting new files in a directory \href{https://github.com/sass/node-sass/issues/1891}{\tt \#1891}

{\ttfamily node-\/sass-\/chokidar} is used here as it addresses these issues.
\end{DoxyItemize}

\subsection*{Adding Images, Fonts, and Files}

With Webpack, using static assets like images and fonts works similarly to C\+SS.

You can $\ast$$\ast${\ttfamily import} a file right in a Java\+Script module$\ast$$\ast$. This tells Webpack to include that file in the bundle. Unlike C\+SS imports, importing a file gives you a string value. This value is the final path you can reference in your code, e.\+g. as the {\ttfamily src} attribute of an image or the {\ttfamily href} of a link to a P\+DF.

To reduce the number of requests to the server, importing images that are less than 10,000 bytes returns a \href{https://developer.mozilla.org/en-US/docs/Web/HTTP/Basics_of_HTTP/Data_URIs}{\tt data U\+RI} instead of a path. This applies to the following file extensions\+: bmp, gif, jpg, jpeg, and png. S\+VG files are excluded due to \href{https://github.com/facebookincubator/create-react-app/issues/1153}{\tt \#1153}.

Here is an example\+:


\begin{DoxyCode}
import React from 'react';
import logo from './logo.png'; // Tell Webpack this JS file uses this image

console.log(logo); // /logo.84287d09.png

function Header() \{
  // Import result is the URL of your image
  return <img src=\{logo\} alt="Logo" />;
\}

export default Header;
\end{DoxyCode}


This ensures that when the project is built, Webpack will correctly move the images into the build folder, and provide us with correct paths.

This works in C\+SS too\+:


\begin{DoxyCode}
.Logo \{
  background-image: url(./logo.png);
\}
\end{DoxyCode}


Webpack finds all relative module references in C\+SS (they start with {\ttfamily ./}) and replaces them with the final paths from the compiled bundle. If you make a typo or accidentally delete an important file, you will see a compilation error, just like when you import a non-\/existent Java\+Script module. The final filenames in the compiled bundle are generated by Webpack from content hashes. If the file content changes in the future, Webpack will give it a different name in production so you don’t need to worry about long-\/term caching of assets.

Please be advised that this is also a custom feature of Webpack.

{\bfseries It is not required for React} but many people enjoy it (and React Native uses a similar mechanism for images).~\newline
 An alternative way of handling static assets is described in the next section.

\subsection*{Using the {\ttfamily public} Folder}

$>$Note\+: this feature is available with {\ttfamily react-\/scripts@0.\+5.\+0} and higher.

\subsubsection*{Changing the H\+T\+ML}

The {\ttfamily public} folder contains the H\+T\+ML file so you can tweak it, for example, to \href{#changing-the-page-title}{\tt set the page title}. The {\ttfamily $<$script$>$} tag with the compiled code will be added to it automatically during the build process.

\subsubsection*{Adding Assets Outside of the Module System}

You can also add other assets to the {\ttfamily public} folder.

Note that we normally encourage you to {\ttfamily import} assets in Java\+Script files instead. For example, see the sections on \href{#adding-a-stylesheet}{\tt adding a stylesheet} and \href{#adding-images-fonts-and-files}{\tt adding images and fonts}. This mechanism provides a number of benefits\+:


\begin{DoxyItemize}
\item Scripts and stylesheets get minified and bundled together to avoid extra network requests.
\item Missing files cause compilation errors instead of 404 errors for your users.
\item Result filenames include content hashes so you don’t need to worry about browsers caching their old versions.
\end{DoxyItemize}

However there is an {\bfseries escape hatch} that you can use to add an asset outside of the module system.

If you put a file into the {\ttfamily public} folder, it will {\bfseries not} be processed by Webpack. Instead it will be copied into the build folder untouched. To reference assets in the {\ttfamily public} folder, you need to use a special variable called {\ttfamily P\+U\+B\+L\+I\+C\+\_\+\+U\+RL}.

Inside {\ttfamily index.\+html}, you can use it like this\+:


\begin{DoxyCode}
<link rel="shortcut icon" href="%PUBLIC\_URL%/favicon.ico">
\end{DoxyCode}


Only files inside the {\ttfamily public} folder will be accessible by {\ttfamily P\+U\+B\+L\+I\+C\+\_\+\+U\+RL\%} prefix. If you need to use a file from {\ttfamily src} or {\ttfamily node\+\_\+modules}, you’ll have to copy it there to explicitly specify your intention to make this file a part of the build.

When you run {\ttfamily npm run build}, Create React App will substitute {\ttfamily P\+U\+B\+L\+I\+C\+\_\+\+U\+RL\%} with a correct absolute path so your project works even if you use client-\/side routing or host it at a non-\/root \mbox{\hyperlink{namespace_u_r_l}{U\+RL}}.

In Java\+Script code, you can use {\ttfamily process.\+env.\+P\+U\+B\+L\+I\+C\+\_\+\+U\+RL} for similar purposes\+:


\begin{DoxyCode}
render() \{
  // Note: this is an escape hatch and should be used sparingly!
  // Normally we recommend using `import` for getting asset URLs
  // as described in “Adding Images and Fonts” above this section.
  return <img src=\{process.env.PUBLIC\_URL + '/img/logo.png'\} />;
\}
\end{DoxyCode}


Keep in mind the downsides of this approach\+:


\begin{DoxyItemize}
\item None of the files in {\ttfamily public} folder get post-\/processed or minified.
\item Missing files will not be called at compilation time, and will cause 404 errors for your users.
\item Result filenames won’t include content hashes so you’ll need to add query arguments or rename them every time they change.
\end{DoxyItemize}

\subsubsection*{When to Use the {\ttfamily public} Folder}

Normally we recommend importing \href{#adding-a-stylesheet}{\tt stylesheets}, \href{#adding-images-fonts-and-files}{\tt images, and fonts} from Java\+Script. The {\ttfamily public} folder is useful as a workaround for a number of less common cases\+:


\begin{DoxyItemize}
\item You need a file with a specific name in the build output, such as \href{https://developer.mozilla.org/en-US/docs/Web/Manifest}{\tt {\ttfamily manifest.\+webmanifest}}.
\item You have thousands of images and need to dynamically reference their paths.
\item You want to include a small script like \href{http://github.hubspot.com/pace/docs/welcome/}{\tt {\ttfamily pace.\+js}} outside of the bundled code.
\item Some library may be incompatible with Webpack and you have no other option but to include it as a {\ttfamily $<$script$>$} tag.
\end{DoxyItemize}

Note that if you add a {\ttfamily $<$script$>$} that declares global variables, you also need to read the next section on using them.

\subsection*{Using Global Variables}

When you include a script in the H\+T\+ML file that defines global variables and try to use one of these variables in the code, the linter will complain because it cannot see the definition of the variable.

You can avoid this by reading the global variable explicitly from the {\ttfamily window} object, for example\+:


\begin{DoxyCode}
const $ = window.$;
\end{DoxyCode}


This makes it obvious you are using a global variable intentionally rather than because of a typo.

Alternatively, you can force the linter to ignore any line by adding {\ttfamily // eslint-\/disable-\/line} after it.

\subsection*{Adding Bootstrap}

You don’t have to use \href{https://react-bootstrap.github.io}{\tt React Bootstrap} together with React but it is a popular library for integrating Bootstrap with React apps. If you need it, you can integrate it with Create React App by following these steps\+:

Install React Bootstrap and Bootstrap from npm. React Bootstrap does not include Bootstrap C\+SS so this needs to be installed as well\+:


\begin{DoxyCode}
npm install --save react-bootstrap bootstrap@3
\end{DoxyCode}


Alternatively you may use {\ttfamily yarn}\+:


\begin{DoxyCode}
yarn add react-bootstrap bootstrap@3
\end{DoxyCode}


Import Bootstrap C\+SS and optionally Bootstrap theme C\+SS in the beginning of your {\ttfamily src/index.\+js} file\+:


\begin{DoxyCode}
import 'bootstrap/dist/css/bootstrap.css';
import 'bootstrap/dist/css/bootstrap-theme.css';
// Put any other imports below so that CSS from your
// components takes precedence over default styles.
\end{DoxyCode}


Import required React Bootstrap components within {\ttfamily src/\+App.\+js} file or your custom component files\+:


\begin{DoxyCode}
import \{ Navbar, Jumbotron, Button \} from 'react-bootstrap';
\end{DoxyCode}


Now you are ready to use the imported React Bootstrap components within your component hierarchy defined in the render method. Here is an example \href{https://gist.githubusercontent.com/gaearon/85d8c067f6af1e56277c82d19fd4da7b/raw/6158dd991b67284e9fc8d70b9d973efe87659d72/App.js}{\tt {\ttfamily App.\+js}} redone using React Bootstrap.

\subsubsection*{Using a Custom Theme}

Sometimes you might need to tweak the visual styles of Bootstrap (or equivalent package).~\newline
 We suggest the following approach\+:


\begin{DoxyItemize}
\item Create a new package that depends on the package you wish to customize, e.\+g. Bootstrap.
\item Add the necessary build steps to tweak the theme, and publish your package on npm.
\item Install your own theme npm package as a dependency of your app.
\end{DoxyItemize}

Here is an example of adding a \href{https://medium.com/@tacomanator/customizing-create-react-app-aa9ffb88165}{\tt customized Bootstrap} that follows these steps.

\subsection*{Adding Flow}

Flow is a static type checker that helps you write code with fewer bugs. Check out this \href{https://medium.com/@preethikasireddy/why-use-static-types-in-javascript-part-1-8382da1e0adb}{\tt introduction to using static types in Java\+Script} if you are new to this concept.

Recent versions of \href{http://flowtype.org/}{\tt Flow} work with Create React App projects out of the box.

To add Flow to a Create React App project, follow these steps\+:


\begin{DoxyEnumerate}
\item Run {\ttfamily npm install -\/-\/save flow-\/bin} (or {\ttfamily yarn add flow-\/bin}).
\item Add {\ttfamily \char`\"{}flow\char`\"{}\+: \char`\"{}flow\char`\"{}} to the {\ttfamily scripts} section of your {\ttfamily package.\+json}.
\item Run {\ttfamily npm run flow init} (or {\ttfamily yarn flow init}) to create a \href{https://flowtype.org/docs/advanced-configuration.html}{\tt {\ttfamily .flowconfig} file} in the root directory.
\item Add {\ttfamily // @flow} to any files you want to type check (for example, to {\ttfamily src/\+App.\+js}).
\end{DoxyEnumerate}

Now you can run {\ttfamily npm run flow} (or {\ttfamily yarn flow}) to check the files for type errors. You can optionally use an I\+DE like \href{https://nuclide.io/docs/languages/flow/}{\tt Nuclide} for a better integrated experience. In the future we plan to integrate it into Create React App even more closely.

To learn more about Flow, check out \href{https://flowtype.org/}{\tt its documentation}.

\subsection*{Adding Custom Environment Variables}

$>$Note\+: this feature is available with {\ttfamily react-\/scripts@0.\+2.\+3} and higher.

Your project can consume variables declared in your environment as if they were declared locally in your JS files. By default you will have {\ttfamily N\+O\+D\+E\+\_\+\+E\+NV} defined for you, and any other environment variables starting with {\ttfamily R\+E\+A\+C\+T\+\_\+\+A\+P\+P\+\_\+}.

{\bfseries The environment variables are embedded during the build time}. Since Create React App produces a static H\+T\+M\+L/\+C\+S\+S/\+JS bundle, it can’t possibly read them at runtime. To read them at runtime, you would need to load H\+T\+ML into memory on the server and replace placeholders in runtime, just like \href{#injecting-data-from-the-server-into-the-page}{\tt described here}. Alternatively you can rebuild the app on the server anytime you change them.

$>$Note\+: You must create custom environment variables beginning with {\ttfamily R\+E\+A\+C\+T\+\_\+\+A\+P\+P\+\_\+}. Any other variables except {\ttfamily N\+O\+D\+E\+\_\+\+E\+NV} will be ignored to avoid accidentally \href{https://github.com/facebookincubator/create-react-app/issues/865#issuecomment-252199527}{\tt exposing a private key on the machine that could have the same name}. Changing any environment variables will require you to restart the development server if it is running.

These environment variables will be defined for you on {\ttfamily process.\+env}. For example, having an environment variable named {\ttfamily R\+E\+A\+C\+T\+\_\+\+A\+P\+P\+\_\+\+S\+E\+C\+R\+E\+T\+\_\+\+C\+O\+DE} will be exposed in your JS as {\ttfamily process.\+env.\+R\+E\+A\+C\+T\+\_\+\+A\+P\+P\+\_\+\+S\+E\+C\+R\+E\+T\+\_\+\+C\+O\+DE}.

There is also a special built-\/in environment variable called {\ttfamily N\+O\+D\+E\+\_\+\+E\+NV}. You can read it from {\ttfamily process.\+env.\+N\+O\+D\+E\+\_\+\+E\+NV}. When you run {\ttfamily npm start}, it is always equal to `\textquotesingle{}development'{\ttfamily , when you run}npm test{\ttfamily it is always equal to}\textquotesingle{}test\textquotesingle{}{\ttfamily , and when you run}npm run build{\ttfamily to make a production bundle, it is always equal to}\textquotesingle{}production\textquotesingle{}{\ttfamily . $\ast$$\ast$\+You cannot override}N\+O\+D\+E\+\_\+\+E\+N\+V\`{} manually.$\ast$$\ast$ This prevents developers from accidentally deploying a slow development build to production.

These environment variables can be useful for displaying information conditionally based on where the project is deployed or consuming sensitive data that lives outside of version control.

First, you need to have environment variables defined. For example, let’s say you wanted to consume a secret defined in the environment inside a {\ttfamily $<$form$>$}\+:


\begin{DoxyCode}
render() \{
  return (
    <div>
      <small>You are running this application in <b>\{process.env.NODE\_ENV\}</b> mode.</small>
      <form>
        <input type="hidden" defaultValue=\{process.env.REACT\_APP\_SECRET\_CODE\} />
      </form>
    </div>
  );
\}
\end{DoxyCode}


During the build, {\ttfamily process.\+env.\+R\+E\+A\+C\+T\+\_\+\+A\+P\+P\+\_\+\+S\+E\+C\+R\+E\+T\+\_\+\+C\+O\+DE} will be replaced with the current value of the {\ttfamily R\+E\+A\+C\+T\+\_\+\+A\+P\+P\+\_\+\+S\+E\+C\+R\+E\+T\+\_\+\+C\+O\+DE} environment variable. Remember that the {\ttfamily N\+O\+D\+E\+\_\+\+E\+NV} variable will be set for you automatically.

When you load the app in the browser and inspect the {\ttfamily $<$input$>$}, you will see its value set to {\ttfamily abcdef}, and the bold text will show the environment provided when using {\ttfamily npm start}\+:


\begin{DoxyCode}
<div>
  <small>You are running this application in <b>development</b> mode.</small>
  <form>
    <input type="hidden" value="abcdef" />
  </form>
</div>
\end{DoxyCode}


The above form is looking for a variable called {\ttfamily R\+E\+A\+C\+T\+\_\+\+A\+P\+P\+\_\+\+S\+E\+C\+R\+E\+T\+\_\+\+C\+O\+DE} from the environment. In order to consume this value, we need to have it defined in the environment. This can be done using two ways\+: either in your shell or in a {\ttfamily .env} file. Both of these ways are described in the next few sections.

Having access to the {\ttfamily N\+O\+D\+E\+\_\+\+E\+NV} is also useful for performing actions conditionally\+:


\begin{DoxyCode}
if (process.env.NODE\_ENV !== 'production') \{
  analytics.disable();
\}
\end{DoxyCode}


When you compile the app with {\ttfamily npm run build}, the minification step will strip out this condition, and the resulting bundle will be smaller.

\subsubsection*{Referencing Environment Variables in the H\+T\+ML}

$>$Note\+: this feature is available with {\ttfamily react-\/scripts@0.\+9.\+0} and higher.

You can also access the environment variables starting with {\ttfamily R\+E\+A\+C\+T\+\_\+\+A\+P\+P\+\_\+} in the {\ttfamily public/index.\+html}. For example\+:


\begin{DoxyCode}
<title>%REACT\_APP\_WEBSITE\_NAME%</title>
\end{DoxyCode}


Note that the caveats from the above section apply\+:


\begin{DoxyItemize}
\item Apart from a few built-\/in variables ({\ttfamily N\+O\+D\+E\+\_\+\+E\+NV} and {\ttfamily P\+U\+B\+L\+I\+C\+\_\+\+U\+RL}), variable names must start with {\ttfamily R\+E\+A\+C\+T\+\_\+\+A\+P\+P\+\_\+} to work.
\item The environment variables are injected at build time. If you need to inject them at runtime, \href{#generating-dynamic-meta-tags-on-the-server}{\tt follow this approach instead}.
\end{DoxyItemize}

\subsubsection*{Adding Temporary Environment Variables In Your Shell}

Defining environment variables can vary between O\+Ses. It’s also important to know that this manner is temporary for the life of the shell session.

\paragraph*{Windows (cmd.\+exe)}


\begin{DoxyCode}
set REACT\_APP\_SECRET\_CODE=abcdef&&npm start
\end{DoxyCode}


(Note\+: the lack of whitespace is intentional.)

\paragraph*{Linux, mac\+OS (Bash)}


\begin{DoxyCode}
REACT\_APP\_SECRET\_CODE=abcdef npm start
\end{DoxyCode}


\subsubsection*{Adding Development Environment Variables In {\ttfamily .env}}

$>$Note\+: this feature is available with {\ttfamily react-\/scripts@0.\+5.\+0} and higher.

To define permanent environment variables, create a file called {\ttfamily .env} in the root of your project\+:


\begin{DoxyCode}
REACT\_APP\_SECRET\_CODE=abcdef
\end{DoxyCode}


{\ttfamily .env} files {\bfseries should be} checked into source control (with the exclusion of {\ttfamily .env$\ast$.local}).

\paragraph*{What other {\ttfamily .env} files are can be used?}

$>$Note\+: this feature is {\bfseries available with {\ttfamily react-\/scripts@1.\+0.\+0} and higher}.


\begin{DoxyItemize}
\item {\ttfamily .env}\+: Default.
\item {\ttfamily .env.\+local}\+: Local overrides. {\bfseries This file is loaded for all environments except test.}
\item {\ttfamily .env.\+development}, {\ttfamily .env.\+test}, {\ttfamily .env.\+production}\+: Environment-\/specific settings.
\item {\ttfamily .env.\+development.\+local}, {\ttfamily .env.\+test.\+local}, {\ttfamily .env.\+production.\+local}\+: Local overrides of environment-\/specific settings.
\end{DoxyItemize}

Files on the left have more priority than files on the right\+:


\begin{DoxyItemize}
\item {\ttfamily npm start}\+: {\ttfamily .env.\+development.\+local}, {\ttfamily .env.\+development}, {\ttfamily .env.\+local}, {\ttfamily .env}
\item {\ttfamily npm run build}\+: {\ttfamily .env.\+production.\+local}, {\ttfamily .env.\+production}, {\ttfamily .env.\+local}, {\ttfamily .env}
\item {\ttfamily npm test}\+: {\ttfamily .env.\+test.\+local}, {\ttfamily .env.\+test}, {\ttfamily .env} (note {\ttfamily .env.\+local} is missing)
\end{DoxyItemize}

These variables will act as the defaults if the machine does not explicitly set them.~\newline
 Please refer to the \href{https://github.com/motdotla/dotenv}{\tt dotenv documentation} for more details.

$>$Note\+: If you are defining environment variables for development, your CI and/or hosting platform will most likely need these defined as well. Consult their documentation how to do this. For example, see the documentation for \href{https://docs.travis-ci.com/user/environment-variables/}{\tt Travis CI} or \href{https://devcenter.heroku.com/articles/config-vars}{\tt Heroku}.

\subsection*{Can I Use Decorators?}

Many popular libraries use \href{https://medium.com/google-developers/exploring-es7-decorators-76ecb65fb841}{\tt decorators} in their documentation.~\newline
 Create React App doesn’t support decorator syntax at the moment because\+:


\begin{DoxyItemize}
\item It is an experimental proposal and is subject to change.
\item The current specification version is not officially supported by Babel.
\item If the specification changes, we won’t be able to write a codemod because we don’t use them internally at Facebook.
\end{DoxyItemize}

However in many cases you can rewrite decorator-\/based code without decorators just as fine.~\newline
 Please refer to these two threads for reference\+:


\begin{DoxyItemize}
\item \href{https://github.com/facebookincubator/create-react-app/issues/214}{\tt \#214}
\item \href{https://github.com/facebookincubator/create-react-app/issues/411}{\tt \#411}
\end{DoxyItemize}

Create React App will add decorator support when the specification advances to a stable stage.

\subsection*{Integrating with an A\+PI Backend}

These tutorials will help you to integrate your app with an A\+PI backend running on another port, using {\ttfamily fetch()} to access it.

\subsubsection*{Node}

Check out \href{https://www.fullstackreact.com/articles/using-create-react-app-with-a-server/}{\tt this tutorial}. You can find the companion Git\+Hub repository \href{https://github.com/fullstackreact/food-lookup-demo}{\tt here}.

\subsubsection*{Ruby on Rails}

Check out \href{https://www.fullstackreact.com/articles/how-to-get-create-react-app-to-work-with-your-rails-api/}{\tt this tutorial}. You can find the companion Git\+Hub repository \href{https://github.com/fullstackreact/food-lookup-demo-rails}{\tt here}.

\subsection*{Proxying A\+PI Requests in Development}

$>$Note\+: this feature is available with {\ttfamily react-\/scripts@0.\+2.\+3} and higher.

People often serve the front-\/end React app from the same host and port as their backend implementation.~\newline
 For example, a production setup might look like this after the app is deployed\+:


\begin{DoxyCode}
/             - static server returns index.html with React app
/todos        - static server returns index.html with React app
/api/todos    - server handles any /api/* requests using the backend implementation
\end{DoxyCode}


Such setup is {\bfseries not} required. However, if you {\bfseries do} have a setup like this, it is convenient to write requests like `fetch('/api/todos\textquotesingle{})\`{} without worrying about redirecting them to another host or port during development.

To tell the development server to proxy any unknown requests to your A\+PI server in development, add a {\ttfamily proxy} field to your {\ttfamily package.\+json}, for example\+:


\begin{DoxyCode}
"proxy": "http://localhost:4000",
\end{DoxyCode}


This way, when you `fetch('/api/todos\textquotesingle{}){\ttfamily in development, the development server will recognize that it’s not a static asset, and will proxy your request to}\href{http://localhost:4000/api/todos}{\tt http\+://localhost\+:4000/api/todos}{\ttfamily as a fallback. The development server will only attempt to send requests without a}text/html\`{} accept header to the proxy.

Conveniently, this avoids \href{http://stackoverflow.com/questions/21854516/understanding-ajax-cors-and-security-considerations}{\tt C\+O\+RS issues} and error messages like this in development\+:


\begin{DoxyCode}
Fetch API cannot load http://localhost:4000/api/todos. No 'Access-Control-Allow-Origin' header is present
       on the requested resource. Origin 'http://localhost:3000' is therefore not allowed access. If an opaque
       response serves your needs, set the request's mode to 'no-cors' to fetch the resource with CORS disabled.
\end{DoxyCode}


Keep in mind that {\ttfamily proxy} only has effect in development (with {\ttfamily npm start}), and it is up to you to ensure that U\+R\+Ls like {\ttfamily /api/todos} point to the right thing in production. You don’t have to use the {\ttfamily /api} prefix. Any unrecognized request without a {\ttfamily text/html} accept header will be redirected to the specified {\ttfamily proxy}.

The {\ttfamily proxy} option supports H\+T\+TP, H\+T\+T\+PS and Web\+Socket connections.~\newline
 If the {\ttfamily proxy} option is {\bfseries not} flexible enough for you, alternatively you can\+:


\begin{DoxyItemize}
\item \href{#configuring-the-proxy-manually}{\tt Configure the proxy yourself}
\item Enable C\+O\+RS on your server (\href{http://enable-cors.org/server_expressjs.html}{\tt here’s how to do it for Express}).
\item Use \href{#adding-custom-environment-variables}{\tt environment variables} to inject the right server host and port into your app.
\end{DoxyItemize}

\subsubsection*{\char`\"{}\+Invalid Host Header\char`\"{} Errors After Configuring Proxy}

When you enable the {\ttfamily proxy} option, you opt into a more strict set of host checks. This is necessary because leaving the backend open to remote hosts makes your computer vulnerable to D\+NS rebinding attacks. The issue is explained in \href{https://medium.com/webpack/webpack-dev-server-middleware-security-issues-1489d950874a}{\tt this article} and \href{https://github.com/webpack/webpack-dev-server/issues/887}{\tt this issue}.

This shouldn’t affect you when developing on {\ttfamily localhost}, but if you develop remotely like \href{https://github.com/facebookincubator/create-react-app/issues/2271}{\tt described here}, you will see this error in the browser after enabling the {\ttfamily proxy} option\+:

$>$Invalid Host header

To work around it, you can specify your public development host in a file called {\ttfamily .env.\+development} in the root of your project\+:


\begin{DoxyCode}
HOST=mypublicdevhost.com
\end{DoxyCode}


If you restart the development server now and load the app from the specified host, it should work.

If you are still having issues or if you’re using a more exotic environment like a cloud editor, you can bypass the host check completely by adding a line to {\ttfamily .env.\+development.\+local}. {\bfseries Note that this is dangerous and exposes your machine to remote code execution from malicious websites\+:}


\begin{DoxyCode}
# NOTE: THIS IS DANGEROUS!
# It exposes your machine to attacks from the websites you visit.
DANGEROUSLY\_DISABLE\_HOST\_CHECK=true
\end{DoxyCode}


We don’t recommend this approach.

\subsubsection*{Configuring the Proxy Manually}

$>$Note\+: this feature is available with {\ttfamily react-\/scripts@1.\+0.\+0} and higher.

If the {\ttfamily proxy} option is {\bfseries not} flexible enough for you, you can specify an object in the following form (in {\ttfamily package.\+json}).~\newline
 You may also specify any configuration value \href{https://github.com/chimurai/http-proxy-middleware#options}{\tt {\ttfamily http-\/proxy-\/middleware}} or \href{https://github.com/nodejitsu/node-http-proxy#options}{\tt {\ttfamily http-\/proxy}} supports. 
\begin{DoxyCode}
\{
  // ...
  "proxy": \{
    "/api": \{
      "target": "<url>",
      "ws": true
      // ...
    \}
  \}
  // ...
\}
\end{DoxyCode}


All requests matching this path will be proxies, no exceptions. This includes requests for {\ttfamily text/html}, which the standard {\ttfamily proxy} option does not proxy.

If you need to specify multiple proxies, you may do so by specifying additional entries. You may also narrow down matches using {\ttfamily $\ast$} and/or {\ttfamily $\ast$$\ast$}, to match the path exactly or any subpath. 
\begin{DoxyCode}
\{
  // ...
  "proxy": \{
    // Matches any request starting with /api
    "/api": \{
      "target": "<url\_1>",
      "ws": true
      // ...
    \},
    // Matches any request starting with /foo
    "/foo": \{
      "target": "<url\_2>",
      "ssl": true,
      "pathRewrite": \{
        "^/foo": "/foo/beta"
      \}
      // ...
    \},
    // Matches /bar/abc.html but not /bar/sub/def.html
    "/bar/*.html": \{
      "target": "<url\_3>",
      // ...
    \},
    // Matches /baz/abc.html and /baz/sub/def.html
    "/baz/**/*.html": \{
      "target": "<url\_4>"
      // ...
    \}
  \}
  // ...
\}
\end{DoxyCode}


\subsubsection*{Configuring a Web\+Socket Proxy}

When setting up a Web\+Socket proxy, there are a some extra considerations to be aware of.

If you’re using a Web\+Socket engine like \href{https://socket.io/}{\tt Socket.\+io}, you must have a Socket.\+io server running that you can use as the proxy target. Socket.\+io will not work with a standard Web\+Socket server. Specifically, don\textquotesingle{}t expect Socket.\+io to work with \href{http://websocket.org/echo.html}{\tt the websocket.\+org echo test}.

There’s some good documentation available for \href{https://socket.io/docs/}{\tt setting up a Socket.\+io server}.

Standard Web\+Sockets {\bfseries will} work with a standard Web\+Socket server as well as the websocket.\+org echo test. You can use libraries like \href{https://github.com/websockets/ws}{\tt ws} for the server, with \href{https://developer.mozilla.org/en-US/docs/Web/API/WebSocket}{\tt native Web\+Sockets in the browser}.

Either way, you can proxy Web\+Socket requests manually in {\ttfamily package.\+json}\+:


\begin{DoxyCode}
\{
  // ...
  "proxy": \{
    "/socket": \{
      // Your compatible WebSocket server
      "target": "ws://<socket\_url>",
      // Tell http-proxy-middleware that this is a WebSocket proxy.
      // Also allows you to proxy WebSocket requests without an additional HTTP request
      // https://github.com/chimurai/http-proxy-middleware#external-websocket-upgrade
      "ws": true
      // ...
    \}
  \}
  // ...
\}
\end{DoxyCode}


\subsection*{Using H\+T\+T\+PS in Development}

$>$Note\+: this feature is available with {\ttfamily react-\/scripts@0.\+4.\+0} and higher.

You may require the dev server to serve pages over H\+T\+T\+PS. One particular case where this could be useful is when using \href{#proxying-api-requests-in-development}{\tt the \char`\"{}proxy\char`\"{} feature} to proxy requests to an A\+PI server when that A\+PI server is itself serving H\+T\+T\+PS.

To do this, set the {\ttfamily H\+T\+T\+PS} environment variable to {\ttfamily true}, then start the dev server as usual with {\ttfamily npm start}\+:

\paragraph*{Windows (cmd.\+exe)}


\begin{DoxyCode}
set HTTPS=true&&npm start
\end{DoxyCode}


(Note\+: the lack of whitespace is intentional.)

\paragraph*{Linux, mac\+OS (Bash)}


\begin{DoxyCode}
HTTPS=true npm start
\end{DoxyCode}


Note that the server will use a self-\/signed certificate, so your web browser will almost definitely display a warning upon accessing the page.

\subsection*{Generating Dynamic {\ttfamily $<$meta$>$} Tags on the Server}

Since Create React App doesn’t support server rendering, you might be wondering how to make {\ttfamily $<$meta$>$} tags dynamic and reflect the current \mbox{\hyperlink{namespace_u_r_l}{U\+RL}}. To solve this, we recommend to add placeholders into the H\+T\+ML, like this\+:


\begin{DoxyCode}
<!doctype html>
<html lang="en">
  <head>
    <meta property="og:title" content="\_\_OG\_TITLE\_\_">
    <meta property="og:description" content="\_\_OG\_DESCRIPTION\_\_">
\end{DoxyCode}


Then, on the server, regardless of the backend you use, you can read {\ttfamily index.\+html} into memory and replace {\ttfamily \+\_\+\+\_\+\+O\+G\+\_\+\+T\+I\+T\+L\+E\+\_\+\+\_\+}, {\ttfamily \+\_\+\+\_\+\+O\+G\+\_\+\+D\+E\+S\+C\+R\+I\+P\+T\+I\+O\+N\+\_\+\+\_\+}, and any other placeholders with values depending on the current \mbox{\hyperlink{namespace_u_r_l}{U\+RL}}. Just make sure to sanitize and escape the interpolated values so that they are safe to embed into H\+T\+M\+L!

If you use a Node server, you can even share the route matching logic between the client and the server. However duplicating it also works fine in simple cases.

\subsection*{Pre-\/\+Rendering into Static H\+T\+ML Files}

If you’re hosting your {\ttfamily build} with a static hosting provider you can use \href{https://www.npmjs.com/package/react-snapshot}{\tt react-\/snapshot} to generate H\+T\+ML pages for each route, or relative link, in your application. These pages will then seamlessly become active, or “hydrated”, when the Java\+Script bundle has loaded.

There are also opportunities to use this outside of static hosting, to take the pressure off the server when generating and caching routes.

The primary benefit of pre-\/rendering is that you get the core content of each page {\itshape with} the H\+T\+ML payload—regardless of whether or not your Java\+Script bundle successfully downloads. It also increases the likelihood that each route of your application will be picked up by search engines.

You can read more about \href{https://medium.com/superhighfives/an-almost-static-stack-6df0a2791319}{\tt zero-\/configuration pre-\/rendering (also called snapshotting) here}.

\subsection*{Injecting Data from the Server into the Page}

Similarly to the previous section, you can leave some placeholders in the H\+T\+ML that inject global variables, for example\+:


\begin{DoxyCode}
<!doctype html>
<html lang="en">
  <head>
    <script>
      window.SERVER\_DATA = \_\_SERVER\_DATA\_\_;
    </script>
\end{DoxyCode}


Then, on the server, you can replace {\ttfamily \+\_\+\+\_\+\+S\+E\+R\+V\+E\+R\+\_\+\+D\+A\+T\+A\+\_\+\+\_\+} with a J\+S\+ON of real data right before sending the response. The client code can then read {\ttfamily window.\+S\+E\+R\+V\+E\+R\+\_\+\+D\+A\+TA} to use it. {\bfseries Make sure to \href{https://medium.com/node-security/the-most-common-xss-vulnerability-in-react-js-applications-2bdffbcc1fa0}{\tt sanitize the J\+S\+ON before sending it to the client} as it makes your app vulnerable to X\+SS attacks.}

\subsection*{Running Tests}

$>$Note\+: this feature is available with {\ttfamily react-\/scripts@0.\+3.\+0} and higher.~\newline
 $>$\href{https://github.com/facebookincubator/create-react-app/blob/master/CHANGELOG.md#migrating-from-023-to-030}{\tt Read the migration guide to learn how to enable it in older projects!}

Create React App uses \href{https://facebook.github.io/jest/}{\tt Jest} as its test runner. To prepare for this integration, we did a \href{https://facebook.github.io/jest/blog/2016/09/01/jest-15.html}{\tt major revamp} of Jest so if you heard bad things about it years ago, give it another try.

Jest is a Node-\/based runner. This means that the tests always run in a Node environment and not in a real browser. This lets us enable fast iteration speed and prevent flakiness.

While Jest provides browser globals such as {\ttfamily window} thanks to \href{https://github.com/tmpvar/jsdom}{\tt jsdom}, they are only approximations of the real browser behavior. Jest is intended to be used for unit tests of your logic and your components rather than the D\+OM quirks.

We recommend that you use a separate tool for browser end-\/to-\/end tests if you need them. They are beyond the scope of Create React App.

\subsubsection*{Filename Conventions}

Jest will look for test files with any of the following popular naming conventions\+:


\begin{DoxyItemize}
\item Files with {\ttfamily .js} suffix in {\ttfamily \+\_\+\+\_\+tests\+\_\+\+\_\+} folders.
\item Files with {\ttfamily .test.\+js} suffix.
\item Files with {\ttfamily .spec.\+js} suffix.
\end{DoxyItemize}

The {\ttfamily .test.\+js} / {\ttfamily .spec.\+js} files (or the {\ttfamily \+\_\+\+\_\+tests\+\_\+\+\_\+} folders) can be located at any depth under the {\ttfamily src} top level folder.

We recommend to put the test files (or {\ttfamily \+\_\+\+\_\+tests\+\_\+\+\_\+} folders) next to the code they are testing so that relative imports appear shorter. For example, if {\ttfamily App.\+test.\+js} and {\ttfamily App.\+js} are in the same folder, the test just needs to `import App from './\+App\textquotesingle{}\`{} instead of a long relative path. Colocation also helps find tests more quickly in larger projects.

\subsubsection*{Command Line Interface}

When you run {\ttfamily npm test}, Jest will launch in the watch mode. Every time you save a file, it will re-\/run the tests, just like {\ttfamily npm start} recompiles the code.

The watcher includes an interactive command-\/line interface with the ability to run all tests, or focus on a search pattern. It is designed this way so that you can keep it open and enjoy fast re-\/runs. You can learn the commands from the “\+Watch Usage” note that the watcher prints after every run\+:



\subsubsection*{Version Control Integration}

By default, when you run {\ttfamily npm test}, Jest will only run the tests related to files changed since the last commit. This is an optimization designed to make your tests run fast regardless of how many tests you have. However it assumes that you don’t often commit the code that doesn’t pass the tests.

Jest will always explicitly mention that it only ran tests related to the files changed since the last commit. You can also press {\ttfamily a} in the watch mode to force Jest to run all tests.

Jest will always run all tests on a \href{#continuous-integration}{\tt continuous integration} server or if the project is not inside a Git or Mercurial repository.

\subsubsection*{Writing Tests}

To create tests, add {\ttfamily it()} (or {\ttfamily test()}) blocks with the name of the test and its code. You may optionally wrap them in {\ttfamily describe()} blocks for logical grouping but this is neither required nor recommended.

Jest provides a built-\/in {\ttfamily expect()} global function for making assertions. A basic test could look like this\+:


\begin{DoxyCode}
import sum from './sum';

it('sums numbers', () => \{
  expect(sum(1, 2)).toEqual(3);
  expect(sum(2, 2)).toEqual(4);
\});
\end{DoxyCode}


All {\ttfamily expect()} matchers supported by Jest are \href{http://facebook.github.io/jest/docs/expect.html}{\tt extensively documented here}.~\newline
 You can also use \href{http://facebook.github.io/jest/docs/expect.html#tohavebeencalled}{\tt {\ttfamily jest.\+fn()} and {\ttfamily expect(fn).to\+Be\+Called()}} to create “spies” or mock functions.

\subsubsection*{Testing Components}

There is a broad spectrum of component testing techniques. They range from a “smoke test” verifying that a component renders without throwing, to shallow rendering and testing some of the output, to full rendering and testing component lifecycle and state changes.

Different projects choose different testing tradeoffs based on how often components change, and how much logic they contain. If you haven’t decided on a testing strategy yet, we recommend that you start with creating simple smoke tests for your components\+:


\begin{DoxyCode}
import React from 'react';
import ReactDOM from 'react-dom';
import App from './App';

it('renders without crashing', () => \{
  const div = document.createElement('div');
  ReactDOM.render(<App />, div);
\});
\end{DoxyCode}


This test mounts a component and makes sure that it didn’t throw during rendering. Tests like this provide a lot value with very little effort so they are great as a starting point, and this is the test you will find in {\ttfamily src/\+App.\+test.\+js}.

When you encounter bugs caused by changing components, you will gain a deeper insight into which parts of them are worth testing in your application. This might be a good time to introduce more specific tests asserting specific expected output or behavior.

If you’d like to test components in isolation from the child components they render, we recommend using \href{http://airbnb.io/enzyme/docs/api/shallow.html}{\tt {\ttfamily shallow()} rendering A\+PI} from \href{http://airbnb.io/enzyme/}{\tt Enzyme}. To install it, run\+:


\begin{DoxyCode}
npm install --save enzyme react-test-renderer
\end{DoxyCode}


Alternatively you may use {\ttfamily yarn}\+:


\begin{DoxyCode}
yarn add enzyme react-test-renderer
\end{DoxyCode}


You can write a smoke test with it too\+:


\begin{DoxyCode}
import React from 'react';
import \{ shallow \} from 'enzyme';
import App from './App';

it('renders without crashing', () => \{
  shallow(<App />);
\});
\end{DoxyCode}


Unlike the previous smoke test using {\ttfamily React\+D\+O\+M.\+render()}, this test only renders {\ttfamily $<$App$>$} and doesn’t go deeper. For example, even if {\ttfamily $<$App$>$} itself renders a {\ttfamily $<$Button$>$} that throws, this test will pass. Shallow rendering is great for isolated unit tests, but you may still want to create some full rendering tests to ensure the components integrate correctly. Enzyme supports \href{http://airbnb.io/enzyme/docs/api/mount.html}{\tt full rendering with {\ttfamily mount()}}, and you can also use it for testing state changes and component lifecycle.

You can read the \href{http://airbnb.io/enzyme/}{\tt Enzyme documentation} for more testing techniques. Enzyme documentation uses Chai and Sinon for assertions but you don’t have to use them because Jest provides built-\/in {\ttfamily expect()} and {\ttfamily jest.\+fn()} for spies.

Here is an example from Enzyme documentation that asserts specific output, rewritten to use Jest matchers\+:


\begin{DoxyCode}
import React from 'react';
import \{ shallow \} from 'enzyme';
import App from './App';

it('renders welcome message', () => \{
  const wrapper = shallow(<App />);
  const welcome = <h2>Welcome to React</h2>;
  // expect(wrapper.contains(welcome)).to.equal(true);
  expect(wrapper.contains(welcome)).toEqual(true);
\});
\end{DoxyCode}


All Jest matchers are \href{http://facebook.github.io/jest/docs/expect.html}{\tt extensively documented here}.~\newline
 Nevertheless you can use a third-\/party assertion library like \href{http://chaijs.com/}{\tt Chai} if you want to, as described below.

Additionally, you might find \href{https://github.com/blainekasten/enzyme-matchers}{\tt jest-\/enzyme} helpful to simplify your tests with readable matchers. The above {\ttfamily contains} code can be written simpler with jest-\/enzyme.


\begin{DoxyCode}
expect(wrapper).toContainReact(welcome)
\end{DoxyCode}


To enable this, install {\ttfamily jest-\/enzyme}\+:


\begin{DoxyCode}
npm install --save jest-enzyme
\end{DoxyCode}


Alternatively you may use {\ttfamily yarn}\+:


\begin{DoxyCode}
yarn add jest-enzyme
\end{DoxyCode}


Import it in \href{#initializing-test-environment}{\tt {\ttfamily src/setup\+Tests.\+js}} to make its matchers available in every test\+:


\begin{DoxyCode}
import 'jest-enzyme';
\end{DoxyCode}


\subsubsection*{Using Third Party Assertion Libraries}

We recommend that you use {\ttfamily expect()} for assertions and {\ttfamily jest.\+fn()} for spies. If you are having issues with them please \href{https://github.com/facebook/jest/issues/new}{\tt file those against Jest}, and we’ll fix them. We intend to keep making them better for React, supporting, for example, \href{https://github.com/facebook/jest/pull/1566}{\tt pretty-\/printing React elements as J\+SX}.

However, if you are used to other libraries, such as \href{http://chaijs.com/}{\tt Chai} and \href{http://sinonjs.org/}{\tt Sinon}, or if you have existing code using them that you’d like to port over, you can import them normally like this\+:


\begin{DoxyCode}
import sinon from 'sinon';
import \{ expect \} from 'chai';
\end{DoxyCode}


and then use them in your tests like you normally do.

\subsubsection*{Initializing Test Environment}

$>$Note\+: this feature is available with {\ttfamily react-\/scripts@0.\+4.\+0} and higher.

If your app uses a browser A\+PI that you need to mock in your tests or if you just need a global setup before running your tests, add a {\ttfamily src/setup\+Tests.\+js} to your project. It will be automatically executed before running your tests.

For example\+:

\#\#\#\# {\ttfamily src/setup\+Tests.\+js} 
\begin{DoxyCode}
const localStorageMock = \{
  getItem: jest.fn(),
  setItem: jest.fn(),
  clear: jest.fn()
\};
global.localStorage = localStorageMock
\end{DoxyCode}


\subsubsection*{Focusing and Excluding Tests}

You can replace {\ttfamily it()} with {\ttfamily xit()} to temporarily exclude a test from being executed.~\newline
 Similarly, {\ttfamily fit()} lets you focus on a specific test without running any other tests.

\subsubsection*{Coverage Reporting}

Jest has an integrated coverage reporter that works well with E\+S6 and requires no configuration.~\newline
 Run {\ttfamily npm test -\/-\/ -\/-\/coverage} (note extra {\ttfamily -\/-\/} in the middle) to include a coverage report like this\+:



Note that tests run much slower with coverage so it is recommended to run it separately from your normal workflow.

\subsubsection*{Continuous Integration}

By default {\ttfamily npm test} runs the watcher with interactive C\+LI. However, you can force it to run tests once and finish the process by setting an environment variable called {\ttfamily CI}.

When creating a build of your application with {\ttfamily npm run build} linter warnings are not checked by default. Like {\ttfamily npm test}, you can force the build to perform a linter warning check by setting the environment variable {\ttfamily CI}. If any warnings are encountered then the build fails.

Popular CI servers already set the environment variable {\ttfamily CI} by default but you can do this yourself too\+:

\subsubsection*{On CI servers}

\paragraph*{Travis CI}


\begin{DoxyEnumerate}
\item Following the \href{https://docs.travis-ci.com/user/getting-started/}{\tt Travis Getting started} guide for syncing your Git\+Hub repository with Travis. You may need to initialize some settings manually in your \href{https://travis-ci.org/profile}{\tt profile} page.
\end{DoxyEnumerate}
\begin{DoxyEnumerate}
\item Add a {\ttfamily .travis.\+yml} file to your git repository. 
\begin{DoxyCode}
language: node\_js
node\_js:
  - 6
cache:
  directories:
    - node\_modules
script:
  - npm test
  - npm run build
\end{DoxyCode}

\end{DoxyEnumerate}
\begin{DoxyEnumerate}
\item Trigger your first build with a git push.
\end{DoxyEnumerate}
\begin{DoxyEnumerate}
\item \href{https://docs.travis-ci.com/user/customizing-the-build/}{\tt Customize your Travis CI Build} if needed.
\end{DoxyEnumerate}

\paragraph*{Circle\+CI}

Follow \href{https://medium.com/@knowbody/circleci-and-zeits-now-sh-c9b7eebcd3c1}{\tt this article} to set up Circle\+CI with a Create React App project.

\subsubsection*{On your own environment}

\subparagraph*{Windows (cmd.\+exe)}


\begin{DoxyCode}
set CI=true&&npm test
\end{DoxyCode}



\begin{DoxyCode}
set CI=true&&npm run build
\end{DoxyCode}


(Note\+: the lack of whitespace is intentional.)

\subparagraph*{Linux, mac\+OS (Bash)}


\begin{DoxyCode}
CI=true npm test
\end{DoxyCode}



\begin{DoxyCode}
CI=true npm run build
\end{DoxyCode}


The test command will force Jest to run tests once instead of launching the watcher.

\begin{quote}
If you find yourself doing this often in development, please \href{https://github.com/facebookincubator/create-react-app/issues/new}{\tt file an issue} to tell us about your use case because we want to make watcher the best experience and are open to changing how it works to accommodate more workflows. \end{quote}


The build command will check for linter warnings and fail if any are found.

\subsubsection*{Disabling jsdom}

By default, the {\ttfamily package.\+json} of the generated project looks like this\+:


\begin{DoxyCode}
"scripts": \{
  "start": "react-scripts start",
  "build": "react-scripts build",
  "test": "react-scripts test --env=jsdom"
\end{DoxyCode}


If you know that none of your tests depend on \href{https://github.com/tmpvar/jsdom}{\tt jsdom}, you can safely remove {\ttfamily -\/-\/env=jsdom}, and your tests will run faster\+:


\begin{DoxyCode}
  "scripts": \{
    "start": "react-scripts start",
    "build": "react-scripts build",
-   "test": "react-scripts test --env=jsdom"
+   "test": "react-scripts test"
\end{DoxyCode}


To help you make up your mind, here is a list of A\+P\+Is that {\bfseries need jsdom}\+:


\begin{DoxyItemize}
\item Any browser globals like {\ttfamily window} and {\ttfamily document}
\item \href{https://facebook.github.io/react/docs/top-level-api.html#reactdom.render}{\tt {\ttfamily React\+D\+O\+M.\+render()}}
\item \href{https://facebook.github.io/react/docs/test-utils.html#renderintodocument}{\tt {\ttfamily Test\+Utils.\+render\+Into\+Document()}} (\href{https://github.com/facebook/react/blob/34761cf9a252964abfaab6faf74d473ad95d1f21/src/test/ReactTestUtils.js#L83-L91}{\tt a shortcut} for the above)
\item \href{http://airbnb.io/enzyme/docs/api/mount.html}{\tt {\ttfamily mount()}} in \href{http://airbnb.io/enzyme/index.html}{\tt Enzyme}
\end{DoxyItemize}

In contrast, {\bfseries jsdom is not needed} for the following A\+P\+Is\+:


\begin{DoxyItemize}
\item \href{https://facebook.github.io/react/docs/test-utils.html#shallow-rendering}{\tt {\ttfamily Test\+Utils.\+create\+Renderer()}} (shallow rendering)
\item \href{http://airbnb.io/enzyme/docs/api/shallow.html}{\tt {\ttfamily shallow()}} in \href{http://airbnb.io/enzyme/index.html}{\tt Enzyme}
\end{DoxyItemize}

Finally, jsdom is also not needed for \href{http://facebook.github.io/jest/blog/2016/07/27/jest-14.html}{\tt snapshot testing}.

\subsubsection*{Snapshot Testing}

Snapshot testing is a feature of Jest that automatically generates text snapshots of your components and saves them on the disk so if the UI output changes, you get notified without manually writing any assertions on the component output. \href{http://facebook.github.io/jest/blog/2016/07/27/jest-14.html}{\tt Read more about snapshot testing.}

\subsubsection*{Editor Integration}

If you use \href{https://code.visualstudio.com}{\tt Visual Studio Code}, there is a \href{https://github.com/orta/vscode-jest}{\tt Jest extension} which works with Create React App out of the box. This provides a lot of I\+D\+E-\/like features while using a text editor\+: showing the status of a test run with potential fail messages inline, starting and stopping the watcher automatically, and offering one-\/click snapshot updates.



\subsection*{Developing Components in Isolation}

Usually, in an app, you have a lot of UI components, and each of them has many different states. For an example, a simple button component could have following states\+:


\begin{DoxyItemize}
\item In a regular state, with a text label.
\item In the disabled mode.
\item In a loading state.
\end{DoxyItemize}

Usually, it’s hard to see these states without running a sample app or some examples.

Create React App doesn’t include any tools for this by default, but you can easily add \href{https://storybook.js.org}{\tt Storybook for React} (\href{https://github.com/storybooks/storybook}{\tt source}) or \href{https://react-styleguidist.js.org/}{\tt React Styleguidist} (\href{https://github.com/styleguidist/react-styleguidist}{\tt source}) to your project. {\bfseries These are third-\/party tools that let you develop components and see all their states in isolation from your app}.



You can also deploy your Storybook or style guide as a static app. This way, everyone in your team can view and review different states of UI components without starting a backend server or creating an account in your app.

\subsubsection*{Getting Started with Storybook}

Storybook is a development environment for React UI components. It allows you to browse a component library, view the different states of each component, and interactively develop and test components.

First, install the following npm package globally\+:


\begin{DoxyCode}
npm install -g @storybook/cli
\end{DoxyCode}


Then, run the following command inside your app’s directory\+:


\begin{DoxyCode}
getstorybook
\end{DoxyCode}


After that, follow the instructions on the screen.

Learn more about React Storybook\+:


\begin{DoxyItemize}
\item Screencast\+: \href{https://egghead.io/lessons/react-getting-started-with-react-storybook}{\tt Getting Started with React Storybook}
\item \href{https://github.com/storybooks/storybook}{\tt Git\+Hub Repo}
\item \href{https://storybook.js.org/basics/introduction/}{\tt Documentation}
\item \href{https://github.com/storybooks/storybook/tree/master/addons/storyshots}{\tt Snapshot Testing UI} with Storybook + addon/storyshot
\end{DoxyItemize}

\subsubsection*{Getting Started with Styleguidist}

Styleguidist combines a style guide, where all your components are presented on a single page with their props documentation and usage examples, with an environment for developing components in isolation, similar to Storybook. In Styleguidist you write examples in Markdown, where each code snippet is rendered as a live editable playground.

First, install Styleguidist\+:


\begin{DoxyCode}
npm install --save react-styleguidist
\end{DoxyCode}


Alternatively you may use {\ttfamily yarn}\+:


\begin{DoxyCode}
yarn add react-styleguidist
\end{DoxyCode}


Then, add these scripts to your {\ttfamily package.\+json}\+:


\begin{DoxyCode}
   "scripts": \{
+    "styleguide": "styleguidist server",
+    "styleguide:build": "styleguidist build",
     "start": "react-scripts start",
\end{DoxyCode}


Then, run the following command inside your app’s directory\+:


\begin{DoxyCode}
npm run styleguide
\end{DoxyCode}


After that, follow the instructions on the screen.

Learn more about React Styleguidist\+:


\begin{DoxyItemize}
\item \href{https://github.com/styleguidist/react-styleguidist}{\tt Git\+Hub Repo}
\item \href{https://react-styleguidist.js.org/docs/getting-started.html}{\tt Documentation}
\end{DoxyItemize}

\subsection*{Making a Progressive Web App}

By default, the production build is a fully functional, offline-\/first \href{https://developers.google.com/web/progressive-web-apps/}{\tt Progressive Web App}.

Progressive Web Apps are faster and more reliable than traditional web pages, and provide an engaging mobile experience\+:


\begin{DoxyItemize}
\item All static site assets are cached so that your page loads fast on subsequent visits, regardless of network connectivity (such as 2G or 3G). Updates are downloaded in the background.
\item Your app will work regardless of network state, even if offline. This means your users will be able to use your app at 10,000 feet and on the Subway.
\item On mobile devices, your app can be added directly to the user\textquotesingle{}s home screen, app icon and all. You can also re-\/engage users using web {\bfseries push notifications}. This eliminates the need for the app store.
\end{DoxyItemize}

The \href{https://github.com/goldhand/sw-precache-webpack-plugin}{\tt {\ttfamily sw-\/precache-\/webpack-\/plugin}} is integrated into production configuration, and it will take care of generating a service worker file that will automatically precache all of your local assets and keep them up to date as you deploy updates. The service worker will use a \href{https://developers.google.com/web/fundamentals/instant-and-offline/offline-cookbook/#cache-falling-back-to-network}{\tt cache-\/first strategy} for handling all requests for local assets, including the initial H\+T\+ML, ensuring that your web app is reliably fast, even on a slow or unreliable network.

If you would prefer not to enable service workers prior to your initial production deployment, then remove the call to {\ttfamily service\+Worker\+Registration.\+register()} from \href{src/index.js}{\tt {\ttfamily src/index.\+js}}.

If you had previously enabled service workers in your production deployment and have decided that you would like to disable them for all your existing users, you can swap out the call to {\ttfamily service\+Worker\+Registration.\+register()} in \href{src/index.js}{\tt {\ttfamily src/index.\+js}} with a call to {\ttfamily service\+Worker\+Registration.\+unregister()}. After the user visits a page that has {\ttfamily service\+Worker\+Registration.\+unregister()}, the service worker will be uninstalled.

\subsubsection*{Offline-\/\+First Considerations}


\begin{DoxyEnumerate}
\item Service workers \href{https://developers.google.com/web/fundamentals/getting-started/primers/service-workers#you_need_https}{\tt require H\+T\+T\+PS}, although to facilitate local testing, that policy \href{http://stackoverflow.com/questions/34160509/options-for-testing-service-workers-via-http/34161385#34161385}{\tt does not apply to {\ttfamily localhost}}. If your production web server does not support H\+T\+T\+PS, then the service worker registration will fail, but the rest of your web app will remain functional.
\end{DoxyEnumerate}
\begin{DoxyEnumerate}
\item Service workers are \href{https://jakearchibald.github.io/isserviceworkerready/}{\tt not currently supported} in all web browsers. Service worker registration \href{src/registerServiceWorker.js}{\tt won\textquotesingle{}t be attempted} on browsers that lack support.
\end{DoxyEnumerate}
\begin{DoxyEnumerate}
\item The service worker is only enabled in the \href{#deployment}{\tt production environment}, e.\+g. the output of {\ttfamily npm run build}. It\textquotesingle{}s recommended that you do not enable an offline-\/first service worker in a development environment, as it can lead to frustration when previously cached assets are used and do not include the latest changes you\textquotesingle{}ve made locally.
\end{DoxyEnumerate}
\begin{DoxyEnumerate}
\item If you {\itshape need} to test your offline-\/first service worker locally, build the application (using {\ttfamily npm run build}) and run a simple http server from your build directory. After running the build script, {\ttfamily create-\/react-\/app} will give instructions for one way to test your production build locally and the \href{#deployment}{\tt deployment instructions} have instructions for using other methods. {\itshape Be sure to always use an incognito window to avoid complications with your browser cache.}
\end{DoxyEnumerate}
\begin{DoxyEnumerate}
\item If possible, configure your production environment to serve the generated {\ttfamily service-\/worker.\+js} \href{http://stackoverflow.com/questions/38843970/service-worker-javascript-update-frequency-every-24-hours}{\tt with H\+T\+TP caching disabled}. If that\textquotesingle{}s not possible—\href{#github-pages}{\tt Git\+Hub Pages}, for instance, does not allow you to change the default 10 minute H\+T\+TP cache lifetime—then be aware that if you visit your production site, and then revisit again before {\ttfamily service-\/worker.\+js} has expired from your H\+T\+TP cache, you\textquotesingle{}ll continue to get the previously cached assets from the service worker. If you have an immediate need to view your updated production deployment, performing a shift-\/refresh will temporarily disable the service worker and retrieve all assets from the network.
\end{DoxyEnumerate}
\begin{DoxyEnumerate}
\item Users aren\textquotesingle{}t always familiar with offline-\/first web apps. It can be useful to \href{https://developers.google.com/web/fundamentals/instant-and-offline/offline-ux#inform_the_user_when_the_app_is_ready_for_offline_consumption}{\tt let the user know} when the service worker has finished populating your caches (showing a \char`\"{}\+This web
app works offline!\char`\"{} message) and also let them know when the service worker has fetched the latest updates that will be available the next time they load the page (showing a \char`\"{}\+New content is available; please refresh.\char`\"{} message). Showing this messages is currently left as an exercise to the developer, but as a starting point, you can make use of the logic included in \href{src/registerServiceWorker.js}{\tt {\ttfamily src/register\+Service\+Worker.\+js}}, which demonstrates which service worker lifecycle events to listen for to detect each scenario, and which as a default, just logs appropriate messages to the Java\+Script console.
\end{DoxyEnumerate}
\begin{DoxyEnumerate}
\item By default, the generated service worker file will not intercept or cache any cross-\/origin traffic, like H\+T\+TP \href{#integrating-with-an-api-backend}{\tt A\+PI requests}, images, or embeds loaded from a different domain. If you would like to use a runtime caching strategy for those requests, you can \href{#npm-run-eject}{\tt {\ttfamily eject}} and then configure the \href{https://github.com/GoogleChrome/sw-precache#runtimecaching-arrayobject}{\tt {\ttfamily runtime\+Caching}} option in the {\ttfamily S\+W\+Precache\+Webpack\+Plugin} section of \href{../config/webpack.config.prod.js}{\tt {\ttfamily webpack.\+config.\+prod.\+js}}.
\end{DoxyEnumerate}

\subsubsection*{Progressive Web App Metadata}

The default configuration includes a web app manifest located at \href{public/manifest.json}{\tt {\ttfamily public/manifest.\+json}}, that you can customize with details specific to your web application.

When a user adds a web app to their homescreen using Chrome or Firefox on Android, the metadata in \href{public/manifest.json}{\tt {\ttfamily manifest.\+json}} determines what icons, names, and branding colors to use when the web app is displayed. \href{https://developers.google.com/web/fundamentals/engage-and-retain/web-app-manifest/}{\tt The Web App Manifest guide} provides more context about what each field means, and how your customizations will affect your users\textquotesingle{} experience.

\subsection*{Analyzing the Bundle Size}

\href{https://www.npmjs.com/package/source-map-explorer}{\tt Source map explorer} analyzes Java\+Script bundles using the source maps. This helps you understand where code bloat is coming from.

To add Source map explorer to a Create React App project, follow these steps\+:


\begin{DoxyCode}
npm install --save source-map-explorer
\end{DoxyCode}


Alternatively you may use {\ttfamily yarn}\+:


\begin{DoxyCode}
yarn add source-map-explorer
\end{DoxyCode}


Then in {\ttfamily package.\+json}, add the following line to {\ttfamily scripts}\+:


\begin{DoxyCode}
   "scripts": \{
+    "analyze": "source-map-explorer build/static/js/main.*",
     "start": "react-scripts start",
     "build": "react-scripts build",
     "test": "react-scripts test --env=jsdom",
\end{DoxyCode}


$>$$\ast$$\ast$\+Note\+:$\ast$$\ast$ \begin{quote}


$>$This doesn\textquotesingle{}t quite work on Windows because it doesn\textquotesingle{}t automatically expand {\ttfamily $\ast$} in the filepath. For now, the workaround is to look at the full hashed filename in {\ttfamily build/static/js} (e.\+g. {\ttfamily main.\+89b7e95a.\+js}) and copy it into {\ttfamily package.\+json} when you\textquotesingle{}re running the analyzer. For example\+:

$>$\`{}\`{}\`{}diff $>$+ \char`\"{}analyze\char`\"{}\+: \char`\"{}source-\/map-\/explorer build/static/js/main.\+89b7e95a.\+js\char`\"{}, $>$\`{}\`{}\`{}

$>$Unfortunately it will be different after every build. You can express support for fixing this on Windows \href{https://github.com/danvk/source-map-explorer/issues/52}{\tt in this issue}. \end{quote}


Then to analyze the bundle run the production build then run the analyze script.


\begin{DoxyCode}
npm run build
npm run analyze
\end{DoxyCode}


\subsection*{Deployment}

{\ttfamily npm run build} creates a {\ttfamily build} directory with a production build of your app. Set up your favourite H\+T\+TP server so that a visitor to your site is served {\ttfamily index.\+html}, and requests to static paths like {\ttfamily /static/js/main.$<$hash$>$.js} are served with the contents of the {\ttfamily /static/js/main.$<$hash$>$.js} file.

\subsubsection*{Static Server}

For environments using \href{https://nodejs.org/}{\tt Node}, the easiest way to handle this would be to install \href{https://github.com/zeit/serve}{\tt serve} and let it handle the rest\+:


\begin{DoxyCode}
npm install -g serve
serve -s build
\end{DoxyCode}


The last command shown above will serve your static site on the port {\bfseries 5000}. Like many of \href{https://github.com/zeit/serve}{\tt serve}’s internal settings, the port can be adjusted using the {\ttfamily -\/p} or {\ttfamily -\/-\/port} flags.

Run this command to get a full list of the options available\+:


\begin{DoxyCode}
serve -h
\end{DoxyCode}


\subsubsection*{Other Solutions}

You don’t necessarily need a static server in order to run a Create React App project in production. It works just as fine integrated into an existing dynamic one.

Here’s a programmatic example using \href{https://nodejs.org/}{\tt Node} and \href{http://expressjs.com/}{\tt Express}\+:


\begin{DoxyCode}
const express = require('express');
const path = require('path');
const app = express();

app.use(express.static(path.join(\_\_dirname, 'build')));

app.get('/', function (req, res) \{
  res.sendFile(path.join(\_\_dirname, 'build', 'index.html'));
\});

app.listen(9000);
\end{DoxyCode}


The choice of your server software isn’t important either. Since Create React App is completely platform-\/agnostic, there’s no need to explicitly use Node.

The {\ttfamily build} folder with static assets is the only output produced by Create React App.

However this is not quite enough if you use client-\/side routing. Read the next section if you want to support U\+R\+Ls like {\ttfamily /todos/42} in your single-\/page app.

\subsubsection*{Serving Apps with Client-\/\+Side Routing}

If you use routers that use the H\+T\+M\+L5 \href{https://developer.mozilla.org/en-US/docs/Web/API/History_API#Adding_and_modifying_history_entries}{\tt {\ttfamily push\+State} history A\+PI} under the hood (for example, \href{https://github.com/ReactTraining/react-router}{\tt React Router} with {\ttfamily browser\+History}), many static file servers will fail. For example, if you used React Router with a route for {\ttfamily /todos/42}, the development server will respond to {\ttfamily localhost\+:3000/todos/42} properly, but an Express serving a production build as above will not.

This is because when there is a fresh page load for a {\ttfamily /todos/42}, the server looks for the file {\ttfamily build/todos/42} and does not find it. The server needs to be configured to respond to a request to {\ttfamily /todos/42} by serving {\ttfamily index.\+html}. For example, we can amend our Express example above to serve {\ttfamily index.\+html} for any unknown paths\+:


\begin{DoxyCode}
 app.use(express.static(path.join(\_\_dirname, 'build')));

-app.get('/', function (req, res) \{
+app.get('/*', function (req, res) \{
   res.sendFile(path.join(\_\_dirname, 'build', 'index.html'));
 \});
\end{DoxyCode}


If you’re using \href{https://httpd.apache.org/}{\tt Apache H\+T\+TP Server}, you need to create a {\ttfamily .htaccess} file in the {\ttfamily public} folder that looks like this\+:


\begin{DoxyCode}
Options -MultiViews
RewriteEngine On
RewriteCond %\{REQUEST\_FILENAME\} !-f
RewriteRule ^ index.html [QSA,L]
\end{DoxyCode}


It will get copied to the {\ttfamily build} folder when you run {\ttfamily npm run build}.

If you’re using \href{http://tomcat.apache.org/}{\tt Apache Tomcat}, you need to follow \href{https://stackoverflow.com/a/41249464/4878474}{\tt this Stack Overflow answer}.

Now requests to {\ttfamily /todos/42} will be handled correctly both in development and in production.

On a production build, and in a browser that supports \href{https://developers.google.com/web/fundamentals/getting-started/primers/service-workers}{\tt service workers}, the service worker will automatically handle all navigation requests, like for {\ttfamily /todos/42}, by serving the cached copy of your {\ttfamily index.\+html}. This service worker navigation routing can be configured or disabled by \href{#npm-run-eject}{\tt {\ttfamily eject}ing} and then modifying the \href{https://github.com/GoogleChrome/sw-precache#navigatefallback-string}{\tt {\ttfamily navigate\+Fallback}} and \href{https://github.com/GoogleChrome/sw-precache#navigatefallbackwhitelist-arrayregexp}{\tt {\ttfamily navigate\+Fallback\+Whitelist}} options of the {\ttfamily S\+W\+Preache\+Plugin} \href{../config/webpack.config.prod.js}{\tt configuration}.

\subsubsection*{Building for Relative Paths}

By default, Create React App produces a build assuming your app is hosted at the server root.~\newline
 To override this, specify the {\ttfamily homepage} in your {\ttfamily package.\+json}, for example\+:


\begin{DoxyCode}
"homepage": "http://mywebsite.com/relativepath",
\end{DoxyCode}


This will let Create React App correctly infer the root path to use in the generated H\+T\+ML file.

\paragraph*{Serving the Same Build from Different Paths}

$>$Note\+: this feature is available with {\ttfamily react-\/scripts@0.\+9.\+0} and higher.

If you are not using the H\+T\+M\+L5 {\ttfamily push\+State} history A\+PI or not using client-\/side routing at all, it is unnecessary to specify the \mbox{\hyperlink{namespace_u_r_l}{U\+RL}} from which your app will be served. Instead, you can put this in your {\ttfamily package.\+json}\+:


\begin{DoxyCode}
"homepage": ".",
\end{DoxyCode}


This will make sure that all the asset paths are relative to {\ttfamily index.\+html}. You will then be able to move your app from {\ttfamily \href{http://mywebsite.com}{\tt http\+://mywebsite.\+com}} to {\ttfamily \href{http://mywebsite.com/relativepath}{\tt http\+://mywebsite.\+com/relativepath}} or even {\ttfamily \href{http://mywebsite.com/relative/path}{\tt http\+://mywebsite.\+com/relative/path}} without having to rebuild it.

\subsubsection*{Azure}

See \href{https://medium.com/@to_pe/deploying-create-react-app-on-microsoft-azure-c0f6686a4321}{\tt this} blog post on how to deploy your React app to \href{https://azure.microsoft.com/}{\tt Microsoft Azure}.

\subsubsection*{Firebase}

Install the Firebase C\+LI if you haven’t already by running {\ttfamily npm install -\/g firebase-\/tools}. Sign up for a \href{https://console.firebase.google.com/}{\tt Firebase account} and create a new project. Run {\ttfamily firebase login} and login with your previous created Firebase account.

Then run the {\ttfamily firebase init} command from your project’s root. You need to choose the {\bfseries Hosting\+: Configure and deploy Firebase Hosting sites} and choose the Firebase project you created in the previous step. You will need to agree with {\ttfamily database.\+rules.\+json} being created, choose {\ttfamily build} as the public directory, and also agree to {\bfseries Configure as a single-\/page app} by replying with {\ttfamily y}.


\begin{DoxyCode}
=== Project Setup

First, let's associate this project directory with a Firebase project.
You can create multiple project aliases by running firebase use --add,
but for now we'll just set up a default project.

? What Firebase project do you want to associate as default? Example app (example-app-fd690)

=== Database Setup

Firebase Realtime Database Rules allow you to define how your data should be
structured and when your data can be read from and written to.

? What file should be used for Database Rules? database.rules.json
✔  Database Rules for example-app-fd690 have been downloaded to database.rules.json.
Future modifications to database.rules.json will update Database Rules when you run
firebase deploy.

=== Hosting Setup

Your public directory is the folder (relative to your project directory) that
will contain Hosting assets to uploaded with firebase deploy. If you
have a build process for your assets, use your build's output directory.

? What do you want to use as your public directory? build
? Configure as a single-page app (rewrite all urls to /index.html)? Yes
✔  Wrote build/index.html

i  Writing configuration info to firebase.json...
i  Writing project information to .firebaserc...

✔  Firebase initialization complete!
\end{DoxyCode}


Now, after you create a production build with {\ttfamily npm run build}, you can deploy it by running {\ttfamily firebase deploy}.


\begin{DoxyCode}
=== Deploying to 'example-app-fd690'...

i  deploying database, hosting
✔  database: rules ready to deploy.
i  hosting: preparing build directory for upload...
Uploading: [==============================          ] 75%✔  hosting: build folder uploaded successfully
✔  hosting: 8 files uploaded successfully
i  starting release process (may take several minutes)...

✔  Deploy complete!

Project Console: https://console.firebase.google.com/project/example-app-fd690/overview
Hosting URL: https://example-app-fd690.firebaseapp.com
\end{DoxyCode}


For more information see \href{https://firebase.google.com/docs/web/setup}{\tt Add Firebase to your Java\+Script Project}.

\subsubsection*{Git\+Hub Pages}

$>$Note\+: this feature is available with {\ttfamily react-\/scripts@0.\+2.\+0} and higher.

\paragraph*{Step 1\+: Add {\ttfamily homepage} to {\ttfamily package.\+json}}

{\bfseries The step below is important!}~\newline
 {\bfseries If you skip it, your app will not deploy correctly.}

Open your {\ttfamily package.\+json} and add a {\ttfamily homepage} field\+:


\begin{DoxyCode}
"homepage": "https://myusername.github.io/my-app",
\end{DoxyCode}


Create React App uses the {\ttfamily homepage} field to determine the root \mbox{\hyperlink{namespace_u_r_l}{U\+RL}} in the built H\+T\+ML file.

\paragraph*{Step 2\+: Install {\ttfamily gh-\/pages} and add {\ttfamily deploy} to {\ttfamily scripts} in {\ttfamily package.\+json}}

Now, whenever you run {\ttfamily npm run build}, you will see a cheat sheet with instructions on how to deploy to Git\+Hub Pages.

To publish it at \href{https://myusername.github.io/my-app}{\tt https\+://myusername.\+github.\+io/my-\/app}, run\+:


\begin{DoxyCode}
npm install --save gh-pages
\end{DoxyCode}


Alternatively you may use {\ttfamily yarn}\+:


\begin{DoxyCode}
yarn add gh-pages
\end{DoxyCode}


Add the following scripts in your {\ttfamily package.\+json}\+:


\begin{DoxyCode}
  "scripts": \{
+   "predeploy": "npm run build",
+   "deploy": "gh-pages -d build",
    "start": "react-scripts start",
    "build": "react-scripts build",
\end{DoxyCode}


The {\ttfamily predeploy} script will run automatically before {\ttfamily deploy} is run.

\paragraph*{Step 3\+: Deploy the site by running {\ttfamily npm run deploy}}

Then run\+:


\begin{DoxyCode}
npm run deploy
\end{DoxyCode}


\paragraph*{Step 4\+: Ensure your project’s settings use {\ttfamily gh-\/pages}}

Finally, make sure {\bfseries Git\+Hub Pages} option in your Git\+Hub project settings is set to use the {\ttfamily gh-\/pages} branch\+:



\paragraph*{Step 5\+: Optionally, configure the domain}

You can configure a custom domain with Git\+Hub Pages by adding a {\ttfamily C\+N\+A\+ME} file to the {\ttfamily public/} folder.

\paragraph*{Notes on client-\/side routing}

Git\+Hub Pages doesn’t support routers that use the H\+T\+M\+L5 {\ttfamily push\+State} history A\+PI under the hood (for example, React Router using {\ttfamily browser\+History}). This is because when there is a fresh page load for a url like {\ttfamily \href{http://user.github.io/todomvc/todos/42}{\tt http\+://user.\+github.\+io/todomvc/todos/42}}, where {\ttfamily /todos/42} is a frontend route, the Git\+Hub Pages server returns 404 because it knows nothing of {\ttfamily /todos/42}. If you want to add a router to a project hosted on Git\+Hub Pages, here are a couple of solutions\+:


\begin{DoxyItemize}
\item You could switch from using H\+T\+M\+L5 history A\+PI to routing with hashes. If you use React Router, you can switch to {\ttfamily hash\+History} for this effect, but the \mbox{\hyperlink{namespace_u_r_l}{U\+RL}} will be longer and more verbose (for example, {\ttfamily \href{http://user.github.io/todomvc/#/todos/42?_k=yknaj}{\tt http\+://user.\+github.\+io/todomvc/\#/todos/42?\+\_\+k=yknaj}}). \href{https://reacttraining.com/react-router/web/api/Router}{\tt Read more} about different history implementations in React Router.
\item Alternatively, you can use a trick to teach Git\+Hub Pages to handle 404 by redirecting to your {\ttfamily index.\+html} page with a special redirect parameter. You would need to add a {\ttfamily 404.\+html} file with the redirection code to the {\ttfamily build} folder before deploying your project, and you’ll need to add code handling the redirect parameter to {\ttfamily index.\+html}. You can find a detailed explanation of this technique \href{https://github.com/rafrex/spa-github-pages}{\tt in this guide}.
\end{DoxyItemize}

\subsubsection*{Heroku}

Use the \href{https://github.com/mars/create-react-app-buildpack}{\tt Heroku Buildpack for Create React App}.~\newline
 You can find instructions in \href{https://blog.heroku.com/deploying-react-with-zero-configuration}{\tt Deploying React with Zero Configuration}.

\paragraph*{Resolving Heroku Deployment Errors}

Sometimes {\ttfamily npm run build} works locally but fails during deploy via Heroku. Following are the most common cases.

\subparagraph*{\char`\"{}\+Module not found\+: Error\+: Cannot resolve \textquotesingle{}file\textquotesingle{} or \textquotesingle{}directory\textquotesingle{}\char`\"{}}

If you get something like this\+:


\begin{DoxyCode}
remote: Failed to create a production build. Reason:
remote: Module not found: Error: Cannot resolve 'file' or 'directory'
MyDirectory in /tmp/build\_1234/src
\end{DoxyCode}


It means you need to ensure that the lettercase of the file or directory you {\ttfamily import} matches the one you see on your filesystem or on Git\+Hub.

This is important because Linux (the operating system used by Heroku) is case sensitive. So {\ttfamily My\+Directory} and {\ttfamily mydirectory} are two distinct directories and thus, even though the project builds locally, the difference in case breaks the {\ttfamily import} statements on Heroku remotes.

\subparagraph*{\char`\"{}\+Could not find a required file.\char`\"{}}

If you exclude or ignore necessary files from the package you will see a error similar this one\+:

\`{}\`{}\`{} remote\+: Could not find a required file. remote\+: Name\+: {\ttfamily index.\+html} remote\+: Searched in\+: /tmp/build\+\_\+a2875fc163b209225122d68916f1d4df/public remote\+: remote\+: npm E\+R\+R! Linux 3.\+13.\+0-\/105-\/generic remote\+: npm E\+R\+R! argv \char`\"{}/tmp/build\+\_\+a2875fc163b209225122d68916f1d4df/.\+heroku/node/bin/node\char`\"{} \char`\"{}/tmp/build\+\_\+a2875fc163b209225122d68916f1d4df/.\+heroku/node/bin/npm\char`\"{} \char`\"{}run\char`\"{} \char`\"{}build\char`\"{} 
\begin{DoxyCode}
In this case, ensure that the file is there with the proper lettercase and that’s not ignored on your local
       `.gitignore` or `~/.gitignore\_global`.

### Modulus

See the [Modulus blog post](http://blog.modulus.io/deploying-react-apps-on-modulus) on how to deploy your
       react app to Modulus.

### Netlify

**To do a manual deploy to Netlify’s CDN:**

```sh
npm install netlify-cli
netlify deploy
\end{DoxyCode}


Choose {\ttfamily build} as the path to deploy.

{\bfseries To setup continuous delivery\+:}

With this setup Netlify will build and deploy when you push to git or open a pull request\+:


\begin{DoxyEnumerate}
\item \href{https://app.netlify.com/signup}{\tt Start a new netlify project}
\item Pick your Git hosting service and select your repository
\item Click {\ttfamily Build your site}
\end{DoxyEnumerate}

{\bfseries Support for client-\/side routing\+:}

To support {\ttfamily push\+State}, make sure to create a {\ttfamily public/\+\_\+redirects} file with the following rewrite rules\+:


\begin{DoxyCode}
/*  /index.html  200
\end{DoxyCode}


When you build the project, Create React App will place the {\ttfamily public} folder contents into the build output.

\subsubsection*{Now}

\href{https://zeit.co/now}{\tt now} offers a zero-\/configuration single-\/command deployment. You can use {\ttfamily now} to deploy your app for free.


\begin{DoxyEnumerate}
\item Install the {\ttfamily now} command-\/line tool either via the recommended \href{https://zeit.co/download}{\tt desktop tool} or via node with {\ttfamily npm install -\/g now}.
\item Build your app by running {\ttfamily npm run build}.
\item Move into the build directory by running {\ttfamily cd build}.
\item Run {\ttfamily now -\/-\/name your-\/project-\/name} from within the build directory. You will see a {\bfseries now.\+sh} \mbox{\hyperlink{namespace_u_r_l}{U\+RL}} in your output like this\+:

\`{}\`{}\`{} $>$ Ready! \href{https://your-project-name-tpspyhtdtk.now.sh}{\tt https\+://your-\/project-\/name-\/tpspyhtdtk.\+now.\+sh} (copied to clipboard) \`{}\`{}\`{}

Paste that \mbox{\hyperlink{namespace_u_r_l}{U\+RL}} into your browser when the build is complete, and you will see your deployed app.
\end{DoxyEnumerate}

Details are available in \href{https://zeit.co/blog/unlimited-static}{\tt this article.}

\subsubsection*{S3 and Cloud\+Front}

See this \href{https://medium.com/@omgwtfmarc/deploying-create-react-app-to-s3-or-cloudfront-48dae4ce0af}{\tt blog post} on how to deploy your React app to Amazon Web Services \href{https://aws.amazon.com/s3}{\tt S3} and \href{https://aws.amazon.com/cloudfront/}{\tt Cloud\+Front}.

\subsubsection*{Surge}

Install the Surge C\+LI if you haven’t already by running {\ttfamily npm install -\/g surge}. Run the {\ttfamily surge} command and log in you or create a new account.

When asked about the project path, make sure to specify the {\ttfamily build} folder, for example\+:


\begin{DoxyCode}
project path: /path/to/project/build
\end{DoxyCode}


Note that in order to support routers that use H\+T\+M\+L5 {\ttfamily push\+State} A\+PI, you may want to rename the {\ttfamily index.\+html} in your build folder to {\ttfamily 200.\+html} before deploying to Surge. This \href{https://surge.sh/help/adding-a-200-page-for-client-side-routing}{\tt ensures that every U\+RL falls back to that file}.

\subsection*{Advanced Configuration}

You can adjust various development and production settings by setting environment variables in your shell or with \href{#adding-development-environment-variables-in-env}{\tt .env}.

\tabulinesep=1mm
\begin{longtabu} spread 0pt [c]{*{4}{|X[-1]}|}
\hline
\rowcolor{\tableheadbgcolor}\textbf{ Variable  }&\textbf{ Development  }&\multicolumn{2}{p{(\linewidth-\tabcolsep*4-\arrayrulewidth*3)*2/4}|}{\cellcolor{\tableheadbgcolor}\textbf{ Pr   }}\\\cline{1-4}
\endfirsthead
\hline
\endfoot
\hline
\rowcolor{\tableheadbgcolor}\textbf{ Variable  }&\textbf{ Development  }&\multicolumn{2}{p{(\linewidth-\tabcolsep*4-\arrayrulewidth*3)*2/4}|}{\cellcolor{\tableheadbgcolor}\textbf{ Pr   }}\\\cline{1-4}
\endhead
B\+R\+O\+W\+S\+ER  &\+:white\+\_\+check\+\_\+mark\+:  &\+:x\+:  &By default, Create React App will open the default system browser, favoring Chrome on mac\+OS. Specify a \href{https://github.com/sindresorhus/opn#app}{\tt browser} to override this behavior, or set it to {\ttfamily none} to disable it completely. If you need to customize the way the browser is launched, you can specify a node script instead. Any arguments passed to {\ttfamily npm start} will also be passed to this script, and the url where your app is served will be the last argument. Your script\textquotesingle{}s file name must have the {\ttfamily .js} extension.   \\\cline{1-4}
H\+O\+ST  &\+:white\+\_\+check\+\_\+mark\+:  &\+:x\+:  &By default, the development web server binds to {\ttfamily localhost}. You may use this variable to specify a different host.   \\\cline{1-4}
P\+O\+RT  &\+:white\+\_\+check\+\_\+mark\+:  &\+:x\+:  &By default, the development web server will attempt to listen on port 3000 or prompt you to attempt the next available port. You may use this variable to specify a different port.   \\\cline{1-4}
H\+T\+T\+PS  &\+:white\+\_\+check\+\_\+mark\+:  &\+:x\+:  &When set to {\ttfamily true}, Create React App will run the development server in {\ttfamily https} mode.   \\\cline{1-4}
P\+U\+B\+L\+I\+C\+\_\+\+U\+RL  &\+:x\+:  &\+:white\+\_\+check\+\_\+mark\+:  &Create React App assumes your application is hosted at the serving web server\textquotesingle{}s root or a subpath as specified in \href{#building-for-relative-paths}{\tt {\ttfamily package.\+json} ({\ttfamily homepage})}. Normally, Create React App ignores the hostname. You may use this variable to force assets to be referenced verbatim to the url you provide (hostname included). This may be particularly useful when using a C\+DN to host your application.   \\\cline{1-4}
CI  &\+:large\+\_\+orange\+\_\+diamond\+:  &\+:white\+\_\+check\+\_\+mark\+:  &When set to {\ttfamily true}, Create React App treats warnings as failures in the build. It also makes the test runner non-\/watching. Most C\+Is set this flag by default.   \\\cline{1-4}
R\+E\+A\+C\+T\+\_\+\+E\+D\+I\+T\+OR  &\+:white\+\_\+check\+\_\+mark\+:  &\+:x\+:  &When an app crashes in development, you will see an error overlay with clickable stack trace. When you click on it, Create React App will try to determine the editor you are using based on currently running processes, and open the relevant source file. You can \href{https://github.com/facebookincubator/create-react-app/issues/2636}{\tt send a pull request to detect your editor of choice}. Setting this environment variable overrides the automatic detection. If you do it, make sure your systems \href{https://en.wikipedia.org/wiki/PATH_(variable)}{\tt P\+A\+TH} environment variable points to your editor’s bin folder.   \\\cline{1-4}
\end{longtabu}


\subsection*{Troubleshooting}

\subsubsection*{{\ttfamily npm start} doesn’t detect changes}

When you save a file while {\ttfamily npm start} is running, the browser should refresh with the updated code.~\newline
 If this doesn’t happen, try one of the following workarounds\+:


\begin{DoxyItemize}
\item If your project is in a Dropbox folder, try moving it out.
\item If the watcher doesn’t see a file called {\ttfamily index.\+js} and you’re referencing it by the folder name, you \href{https://github.com/facebookincubator/create-react-app/issues/1164}{\tt need to restart the watcher} due to a Webpack bug.
\item Some editors like Vim and IntelliJ have a “safe write” feature that currently breaks the watcher. You will need to disable it. Follow the instructions in \href{https://webpack.js.org/guides/development/#adjusting-your-text-editor}{\tt “\+Adjusting Your Text Editor”}.
\item If your project path contains parentheses, try moving the project to a path without them. This is caused by a \href{https://github.com/webpack/watchpack/issues/42}{\tt Webpack watcher bug}.
\item On Linux and mac\+OS, you might need to \href{https://webpack.github.io/docs/troubleshooting.html#not-enough-watchers}{\tt tweak system settings} to allow more watchers.
\item If the project runs inside a virtual machine such as (a Vagrant provisioned) Virtual\+Box, create an {\ttfamily .env} file in your project directory if it doesn’t exist, and add {\ttfamily C\+H\+O\+K\+I\+D\+A\+R\+\_\+\+U\+S\+E\+P\+O\+L\+L\+I\+NG=true} to it. This ensures that the next time you run {\ttfamily npm start}, the watcher uses the polling mode, as necessary inside a VM.
\end{DoxyItemize}

If none of these solutions help please leave a comment \href{https://github.com/facebookincubator/create-react-app/issues/659}{\tt in this thread}.

\subsubsection*{{\ttfamily npm test} hangs on mac\+OS Sierra}

If you run {\ttfamily npm test} and the console gets stuck after printing {\ttfamily react-\/scripts test -\/-\/env=jsdom} to the console there might be a problem with your \href{https://facebook.github.io/watchman/}{\tt Watchman} installation as described in \href{https://github.com/facebookincubator/create-react-app/issues/713}{\tt facebookincubator/create-\/react-\/app\#713}.

We recommend deleting {\ttfamily node\+\_\+modules} in your project and running {\ttfamily npm install} (or {\ttfamily yarn} if you use it) first. If it doesn\textquotesingle{}t help, you can try one of the numerous workarounds mentioned in these issues\+:


\begin{DoxyItemize}
\item \href{https://github.com/facebook/jest/issues/1767}{\tt facebook/jest\#1767}
\item \href{https://github.com/facebook/watchman/issues/358}{\tt facebook/watchman\#358}
\item \href{https://github.com/ember-cli/ember-cli/issues/6259}{\tt ember-\/cli/ember-\/cli\#6259}
\end{DoxyItemize}

It is reported that installing Watchman 4.\+7.\+0 or newer fixes the issue. If you use \href{http://brew.sh/}{\tt Homebrew}, you can run these commands to update it\+:


\begin{DoxyCode}
watchman shutdown-server
brew update
brew reinstall watchman
\end{DoxyCode}


You can find \href{https://facebook.github.io/watchman/docs/install.html#build-install}{\tt other installation methods} on the Watchman documentation page.

If this still doesn’t help, try running {\ttfamily launchctl unload -\/F $\sim$/\+Library/\+Launch\+Agents/com.github.\+facebook.\+watchman.\+plist}.

There are also reports that {\itshape uninstalling} Watchman fixes the issue. So if nothing else helps, remove it from your system and try again.

\subsubsection*{{\ttfamily npm run build} exits too early}

It is reported that {\ttfamily npm run build} can fail on machines with limited memory and no swap space, which is common in cloud environments. Even with small projects this command can increase R\+AM usage in your system by hundreds of megabytes, so if you have less than 1 GB of available memory your build is likely to fail with the following message\+:

\begin{quote}
The build failed because the process exited too early. This probably means the system ran out of memory or someone called {\ttfamily kill -\/9} on the process. \end{quote}


If you are completely sure that you didn\textquotesingle{}t terminate the process, consider \href{https://www.digitalocean.com/community/tutorials/how-to-add-swap-on-ubuntu-14-04}{\tt adding some swap space} to the machine you’re building on, or build the project locally.

\subsubsection*{{\ttfamily npm run build} fails on Heroku}

This may be a problem with case sensitive filenames. Please refer to \href{#resolving-heroku-deployment-errors}{\tt this section}.

\subsubsection*{Moment.\+js locales are missing}

If you use a \href{https://momentjs.com/}{\tt Moment.\+js}, you might notice that only the English locale is available by default. This is because the locale files are large, and you probably only need a subset of \href{https://momentjs.com/#multiple-locale-support}{\tt all the locales provided by Moment.\+js}.

To add a specific Moment.\+js locale to your bundle, you need to import it explicitly.~\newline
 For example\+:


\begin{DoxyCode}
import moment from 'moment';
import 'moment/locale/fr';
\end{DoxyCode}


If import multiple locales this way, you can later switch between them by calling {\ttfamily moment.\+locale()} with the locale name\+:


\begin{DoxyCode}
import moment from 'moment';
import 'moment/locale/fr';
import 'moment/locale/es';

// ...

moment.locale('fr');
\end{DoxyCode}


This will only work for locales that have been explicitly imported before.

\subsection*{Something Missing?}

If you have ideas for more “\+How To” recipes that should be on this page, \href{https://github.com/facebookincubator/create-react-app/issues}{\tt let us know} or https\+://github.com/facebookincubator/create-\/react-\/app/edit/master/packages/react-\/scripts/template/\+R\+E\+A\+D\+M\+E.\+md \char`\"{}contribute some!\char`\"{} 