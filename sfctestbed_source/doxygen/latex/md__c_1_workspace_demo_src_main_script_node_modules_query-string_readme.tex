\begin{quote}
Parse and stringify \mbox{\hyperlink{namespace_u_r_l}{U\+RL}} \href{http://en.wikipedia.org/wiki/Query_string}{\tt query strings} \end{quote}




{\bfseries 🔥 Want to strengthen your core Java\+Script skills and master E\+S6?}~\newline
I would personally recommend this awesome \href{https://ES6.io/friend/AWESOME}{\tt E\+S6 course} by Wes Bos. You might also like his \href{https://ReactForBeginners.com/friend/AWESOME}{\tt React course}.





\subsection*{Install}


\begin{DoxyCode}
$ npm install --save query-string
\end{DoxyCode}


\subsection*{Usage}


\begin{DoxyCode}
const queryString = require('query-string');

console.log(location.search);
//=> '?foo=bar'

const parsed = queryString.parse(location.search);
console.log(parsed);
//=> \{foo: 'bar'\}

console.log(location.hash);
//=> '#token=bada55cafe'

const parsedHash = queryString.parse(location.hash);
console.log(parsedHash);
//=> \{token: 'bada55cafe'\}

parsed.foo = 'unicorn';
parsed.ilike = 'pizza';

const stringified = queryString.stringify(parsed);
//=> 'foo=unicorn&ilike=pizza'

location.search = stringified;
// note that `location.search` automatically prepends a question mark
console.log(location.search);
//=> '?foo=unicorn&ilike=pizza'
\end{DoxyCode}


\subsection*{A\+PI}

\subsubsection*{.parse({\itshape string}, {\itshape \mbox{[}options\mbox{]}})}

Parse a query string into an object. Leading {\ttfamily ?} or {\ttfamily \#} are ignored, so you can pass {\ttfamily location.\+search} or {\ttfamily location.\+hash} directly.

The returned object is created with \href{https://developer.mozilla.org/en-US/docs/Web/JavaScript/Reference/Global_Objects/Object/create}{\tt {\ttfamily Object.\+create(null)}} and thus does not have a {\ttfamily prototype}.

\paragraph*{array\+Format}

Type\+: {\ttfamily string}~\newline
 Default\+: `\textquotesingle{}none'\`{}

Supports both {\ttfamily index} for an indexed array representation or {\ttfamily bracket} for a {\itshape bracketed} array representation.


\begin{DoxyItemize}
\item {\ttfamily bracket}\+: stands for parsing correctly arrays with bracket representation on the query string, such as\+:
\end{DoxyItemize}


\begin{DoxyCode}
queryString.parse('foo[]=1&foo[]=2&foo[]=3', \{arrayFormat: 'bracket'\});
//=> foo: [1,2,3]
\end{DoxyCode}



\begin{DoxyItemize}
\item {\ttfamily index}\+: stands for parsing taking the index into account, such as\+:
\end{DoxyItemize}


\begin{DoxyCode}
queryString.parse('foo[0]=1&foo[1]=2&foo[3]=3', \{arrayFormat: 'index'\});
//=> foo: [1,2,3]
\end{DoxyCode}



\begin{DoxyItemize}
\item {\ttfamily none}\+: is the {\bfseries default} option and removes any bracket representation, such as\+:
\end{DoxyItemize}


\begin{DoxyCode}
queryString.parse('foo=1&foo=2&foo=3');
//=> foo: [1,2,3]
\end{DoxyCode}


\subsubsection*{.stringify({\itshape object}, {\itshape \mbox{[}options\mbox{]}})}

Stringify an object into a query string, sorting the keys.

\paragraph*{strict}

Type\+: {\ttfamily boolean}~\newline
 Default\+: {\ttfamily true}

Strictly encode U\+RI components with \href{https://github.com/kevva/strict-uri-encode}{\tt strict-\/uri-\/encode}. It uses \href{https://developer.mozilla.org/en/docs/Web/JavaScript/Reference/Global_Objects/encodeURIComponent}{\tt encode\+U\+R\+I\+Component} if set to false. You probably \href{https://github.com/sindresorhus/query-string/issues/42}{\tt don\textquotesingle{}t care} about this option.

\paragraph*{encode}

Type\+: {\ttfamily boolean}~\newline
 Default\+: {\ttfamily true}

\href{https://developer.mozilla.org/en/docs/Web/JavaScript/Reference/Global_Objects/encodeURIComponent}{\tt U\+RL encode} the keys and values.

\paragraph*{array\+Format}

Type\+: {\ttfamily string}~\newline
 Default\+: `\textquotesingle{}none'\`{}

Supports both {\ttfamily index} for an indexed array representation or {\ttfamily bracket} for a {\itshape bracketed} array representation.


\begin{DoxyItemize}
\item {\ttfamily bracket}\+: stands for parsing correctly arrays with bracket representation on the query string, such as\+:
\end{DoxyItemize}


\begin{DoxyCode}
queryString.stringify(\{foo: [1,2,3]\}, \{arrayFormat: 'bracket'\});
// => foo[]=1&foo[]=2&foo[]=3
\end{DoxyCode}



\begin{DoxyItemize}
\item {\ttfamily index}\+: stands for parsing taking the index into account, such as\+:
\end{DoxyItemize}


\begin{DoxyCode}
queryString.stringify(\{foo: [1,2,3]\}, \{arrayFormat: 'index'\});
// => foo[0]=1&foo[1]=2&foo[3]=3
\end{DoxyCode}



\begin{DoxyItemize}
\item {\ttfamily none}\+: is the {\bfseries default} option and removes any bracket representation, such as\+:
\end{DoxyItemize}


\begin{DoxyCode}
queryString.stringify(\{foo: [1,2,3]\});
// => foo=1&foo=2&foo=3
\end{DoxyCode}


\subsubsection*{.extract({\itshape string})}

Extract a query string from a \mbox{\hyperlink{namespace_u_r_l}{U\+RL}} that can be passed into {\ttfamily .parse()}.

\subsection*{Nesting}

This module intentionally doesn\textquotesingle{}t support nesting as it\textquotesingle{}s not spec\textquotesingle{}d and varies between implementations, which causes a lot of \href{https://github.com/visionmedia/node-querystring/issues}{\tt edge cases}.

You\textquotesingle{}re much better off just converting the object to a J\+S\+ON string\+:


\begin{DoxyCode}
queryString.stringify(\{
    foo: 'bar',
    nested: JSON.stringify(\{
        unicorn: 'cake'
    \})
\});
//=> 'foo=bar&nested=%7B%22unicorn%22%3A%22cake%22%7D'
\end{DoxyCode}


However, there is support for multiple instances of the same key\+:


\begin{DoxyCode}
queryString.parse('likes=cake&name=bob&likes=icecream');
//=> \{likes: ['cake', 'icecream'], name: 'bob'\}

queryString.stringify(\{color: ['taupe', 'chartreuse'], id: '515'\});
//=> 'color=chartreuse&color=taupe&id=515'
\end{DoxyCode}


\subsection*{Falsy values}

Sometimes you want to unset a key, or maybe just make it present without assigning a value to it. Here is how falsy values are stringified\+:


\begin{DoxyCode}
queryString.stringify(\{foo: false\});
//=> 'foo=false'

queryString.stringify(\{foo: null\});
//=> 'foo'

queryString.stringify(\{foo: undefined\});
//=> ''
\end{DoxyCode}


\subsection*{License}

M\+IT © \href{https://sindresorhus.com}{\tt Sindre Sorhus} 