\subsection*{get color and style in your node.\+js console}



\subsection*{Installation}

\begin{DoxyVerb}npm install colors
\end{DoxyVerb}


\subsection*{colors and styles!}

\subsubsection*{text colors}


\begin{DoxyItemize}
\item black
\item red
\item green
\item yellow
\item blue
\item magenta
\item cyan
\item white
\item gray
\item grey
\end{DoxyItemize}

\subsubsection*{background colors}


\begin{DoxyItemize}
\item bg\+Black
\item bg\+Red
\item bg\+Green
\item bg\+Yellow
\item bg\+Blue
\item bg\+Magenta
\item bg\+Cyan
\item bg\+White
\end{DoxyItemize}

\subsubsection*{styles}


\begin{DoxyItemize}
\item reset
\item bold
\item dim
\item italic
\item underline
\item inverse
\item hidden
\item strikethrough
\end{DoxyItemize}

\subsubsection*{extras}


\begin{DoxyItemize}
\item rainbow
\item zebra
\item america
\item trap
\item random
\end{DoxyItemize}

\subsection*{Usage}

By popular demand, {\ttfamily colors} now ships with two types of usages!

The super nifty way


\begin{DoxyCode}
var colors = require('colors');

console.log('hello'.green); // outputs green text
console.log('i like cake and pies'.underline.red) // outputs red underlined text
console.log('inverse the color'.inverse); // inverses the color
console.log('OMG Rainbows!'.rainbow); // rainbow
console.log('Run the trap'.trap); // Drops the bass
\end{DoxyCode}


or a slightly less nifty way which doesn\textquotesingle{}t extend {\ttfamily String.\+prototype}


\begin{DoxyCode}
var colors = require('colors/safe');

console.log(colors.green('hello')); // outputs green text
console.log(colors.red.underline('i like cake and pies')) // outputs red underlined text
console.log(colors.inverse('inverse the color')); // inverses the color
console.log(colors.rainbow('OMG Rainbows!')); // rainbow
console.log(colors.trap('Run the trap')); // Drops the bass
\end{DoxyCode}


I prefer the first way. Some people seem to be afraid of extending {\ttfamily String.\+prototype} and prefer the second way.

If you are writing good code you will never have an issue with the first approach. If you really don\textquotesingle{}t want to touch {\ttfamily String.\+prototype}, the second usage will not touch {\ttfamily String} native object.

\subsection*{Disabling Colors}

To disable colors you can pass the following arguments in the command line to your application\+:


\begin{DoxyCode}
node myapp.js --no-color
\end{DoxyCode}


\subsection*{Console.\+log \href{http://nodejs.org/docs/latest/api/console.html#console_console_log_data}{\tt string substitution}}


\begin{DoxyCode}
var name = 'Marak';
console.log(colors.green('Hello %s'), name);
// outputs -> 'Hello Marak'
\end{DoxyCode}


\subsection*{Custom themes}

\subsubsection*{Using standard A\+PI}


\begin{DoxyCode}
var colors = require('colors');

colors.setTheme(\{
  silly: 'rainbow',
  input: 'grey',
  verbose: 'cyan',
  prompt: 'grey',
  info: 'green',
  data: 'grey',
  help: 'cyan',
  warn: 'yellow',
  debug: 'blue',
  error: 'red'
\});

// outputs red text
console.log("this is an error".error);

// outputs yellow text
console.log("this is a warning".warn);
\end{DoxyCode}


\subsubsection*{Using string safe A\+PI}


\begin{DoxyCode}
var colors = require('colors/safe');

// set single property
var error = colors.red;
error('this is red');

// set theme
colors.setTheme(\{
  silly: 'rainbow',
  input: 'grey',
  verbose: 'cyan',
  prompt: 'grey',
  info: 'green',
  data: 'grey',
  help: 'cyan',
  warn: 'yellow',
  debug: 'blue',
  error: 'red'
\});

// outputs red text
console.log(colors.error("this is an error"));

// outputs yellow text
console.log(colors.warn("this is a warning"));
\end{DoxyCode}


You can also combine them\+:


\begin{DoxyCode}
var colors = require('colors');

colors.setTheme(\{
  custom: ['red', 'underline']
\});

console.log('test'.custom);
\end{DoxyCode}


{\itshape Protip\+: There is a secret undocumented style in {\ttfamily colors}. If you find the style you can summon him.} 