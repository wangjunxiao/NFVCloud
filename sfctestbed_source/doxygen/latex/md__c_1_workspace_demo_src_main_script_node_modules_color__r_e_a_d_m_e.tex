\begin{quote}
Java\+Script library for color conversion and manipulation with support for C\+SS color strings. \end{quote}



\begin{DoxyCode}
var color = Color("#7743CE");

color.alpha(0.5).lighten(0.5);

console.log(color.hslString());  // "hsla(262, 59%, 81%, 0.5)"
\end{DoxyCode}


\subsection*{Install}


\begin{DoxyCode}
$ npm install color
\end{DoxyCode}


\subsection*{Usage}


\begin{DoxyCode}
var Color = require("color")
\end{DoxyCode}


\subsubsection*{Setters}


\begin{DoxyCode}
var color = Color("rgb(255, 255, 255)")
var color = Color(\{r: 255, g: 255, b: 255\})
var color = Color().rgb(255, 255, 255)
var color = Color().rgb([255, 255, 255])
\end{DoxyCode}
 Pass any valid C\+SS color string into {\ttfamily Color()} or a hash of values. Also load in color values with {\ttfamily rgb()}, {\ttfamily hsl()}, {\ttfamily hsv()}, {\ttfamily hwb()}, and {\ttfamily cmyk()}.


\begin{DoxyCode}
color.red(120)
\end{DoxyCode}
 Set the values for individual channels with {\ttfamily alpha}, {\ttfamily red}, {\ttfamily green}, {\ttfamily blue}, {\ttfamily hue}, {\ttfamily saturation} (hsl), {\ttfamily saturationv} (hsv), {\ttfamily lightness}, {\ttfamily whiteness}, {\ttfamily blackness}, {\ttfamily cyan}, {\ttfamily magenta}, {\ttfamily yellow}, {\ttfamily black}

\subsubsection*{Getters}


\begin{DoxyCode}
color.rgb()       // \{r: 255, g: 255, b: 255\}
\end{DoxyCode}
 Get a hash of the rgb values with {\ttfamily rgb()}, similarly for {\ttfamily hsl()}, {\ttfamily hsv()}, and {\ttfamily cmyk()}


\begin{DoxyCode}
color.rgbArray()  // [255, 255, 255]
\end{DoxyCode}
 Get an array of the values with {\ttfamily rgb\+Array()}, {\ttfamily hsl\+Array()}, {\ttfamily hsv\+Array()}, and {\ttfamily cmyk\+Array()}.


\begin{DoxyCode}
color.red()       // 255
\end{DoxyCode}
 Get the value for an individual channel.

\subsubsection*{C\+SS Strings}


\begin{DoxyCode}
color.hslString()  // "hsl(320, 50%, 100%)"
\end{DoxyCode}


Different C\+SS String formats for the color are on {\ttfamily hex\+String}, {\ttfamily rgb\+String}, {\ttfamily percent\+String}, {\ttfamily hsl\+String}, {\ttfamily hwb\+String}, and {\ttfamily keyword} (undefined if it\textquotesingle{}s not a keyword color). {\ttfamily \char`\"{}rgba\char`\"{}} and {\ttfamily \char`\"{}hsla\char`\"{}} are used if the current alpha value of the color isn\textquotesingle{}t {\ttfamily 1}.

\subsubsection*{Luminosity}


\begin{DoxyCode}
color.luminosity();  // 0.412
\end{DoxyCode}
 The \href{http://www.w3.org/TR/WCAG20/#relativeluminancedef}{\tt W\+C\+AG luminosity} of the color. 0 is black, 1 is white.


\begin{DoxyCode}
color.contrast(Color("blue"))  // 12
\end{DoxyCode}
 The \href{http://www.w3.org/TR/WCAG20/#contrast-ratiodef}{\tt W\+C\+AG contrast ratio} to another color, from 1 (same color) to 21 (contrast b/w white and black).


\begin{DoxyCode}
color.light();  // true
color.dark();   // false
\end{DoxyCode}
 Get whether the color is \char`\"{}light\char`\"{} or \char`\"{}dark\char`\"{}, useful for deciding text color.

\subsubsection*{Manipulation}


\begin{DoxyCode}
color.negate()         // rgb(0, 100, 255) -> rgb(255, 155, 0)

color.lighten(0.5)     // hsl(100, 50%, 50%) -> hsl(100, 50%, 75%)
color.darken(0.5)      // hsl(100, 50%, 50%) -> hsl(100, 50%, 25%)

color.saturate(0.5)    // hsl(100, 50%, 50%) -> hsl(100, 75%, 50%)
color.desaturate(0.5)  // hsl(100, 50%, 50%) -> hsl(100, 25%, 50%)
color.greyscale()      // #5CBF54 -> #969696

color.whiten(0.5)      // hwb(100, 50%, 50%) -> hwb(100, 75%, 50%)
color.blacken(0.5)     // hwb(100, 50%, 50%) -> hwb(100, 50%, 75%)

color.clearer(0.5)     // rgba(10, 10, 10, 0.8) -> rgba(10, 10, 10, 0.4)
color.opaquer(0.5)     // rgba(10, 10, 10, 0.8) -> rgba(10, 10, 10, 1.0)

color.rotate(180)      // hsl(60, 20%, 20%) -> hsl(240, 20%, 20%)
color.rotate(-90)      // hsl(60, 20%, 20%) -> hsl(330, 20%, 20%)

color.mix(Color("yellow"))        // cyan -> rgb(128, 255, 128)
color.mix(Color("yellow"), 0.3)   // cyan -> rgb(77, 255, 179)

// chaining
color.green(100).greyscale().lighten(0.6)
\end{DoxyCode}


\subsubsection*{Clone}

You can can create a copy of an existing color object using {\ttfamily clone()}\+:


\begin{DoxyCode}
color.clone() // -> New color object
\end{DoxyCode}


And more to come...

\subsection*{Propers}

The A\+PI was inspired by \href{https://github.com/brehaut/color-js}{\tt color-\/js}. Manipulation functions by C\+SS tools like Sass, L\+E\+SS, and Stylus. 