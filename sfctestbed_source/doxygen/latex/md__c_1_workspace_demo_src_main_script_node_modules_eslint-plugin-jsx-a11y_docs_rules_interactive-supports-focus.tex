Elements with an interactive role and interaction handlers (mouse or key press) must be focusable.

\subsection*{How do I resolve this error?}

\subsubsection*{Case\+: This element is a stand-\/alone control like a button, a link or a form element}

Add the {\ttfamily tab\+Index} property to your component. A value of zero indicates that this element can be tabbed to.


\begin{DoxyCode}
<div
  role="button"
  tabIndex=\{0\} />
\end{DoxyCode}


-- or --

Replace the component with one that renders semantic html element like {\ttfamily $<$button$>$}, {\ttfamily $<$a href$>$} or {\ttfamily $<$input$>$} -- whichever fits your purpose.

Generally buttons, links and form elements should be reachable via tab key presses. An element that can be tabbed to is said to be in the {\itshape tab ring}.

\subsubsection*{Case\+: This element is part of a group of buttons, links, menu items, etc}

One item in a group should have a tabindex of zero, the rest should have a tabindex of -\/1. A value of zero makes the element {\itshape tabbable}. A value of -\/1 makes the element {\itshape focusable}.


\begin{DoxyCode}
<div role="menu">
  <div role="menuitem" tabIndex="0">Open</div>
  <div role="menuitem" tabIndex="-1">Save</div>
  <div role="menuitem" tabIndex="-1">Close</div>
</div>
\end{DoxyCode}


In the example above, the first item in the group can be tabbed to. The developer provides the ability to traverse to the subsequent items via the up/down/left/right arrow keys. Traversing via arrow keys is not provided by the browser or the assistive technology. See \href{https://www.w3.org/TR/wai-aria-practices-1.1/#kbd_generalnav}{\tt Fundamental Keyboard Navigation Conventions} for information about established traversal behaviors for various UI widgets.

\subsubsection*{Case\+: This element is not a button, link, menuitem, etc. It is catching bubbled events from elements that it contains}

If your element is catching bubbled click or key events from descendant elements, then the proper role for this element is {\ttfamily presentation}.


\begin{DoxyCode}
<div
  onClick=\{onClickHandler\}
  role="presentation">
  <button>Save</button>
</div>
\end{DoxyCode}


Marking an element with the role {\ttfamily presentation} indicates to assistive technology that this element should be ignored; it exists to support the web application and is not meant for humans to interact with directly.

\subsubsection*{References}


\begin{DoxyEnumerate}
\item \href{https://github.com/GoogleChrome/accessibility-developer-tools/wiki/Audit-Rules#ax_focus_02}{\tt A\+X\+\_\+\+F\+O\+C\+U\+S\+\_\+02}
\end{DoxyEnumerate}
\begin{DoxyEnumerate}
\item \href{https://developer.mozilla.org/en-US/docs/Web/Accessibility/ARIA/ARIA_Techniques/Using_the_button_role#Keyboard_and_focus}{\tt Mozilla Developer Network -\/ A\+R\+IA Techniques}
\end{DoxyEnumerate}
\begin{DoxyEnumerate}
\item \href{https://www.w3.org/TR/wai-aria-practices-1.1/#kbd_generalnav}{\tt Fundamental Keyboard Navigation Conventions}
\end{DoxyEnumerate}
\begin{DoxyEnumerate}
\item \href{https://www.w3.org/TR/wai-aria-practices-1.1/#aria_ex}{\tt W\+A\+I-\/\+A\+R\+IA Authoring Practices Guide -\/ Design Patterns and Widgets}
\end{DoxyEnumerate}

\subsection*{Rule details}

This rule takes no arguments.

\#\#\# Succeed 
\begin{DoxyCode}
<div aria-hidden onClick=\{() => void 0\} />

<span onClick="doSomething();" tabIndex="0" role="button">Click me!</span>

<span onClick="doSomething();" tabIndex="-1" role="menuitem">Click me too!</span>

<a href="javascript:void(0);" onClick="doSomething();">Click ALL the things!</a>

<button onClick="doSomething();">Click the button :)</button>
\end{DoxyCode}


\subsubsection*{Fail}


\begin{DoxyCode}
<span onclick="submitForm();" role="button">Submit</span>

<a onclick="showNextPage();" role="button">Next page</a>
\end{DoxyCode}
 