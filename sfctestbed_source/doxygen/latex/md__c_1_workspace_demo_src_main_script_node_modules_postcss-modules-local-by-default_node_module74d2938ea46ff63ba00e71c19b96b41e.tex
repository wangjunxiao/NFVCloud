A Post\+C\+SS plugin is a function that receives and, usually, transforms a C\+SS A\+ST from the Post\+C\+SS parser.

The rules below are {\itshape mandatory} for all Post\+C\+SS plugins.

See also \href{http://blog.clojurewerkz.org/blog/2013/04/20/how-to-make-your-open-source-project-really-awesome/}{\tt Clojure\+Werkz’s recommendations} for open source projects.

\subsection*{1. A\+PI}

\subsubsection*{1.\+1 Clear name with {\ttfamily postcss-\/} prefix}

The plugin’s purpose should be clear just by reading its name. If you wrote a transpiler for C\+SS 4 Custom Media, {\ttfamily postcss-\/custom-\/media} would be a good name. If you wrote a plugin to support mixins, {\ttfamily postcss-\/mixins} would be a good name.

The prefix {\ttfamily postcss-\/} shows that the plugin is part of the Post\+C\+SS ecosystem.

This rule is not mandatory for plugins that can run as independent tools, without the user necessarily knowing that it is powered by Post\+C\+SS — for example, \href{http://cssnext.io/}{\tt cssnext} and \href{https://github.com/postcss/autoprefixer}{\tt Autoprefixer}.

\subsubsection*{1.\+2. Do one thing, and do it well}

Do not create multitool plugins. Several small, one-\/purpose plugins bundled into a plugin pack is usually a better solution.

For example, \href{http://cssnext.io/}{\tt cssnext} contains many small plugins, one for each W3C specification. And \href{https://github.com/ben-eb/cssnano}{\tt cssnano} contains a separate plugin for each of its optimization.

\subsubsection*{1.\+3. Do not use mixins}

Preprocessors libraries like Compass provide an A\+PI with mixins.

Post\+C\+SS plugins are different. A plugin cannot be just a set of mixins for \href{https://github.com/postcss/postcss-mixins}{\tt postcss-\/mixins}.

To achieve your goal, consider transforming valid C\+SS or using custom at-\/rules and custom properties.

\subsubsection*{1.\+4. Create plugin by {\ttfamily postcss.\+plugin}}

By wrapping your function in this method, you are hooking into a common plugin A\+PI\+:


\begin{DoxyCode}
module.exports = postcss.plugin('plugin-name', function (opts) \{
    return function (root, result) \{
        // Plugin code
    \};
\});
\end{DoxyCode}


\subsection*{2. Processing}

\subsubsection*{2.\+1. Plugin must be tested}

A CI service like \href{https://travis-ci.org/}{\tt Travis} is also recommended for testing code in different environments. You should test in (at least) Node.\+js \href{https://github.com/nodejs/LTS}{\tt active L\+TS} and current stable version.

\subsubsection*{2.\+2. Use asynchronous methods whenever possible}

For example, use {\ttfamily fs.\+write\+File} instead of {\ttfamily fs.\+write\+File\+Sync}\+:


\begin{DoxyCode}
postcss.plugin('plugin-sprite', function (opts) \{
    return function (root, result) \{

        return new Promise(function (resolve, reject) \{
            var sprite = makeSprite();
            fs.writeFile(opts.file, function (err) \{
                if ( err ) return reject(err);
                resolve();
            \})
        \});

    \};
\});
\end{DoxyCode}


\subsubsection*{2.\+3. Set {\ttfamily node.\+source} for new nodes}

Every node must have a relevant {\ttfamily source} so Post\+C\+SS can generate an accurate source map.

So if you add new declaration based on some existing declaration, you should clone the existing declaration in order to save that original {\ttfamily source}.


\begin{DoxyCode}
if ( needPrefix(decl.prop) ) \{
    decl.cloneBefore(\{ prop: '-webkit-' + decl.prop \});
\}
\end{DoxyCode}


You can also set {\ttfamily source} directly, copying from some existing node\+:


\begin{DoxyCode}
if ( decl.prop === 'animation' ) \{
    var keyframe = createAnimationByName(decl.value);
    keyframes.source = decl.source;
    decl.root().append(keyframes);
\}
\end{DoxyCode}


\subsubsection*{2.\+4. Use only the public Post\+C\+SS A\+PI}

Post\+C\+SS plugins must not rely on undocumented properties or methods, which may be subject to change in any minor release. The public A\+PI is described in \href{http://api.postcss.org/}{\tt A\+PI docs}.

\subsection*{3. Errors}

\subsubsection*{3.\+1. Use {\ttfamily node.\+error} on C\+SS relevant errors}

If you have an error because of input C\+SS (like an unknown name in a mixin plugin) you should use {\ttfamily node.\+error} to create an error that includes source position\+:


\begin{DoxyCode}
if ( typeof mixins[name] === 'undefined' ) \{
    throw decl.error('Unknown mixin ' + name, \{ plugin: 'postcss-mixins' \});
\}
\end{DoxyCode}


\subsubsection*{3.\+2. Use {\ttfamily result.\+warn} for warnings}

Do not print warnings with {\ttfamily console.\+log} or {\ttfamily console.\+warn}, because some Post\+C\+SS runner may not allow console output.


\begin{DoxyCode}
if ( outdated(decl.prop) ) \{
    result.warn(decl.prop + ' is outdated', \{ node: decl \});
\}
\end{DoxyCode}


If C\+SS input is a source of the warning, the plugin must set the {\ttfamily node} option.

\subsection*{4. Documentation}

\subsubsection*{4.\+1. Document your plugin in English}

Post\+C\+SS plugins must have their {\ttfamily R\+E\+A\+D\+M\+E.\+md} written in English. Do not be afraid of your English skills, as the open source community will fix your errors.

Of course, you are welcome to write documentation in other languages; just name them appropriately (e.\+g.\+ $<$tt$>$R\+E\+A\+D\+M\+E.\+ja.\+md).

\subsubsection*{4.\+2. Include input and output examples}

The plugin\textquotesingle{}s {\ttfamily R\+E\+A\+D\+M\+E.\+md} must contain example input and output C\+SS. A clear example is the best way to describe how your plugin works.

The first section of the {\ttfamily R\+E\+A\+D\+M\+E.\+md} is a good place to put examples. See \href{https://github.com/iamvdo/postcss-opacity}{\tt postcss-\/opacity} for an example.

Of course, this guideline does not apply if your plugin does not transform the C\+SS.

\subsubsection*{4.\+3. Maintain a changelog}

Post\+C\+SS plugins must describe the changes of all their releases in a separate file, such as {\ttfamily C\+H\+A\+N\+G\+E\+L\+O\+G.\+md}, {\ttfamily History.\+md}, or \href{https://help.github.com/articles/creating-releases/}{\tt Git\+Hub Releases}. Visit \href{http://keepachangelog.com/}{\tt Keep A Changelog} for more information about how to write one of these.

Of course, you should be using \href{http://semver.org/}{\tt Sem\+Ver}.

\subsubsection*{4.\+4. Include {\ttfamily postcss-\/plugin} keyword in {\ttfamily package.\+json}}

Post\+C\+SS plugins written for npm must have the {\ttfamily postcss-\/plugin} keyword in their {\ttfamily package.\+json}. This special keyword will be useful for feedback about the Post\+C\+SS ecosystem.

For packages not published to npm, this is not mandatory, but is recommended if the package format can contain keywords. 