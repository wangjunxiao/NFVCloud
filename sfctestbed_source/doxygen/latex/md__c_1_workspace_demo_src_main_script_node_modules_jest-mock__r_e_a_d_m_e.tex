\subsection*{A\+PI}

\subsubsection*{{\ttfamily constructor(global)}}

Creates a new module mocker that generates mocks as if they were created in an environment with the given global object.

\subsubsection*{{\ttfamily generate\+From\+Metadata(metadata)}}

Generates a mock based on the given metadata (Metadata for the mock in the schema returned by the get\+Metadata method of this module). Mocks treat functions specially, and all mock functions have additional members, described in the documentation for {\ttfamily fn} in this module.

One important note\+: function prototypes are handled specially by this mocking framework. For functions with prototypes, when called as a constructor, the mock will install mocked function members on the instance. This allows different instances of the same constructor to have different values for its mocks member and its return values.

\subsubsection*{{\ttfamily get\+Metadata(component)}}

Inspects the argument and returns its schema in the following recursive format\+:


\begin{DoxyCode}
\{
  type: ...
  members : \{\}
\}
\end{DoxyCode}


Where type is one of {\ttfamily array}, {\ttfamily object}, {\ttfamily function}, or {\ttfamily ref}, and members is an optional dictionary where the keys are member names and the values are metadata objects. Function prototypes are defined simply by defining metadata for the {\ttfamily member.\+prototype} of the function. The type of a function prototype should always be {\ttfamily object}. For instance, a simple class might be defined like this\+:


\begin{DoxyCode}
\{
  type: 'function',
  members: \{
    staticMethod: \{type: 'function'\},
    prototype: \{
      type: 'object',
      members: \{
        instanceMethod: \{type: 'function'\}
      \}
    \}
  \}
\}
\end{DoxyCode}


Metadata may also contain references to other objects defined within the same metadata object. The metadata for the referent must be marked with {\ttfamily ref\+ID} key and an arbitrary value. The referrer must be marked with a {\ttfamily ref} key that has the same value as object with ref\+ID that it refers to. For instance, this metadata blob\+:


\begin{DoxyCode}
\{
  type: 'object',
  refID: 1,
  members: \{
    self: \{ref: 1\}
  \}
\}
\end{DoxyCode}


defines an object with a slot named {\ttfamily self} that refers back to the object.

\subsubsection*{{\ttfamily fn}}

Generates a stand-\/alone function with members that help drive unit tests or confirm expectations. Specifically, functions returned by this method have the following members\+:

\subparagraph*{{\ttfamily .mock}}

An object with two members, {\ttfamily calls}, and {\ttfamily instances}, which are both lists. The items in the {\ttfamily calls} list are the arguments with which the function was called. The \char`\"{}instances\char`\"{} list stores the value of \textquotesingle{}this\textquotesingle{} for each call to the function. This is useful for retrieving instances from a constructor.

\subparagraph*{{\ttfamily .mock\+Return\+Value\+Once(value)}}

Pushes the given value onto a F\+I\+FO queue of return values for the function.

\subparagraph*{{\ttfamily .mock\+Return\+Value(value)}}

Sets the default return value for the function.

\subparagraph*{{\ttfamily .mock\+Implementation\+Once(function)}}

Pushes the given mock implementation onto a F\+I\+FO queue of mock implementations for the function.

\subparagraph*{{\ttfamily .mock\+Implementation(function)}}

Sets the default mock implementation for the function.

\subparagraph*{{\ttfamily .mock\+Return\+This()}}

Syntactic sugar for .mock\+Implementation(function() \{return this;\})

In case both {\ttfamily mock\+Implementation\+Once()/mock\+Implementation()} and {\ttfamily mock\+Return\+Value\+Once()/mock\+Return\+Value()} are called. The priority of which to use is based on what is the last call\+:
\begin{DoxyItemize}
\item if the last call is mock\+Return\+Value\+Once() or mock\+Return\+Value(), use the specific return value or default return value. If specific return values are used up or no default return value is set, fall back to try mock\+Implementation();
\item if the last call is mock\+Implementation\+Once() or mock\+Implementation(), run the specific implementation and return the result or run default implementation and return the result. 
\end{DoxyItemize}