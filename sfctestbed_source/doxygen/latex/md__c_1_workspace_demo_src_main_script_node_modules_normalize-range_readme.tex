Utility for normalizing a numeric range, with a wrapping function useful for polar coordinates.

\href{https://travis-ci.org/jamestalmage/normalize-range}{\tt } \href{https://coveralls.io/github/jamestalmage/normalize-range?branch=master}{\tt } \href{https://codeclimate.com/github/jamestalmage/normalize-range}{\tt } \href{https://david-dm.org/jamestalmage/normalize-range}{\tt } \href{https://david-dm.org/jamestalmage/normalize-range#info=devDependencies}{\tt }

\href{https://nodei.co/npm/normalize-range/}{\tt }

\subsection*{Usage}


\begin{DoxyCode}
var nr = require('normalize-range');

nr.wrap(0, 360, 400);
//=> 40

nr.wrap(0, 360, -90);
//=> 270

nr.limit(0, 100, 500);
//=> 100

nr.limit(0, 100, -20);
//=> 0

// There is a convenient currying function
var wrapAngle = nr.curry(0, 360).wrap;
var limitTo10 = nr.curry(0, 10).limit;

wrapAngle(-30);
//=> 330
\end{DoxyCode}
 \subsection*{A\+PI}

\subsubsection*{wrap(min, max, value)}

Normalizes a values that \char`\"{}wraps around\char`\"{}. For example, in a polar coordinate system, 270˚ can also be represented as -\/90˚. For wrapping purposes we assume {\ttfamily max} is functionally equivalent to {\ttfamily min}, and that {\ttfamily wrap(max + 1) === wrap(min + 1)}. Wrap always assumes that {\ttfamily min} is {\itshape inclusive}, and {\ttfamily max} is {\itshape exclusive}. In other words, if {\ttfamily value === max} the function will wrap it, and return {\ttfamily min}, but {\ttfamily min} will not be wrapped.


\begin{DoxyCode}
nr.wrap(0, 360, 0) === 0;
nr.wrap(0, 360, 360) === 0;
nr.wrap(0, 360, 361) === 1;
nr.wrap(0, 360, -1) === 359;
\end{DoxyCode}


You are not restricted to whole numbers, and ranges can be negative.


\begin{DoxyCode}
var π = Math.PI;
var radianRange = nr.curry(-π, π);

redianRange.wrap(0) === 0;
nr.wrap(π) === -π;
nr.wrap(4 * π / 3) === -2 * π / 3;
\end{DoxyCode}


\subsubsection*{limit(min, max, value)}

Normalize the value by bringing it within the range. If {\ttfamily value} is greater than {\ttfamily max}, {\ttfamily max} will be returned. If {\ttfamily value} is less than {\ttfamily min}, {\ttfamily min} will be returned. Otherwise, {\ttfamily value} is returned unaltered. Both ends of this range are {\itshape inclusive}.

\subsubsection*{test(min, max, value, \mbox{[}min\+Exclusive\mbox{]}, \mbox{[}max\+Exclusive\mbox{]})}

Returns {\ttfamily true} if {\ttfamily value} is within the range, {\ttfamily false} otherwise. It defaults to {\ttfamily inclusive} on both ends of the range, but that can be changed by setting {\ttfamily min\+Exclusive} and/or {\ttfamily max\+Exclusive} to a truthy value.

\subsubsection*{validate(min, max, value, \mbox{[}min\+Exclusive\mbox{]}, \mbox{[}max\+Exclusive\mbox{]})}

Returns {\ttfamily value} or throws an error if {\ttfamily value} is outside the specified range.

\subsubsection*{name(min, max, value, \mbox{[}min\+Exclusive\mbox{]}, \mbox{[}max\+Exclusive\mbox{]})}

Returns a string representing this range in \href{https://en.wikipedia.org/wiki/Interval_(mathematics)#Classification_of_intervals}{\tt range notation}.

\subsubsection*{curry(min, max, \mbox{[}min\+Exclusive\mbox{]}, \mbox{[}max\+Exclusive\mbox{]})}

Convenience method for currying all method arguments except {\ttfamily value}.


\begin{DoxyCode}
var angle = require('normalize-range').curry(-180, 180, false, true);

angle.wrap(270)
//=> -90

angle.limit(200)
//=> 180

angle.test(0)
//=> true

angle.validate(300)
//=> throws an Error

angle.toString() // or angle.name()
//=> "[-180,180)"
\end{DoxyCode}


\paragraph*{min}

{\itshape Required} ~\newline
Type\+: {\ttfamily number}

The minimum value (inclusive) of the range.

\paragraph*{max}

{\itshape Required} ~\newline
Type\+: {\ttfamily number}

The maximum value (exclusive) of the range.

\paragraph*{value}

{\itshape Required} ~\newline
Type\+: {\ttfamily number}

The value to be normalized.

\paragraph*{returns}

Type\+: {\ttfamily number}

The normalized value.

\subsection*{Building and Releasing}


\begin{DoxyItemize}
\item {\ttfamily npm test}\+: tests, linting, coverage and style checks.
\item {\ttfamily npm run watch}\+: autotest mode for active development.
\item {\ttfamily npm run debug}\+: run tests without coverage (istanbul can obscure line \#\textquotesingle{}s)
\end{DoxyItemize}

Release via {\ttfamily cut-\/release} tool.

\subsection*{License}

M\+IT © \href{http://github.com/jamestalmage}{\tt James Talmage} 