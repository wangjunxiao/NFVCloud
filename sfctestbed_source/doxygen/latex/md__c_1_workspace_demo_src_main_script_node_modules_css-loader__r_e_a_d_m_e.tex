\href{https://npmjs.com/package/css-loader}{\tt } \href{https://nodejs.org}{\tt } \href{https://david-dm.org/webpack-contrib/css-loader}{\tt } \href{https://travis-ci.org/webpack-contrib/css-loader}{\tt } \href{https://codecov.io/gh/webpack-contrib/css-loader}{\tt } \href{https://gitter.im/webpack/webpack}{\tt }

  \href{https://github.com/webpack/webpack}{\tt } \section*{C\+SS Loader}

 

\subsection*{Install}


\begin{DoxyCode}
npm install --save-dev css-loader
\end{DoxyCode}


\subsection*{Usage}

The {\ttfamily css-\/loader} interprets {\ttfamily @import} and {\ttfamily url()} like {\ttfamily import/require()} and will resolve them.

Good loaders for requiring your assets are the \href{https://github.com/webpack/file-loader}{\tt file-\/loader} and the \href{https://github.com/webpack/url-loader}{\tt url-\/loader} which you should specify in your config (see \href{https://github.com/michael-ciniawsky/css-loader#assets}{\tt below}).

{\bfseries file.\+js} 
\begin{DoxyCode}
import css from 'file.css';
\end{DoxyCode}


{\bfseries webpack.\+config.\+js} 
\begin{DoxyCode}
module.exports = \{
  module: \{
    rules: [
      \{
        test: /\(\backslash\).css$/,
        use: [ 'style-loader', 'css-loader' ]
      \}
    ]
  \}
\}
\end{DoxyCode}


\subsubsection*{{\ttfamily to\+String}}

You can also use the css-\/loader results directly as string, such as in Angular\textquotesingle{}s component style.

{\bfseries webpack.\+config.\+js} 
\begin{DoxyCode}
\{
   test: /\(\backslash\).css$/,
   use: [
     'to-string-loader',
     'css-loader'
   ]
\}
\end{DoxyCode}


or


\begin{DoxyCode}
const css = require('./test.css').toString();

console.log(css); // \{String\}
\end{DoxyCode}


If there are Source\+Maps, they will also be included in the result string.

If, for one reason or another, you need to extract C\+SS as a plain string resource (i.\+e. not wrapped in a JS module) you might want to check out the \href{https://github.com/peerigon/extract-loader}{\tt extract-\/loader}. It\textquotesingle{}s useful when you, for instance, need to post process the C\+SS as a string.

{\bfseries webpack.\+config.\+js} 
\begin{DoxyCode}
\{
   test: /\(\backslash\).css$/,
   use: [
     'handlebars-loader', // handlebars loader expects raw resource string
     'extract-loader',
     'css-loader'
   ]
\}
\end{DoxyCode}


\subsection*{Options}

\tabulinesep=1mm
\begin{longtabu} spread 0pt [c]{*{4}{|X[-1]}|}
\hline
\rowcolor{\tableheadbgcolor}\textbf{ Name  }&\textbf{ Type  }&\textbf{ Default  }&\textbf{ Description   }\\\cline{1-4}
\endfirsthead
\hline
\endfoot
\hline
\rowcolor{\tableheadbgcolor}\textbf{ Name  }&\textbf{ Type  }&\textbf{ Default  }&\textbf{ Description   }\\\cline{1-4}
\endhead
$\ast$$\ast${\ttfamily root}$\ast$$\ast$  &{\ttfamily \{String\}}  &{\ttfamily /}  &Path to resolve U\+R\+Ls, U\+R\+Ls starting with {\ttfamily /} will not be translated   \\\cline{1-4}
$\ast$$\ast${\ttfamily url}$\ast$$\ast$  &{\ttfamily \{Boolean\}}  &{\ttfamily true}  &Enable/\+Disable {\ttfamily url()} handling   \\\cline{1-4}
$\ast$$\ast${\ttfamily alias}$\ast$$\ast$  &{\ttfamily \{Object\}}  &{\ttfamily \{\}}  &Create aliases to import certain modules more easily   \\\cline{1-4}
$\ast$$\ast${\ttfamily import}$\ast$$\ast$  &{\ttfamily \{Boolean\}}  &{\ttfamily true}  &Enable/\+Disable  handling   \\\cline{1-4}
$\ast$$\ast${\ttfamily modules}$\ast$$\ast$  &{\ttfamily \{Boolean\}}  &{\ttfamily false}  &Enable/\+Disable C\+SS Modules   \\\cline{1-4}
$\ast$$\ast${\ttfamily minimize}$\ast$$\ast$  &{\ttfamily \{Boolean\textbackslash{}$\vert$\+Object\}}  &{\ttfamily false}  &Enable/\+Disable minification   \\\cline{1-4}
$\ast$$\ast${\ttfamily source\+Map}$\ast$$\ast$  &{\ttfamily \{Boolean\}}  &{\ttfamily false}  &Enable/\+Disable Sourcemaps   \\\cline{1-4}
$\ast$$\ast${\ttfamily camel\+Case}$\ast$$\ast$  &{\ttfamily \{Boolean\textbackslash{}$\vert$\+String\}}  &{\ttfamily false}  &Export Classnames in Camel\+Case   \\\cline{1-4}
$\ast$$\ast${\ttfamily import\+Loaders}$\ast$$\ast$  &{\ttfamily \{Number\}}  &{\ttfamily 0}  &Number of loaders applied before C\+SS loader   \\\cline{1-4}
\end{longtabu}


\subsubsection*{{\ttfamily root}}

For U\+R\+Ls that start with a {\ttfamily /}, the default behavior is to not translate them.

{\ttfamily url(/image.png) =$>$ url(/image.png)}

If a {\ttfamily root} query parameter is set, however, it will be prepended to the \mbox{\hyperlink{namespace_u_r_l}{U\+RL}} and then translated.

{\bfseries webpack.\+config.\+js} 
\begin{DoxyCode}
\{
  loader: 'css-loader',
  options: \{ root: '.' \}
\}
\end{DoxyCode}


{\ttfamily url(/image.png)} =$>$ `require('./image.png\textquotesingle{})\`{}

Using \textquotesingle{}Root-\/relative\textquotesingle{} urls is not recommended. You should only use it for legacy C\+SS files.

\subsubsection*{{\ttfamily url}}

To disable {\ttfamily url()} resolving by {\ttfamily css-\/loader} set the option to {\ttfamily false}.

To be compatible with existing css files (if not in C\+SS Module mode).


\begin{DoxyCode}
url(image.png) => require('./image.png')
url(~module/image.png) => require('module/image.png')
\end{DoxyCode}


\subsubsection*{{\ttfamily alias}}

Rewrite your urls with alias, this is useful when it\textquotesingle{}s hard to change url paths of your input files, for example, when you\textquotesingle{}re using some css / sass files in another package (bootstrap, ratchet, font-\/awesome, etc.).

{\ttfamily css-\/loader}\textquotesingle{}s {\ttfamily alias} follows the same syntax as webpack\textquotesingle{}s {\ttfamily resolve.\+alias}, you can see the details at the \href{https://webpack.js.org/configuration/resolve/#resolve-alias}{\tt resolve docs}

{\bfseries file.\+scss} 
\begin{DoxyCode}
@charset "UTF-8";
@import "bootstrap";
\end{DoxyCode}


{\bfseries webpack.\+config.\+js} 
\begin{DoxyCode}
\{
  test: /\(\backslash\).scss$/,
  use: [
    \{
      loader: "style-loader"
    \},
    \{
      loader: "css-loader",
      options: \{
        alias: \{
          "../fonts/bootstrap": "bootstrap-sass/assets/fonts/bootstrap"
        \}
      \}
    \},
    \{
      loader: "sass-loader",
      options: \{
        includePaths: [
          path.resolve("./node\_modules/bootstrap-sass/assets/stylesheets")
        ]
      \}
    \}
  ]
\}
\end{DoxyCode}


Check out this \href{https://github.com/bbtfr/webpack2-bootstrap-sass-sample}{\tt working bootstrap example}.

\subsubsection*{{\ttfamily import}}

To disable {\ttfamily @import} resolving by {\ttfamily css-\/loader} set the option to {\ttfamily false}


\begin{DoxyCode}
@import url('https://fonts.googleapis.com/css?family=Roboto');
\end{DoxyCode}


\begin{quote}
\+\_\+⚠️ Use with caution, since this disables resolving for {\bfseries all} {\ttfamily @import}s, including css modules `composes\+: xxx from \textquotesingle{}path/to/file.\+css'\`{} feature.\+\_\+ \end{quote}


\subsubsection*{\href{https://github.com/css-modules/css-modules}{\tt {\ttfamily modules}}}

The query parameter {\ttfamily modules} enables the {\bfseries C\+SS Modules} spec.

This enables local scoped C\+SS by default. (You can switch it off with {\ttfamily \+:global(...)} or {\ttfamily \+:global} for selectors and/or rules.).

\paragraph*{{\ttfamily Scope}}

By default C\+SS exports all classnames into a global selector scope. Styles can be locally scoped to avoid globally scoping styles.

The syntax {\ttfamily \+:local(.class\+Name)} can be used to declare {\ttfamily class\+Name} in the local scope. The local identifiers are exported by the module.

With {\ttfamily \+:local} (without brackets) local mode can be switched on for this selector. {\ttfamily \+:global(.class\+Name)} can be used to declare an explicit global selector. With {\ttfamily \+:global} (without brackets) global mode can be switched on for this selector.

The loader replaces local selectors with unique identifiers. The choosen unique identifiers are exported by the module.


\begin{DoxyCode}
:local(.className) \{ background: red; \}
:local .className \{ color: green; \}
:local(.className .subClass) \{ color: green; \}
:local .className .subClass :global(.global-class-name) \{ color: blue; \}
\end{DoxyCode}



\begin{DoxyCode}
.\_23\_aKvs-b8bW2Vg3fwHozO \{ background: red; \}
.\_23\_aKvs-b8bW2Vg3fwHozO \{ color: green; \}
.\_23\_aKvs-b8bW2Vg3fwHozO .\_13LGdX8RMStbBE9w-t0gZ1 \{ color: green; \}
.\_23\_aKvs-b8bW2Vg3fwHozO .\_13LGdX8RMStbBE9w-t0gZ1 .global-class-name \{ color: blue; \}
\end{DoxyCode}


\begin{quote}
\+:information\+\_\+source\+: Identifiers are exported \end{quote}



\begin{DoxyCode}
exports.locals = \{
  className: '\_23\_aKvs-b8bW2Vg3fwHozO',
  subClass: '\_13LGdX8RMStbBE9w-t0gZ1'
\}
\end{DoxyCode}


Camel\+Case is recommended for local selectors. They are easier to use in the within the imported JS module.

{\ttfamily url()} U\+R\+Ls in block scoped ({\ttfamily \+:local .abc}) rules behave like requests in modules.


\begin{DoxyCode}
file.png => ./file.png
~module/file.png => module/file.png
\end{DoxyCode}


You can use {\ttfamily \+:local(\#some\+Id)}, but this is not recommended. Use classes instead of ids. You can configure the generated ident with the {\ttfamily local\+Ident\+Name} query parameter (default {\ttfamily \mbox{[}hash\+:base64\mbox{]}}).

{\bfseries webpack.\+config.\+js} 
\begin{DoxyCode}
\{
  test: /\(\backslash\).css$/,
  use: [
    \{
      loader: 'css-loader',
      options: \{
        modules: true,
        localIdentName: '[path][name]\_\_[local]--[hash:base64:5]'
      \}
    \}
  ]
\}
\end{DoxyCode}


You can also specify the absolute path to your custom {\ttfamily get\+Local\+Ident} function to generate classname based on a different schema. This requires {\ttfamily webpack $>$= 2.\+2.\+1} (it supports functions in the {\ttfamily options} object).

{\bfseries webpack.\+config.\+js} 
\begin{DoxyCode}
\{
  loader: 'css-loader',
  options: \{
    modules: true,
    localIdentName: '[path][name]\_\_[local]--[hash:base64:5]',
    getLocalIdent: (context, localIdentName, localName, options) => \{
      return 'whatever\_random\_class\_name'
    \}
  \}
\}
\end{DoxyCode}


\begin{quote}
\+:information\+\_\+source\+: For prerendering with extract-\/text-\/webpack-\/plugin you should use {\ttfamily css-\/loader/locals} instead of {\ttfamily style-\/loader!css-\/loader} {\bfseries in the prerendering bundle}. It doesn\textquotesingle{}t embed C\+SS but only exports the identifier mappings. \end{quote}


\paragraph*{{\ttfamily Composing}}

When declaring a local classname you can compose a local class from another local classname.


\begin{DoxyCode}
:local(.className) \{
  background: red;
  color: yellow;
\}

:local(.subClass) \{
  composes: className;
  background: blue;
\}
\end{DoxyCode}


This doesn\textquotesingle{}t result in any change to the C\+SS itself but exports multiple classnames.


\begin{DoxyCode}
exports.locals = \{
  className: '\_23\_aKvs-b8bW2Vg3fwHozO',
  subClass: '\_13LGdX8RMStbBE9w-t0gZ1 \_23\_aKvs-b8bW2Vg3fwHozO'
\}
\end{DoxyCode}



\begin{DoxyCode}
.\_23\_aKvs-b8bW2Vg3fwHozO \{
  background: red;
  color: yellow;
\}

.\_13LGdX8RMStbBE9w-t0gZ1 \{
  background: blue;
\}
\end{DoxyCode}


\paragraph*{{\ttfamily Importing}}

To import a local classname from another module.


\begin{DoxyCode}
:local(.continueButton) \{
  composes: button from 'library/button.css';
  background: red;
\}
\end{DoxyCode}



\begin{DoxyCode}
:local(.nameEdit) \{
  composes: edit highlight from './edit.css';
  background: red;
\}
\end{DoxyCode}


To import from multiple modules use multiple {\ttfamily composes\+:} rules.


\begin{DoxyCode}
:local(.className) \{
  composes: edit hightlight from './edit.css';
  composes: button from 'module/button.css';
  composes: classFromThisModule;
  background: red;
\}
\end{DoxyCode}


\subsubsection*{{\ttfamily minimize}}

By default the css-\/loader minimizes the css if specified by the module system.

In some cases the minification is destructive to the css, so you can provide your own options to the minifier if needed. cssnano is used for minification and you find a \href{http://cssnano.co/options/}{\tt list of options here}.

You can also disable or enforce minification with the {\ttfamily minimize} query parameter.

{\bfseries webpack.\+config.\+js} 
\begin{DoxyCode}
\{
  loader: 'css-loader',
  options: \{
    minimize: true || \{/* CSSNano Options */\}
  \}
\}
\end{DoxyCode}


\subsubsection*{{\ttfamily source\+Map}}

To include source maps set the {\ttfamily source\+Map} option.

I. e. the extract-\/text-\/webpack-\/plugin can handle them.

They are not enabled by default because they expose a runtime overhead and increase in bundle size (JS source maps do not). In addition to that relative paths are buggy and you need to use an absolute public path which include the server \mbox{\hyperlink{namespace_u_r_l}{U\+RL}}.

{\bfseries webpack.\+config.\+js} 
\begin{DoxyCode}
\{
  loader: 'css-loader',
  options: \{
    sourceMap: true
  \}
\}
\end{DoxyCode}


\subsubsection*{{\ttfamily camel\+Case}}

By default, the exported J\+S\+ON keys mirror the class names. If you want to camelize class names (useful in JS), pass the query parameter {\ttfamily camel\+Case} to css-\/loader.

\tabulinesep=1mm
\begin{longtabu} spread 0pt [c]{*{3}{|X[-1]}|}
\hline
\rowcolor{\tableheadbgcolor}\textbf{ Name  }&\textbf{ Type  }&\textbf{ Description   }\\\cline{1-3}
\endfirsthead
\hline
\endfoot
\hline
\rowcolor{\tableheadbgcolor}\textbf{ Name  }&\textbf{ Type  }&\textbf{ Description   }\\\cline{1-3}
\endhead
$\ast$$\ast${\ttfamily true}$\ast$$\ast$  &{\ttfamily \{Boolean\}}  &Class names will be camelized   \\\cline{1-3}
$\ast$$\ast$`\textquotesingle{}dashes'{\ttfamily $\ast$$\ast$ $<$/td$>$ $<$td class=\char`\"{}markdown\+Table\+Body\+Center\char`\"{}$>$}\{String\}\`{}  &Only dashes in class names will be camelized   \\\cline{1-3}
$\ast$$\ast$`\textquotesingle{}only'{\ttfamily $\ast$$\ast$ $<$/td$>$ $<$td class=\char`\"{}markdown\+Table\+Body\+Center\char`\"{}$>$}\{String\}{\ttfamily $<$/td$>$ $<$td class=\char`\"{}markdown\+Table\+Body\+Left\char`\"{}$>$ Introduced in}0.\+27.\+1\`{}. Class names will be camelized, the original class name will be removed from the locals   \\\cline{1-3}
$\ast$$\ast$`\textquotesingle{}dashes\+Only'{\ttfamily $\ast$$\ast$ $<$/td$>$ $<$td class=\char`\"{}markdown\+Table\+Body\+Center\char`\"{}$>$}\{String\}{\ttfamily $<$/td$>$ $<$td class=\char`\"{}markdown\+Table\+Body\+Left\char`\"{}$>$ Introduced in}0.\+27.\+1\`{}. Dashes in class names will be camelized, the original class name will be removed from the locals   \\\cline{1-3}
\end{longtabu}


{\bfseries file.\+css} 
\begin{DoxyCode}
.class-name \{\}
\end{DoxyCode}


{\bfseries file.\+js} 
\begin{DoxyCode}
import \{ className \} from 'file.css';
\end{DoxyCode}


{\bfseries webpack.\+config.\+js} 
\begin{DoxyCode}
\{
  loader: 'css-loader',
  options: \{
    camelCase: true
  \}
\}
\end{DoxyCode}


\subsubsection*{{\ttfamily import\+Loaders}}

The query parameter {\ttfamily import\+Loaders} allows to configure how many loaders before {\ttfamily css-\/loader} should be applied to {\ttfamily @import}ed resources.

{\bfseries webpack.\+config.\+js} 
\begin{DoxyCode}
\{
  test: /\(\backslash\).css$/,
  use: [
    'style-loader',
    \{
      loader: 'css-loader',
      options: \{
        importLoaders: 1 // 0 => no loaders (default); 1 => postcss-loader; 2 => postcss-loader,
       sass-loader
      \}
    \},
    'postcss-loader',
    'sass-loader'
  ]
\}
\end{DoxyCode}


This may change in the future, when the module system (i. e. webpack) supports loader matching by origin.

\subsection*{Examples}

\subsubsection*{Assets}

The following {\ttfamily webpack.\+config.\+js} can load C\+SS files, embed small P\+N\+G/\+J\+P\+G/\+G\+I\+F/\+S\+VG images as well as fonts as \href{https://tools.ietf.org/html/rfc2397}{\tt Data U\+R\+Ls} and copy larger files to the output directory.

{\bfseries webpack.\+config.\+js} 
\begin{DoxyCode}
module.exports = \{
  module: \{
    rules: [
      \{
        test: /\(\backslash\).css$/,
        use: [ 'style-loader', 'css-loader' ]
      \},
      \{
        test: /\(\backslash\).(png|jpg|gif|svg|eot|ttf|woff|woff2)$/,
        loader: 'url-loader',
        options: \{
          limit: 10000
        \}
      \}
    ]
  \}
\}
\end{DoxyCode}


\subsubsection*{Extract}

For production builds it\textquotesingle{}s recommended to extract the C\+SS from your bundle being able to use parallel loading of C\+S\+S/\+JS resources later on. This can be achieved by using the \href{https://github.com/webpack-contrib/extract-text-webpack-plugin}{\tt extract-\/text-\/webpack-\/plugin} to extract the C\+SS when running in production mode.

{\bfseries webpack.\+config.\+js} 
\begin{DoxyCode}
const env = process.env.NODE\_ENV

const ExtractTextPlugin = require('extract-text-webpack-plugin')

module.exports = \{
  module: \{
    rules: [
      \{
        test: /\(\backslash\).css$/,
        use: env === 'production'
          ? ExtractTextPlugin.extract(\{
              fallback: 'style-loader',
              use: [ 'css-loader' ]
          \})
          : [ 'style-loader', 'css-loader' ]
      \},
    ]
  \},
  plugins: env === 'production'
    ? [
        new ExtractTextPlugin(\{
          filename: '[name].css'
        \})
      ]
    : []
  ]
\}
\end{DoxyCode}


\subsection*{Maintainers}

\tabulinesep=1mm
\begin{longtabu} spread 0pt [c]{*{0}{|X[-1]}|}
\hline
\end{longtabu}


  \href{https://github.com/bebraw}{\tt Juho Vepsäläinen}  

  \href{https://github.com/d3viant0ne}{\tt Joshua Wiens}  

  \href{https://github.com/SpaceK33z}{\tt Kees Kluskens}  

  \href{https://github.com/TheLarkInn}{\tt Sean Larkin}   

  \href{https://github.com/michael-ciniawsky}{\tt Michael Ciniawsky}  

  \href{https://github.com/evilebottnawi}{\tt Evilebot Tnawi}  

  \href{https://github.com/joscha}{\tt Joscha Feth}   $<$tbody$>$ 