A Post\+C\+SS runner is a tool that processes C\+SS through a user-\/defined list of plugins; for example, \href{https://github.com/postcss/postcss-cli}{\tt {\ttfamily postcss-\/cli}} or \href{https://github.com/w0rm/gulp-postcss}{\tt {\ttfamily gulp‑postcss}}. These rules are mandatory for any such runners.

For single-\/plugin tools, like \href{https://github.com/sindresorhus/gulp-autoprefixer}{\tt {\ttfamily gulp-\/autoprefixer}}, these rules are not mandatory but are highly recommended.

See also \href{http://blog.clojurewerkz.org/blog/2013/04/20/how-to-make-your-open-source-project-really-awesome/}{\tt Clojure\+Werkz’s recommendations} for open source projects.

\subsection*{1. A\+PI}

\subsubsection*{1.\+1. Accept functions in plugin parameters}

If your runner uses a config file, it must be written in Java\+Script, so that it can support plugins which accept a function, such as \mbox{[}{\ttfamily postcss-\/assets}\mbox{]}\+:


\begin{DoxyCode}
module.exports = [
    require('postcss-assets')(\{
        cachebuster: function (file) \{
            return fs.statSync(file).mtime.getTime().toString(16);
        \}
    \})
];
\end{DoxyCode}


\subsection*{2. Processing}

\subsubsection*{2.\+1. Set {\ttfamily from} and {\ttfamily to} processing options}

To ensure that Post\+C\+SS generates source maps and displays better syntax errors, runners must specify the {\ttfamily from} and {\ttfamily to} options. If your runner does not handle writing to disk (for example, a gulp transform), you should set both options to point to the same file\+:


\begin{DoxyCode}
processor.process(\{ from: file.path, to: file.path \});
\end{DoxyCode}


\subsubsection*{2.\+2. Use only the asynchronous A\+PI}

Post\+C\+SS runners must use only the asynchronous A\+PI. The synchronous A\+PI is provided only for debugging, is slower, and can’t work with asynchronous plugins.


\begin{DoxyCode}
processor.process(opts).then(function (result) \{
    // processing is finished
\});
\end{DoxyCode}


\subsubsection*{2.\+3. Use only the public Post\+C\+SS A\+PI}

Post\+C\+SS runners must not rely on undocumented properties or methods, which may be subject to change in any minor release. The public A\+PI is described in \href{http://api.postcss.org/}{\tt A\+PI docs}.

\subsection*{3. Output}

\subsubsection*{3.\+1. Don’t show JS stack for {\ttfamily Css\+Syntax\+Error}}

Post\+C\+SS runners must not show a stack trace for C\+SS syntax errors, as the runner can be used by developers who are not familiar with Java\+Script. Instead, handle such errors gracefully\+:


\begin{DoxyCode}
processor.process(opts).catch(function (error) \{
    if ( error.name === 'CssSyntaxError' ) \{
        process.stderr.write(error.message + error.showSourceCode());
    \} else \{
        throw error;
    \}
\});
\end{DoxyCode}


\subsubsection*{3.\+2. Display {\ttfamily result.\+warnings()}}

Post\+C\+SS runners must output warnings from {\ttfamily result.\+warnings()}\+:


\begin{DoxyCode}
result.warnings().forEach(function (warn) \{
    process.stderr.write(warn.toString());
\});
\end{DoxyCode}


See also \href{https://github.com/davidtheclark/postcss-log-warnings}{\tt postcss-\/log-\/warnings} and \href{https://github.com/postcss/postcss-messages}{\tt postcss-\/messages} plugins.

\subsubsection*{3.\+3. Allow the user to write source maps to different files}

Post\+C\+SS by default will inline source maps in the generated file; however, Post\+C\+SS runners must provide an option to save the source map in a different file\+:


\begin{DoxyCode}
if ( result.map ) \{
    fs.writeFile(opts.to + '.map', result.map.toString());
\}
\end{DoxyCode}


\subsection*{4. Documentation}

\subsubsection*{4.\+1. Document your runner in English}

Post\+C\+SS runners must have their {\ttfamily R\+E\+A\+D\+M\+E.\+md} written in English. Do not be afraid of your English skills, as the open source community will fix your errors.

Of course, you are welcome to write documentation in other languages; just name them appropriately (e.\+g.\+ $<$tt$>$R\+E\+A\+D\+M\+E.\+ja.\+md).

\subsubsection*{4.\+2. Maintain a changelog}

Post\+C\+SS runners must describe changes of all releases in a separate file, such as {\ttfamily Change\+Log.\+md}, {\ttfamily History.\+md}, or with \href{https://help.github.com/articles/creating-releases/}{\tt Git\+Hub Releases}. Visit \href{http://keepachangelog.com/}{\tt Keep A Changelog} for more information on how to write one of these.

Of course you should use \href{http://semver.org/}{\tt Sem\+Ver}.

\subsubsection*{4.\+3. {\ttfamily postcss-\/runner} keyword in {\ttfamily package.\+json}}

Post\+C\+SS runners written for npm must have the {\ttfamily postcss-\/runner} keyword in their {\ttfamily package.\+json}. This special keyword will be useful for feedback about the Post\+C\+SS ecosystem.

For packages not published to npm, this is not mandatory, but recommended if the package format is allowed to contain keywords. 