\subsection*{Purpose}

To make it easier for Facebook to share and consume our own Java\+Script. Primarily this will allow us to ship code without worrying too much about where it lives, keeping with the spirit of {\ttfamily @provides\+Module} but working in the broader Java\+Script ecosystem.

{\bfseries Note\+:} If you are consuming the code here and you are not also a Facebook project, be prepared for a bad time. A\+P\+Is may appear or disappear and we may not follow semver strictly, though we will do our best to. This library is being published with our use cases in mind and is not necessarily meant to be consumed by the broader public. In order for us to move fast and ship projects like React and Relay, we\textquotesingle{}ve made the decision to not support everybody. We probably won\textquotesingle{}t take your feature requests unless they align with our needs. There will be overlap in functionality here and in other open source projects.

\subsection*{Usage}

Any {\ttfamily @provides\+Module} modules that are used by your project should be added to {\ttfamily src/}. They will be built and added to {\ttfamily module-\/map.\+json}. This file will contain a map from {\ttfamily @provides\+Module} name to what will be published as {\ttfamily fbjs}. The {\ttfamily module-\/map.\+json} file can then be consumed in your own project, along with the \href{https://github.com/facebook/fbjs/blob/master/babel-preset/plugins/rewrite-modules.js}{\tt rewrite-\/modules} Babel plugin (which we\textquotesingle{}ll publish with this), to rewrite requires in your own project. Then, just make sure {\ttfamily fbjs} is a dependency in your {\ttfamily package.\+json} and your package will consume the shared code.


\begin{DoxyCode}
// Before transform
const emptyFunction = require('emptyFunction');
// After transform
const emptyFunction = require('fbjs/lib/emptyFunction');
\end{DoxyCode}


See React for an example of this. {\itshape Coming soon!}

\subsection*{Building}

It\textquotesingle{}s as easy as just running gulp. This assumes you\textquotesingle{}ve also done {\ttfamily npm install -\/g gulp}.


\begin{DoxyCode}
gulp
\end{DoxyCode}


Alternatively {\ttfamily npm run build} will also work.

\subsubsection*{Layout}

Right now these packages represent a subset of packages that we use internally at Facebook. Mostly these are support libraries used when shipping larger libraries, like React and Relay, or products. Each of these packages is in its own directory under {\ttfamily src/}.

\subsubsection*{Process}

Since we use {\ttfamily @provides\+Module}, we need to rewrite requires to be relative. Thanks to {\ttfamily @provides\+Module} requiring global uniqueness, we can do this easily. Eventually we\textquotesingle{}ll try to make this part of the process go away by making more projects use Common\+JS.

\subsection*{T\+O\+DO}


\begin{DoxyItemize}
\item Flow\+: Ideally we\textquotesingle{}d ship our original files with type annotations, however that\textquotesingle{}s not doable right now. We have a couple options\+:
\begin{DoxyItemize}
\item Make sure our transpilation step converts inline type annotations to the comment format.
\item Make our build process also build Flow interface files which we can ship to npm.
\end{DoxyItemize}
\item Split into multiple packages. This will be better for more concise versioning, otherwise we\textquotesingle{}ll likely just be shipping lots of major versions. 
\end{DoxyItemize}