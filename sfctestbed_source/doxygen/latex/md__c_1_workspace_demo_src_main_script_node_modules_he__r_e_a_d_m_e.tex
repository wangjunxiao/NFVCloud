{\itshape he} (for “\+H\+T\+ML entities”) is a robust H\+T\+ML entity encoder/decoder written in Java\+Script. It supports \href{https://html.spec.whatwg.org/multipage/syntax.html#named-character-references}{\tt all standardized named character references as per H\+T\+ML}, handles \href{https://mathiasbynens.be/notes/ambiguous-ampersands}{\tt ambiguous ampersands} and other edge cases \href{https://html.spec.whatwg.org/multipage/syntax.html#tokenizing-character-references}{\tt just like a browser would}, has an extensive test suite, and — contrary to many other Java\+Script solutions — {\itshape he} handles astral Unicode symbols just fine. \href{https://mothereff.in/html-entities}{\tt An online demo is available.}

\subsection*{Installation}

Via \href{https://www.npmjs.com/}{\tt npm}\+:


\begin{DoxyCode}
npm install he
\end{DoxyCode}


Via \href{http://bower.io/}{\tt Bower}\+:


\begin{DoxyCode}
bower install he
\end{DoxyCode}


Via \href{https://github.com/component/component}{\tt Component}\+:


\begin{DoxyCode}
component install mathiasbynens/he
\end{DoxyCode}


In a browser\+:


\begin{DoxyCode}
<script src="he.js"></script>
\end{DoxyCode}


In \href{https://nodejs.org/}{\tt Node.\+js}, \href{https://iojs.org/}{\tt io.\+js}, \href{http://narwhaljs.org/}{\tt Narwhal}, and \href{http://ringojs.org/}{\tt Ringo\+JS}\+:


\begin{DoxyCode}
var he = require('he');
\end{DoxyCode}


In \href{http://www.mozilla.org/rhino/}{\tt Rhino}\+:


\begin{DoxyCode}
load('he.js');
\end{DoxyCode}


Using an A\+MD loader like \href{http://requirejs.org/}{\tt Require\+JS}\+:


\begin{DoxyCode}
require(
  \{
    'paths': \{
      'he': 'path/to/he'
    \}
  \},
  ['he'],
  function(he) \{
    console.log(he);
  \}
);
\end{DoxyCode}


\subsection*{A\+PI}

\subsubsection*{{\ttfamily he.\+version}}

A string representing the semantic version number.

\subsubsection*{{\ttfamily he.\+encode(text, options)}}

This function takes a string of text and encodes (by default) any symbols that aren’t printable A\+S\+C\+II symbols and {\ttfamily \&}, {\ttfamily $<$}, {\ttfamily $>$}, {\ttfamily "}, `'{\ttfamily , and} \`{}\`{}, replacing them with character references.


\begin{DoxyCode}
he.encode('foo © bar ≠ baz 𝌆 qux');
// → 'foo &#xA9; bar &#x2260; baz &#x1D306; qux'
\end{DoxyCode}


As long as the input string contains \href{https://html.spec.whatwg.org/multipage/parsing.html#preprocessing-the-input-stream}{\tt allowed code points} only, the return value of this function is always valid H\+T\+ML. Any \href{https://html.spec.whatwg.org/multipage/syntax.html#table-charref-overrides}{\tt (invalid) code points that cannot be represented using a character reference} in the input are not encoded\+:


\begin{DoxyCode}
he.encode('foo \(\backslash\)0 bar');
// → 'foo \(\backslash\)0 bar'
\end{DoxyCode}


However, enabling \href{https://github.com/mathiasbynens/he#strict}{\tt the {\ttfamily strict} option} causes invalid code points to throw an exception. With {\ttfamily strict} enabled, {\ttfamily he.\+encode} either throws (if the input contains invalid code points) or returns a string of valid H\+T\+ML.

The {\ttfamily options} object is optional. It recognizes the following properties\+:

\paragraph*{{\ttfamily use\+Named\+References}}

The default value for the {\ttfamily use\+Named\+References} option is {\ttfamily false}. This means that {\ttfamily encode()} will not use any named character references (e.\+g. {\ttfamily \copyright{}}) in the output — hexadecimal escapes (e.\+g. {\ttfamily \&\#x\+A9;}) will be used instead. Set it to {\ttfamily true} to enable the use of named references.

{\bfseries Note that if compatibility with older browsers is a concern, this option should remain disabled.}


\begin{DoxyCode}
// Using the global default setting (defaults to `false`):
he.encode('foo © bar ≠ baz 𝌆 qux');
// → 'foo &#xA9; bar &#x2260; baz &#x1D306; qux'

// Passing an `options` object to `encode`, to explicitly disallow named references:
he.encode('foo © bar ≠ baz 𝌆 qux', \{
  'useNamedReferences': false
\});
// → 'foo &#xA9; bar &#x2260; baz &#x1D306; qux'

// Passing an `options` object to `encode`, to explicitly allow named references:
he.encode('foo © bar ≠ baz 𝌆 qux', \{
  'useNamedReferences': true
\});
// → 'foo &copy; bar &ne; baz &#x1D306; qux'
\end{DoxyCode}


\paragraph*{{\ttfamily decimal}}

The default value for the {\ttfamily decimal} option is {\ttfamily false}. If the option is enabled, {\ttfamily encode} will generally use decimal escapes (e.\+g. {\ttfamily \&\#169;}) rather than hexadecimal escapes (e.\+g. {\ttfamily \&\#x\+A9;}). Beside of this replacement, the basic behavior remains the same when combined with other options. For example\+: if both options {\ttfamily use\+Named\+References} and {\ttfamily decimal} are enabled, named references (e.\+g. {\ttfamily \copyright{}}) are used over decimal escapes. H\+T\+ML entities without a named reference are encoded using decimal escapes.


\begin{DoxyCode}
// Using the global default setting (defaults to `false`):
he.encode('foo © bar ≠ baz 𝌆 qux');
// → 'foo &#xA9; bar &#x2260; baz &#x1D306; qux'

// Passing an `options` object to `encode`, to explicitly disable decimal escapes:
he.encode('foo © bar ≠ baz 𝌆 qux', \{
  'decimal': false
\});
// → 'foo &#xA9; bar &#x2260; baz &#x1D306; qux'

// Passing an `options` object to `encode`, to explicitly enable decimal escapes:
he.encode('foo © bar ≠ baz 𝌆 qux', \{
  'decimal': true
\});
// → 'foo &#169; bar &#8800; baz &#119558; qux'

// Passing an `options` object to `encode`, to explicitly allow named references and decimal escapes:
he.encode('foo © bar ≠ baz 𝌆 qux', \{
  'useNamedReferences': true,
  'decimal': true
\});
// → 'foo &copy; bar &ne; baz &#119558; qux'
\end{DoxyCode}


\paragraph*{{\ttfamily encode\+Everything}}

The default value for the {\ttfamily encode\+Everything} option is {\ttfamily false}. This means that {\ttfamily encode()} will not use any character references for printable A\+S\+C\+II symbols that don’t need escaping. Set it to {\ttfamily true} to encode every symbol in the input string. When set to {\ttfamily true}, this option takes precedence over {\ttfamily allow\+Unsafe\+Symbols} (i.\+e. setting the latter to {\ttfamily true} in such a case has no effect).


\begin{DoxyCode}
// Using the global default setting (defaults to `false`):
he.encode('foo © bar ≠ baz 𝌆 qux');
// → 'foo &#xA9; bar &#x2260; baz &#x1D306; qux'

// Passing an `options` object to `encode`, to explicitly encode all symbols:
he.encode('foo © bar ≠ baz 𝌆 qux', \{
  'encodeEverything': true
\});
// →
       '&#x66;&#x6F;&#x6F;&#x20;&#xA9;&#x20;&#x62;&#x61;&#x72;&#x20;&#x2260;&#x20;&#x62;&#x61;&#x7A;&#x20;&#x1D306;&#x20;&#x71;&#x75;&#x78;'

// This setting can be combined with the `useNamedReferences` option:
he.encode('foo © bar ≠ baz 𝌆 qux', \{
  'encodeEverything': true,
  'useNamedReferences': true
\});
// →
       '&#x66;&#x6F;&#x6F;&#x20;&copy;&#x20;&#x62;&#x61;&#x72;&#x20;&ne;&#x20;&#x62;&#x61;&#x7A;&#x20;&#x1D306;&#x20;&#x71;&#x75;&#x78;'
\end{DoxyCode}


\paragraph*{{\ttfamily strict}}

The default value for the {\ttfamily strict} option is {\ttfamily false}. This means that {\ttfamily encode()} will encode any H\+T\+ML text content you feed it, even if it contains any symbols that cause \href{https://html.spec.whatwg.org/multipage/parsing.html#preprocessing-the-input-stream}{\tt parse errors}. To throw an error when such invalid H\+T\+ML is encountered, set the {\ttfamily strict} option to {\ttfamily true}. This option makes it possible to use {\itshape he} as part of H\+T\+ML parsers and H\+T\+ML validators.


\begin{DoxyCode}
// Using the global default setting (defaults to `false`, i.e. error-tolerant mode):
he.encode('\(\backslash\)x01');
// → '&#x1;'

// Passing an `options` object to `encode`, to explicitly enable error-tolerant mode:
he.encode('\(\backslash\)x01', \{
  'strict': false
\});
// → '&#x1;'

// Passing an `options` object to `encode`, to explicitly enable strict mode:
he.encode('\(\backslash\)x01', \{
  'strict': true
\});
// → Parse error
\end{DoxyCode}


\paragraph*{{\ttfamily allow\+Unsafe\+Symbols}}

The default value for the {\ttfamily allow\+Unsafe\+Symbols} option is {\ttfamily false}. This means that characters that are unsafe for use in H\+T\+ML content ({\ttfamily \&}, {\ttfamily $<$}, {\ttfamily $>$}, {\ttfamily "}, `'{\ttfamily , and} \`{}{\ttfamily ) will be encoded. When set to}true{\ttfamily , only non-\/\+A\+S\+C\+II characters will be encoded. If the}encode\+Everything{\ttfamily option is set to}true\`{}, this option will be ignored.


\begin{DoxyCode}
he.encode('foo © and & ampersand', \{
  'allowUnsafeSymbols': true
\});
// → 'foo &#xA9; and & ampersand'
\end{DoxyCode}


\paragraph*{Overriding default {\ttfamily encode} options globally}

The global default setting can be overridden by modifying the {\ttfamily he.\+encode.\+options} object. This saves you from passing in an {\ttfamily options} object for every call to {\ttfamily encode} if you want to use the non-\/default setting.


\begin{DoxyCode}
// Read the global default setting:
he.encode.options.useNamedReferences;
// → `false` by default

// Override the global default setting:
he.encode.options.useNamedReferences = true;

// Using the global default setting, which is now `true`:
he.encode('foo © bar ≠ baz 𝌆 qux');
// → 'foo &copy; bar &ne; baz &#x1D306; qux'
\end{DoxyCode}


\subsubsection*{{\ttfamily he.\+decode(html, options)}}

This function takes a string of H\+T\+ML and decodes any named and numerical character references in it using \href{https://html.spec.whatwg.org/multipage/syntax.html#tokenizing-character-references}{\tt the algorithm described in section 12.\+2.\+4.\+69 of the H\+T\+ML spec}.


\begin{DoxyCode}
he.decode('foo &copy; bar &ne; baz &#x1D306; qux');
// → 'foo © bar ≠ baz 𝌆 qux'
\end{DoxyCode}


The {\ttfamily options} object is optional. It recognizes the following properties\+:

\paragraph*{{\ttfamily is\+Attribute\+Value}}

The default value for the {\ttfamily is\+Attribute\+Value} option is {\ttfamily false}. This means that {\ttfamily decode()} will decode the string as if it were used in \href{https://html.spec.whatwg.org/multipage/syntax.html#data-state}{\tt a text context in an H\+T\+ML document}. H\+T\+ML has different rules for \href{https://html.spec.whatwg.org/multipage/syntax.html#character-reference-in-attribute-value-state}{\tt parsing character references in attribute values} — set this option to {\ttfamily true} to treat the input string as if it were used as an attribute value.


\begin{DoxyCode}
// Using the global default setting (defaults to `false`, i.e. HTML text context):
he.decode('foo&ampbar');
// → 'foo&bar'

// Passing an `options` object to `decode`, to explicitly assume an HTML text context:
he.decode('foo&ampbar', \{
  'isAttributeValue': false
\});
// → 'foo&bar'

// Passing an `options` object to `decode`, to explicitly assume an HTML attribute value context:
he.decode('foo&ampbar', \{
  'isAttributeValue': true
\});
// → 'foo&ampbar'
\end{DoxyCode}


\paragraph*{{\ttfamily strict}}

The default value for the {\ttfamily strict} option is {\ttfamily false}. This means that {\ttfamily decode()} will decode any H\+T\+ML text content you feed it, even if it contains any entities that cause \href{https://html.spec.whatwg.org/multipage/syntax.html#tokenizing-character-references}{\tt parse errors}. To throw an error when such invalid H\+T\+ML is encountered, set the {\ttfamily strict} option to {\ttfamily true}. This option makes it possible to use {\itshape he} as part of H\+T\+ML parsers and H\+T\+ML validators.


\begin{DoxyCode}
// Using the global default setting (defaults to `false`, i.e. error-tolerant mode):
he.decode('foo&ampbar');
// → 'foo&bar'

// Passing an `options` object to `decode`, to explicitly enable error-tolerant mode:
he.decode('foo&ampbar', \{
  'strict': false
\});
// → 'foo&bar'

// Passing an `options` object to `decode`, to explicitly enable strict mode:
he.decode('foo&ampbar', \{
  'strict': true
\});
// → Parse error
\end{DoxyCode}


\paragraph*{Overriding default {\ttfamily decode} options globally}

The global default settings for the {\ttfamily decode} function can be overridden by modifying the {\ttfamily he.\+decode.\+options} object. This saves you from passing in an {\ttfamily options} object for every call to {\ttfamily decode} if you want to use a non-\/default setting.


\begin{DoxyCode}
// Read the global default setting:
he.decode.options.isAttributeValue;
// → `false` by default

// Override the global default setting:
he.decode.options.isAttributeValue = true;

// Using the global default setting, which is now `true`:
he.decode('foo&ampbar');
// → 'foo&ampbar'
\end{DoxyCode}


\subsubsection*{{\ttfamily he.\+escape(text)}}

This function takes a string of text and escapes it for use in text contexts in X\+ML or H\+T\+ML documents. Only the following characters are escaped\+: {\ttfamily \&}, {\ttfamily $<$}, {\ttfamily $>$}, {\ttfamily "}, `'{\ttfamily , and} \`{}\`{}.


\begin{DoxyCode}
he.escape('<img src=\(\backslash\)'x\(\backslash\)' onerror="prompt(1)">');
// → '&lt;img src=&#x27;x&#x27; onerror=&quot;prompt(1)&quot;&gt;'
\end{DoxyCode}


\subsubsection*{{\ttfamily he.\+unescape(html, options)}}

{\ttfamily he.\+unescape} is an alias for {\ttfamily he.\+decode}. It takes a string of H\+T\+ML and decodes any named and numerical character references in it.

\subsubsection*{Using the {\ttfamily he} binary}

To use the {\ttfamily he} binary in your shell, simply install {\itshape he} globally using npm\+:


\begin{DoxyCode}
npm install -g he
\end{DoxyCode}


After that you will be able to encode/decode H\+T\+ML entities from the command line\+:


\begin{DoxyCode}
$ he --encode 'föo ♥ bår 𝌆 baz'
f&#xF6;o &#x2665; b&#xE5;r &#x1D306; baz

$ he --encode --use-named-refs 'föo ♥ bår 𝌆 baz'
f&ouml;o &hearts; b&aring;r &#x1D306; baz

$ he --decode 'f&ouml;o &hearts; b&aring;r &#x1D306; baz'
föo ♥ bår 𝌆 baz
\end{DoxyCode}


Read a local text file, encode it for use in an H\+T\+ML text context, and save the result to a new file\+:


\begin{DoxyCode}
$ he --encode < foo.txt > foo-escaped.html
\end{DoxyCode}


Or do the same with an online text file\+:


\begin{DoxyCode}
$ curl -sL "http://git.io/HnfEaw" | he --encode > escaped.html
\end{DoxyCode}


Or, the opposite — read a local file containing a snippet of H\+T\+ML in a text context, decode it back to plain text, and save the result to a new file\+:


\begin{DoxyCode}
$ he --decode < foo-escaped.html > foo.txt
\end{DoxyCode}


Or do the same with an online H\+T\+ML snippet\+:


\begin{DoxyCode}
$ curl -sL "http://git.io/HnfEaw" | he --decode > decoded.txt
\end{DoxyCode}


See {\ttfamily he -\/-\/help} for the full list of options.

\subsection*{Support}

{\itshape he} has been tested in at least\+:


\begin{DoxyItemize}
\item Chrome 27-\/50
\item Firefox 3-\/45
\item Safari 4-\/9
\item Opera 10-\/12, 15–37
\item IE 6–11
\item Edge
\item Narwhal 0.\+3.\+2
\item Node.\+js v0.\+10, v0.\+12, v4, v5
\item Phantom\+JS 1.\+9.\+0
\item Rhino 1.\+7\+R\+C4
\item Ringo\+JS 0.\+8-\/0.\+11
\end{DoxyItemize}

\subsection*{Unit tests \& code coverage}

After cloning this repository, run {\ttfamily npm install} to install the dependencies needed for he development and testing. You may want to install Istanbul {\itshape globally} using {\ttfamily npm install istanbul -\/g}.

Once that’s done, you can run the unit tests in Node using {\ttfamily npm test} or {\ttfamily node tests/tests.\+js}. To run the tests in Rhino, Ringo, Narwhal, and web browsers as well, use {\ttfamily grunt test}.

To generate the code coverage report, use {\ttfamily grunt cover}.

\subsection*{Acknowledgements}

Thanks to \href{https://simon.html5.org/}{\tt Simon Pieters} (\href{https://twitter.com/zcorpan}{\tt }) for the many suggestions.

\subsection*{Author}

\tabulinesep=1mm
\begin{longtabu} spread 0pt [c]{*{1}{|X[-1]}|}
\hline
\rowcolor{\tableheadbgcolor}\textbf{ \mbox{[}!   }\\\cline{1-1}
\endfirsthead
\hline
\endfoot
\hline
\rowcolor{\tableheadbgcolor}\textbf{ \mbox{[}!   }\\\cline{1-1}
\endhead
\href{https://mathiasbynens.be/}{\tt Mathias Bynens}   \\\cline{1-1}
\end{longtabu}


\subsection*{License}

{\itshape he} is available under the \href{https://mths.be/mit}{\tt M\+IT} license. 