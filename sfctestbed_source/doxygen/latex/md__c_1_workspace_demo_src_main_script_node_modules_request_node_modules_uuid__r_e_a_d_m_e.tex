Simple, fast generation of \href{http://www.ietf.org/rfc/rfc4122.txt}{\tt R\+F\+C4122} U\+U\+I\+DS.

Features\+:


\begin{DoxyItemize}
\item Support for version 1, 4 and 5 U\+U\+I\+Ds
\item Cross-\/platform
\item Uses cryptographically-\/strong random number A\+P\+Is (when available)
\item Zero-\/dependency, small footprint (... but not \href{https://gist.github.com/982883}{\tt this small})
\end{DoxyItemize}

\subsection*{Quickstart -\/ Common\+JS (Recommended)}


\begin{DoxyCode}
npm install uuid
\end{DoxyCode}


Then generate your uuid version of choice ...

Version 1 (timestamp)\+:


\begin{DoxyCode}
const uuidv1 = require('uuid/v1');
uuidv1(); // -> '6c84fb90-12c4-11e1-840d-7b25c5ee775a'
\end{DoxyCode}


Version 4 (random)\+:


\begin{DoxyCode}
const uuidv4 = require('uuid/v4');
uuidv4(); // -> '110ec58a-a0f2-4ac4-8393-c866d813b8d1'
\end{DoxyCode}


Version 5 (namespace)\+:


\begin{DoxyCode}
const uuidv5 = require('uuid/v5');

// ... using predefined DNS namespace (for domain names)
uuidv5('hello.example.com', uuidv5.DNS)); // -> 'fdda765f-fc57-5604-a269-52a7df8164ec'

// ... using predefined URL namespace (for, well, URLs)
uuidv5('http://example.com/hello', uuidv5.URL); // -> '3bbcee75-cecc-5b56-8031-b6641c1ed1f1'

// ... using a custom namespace
const MY\_NAMESPACE = '<UUID string you previously generated elsewhere>';
uuidv5('Hello, World!', MY\_NAMESPACE); // -> '90123e1c-7512-523e-bb28-76fab9f2f73d'
\end{DoxyCode}


\subsection*{Quickstart -\/ Browser-\/ready Versions}

Browser-\/ready versions of this module are available via \href{https://github.com/jfhbrook/wzrd.in}{\tt wzrd.\+in}.

For version 1 uuids\+:


\begin{DoxyCode}
<script src="http://wzrd.in/standalone/uuid%2Fv1@latest"></script>
<script>
uuidv1(); // -> v1 UUID
</script>
\end{DoxyCode}


For version 4 uuids\+:


\begin{DoxyCode}
<script src="http://wzrd.in/standalone/uuid%2Fv4@latest"></script>
<script>
uuidv4(); // -> v4 UUID
</script>
\end{DoxyCode}


For version 5 uuids\+:


\begin{DoxyCode}
<script src="http://wzrd.in/standalone/uuid%2Fv5@latest"></script>
<script>
uuidv5('http://example.com/hello', uuidv5.URL); // -> v5 UUID
</script>
\end{DoxyCode}


\subsection*{A\+PI}

\subsubsection*{Version 1}


\begin{DoxyCode}
const uuidv1 = require('uuid/v1');

// Allowed arguments
uuidv1();
uuidv1(options);
uuidv1(options, buffer, offset);
\end{DoxyCode}


Generate and return a R\+F\+C4122 v1 (timestamp-\/based) U\+U\+ID.


\begin{DoxyItemize}
\item {\ttfamily options} -\/ (Object) Optional uuid state to apply. Properties may include\+:
\begin{DoxyItemize}
\item {\ttfamily node} -\/ (Array) Node id as Array of 6 bytes (per 4.\+1.\+6). Default\+: Randomly generated ID. See note 1.
\item {\ttfamily clockseq} -\/ (Number between 0 -\/ 0x3fff) R\+FC clock sequence. Default\+: An internally maintained clockseq is used.
\item {\ttfamily msecs} -\/ (Number $\vert$ Date) Time in milliseconds since unix Epoch. Default\+: The current time is used.
\item {\ttfamily nsecs} -\/ (Number between 0-\/9999) additional time, in 100-\/nanosecond units. Ignored if {\ttfamily msecs} is unspecified. Default\+: internal uuid counter is used, as per 4.\+2.\+1.\+2.
\end{DoxyItemize}
\item {\ttfamily buffer} -\/ (Array $\vert$ Buffer) Array or buffer where U\+U\+ID bytes are to be written.
\item {\ttfamily offset} -\/ (Number) Starting index in {\ttfamily buffer} at which to begin writing.
\end{DoxyItemize}

Returns {\ttfamily buffer}, if specified, otherwise the string form of the U\+U\+ID

Note\+: The $<$node$>$ id is generated guaranteed to stay constant for the lifetime of the current JS runtime. (Future versions of this module may use persistent storage mechanisms to extend this guarantee.)

Example\+: Generate string U\+U\+ID with fully-\/specified options


\begin{DoxyCode}
uuidv1(\{
  node: [0x01, 0x23, 0x45, 0x67, 0x89, 0xab],
  clockseq: 0x1234,
  msecs: new Date('2011-11-01').getTime(),
  nsecs: 5678
\});   // -> "710b962e-041c-11e1-9234-0123456789ab"
\end{DoxyCode}


Example\+: In-\/place generation of two binary I\+Ds


\begin{DoxyCode}
// Generate two ids in an array
const arr = new Array(32); // -> []
uuidv1(null, arr, 0);   // -> [02 a2 ce 90 14 32 11 e1 85 58 0b 48 8e 4f c1 15]
uuidv1(null, arr, 16);  // -> [02 a2 ce 90 14 32 11 e1 85 58 0b 48 8e 4f c1 15 02 a3 1c b0 14 32 11 e1 85
       58 0b 48 8e 4f c1 15]
\end{DoxyCode}


\subsubsection*{Version 4}


\begin{DoxyCode}
const uuidv4 = require('uuid/v4')

// Allowed arguments
uuidv4();
uuidv4(options);
uuidv4(options, buffer, offset);
\end{DoxyCode}


Generate and return a R\+F\+C4122 v4 U\+U\+ID.


\begin{DoxyItemize}
\item {\ttfamily options} -\/ (Object) Optional uuid state to apply. Properties may include\+:
\begin{DoxyItemize}
\item {\ttfamily random} -\/ (Number\mbox{[}16\mbox{]}) Array of 16 numbers (0-\/255) to use in place of randomly generated values
\item {\ttfamily rng} -\/ (Function) Random \# generator function that returns an Array\mbox{[}16\mbox{]} of byte values (0-\/255)
\end{DoxyItemize}
\item {\ttfamily buffer} -\/ (Array $\vert$ Buffer) Array or buffer where U\+U\+ID bytes are to be written.
\item {\ttfamily offset} -\/ (Number) Starting index in {\ttfamily buffer} at which to begin writing.
\end{DoxyItemize}

Returns {\ttfamily buffer}, if specified, otherwise the string form of the U\+U\+ID

Example\+: Generate string U\+U\+ID with fully-\/specified options


\begin{DoxyCode}
uuid.v4(\{
  random: [
    0x10, 0x91, 0x56, 0xbe, 0xc4, 0xfb, 0xc1, 0xea,
    0x71, 0xb4, 0xef, 0xe1, 0x67, 0x1c, 0x58, 0x36
  ]
\});
// -> "109156be-c4fb-41ea-b1b4-efe1671c5836"
\end{DoxyCode}


Example\+: Generate two I\+Ds in a single buffer


\begin{DoxyCode}
const buffer = new Array(32); // (or 'new Buffer' in node.js)
uuid.v4(null, buffer, 0);
uuid.v4(null, buffer, 16);
\end{DoxyCode}


\subsubsection*{Version 5}


\begin{DoxyCode}
const uuidv5 = require('uuid/v4');

// Allowed arguments
uuidv5(name, namespace);
uuidv5(name, namespace, buffer);
uuidv5(name, namespace, buffer, offset);
\end{DoxyCode}


Generate and return a R\+F\+C4122 v4 U\+U\+ID.


\begin{DoxyItemize}
\item {\ttfamily name} -\/ (String $\vert$ Array\mbox{[}\mbox{]}) \char`\"{}name\char`\"{} to create U\+U\+ID with
\item {\ttfamily namespace} -\/ (String $\vert$ Array\mbox{[}\mbox{]}) \char`\"{}namespace\char`\"{} U\+U\+ID either as a String or Array\mbox{[}16\mbox{]} of byte values
\item {\ttfamily buffer} -\/ (Array $\vert$ Buffer) Array or buffer where U\+U\+ID bytes are to be written.
\item {\ttfamily offset} -\/ (Number) Starting index in {\ttfamily buffer} at which to begin writing. Default = 0
\end{DoxyItemize}

Returns {\ttfamily buffer}, if specified, otherwise the string form of the U\+U\+ID

Example\+:


\begin{DoxyCode}
// Generate a unique  namespace (typically you would do this once, outside of
// your project, then bake this value into your code)
const uuidv4 = require('uuid/v4');
const MY\_NAMESPACE = uuidv4();  //

// Generate a couple namespace uuids
const uuidv5 = require('uuid/v5');
uuidv5('hello', MY\_NAMESPACE);
uuidv5('world', MY\_NAMESPACE);
\end{DoxyCode}


\subsection*{Testing}


\begin{DoxyCode}
npm test
\end{DoxyCode}


\subsection*{Deprecated / Browser-\/ready A\+PI}

The A\+PI below is available for legacy purposes and is not expected to be available post-\/3.\+X


\begin{DoxyCode}
const uuid = require('uuid');

uuid.v1(...); // alias of uuid/v1
uuid.v4(...); // alias of uuid/v4
uuid(...);    // alias of uuid/v4

// uuid.v5() is not supported in this API
\end{DoxyCode}


\subsection*{Legacy node-\/uuid package}

The code for the legacy node-\/uuid package is available in the {\ttfamily node-\/uuid} branch. 