\href{https://travis-ci.org/caolan/async}{\tt } \href{https://www.npmjs.org/package/async}{\tt } \href{https://coveralls.io/r/caolan/async?branch=master}{\tt } \href{https://gitter.im/caolan/async?utm_source=badge&utm_medium=badge&utm_campaign=pr-badge&utm_content=badge}{\tt }

Async is a utility module which provides straight-\/forward, powerful functions for working with asynchronous Java\+Script. Although originally designed for use with \href{http://nodejs.org}{\tt Node.\+js} and installable via {\ttfamily npm install async}, it can also be used directly in the browser.

Async is also installable via\+:


\begin{DoxyItemize}
\item \href{http://bower.io/}{\tt bower}\+: {\ttfamily bower install async}
\item \href{https://github.com/component/component}{\tt component}\+: {\ttfamily component install caolan/async}
\item \href{http://jamjs.org/}{\tt jam}\+: {\ttfamily jam install async}
\item \href{http://spmjs.io/}{\tt spm}\+: {\ttfamily spm install async}
\end{DoxyItemize}

Async provides around 20 functions that include the usual \textquotesingle{}functional\textquotesingle{} suspects ({\ttfamily map}, {\ttfamily reduce}, {\ttfamily filter}, {\ttfamily each}…) as well as some common patterns for asynchronous control flow ({\ttfamily parallel}, {\ttfamily series}, {\ttfamily waterfall}…). All these functions assume you follow the Node.\+js convention of providing a single callback as the last argument of your {\ttfamily async} function.

\subsection*{Quick Examples}


\begin{DoxyCode}
async.map(['file1','file2','file3'], fs.stat, function(err, results)\{
    // results is now an array of stats for each file
\});

async.filter(['file1','file2','file3'], fs.exists, function(results)\{
    // results now equals an array of the existing files
\});

async.parallel([
    function()\{ ... \},
    function()\{ ... \}
], callback);

async.series([
    function()\{ ... \},
    function()\{ ... \}
]);
\end{DoxyCode}


There are many more functions available so take a look at the docs below for a full list. This module aims to be comprehensive, so if you feel anything is missing please create a Git\+Hub issue for it.

\subsection*{Common Pitfalls \textsubscript{\href{http://stackoverflow.com/questions/tagged/async.js}{\tt (Stack\+Overflow)}}}

\subsubsection*{Synchronous iteration functions}

If you get an error like {\ttfamily Range\+Error\+: Maximum call stack size exceeded.} or other stack overflow issues when using async, you are likely using a synchronous iterator. By {\itshape synchronous} we mean a function that calls its callback on the same tick in the javascript event loop, without doing any I/O or using any timers. Calling many callbacks iteratively will quickly overflow the stack. If you run into this issue, just defer your callback with {\ttfamily async.\+set\+Immediate} to start a new call stack on the next tick of the event loop.

This can also arise by accident if you callback early in certain cases\+:


\begin{DoxyCode}
async.eachSeries(hugeArray, function iterator(item, callback) \{
  if (inCache(item)) \{
    callback(null, cache[item]); // if many items are cached, you'll overflow
  \} else \{
    doSomeIO(item, callback);
  \}
\}, function done() \{
  //...
\});
\end{DoxyCode}


Just change it to\+:


\begin{DoxyCode}
async.eachSeries(hugeArray, function iterator(item, callback) \{
  if (inCache(item)) \{
    async.setImmediate(function () \{
      callback(null, cache[item]);
    \});
  \} else \{
    doSomeIO(item, callback);
  //...
\end{DoxyCode}


Async guards against synchronous functions in some, but not all, cases. If you are still running into stack overflows, you can defer as suggested above, or wrap functions with \href{#ensureAsync}{\tt {\ttfamily async.\+ensure\+Async}} Functions that are asynchronous by their nature do not have this problem and don\textquotesingle{}t need the extra callback deferral.

If Java\+Script\textquotesingle{}s event loop is still a bit nebulous, check out \href{http://blog.carbonfive.com/2013/10/27/the-javascript-event-loop-explained/}{\tt this article} or \href{http://2014.jsconf.eu/speakers/philip-roberts-what-the-heck-is-the-event-loop-anyway.html}{\tt this talk} for more detailed information about how it works.

\subsubsection*{Multiple callbacks}

Make sure to always {\ttfamily return} when calling a callback early, otherwise you will cause multiple callbacks and unpredictable behavior in many cases.


\begin{DoxyCode}
async.waterfall([
    function (callback) \{
        getSomething(options, function (err, result) \{
          if (err) \{
            callback(new Error("failed getting something:" + err.message));
            // we should return here
          \}
          // since we did not return, this callback still will be called and
          // `processData` will be called twice
          callback(null, result);
        \});
    \},
    processData
], done)
\end{DoxyCode}


It is always good practice to {\ttfamily return callback(err, result)} whenever a callback call is not the last statement of a function.

\subsubsection*{Binding a context to an iterator}

This section is really about {\ttfamily bind}, not about {\ttfamily async}. If you are wondering how to make {\ttfamily async} execute your iterators in a given context, or are confused as to why a method of another library isn\textquotesingle{}t working as an iterator, study this example\+:

\`{}\`{}\`{}js // Here is a simple object with an (unnecessarily roundabout) squaring method var Async\+Squaring\+Library = \{ square\+Exponent\+: 2, square\+: function(number, callback)\{ var result = Math.\+pow(number, this.\+square\+Exponent); set\+Timeout(function()\{ callback(null, result); \}, 200); \} \};

async.\+map(\mbox{[}1, 2, 3\mbox{]}, Async\+Squaring\+Library.\+square, function(err, result)\{ // result is \mbox{[}NaN, NaN, NaN\mbox{]} // This fails because the {\ttfamily this.\+square\+Exponent} expression in the square // function is not evaluated in the context of Async\+Squaring\+Library, and is // therefore undefined. \});

async.\+map(\mbox{[}1, 2, 3\mbox{]}, Async\+Squaring\+Library.\+square.\+bind(\+Async\+Squaring\+Library), function(err, result)\{ // result is \mbox{[}1, 4, 9\mbox{]} // With the help of bind we can attach a context to the iterator before // passing it to async. Now the square function will be executed in its // \textquotesingle{}home\textquotesingle{} Async\+Squaring\+Library context and the value of {\ttfamily this.\+square\+Exponent} // will be as expected. \}); 
\begin{DoxyCode}
## Download

The source is available for download from
[GitHub](https://github.com/caolan/async/blob/master/lib/async.js).
Alternatively, you can install using Node Package Manager (`npm`):

    npm install async

As well as using Bower:

    bower install async

\_\_Development:\_\_ [async.js](https://github.com/caolan/async/raw/master/lib/async.js) - 29.6kb Uncompressed

## In the Browser

So far it's been tested in IE6, IE7, IE8, FF3.6 and Chrome 5.

Usage:

```html
<script type="text/javascript" src="async.js"></script>
<script type="text/javascript">

    async.map(data, asyncProcess, function(err, results)\{
        alert(results);
    \});

</script>
\end{DoxyCode}


\subsection*{Documentation}

Some functions are also available in the following forms\+:
\begin{DoxyItemize}
\item {\ttfamily $<$name$>$Series} -\/ the same as {\ttfamily $<$name$>$} but runs only a single async operation at a time
\item {\ttfamily $<$name$>$Limit} -\/ the same as {\ttfamily $<$name$>$} but runs a maximum of {\ttfamily limit} async operations at a time
\end{DoxyItemize}

\subsubsection*{Collections}


\begin{DoxyItemize}
\item \href{#each}{\tt {\ttfamily each}}, {\ttfamily each\+Series}, {\ttfamily each\+Limit}
\item \href{#forEachOf}{\tt {\ttfamily for\+Each\+Of}}, {\ttfamily for\+Each\+Of\+Series}, {\ttfamily for\+Each\+Of\+Limit}
\item \href{#map}{\tt {\ttfamily map}}, {\ttfamily map\+Series}, {\ttfamily map\+Limit}
\item \href{#filter}{\tt {\ttfamily filter}}, {\ttfamily filter\+Series}, {\ttfamily filter\+Limit}
\item \href{#reject}{\tt {\ttfamily reject}}, {\ttfamily reject\+Series}, {\ttfamily reject\+Limit}
\item \href{#reduce}{\tt {\ttfamily reduce}}, \href{#reduceRight}{\tt {\ttfamily reduce\+Right}}
\item \href{#detect}{\tt {\ttfamily detect}}, {\ttfamily detect\+Series}, {\ttfamily detect\+Limit}
\item \href{#sortBy}{\tt {\ttfamily sort\+By}}
\item \href{#some}{\tt {\ttfamily some}}, {\ttfamily some\+Limit}
\item \href{#every}{\tt {\ttfamily every}}, {\ttfamily every\+Limit}
\item \href{#concat}{\tt {\ttfamily concat}}, {\ttfamily concat\+Series}
\end{DoxyItemize}

\subsubsection*{Control Flow}


\begin{DoxyItemize}
\item \href{#seriestasks-callback}{\tt {\ttfamily series}}
\item \href{#parallel}{\tt {\ttfamily parallel}}, {\ttfamily parallel\+Limit}
\item \href{#whilst}{\tt {\ttfamily whilst}}, \href{#doWhilst}{\tt {\ttfamily do\+Whilst}}
\item \href{#until}{\tt {\ttfamily until}}, \href{#doUntil}{\tt {\ttfamily do\+Until}}
\item \href{#during}{\tt {\ttfamily during}}, \href{#doDuring}{\tt {\ttfamily do\+During}}
\item \href{#forever}{\tt {\ttfamily forever}}
\item \href{#waterfall}{\tt {\ttfamily waterfall}}
\item \href{#compose}{\tt {\ttfamily compose}}
\item \href{#seq}{\tt {\ttfamily seq}}
\item \href{#applyEach}{\tt {\ttfamily apply\+Each}}, {\ttfamily apply\+Each\+Series}
\item \href{#queue}{\tt {\ttfamily queue}}, \href{#priorityQueue}{\tt {\ttfamily priority\+Queue}}
\item \href{#cargo}{\tt {\ttfamily cargo}}
\item \href{#auto}{\tt {\ttfamily auto}}
\item \href{#retry}{\tt {\ttfamily retry}}
\item \href{#iterator}{\tt {\ttfamily iterator}}
\item \href{#times}{\tt {\ttfamily times}}, {\ttfamily times\+Series}, {\ttfamily times\+Limit}
\end{DoxyItemize}

\subsubsection*{Utils}


\begin{DoxyItemize}
\item \href{#apply}{\tt {\ttfamily apply}}
\item \href{#nextTick}{\tt {\ttfamily next\+Tick}}
\item \href{#memoize}{\tt {\ttfamily memoize}}
\item \href{#unmemoize}{\tt {\ttfamily unmemoize}}
\item \href{#ensureAsync}{\tt {\ttfamily ensure\+Async}}
\item \href{#constant}{\tt {\ttfamily constant}}
\item \href{#asyncify}{\tt {\ttfamily asyncify}}
\item \href{#wrapSync}{\tt {\ttfamily wrap\+Sync}}
\item \href{#log}{\tt {\ttfamily log}}
\item \href{#dir}{\tt {\ttfamily dir}}
\item \href{#noConflict}{\tt {\ttfamily no\+Conflict}}
\end{DoxyItemize}

\subsection*{Collections}

\label{_forEach}%
 \label{_each}%
 \subsubsection*{each(arr, iterator, \mbox{[}callback\mbox{]})}

Applies the function {\ttfamily iterator} to each item in {\ttfamily arr}, in parallel. The {\ttfamily iterator} is called with an item from the list, and a callback for when it has finished. If the {\ttfamily iterator} passes an error to its {\ttfamily callback}, the main {\ttfamily callback} (for the {\ttfamily each} function) is immediately called with the error.

Note, that since this function applies {\ttfamily iterator} to each item in parallel, there is no guarantee that the iterator functions will complete in order.

{\bfseries Arguments}


\begin{DoxyItemize}
\item {\ttfamily arr} -\/ An array to iterate over.
\item {\ttfamily iterator(item, callback)} -\/ A function to apply to each item in {\ttfamily arr}. The iterator is passed a {\ttfamily callback(err)} which must be called once it has completed. If no error has occurred, the {\ttfamily callback} should be run without arguments or with an explicit {\ttfamily null} argument. The array index is not passed to the iterator. If you need the index, use \href{#forEachOf}{\tt {\ttfamily for\+Each\+Of}}.
\item {\ttfamily callback(err)} -\/ {\itshape Optional} A callback which is called when all {\ttfamily iterator} functions have finished, or an error occurs.
\end{DoxyItemize}

{\bfseries Examples}


\begin{DoxyCode}
// assuming openFiles is an array of file names and saveFile is a function
// to save the modified contents of that file:

async.each(openFiles, saveFile, function(err)\{
    // if any of the saves produced an error, err would equal that error
\});
\end{DoxyCode}



\begin{DoxyCode}
// assuming openFiles is an array of file names

async.each(openFiles, function(file, callback) \{

  // Perform operation on file here.
  console.log('Processing file ' + file);

  if( file.length > 32 ) \{
    console.log('This file name is too long');
    callback('File name too long');
  \} else \{
    // Do work to process file here
    console.log('File processed');
    callback();
  \}
\}, function(err)\{
    // if any of the file processing produced an error, err would equal that error
    if( err ) \{
      // One of the iterations produced an error.
      // All processing will now stop.
      console.log('A file failed to process');
    \} else \{
      console.log('All files have been processed successfully');
    \}
\});
\end{DoxyCode}


{\bfseries Related}


\begin{DoxyItemize}
\item each\+Series(arr, iterator, \mbox{[}callback\mbox{]})
\item each\+Limit(arr, limit, iterator, \mbox{[}callback\mbox{]}) 


\end{DoxyItemize}

\label{_forEachOf}%
 \label{_eachOf}%


\subsubsection*{for\+Each\+Of(obj, iterator, \mbox{[}callback\mbox{]})}

Like {\ttfamily each}, except that it iterates over objects, and passes the key as the second argument to the iterator.

{\bfseries Arguments}


\begin{DoxyItemize}
\item {\ttfamily obj} -\/ An object or array to iterate over.
\item {\ttfamily iterator(item, key, callback)} -\/ A function to apply to each item in {\ttfamily obj}. The {\ttfamily key} is the item\textquotesingle{}s key, or index in the case of an array. The iterator is passed a {\ttfamily callback(err)} which must be called once it has completed. If no error has occurred, the callback should be run without arguments or with an explicit {\ttfamily null} argument.
\item {\ttfamily callback(err)} -\/ {\itshape Optional} A callback which is called when all {\ttfamily iterator} functions have finished, or an error occurs.
\end{DoxyItemize}

{\bfseries Example}


\begin{DoxyCode}
var obj = \{dev: "/dev.json", test: "/test.json", prod: "/prod.json"\};
var configs = \{\};

async.forEachOf(obj, function (value, key, callback) \{
  fs.readFile(\_\_dirname + value, "utf8", function (err, data) \{
    if (err) return callback(err);
    try \{
      configs[key] = JSON.parse(data);
    \} catch (e) \{
      return callback(e);
    \}
    callback();
  \})
\}, function (err) \{
  if (err) console.error(err.message);
  // configs is now a map of JSON data
  doSomethingWith(configs);
\})
\end{DoxyCode}


{\bfseries Related}


\begin{DoxyItemize}
\item for\+Each\+Of\+Series(obj, iterator, \mbox{[}callback\mbox{]})
\item for\+Each\+Of\+Limit(obj, limit, iterator, \mbox{[}callback\mbox{]}) 


\end{DoxyItemize}

\label{_map}%
 \subsubsection*{map(arr, iterator, \mbox{[}callback\mbox{]})}

Produces a new array of values by mapping each value in {\ttfamily arr} through the {\ttfamily iterator} function. The {\ttfamily iterator} is called with an item from {\ttfamily arr} and a callback for when it has finished processing. Each of these callback takes 2 arguments\+: an {\ttfamily error}, and the transformed item from {\ttfamily arr}. If {\ttfamily iterator} passes an error to its callback, the main {\ttfamily callback} (for the {\ttfamily map} function) is immediately called with the error.

Note, that since this function applies the {\ttfamily iterator} to each item in parallel, there is no guarantee that the {\ttfamily iterator} functions will complete in order. However, the results array will be in the same order as the original {\ttfamily arr}.

{\bfseries Arguments}


\begin{DoxyItemize}
\item {\ttfamily arr} -\/ An array to iterate over.
\item {\ttfamily iterator(item, callback)} -\/ A function to apply to each item in {\ttfamily arr}. The iterator is passed a {\ttfamily callback(err, transformed)} which must be called once it has completed with an error (which can be {\ttfamily null}) and a transformed item.
\item {\ttfamily callback(err, results)} -\/ {\itshape Optional} A callback which is called when all {\ttfamily iterator} functions have finished, or an error occurs. Results is an array of the transformed items from the {\ttfamily arr}.
\end{DoxyItemize}

{\bfseries Example}


\begin{DoxyCode}
async.map(['file1','file2','file3'], fs.stat, function(err, results)\{
    // results is now an array of stats for each file
\});
\end{DoxyCode}


{\bfseries Related}
\begin{DoxyItemize}
\item map\+Series(arr, iterator, \mbox{[}callback\mbox{]})
\item map\+Limit(arr, limit, iterator, \mbox{[}callback\mbox{]}) 


\end{DoxyItemize}

\label{_select}%
 \label{_filter}%
 \subsubsection*{filter(arr, iterator, \mbox{[}callback\mbox{]})}

{\bfseries Alias\+:} {\ttfamily select}

Returns a new array of all the values in {\ttfamily arr} which pass an async truth test. {\itshape The callback for each {\ttfamily iterator} call only accepts a single argument of {\ttfamily true} or {\ttfamily false}; it does not accept an error argument first!} This is in-\/line with the way node libraries work with truth tests like {\ttfamily fs.\+exists}. This operation is performed in parallel, but the results array will be in the same order as the original.

{\bfseries Arguments}


\begin{DoxyItemize}
\item {\ttfamily arr} -\/ An array to iterate over.
\item {\ttfamily iterator(item, callback)} -\/ A truth test to apply to each item in {\ttfamily arr}. The {\ttfamily iterator} is passed a {\ttfamily callback(truth\+Value)}, which must be called with a boolean argument once it has completed.
\item {\ttfamily callback(results)} -\/ {\itshape Optional} A callback which is called after all the {\ttfamily iterator} functions have finished.
\end{DoxyItemize}

{\bfseries Example}


\begin{DoxyCode}
async.filter(['file1','file2','file3'], fs.exists, function(results)\{
    // results now equals an array of the existing files
\});
\end{DoxyCode}


{\bfseries Related}


\begin{DoxyItemize}
\item filter\+Series(arr, iterator, \mbox{[}callback\mbox{]})
\item filter\+Limit(arr, limit, iterator, \mbox{[}callback\mbox{]}) 


\end{DoxyItemize}

\label{_reject}%
 \subsubsection*{reject(arr, iterator, \mbox{[}callback\mbox{]})}

The opposite of \href{#filter}{\tt {\ttfamily filter}}. Removes values that pass an {\ttfamily async} truth test.

{\bfseries Related}


\begin{DoxyItemize}
\item reject\+Series(arr, iterator, \mbox{[}callback\mbox{]})
\item reject\+Limit(arr, limit, iterator, \mbox{[}callback\mbox{]}) 


\end{DoxyItemize}

\label{_reduce}%
 \subsubsection*{reduce(arr, memo, iterator, \mbox{[}callback\mbox{]})}

{\bfseries Aliases\+:} {\ttfamily inject}, {\ttfamily foldl}

Reduces {\ttfamily arr} into a single value using an async {\ttfamily iterator} to return each successive step. {\ttfamily memo} is the initial state of the reduction. This function only operates in series.

For performance reasons, it may make sense to split a call to this function into a parallel map, and then use the normal {\ttfamily Array.\+prototype.\+reduce} on the results. This function is for situations where each step in the reduction needs to be async; if you can get the data before reducing it, then it\textquotesingle{}s probably a good idea to do so.

{\bfseries Arguments}


\begin{DoxyItemize}
\item {\ttfamily arr} -\/ An array to iterate over.
\item {\ttfamily memo} -\/ The initial state of the reduction.
\item {\ttfamily iterator(memo, item, callback)} -\/ A function applied to each item in the array to produce the next step in the reduction. The {\ttfamily iterator} is passed a {\ttfamily callback(err, reduction)} which accepts an optional error as its first argument, and the state of the reduction as the second. If an error is passed to the callback, the reduction is stopped and the main {\ttfamily callback} is immediately called with the error.
\item {\ttfamily callback(err, result)} -\/ {\itshape Optional} A callback which is called after all the {\ttfamily iterator} functions have finished. Result is the reduced value.
\end{DoxyItemize}

{\bfseries Example}


\begin{DoxyCode}
async.reduce([1,2,3], 0, function(memo, item, callback)\{
    // pointless async:
    process.nextTick(function()\{
        callback(null, memo + item)
    \});
\}, function(err, result)\{
    // result is now equal to the last value of memo, which is 6
\});
\end{DoxyCode}
 



\label{_reduceRight}%
 \subsubsection*{reduce\+Right(arr, memo, iterator, \mbox{[}callback\mbox{]})}

{\bfseries Alias\+:} {\ttfamily foldr}

Same as \href{#reduce}{\tt {\ttfamily reduce}}, only operates on {\ttfamily arr} in reverse order.





\label{_detect}%
 \subsubsection*{detect(arr, iterator, \mbox{[}callback\mbox{]})}

Returns the first value in {\ttfamily arr} that passes an async truth test. The {\ttfamily iterator} is applied in parallel, meaning the first iterator to return {\ttfamily true} will fire the detect {\ttfamily callback} with that result. That means the result might not be the first item in the original {\ttfamily arr} (in terms of order) that passes the test.

If order within the original {\ttfamily arr} is important, then look at \href{#detectSeries}{\tt {\ttfamily detect\+Series}}.

{\bfseries Arguments}


\begin{DoxyItemize}
\item {\ttfamily arr} -\/ An array to iterate over.
\item {\ttfamily iterator(item, callback)} -\/ A truth test to apply to each item in {\ttfamily arr}. The iterator is passed a {\ttfamily callback(truth\+Value)} which must be called with a boolean argument once it has completed. {\bfseries Note\+: this callback does not take an error as its first argument.}
\item {\ttfamily callback(result)} -\/ {\itshape Optional} A callback which is called as soon as any iterator returns {\ttfamily true}, or after all the {\ttfamily iterator} functions have finished. Result will be the first item in the array that passes the truth test (iterator) or the value {\ttfamily undefined} if none passed. {\bfseries Note\+: this callback does not take an error as its first argument.}
\end{DoxyItemize}

{\bfseries Example}


\begin{DoxyCode}
async.detect(['file1','file2','file3'], fs.exists, function(result)\{
    // result now equals the first file in the list that exists
\});
\end{DoxyCode}


{\bfseries Related}


\begin{DoxyItemize}
\item detect\+Series(arr, iterator, \mbox{[}callback\mbox{]})
\item detect\+Limit(arr, limit, iterator, \mbox{[}callback\mbox{]}) 


\end{DoxyItemize}

\label{_sortBy}%
 \subsubsection*{sort\+By(arr, iterator, \mbox{[}callback\mbox{]})}

Sorts a list by the results of running each {\ttfamily arr} value through an async {\ttfamily iterator}.

{\bfseries Arguments}


\begin{DoxyItemize}
\item {\ttfamily arr} -\/ An array to iterate over.
\item {\ttfamily iterator(item, callback)} -\/ A function to apply to each item in {\ttfamily arr}. The iterator is passed a {\ttfamily callback(err, sort\+Value)} which must be called once it has completed with an error (which can be {\ttfamily null}) and a value to use as the sort criteria.
\item {\ttfamily callback(err, results)} -\/ {\itshape Optional} A callback which is called after all the {\ttfamily iterator} functions have finished, or an error occurs. Results is the items from the original {\ttfamily arr} sorted by the values returned by the {\ttfamily iterator} calls.
\end{DoxyItemize}

{\bfseries Example}


\begin{DoxyCode}
async.sortBy(['file1','file2','file3'], function(file, callback)\{
    fs.stat(file, function(err, stats)\{
        callback(err, stats.mtime);
    \});
\}, function(err, results)\{
    // results is now the original array of files sorted by
    // modified date
\});
\end{DoxyCode}


{\bfseries Sort Order}

By modifying the callback parameter the sorting order can be influenced\+:


\begin{DoxyCode}
//ascending order
async.sortBy([1,9,3,5], function(x, callback)\{
    callback(null, x);
\}, function(err,result)\{
    //result callback
\} );

//descending order
async.sortBy([1,9,3,5], function(x, callback)\{
    callback(null, x*-1);    //<- x*-1 instead of x, turns the order around
\}, function(err,result)\{
    //result callback
\} );
\end{DoxyCode}
 



\label{_some}%
 \subsubsection*{some(arr, iterator, \mbox{[}callback\mbox{]})}

{\bfseries Alias\+:} {\ttfamily any}

Returns {\ttfamily true} if at least one element in the {\ttfamily arr} satisfies an async test. {\itshape The callback for each iterator call only accepts a single argument of {\ttfamily true} or {\ttfamily false}; it does not accept an error argument first!} This is in-\/line with the way node libraries work with truth tests like {\ttfamily fs.\+exists}. Once any iterator call returns {\ttfamily true}, the main {\ttfamily callback} is immediately called.

{\bfseries Arguments}


\begin{DoxyItemize}
\item {\ttfamily arr} -\/ An array to iterate over.
\item {\ttfamily iterator(item, callback)} -\/ A truth test to apply to each item in the array in parallel. The iterator is passed a {\ttfamily callback(truth\+Value)}\`{} which must be called with a boolean argument once it has completed.
\item {\ttfamily callback(result)} -\/ {\itshape Optional} A callback which is called as soon as any iterator returns {\ttfamily true}, or after all the iterator functions have finished. Result will be either {\ttfamily true} or {\ttfamily false} depending on the values of the async tests.

{\bfseries Note\+: the callbacks do not take an error as their first argument.} {\bfseries Example}
\end{DoxyItemize}


\begin{DoxyCode}
async.some(['file1','file2','file3'], fs.exists, function(result)\{
    // if result is true then at least one of the files exists
\});
\end{DoxyCode}


{\bfseries Related}


\begin{DoxyItemize}
\item some\+Limit(arr, limit, iterator, callback) 


\end{DoxyItemize}

\label{_every}%
 \subsubsection*{every(arr, iterator, \mbox{[}callback\mbox{]})}

{\bfseries Alias\+:} {\ttfamily all}

Returns {\ttfamily true} if every element in {\ttfamily arr} satisfies an async test. {\itshape The callback for each {\ttfamily iterator} call only accepts a single argument of {\ttfamily true} or {\ttfamily false}; it does not accept an error argument first!} This is in-\/line with the way node libraries work with truth tests like {\ttfamily fs.\+exists}.

{\bfseries Arguments}


\begin{DoxyItemize}
\item {\ttfamily arr} -\/ An array to iterate over.
\item {\ttfamily iterator(item, callback)} -\/ A truth test to apply to each item in the array in parallel. The iterator is passed a {\ttfamily callback(truth\+Value)} which must be called with a boolean argument once it has completed.
\item {\ttfamily callback(result)} -\/ {\itshape Optional} A callback which is called as soon as any iterator returns {\ttfamily false}, or after all the iterator functions have finished. Result will be either {\ttfamily true} or {\ttfamily false} depending on the values of the async tests.

{\bfseries Note\+: the callbacks do not take an error as their first argument.}
\end{DoxyItemize}

{\bfseries Example}


\begin{DoxyCode}
async.every(['file1','file2','file3'], fs.exists, function(result)\{
    // if result is true then every file exists
\});
\end{DoxyCode}


{\bfseries Related}


\begin{DoxyItemize}
\item every\+Limit(arr, limit, iterator, callback) 


\end{DoxyItemize}

\label{_concat}%
 \subsubsection*{concat(arr, iterator, \mbox{[}callback\mbox{]})}

Applies {\ttfamily iterator} to each item in {\ttfamily arr}, concatenating the results. Returns the concatenated list. The {\ttfamily iterator}s are called in parallel, and the results are concatenated as they return. There is no guarantee that the results array will be returned in the original order of {\ttfamily arr} passed to the {\ttfamily iterator} function.

{\bfseries Arguments}


\begin{DoxyItemize}
\item {\ttfamily arr} -\/ An array to iterate over.
\item {\ttfamily iterator(item, callback)} -\/ A function to apply to each item in {\ttfamily arr}. The iterator is passed a {\ttfamily callback(err, results)} which must be called once it has completed with an error (which can be {\ttfamily null}) and an array of results.
\item {\ttfamily callback(err, results)} -\/ {\itshape Optional} A callback which is called after all the {\ttfamily iterator} functions have finished, or an error occurs. Results is an array containing the concatenated results of the {\ttfamily iterator} function.
\end{DoxyItemize}

{\bfseries Example}


\begin{DoxyCode}
async.concat(['dir1','dir2','dir3'], fs.readdir, function(err, files)\{
    // files is now a list of filenames that exist in the 3 directories
\});
\end{DoxyCode}


{\bfseries Related}


\begin{DoxyItemize}
\item concat\+Series(arr, iterator, \mbox{[}callback\mbox{]})
\end{DoxyItemize}

\subsection*{Control Flow}

\label{_series}%
 \subsubsection*{series(tasks, \mbox{[}callback\mbox{]})}

Run the functions in the {\ttfamily tasks} array in series, each one running once the previous function has completed. If any functions in the series pass an error to its callback, no more functions are run, and {\ttfamily callback} is immediately called with the value of the error. Otherwise, {\ttfamily callback} receives an array of results when {\ttfamily tasks} have completed.

It is also possible to use an object instead of an array. Each property will be run as a function, and the results will be passed to the final {\ttfamily callback} as an object instead of an array. This can be a more readable way of handling results from \href{#series}{\tt {\ttfamily series}}.

{\bfseries Note} that while many implementations preserve the order of object properties, the \href{http://www.ecma-international.org/ecma-262/5.1/#sec-8.6}{\tt E\+C\+M\+A\+Script Language Specification} explicitly states that

\begin{quote}
The mechanics and order of enumerating the properties is not specified. \end{quote}


So if you rely on the order in which your series of functions are executed, and want this to work on all platforms, consider using an array.

{\bfseries Arguments}


\begin{DoxyItemize}
\item {\ttfamily tasks} -\/ An array or object containing functions to run, each function is passed a {\ttfamily callback(err, result)} it must call on completion with an error {\ttfamily err} (which can be {\ttfamily null}) and an optional {\ttfamily result} value.
\item {\ttfamily callback(err, results)} -\/ An optional callback to run once all the functions have completed. This function gets a results array (or object) containing all the result arguments passed to the {\ttfamily task} callbacks.
\end{DoxyItemize}

{\bfseries Example}


\begin{DoxyCode}
async.series([
    function(callback)\{
        // do some stuff ...
        callback(null, 'one');
    \},
    function(callback)\{
        // do some more stuff ...
        callback(null, 'two');
    \}
],
// optional callback
function(err, results)\{
    // results is now equal to ['one', 'two']
\});


// an example using an object instead of an array
async.series(\{
    one: function(callback)\{
        setTimeout(function()\{
            callback(null, 1);
        \}, 200);
    \},
    two: function(callback)\{
        setTimeout(function()\{
            callback(null, 2);
        \}, 100);
    \}
\},
function(err, results) \{
    // results is now equal to: \{one: 1, two: 2\}
\});
\end{DoxyCode}
 



\label{_parallel}%
 \subsubsection*{parallel(tasks, \mbox{[}callback\mbox{]})}

Run the {\ttfamily tasks} array of functions in parallel, without waiting until the previous function has completed. If any of the functions pass an error to its callback, the main {\ttfamily callback} is immediately called with the value of the error. Once the {\ttfamily tasks} have completed, the results are passed to the final {\ttfamily callback} as an array.

{\bfseries Note\+:} {\ttfamily parallel} is about kicking-\/off I/O tasks in parallel, not about parallel execution of code. If your tasks do not use any timers or perform any I/O, they will actually be executed in series. Any synchronous setup sections for each task will happen one after the other. Java\+Script remains single-\/threaded.

It is also possible to use an object instead of an array. Each property will be run as a function and the results will be passed to the final {\ttfamily callback} as an object instead of an array. This can be a more readable way of handling results from \href{#parallel}{\tt {\ttfamily parallel}}.

{\bfseries Arguments}


\begin{DoxyItemize}
\item {\ttfamily tasks} -\/ An array or object containing functions to run. Each function is passed a {\ttfamily callback(err, result)} which it must call on completion with an error {\ttfamily err} (which can be {\ttfamily null}) and an optional {\ttfamily result} value.
\item {\ttfamily callback(err, results)} -\/ An optional callback to run once all the functions have completed successfully. This function gets a results array (or object) containing all the result arguments passed to the task callbacks.
\end{DoxyItemize}

{\bfseries Example}


\begin{DoxyCode}
async.parallel([
    function(callback)\{
        setTimeout(function()\{
            callback(null, 'one');
        \}, 200);
    \},
    function(callback)\{
        setTimeout(function()\{
            callback(null, 'two');
        \}, 100);
    \}
],
// optional callback
function(err, results)\{
    // the results array will equal ['one','two'] even though
    // the second function had a shorter timeout.
\});


// an example using an object instead of an array
async.parallel(\{
    one: function(callback)\{
        setTimeout(function()\{
            callback(null, 1);
        \}, 200);
    \},
    two: function(callback)\{
        setTimeout(function()\{
            callback(null, 2);
        \}, 100);
    \}
\},
function(err, results) \{
    // results is now equals to: \{one: 1, two: 2\}
\});
\end{DoxyCode}


{\bfseries Related}


\begin{DoxyItemize}
\item parallel\+Limit(tasks, limit, \mbox{[}callback\mbox{]}) 


\end{DoxyItemize}

\label{_whilst}%
 \subsubsection*{whilst(test, fn, callback)}

Repeatedly call {\ttfamily fn}, while {\ttfamily test} returns {\ttfamily true}. Calls {\ttfamily callback} when stopped, or an error occurs.

{\bfseries Arguments}


\begin{DoxyItemize}
\item {\ttfamily test()} -\/ synchronous truth test to perform before each execution of {\ttfamily fn}.
\item {\ttfamily fn(callback)} -\/ A function which is called each time {\ttfamily test} passes. The function is passed a {\ttfamily callback(err)}, which must be called once it has completed with an optional {\ttfamily err} argument.
\item {\ttfamily callback(err, \mbox{[}results\mbox{]})} -\/ A callback which is called after the test function has failed and repeated execution of {\ttfamily fn} has stopped. {\ttfamily callback} will be passed an error and any arguments passed to the final {\ttfamily fn}\textquotesingle{}s callback.
\end{DoxyItemize}

{\bfseries Example}


\begin{DoxyCode}
var count = 0;

async.whilst(
    function () \{ return count < 5; \},
    function (callback) \{
        count++;
        setTimeout(function () \{
            callback(null, count);
        \}, 1000);
    \},
    function (err, n) \{
        // 5 seconds have passed, n = 5
    \}
);
\end{DoxyCode}
 



\label{_doWhilst}%
 \subsubsection*{do\+Whilst(fn, test, callback)}

The post-\/check version of \href{#whilst}{\tt {\ttfamily whilst}}. To reflect the difference in the order of operations, the arguments {\ttfamily test} and {\ttfamily fn} are switched.

{\ttfamily do\+Whilst} is to {\ttfamily whilst} as {\ttfamily do while} is to {\ttfamily while} in plain Java\+Script. 



\label{_until}%
 \subsubsection*{until(test, fn, callback)}

Repeatedly call {\ttfamily fn} until {\ttfamily test} returns {\ttfamily true}. Calls {\ttfamily callback} when stopped, or an error occurs. {\ttfamily callback} will be passed an error and any arguments passed to the final {\ttfamily fn}\textquotesingle{}s callback.

The inverse of \href{#whilst}{\tt {\ttfamily whilst}}. 



\label{_doUntil}%
 \subsubsection*{do\+Until(fn, test, callback)}

Like \href{#doWhilst}{\tt {\ttfamily do\+Whilst}}, except the {\ttfamily test} is inverted. Note the argument ordering differs from {\ttfamily until}. 



\label{_during}%
 \subsubsection*{during(test, fn, callback)}

Like \href{#whilst}{\tt {\ttfamily whilst}}, except the {\ttfamily test} is an asynchronous function that is passed a callback in the form of {\ttfamily function (err, truth)}. If error is passed to {\ttfamily test} or {\ttfamily fn}, the main callback is immediately called with the value of the error.

{\bfseries Example}


\begin{DoxyCode}
var count = 0;

async.during(
    function (callback) \{
      return callback(null, count < 5);
    \},
    function (callback) \{
        count++;
        setTimeout(callback, 1000);
    \},
    function (err) \{
        // 5 seconds have passed
    \}
);
\end{DoxyCode}
 



\label{_doDuring}%
 \subsubsection*{do\+During(fn, test, callback)}

The post-\/check version of \href{#during}{\tt {\ttfamily during}}. To reflect the difference in the order of operations, the arguments {\ttfamily test} and {\ttfamily fn} are switched.

Also a version of \href{#doWhilst}{\tt {\ttfamily do\+Whilst}} with asynchronous {\ttfamily test} function. 



\label{_forever}%
 \subsubsection*{forever(fn, \mbox{[}errback\mbox{]})}

Calls the asynchronous function {\ttfamily fn} with a callback parameter that allows it to call itself again, in series, indefinitely.

If an error is passed to the callback then {\ttfamily errback} is called with the error, and execution stops, otherwise it will never be called.


\begin{DoxyCode}
async.forever(
    function(next) \{
        // next is suitable for passing to things that need a callback(err [, whatever]);
        // it will result in this function being called again.
    \},
    function(err) \{
        // if next is called with a value in its first parameter, it will appear
        // in here as 'err', and execution will stop.
    \}
);
\end{DoxyCode}
 



\label{_waterfall}%
 \subsubsection*{waterfall(tasks, \mbox{[}callback\mbox{]})}

Runs the {\ttfamily tasks} array of functions in series, each passing their results to the next in the array. However, if any of the {\ttfamily tasks} pass an error to their own callback, the next function is not executed, and the main {\ttfamily callback} is immediately called with the error.

{\bfseries Arguments}


\begin{DoxyItemize}
\item {\ttfamily tasks} -\/ An array of functions to run, each function is passed a {\ttfamily callback(err, result1, result2, ...)} it must call on completion. The first argument is an error (which can be {\ttfamily null}) and any further arguments will be passed as arguments in order to the next task.
\item {\ttfamily callback(err, \mbox{[}results\mbox{]})} -\/ An optional callback to run once all the functions have completed. This will be passed the results of the last task\textquotesingle{}s callback.
\end{DoxyItemize}

{\bfseries Example}


\begin{DoxyCode}
async.waterfall([
    function(callback) \{
        callback(null, 'one', 'two');
    \},
    function(arg1, arg2, callback) \{
      // arg1 now equals 'one' and arg2 now equals 'two'
        callback(null, 'three');
    \},
    function(arg1, callback) \{
        // arg1 now equals 'three'
        callback(null, 'done');
    \}
], function (err, result) \{
    // result now equals 'done'
\});
\end{DoxyCode}
 Or, with named functions\+:


\begin{DoxyCode}
async.waterfall([
    myFirstFunction,
    mySecondFunction,
    myLastFunction,
], function (err, result) \{
    // result now equals 'done'
\});
function myFirstFunction(callback) \{
  callback(null, 'one', 'two');
\}
function mySecondFunction(arg1, arg2, callback) \{
  // arg1 now equals 'one' and arg2 now equals 'two'
  callback(null, 'three');
\}
function myLastFunction(arg1, callback) \{
  // arg1 now equals 'three'
  callback(null, 'done');
\}
\end{DoxyCode}


Or, if you need to pass any argument to the first function\+:


\begin{DoxyCode}
async.waterfall([
    async.apply(myFirstFunction, 'zero'),
    mySecondFunction,
    myLastFunction,
], function (err, result) \{
    // result now equals 'done'
\});
function myFirstFunction(arg1, callback) \{
  // arg1 now equals 'zero'
  callback(null, 'one', 'two');
\}
function mySecondFunction(arg1, arg2, callback) \{
  // arg1 now equals 'one' and arg2 now equals 'two'
  callback(null, 'three');
\}
function myLastFunction(arg1, callback) \{
  // arg1 now equals 'three'
  callback(null, 'done');
\}
\end{DoxyCode}
 

 \label{_compose}%
 \subsubsection*{compose(fn1, fn2...)}

Creates a function which is a composition of the passed asynchronous functions. Each function consumes the return value of the function that follows. Composing functions {\ttfamily f()}, {\ttfamily g()}, and {\ttfamily h()} would produce the result of {\ttfamily f(g(h()))}, only this version uses callbacks to obtain the return values.

Each function is executed with the {\ttfamily this} binding of the composed function.

{\bfseries Arguments}


\begin{DoxyItemize}
\item {\ttfamily functions...} -\/ the asynchronous functions to compose
\end{DoxyItemize}

{\bfseries Example}


\begin{DoxyCode}
function add1(n, callback) \{
    setTimeout(function () \{
        callback(null, n + 1);
    \}, 10);
\}

function mul3(n, callback) \{
    setTimeout(function () \{
        callback(null, n * 3);
    \}, 10);
\}

var add1mul3 = async.compose(mul3, add1);

add1mul3(4, function (err, result) \{
   // result now equals 15
\});
\end{DoxyCode}
 

 \label{_seq}%
 \subsubsection*{seq(fn1, fn2...)}

Version of the compose function that is more natural to read. Each function consumes the return value of the previous function. It is the equivalent of \href{#compose}{\tt {\ttfamily compose}} with the arguments reversed.

Each function is executed with the {\ttfamily this} binding of the composed function.

{\bfseries Arguments}


\begin{DoxyItemize}
\item {\ttfamily functions...} -\/ the asynchronous functions to compose
\end{DoxyItemize}

{\bfseries Example}


\begin{DoxyCode}
// Requires lodash (or underscore), express3 and dresende's orm2.
// Part of an app, that fetches cats of the logged user.
// This example uses `seq` function to avoid overnesting and error
// handling clutter.
app.get('/cats', function(request, response) \{
  var User = request.models.User;
  async.seq(
    \_.bind(User.get, User),  // 'User.get' has signature (id, callback(err, data))
    function(user, fn) \{
      user.getCats(fn);      // 'getCats' has signature (callback(err, data))
    \}
  )(req.session.user\_id, function (err, cats) \{
    if (err) \{
      console.error(err);
      response.json(\{ status: 'error', message: err.message \});
    \} else \{
      response.json(\{ status: 'ok', message: 'Cats found', data: cats \});
    \}
  \});
\});
\end{DoxyCode}
 

 \label{_applyEach}%
 \subsubsection*{apply\+Each(fns, args..., callback)}

Applies the provided arguments to each function in the array, calling {\ttfamily callback} after all functions have completed. If you only provide the first argument, then it will return a function which lets you pass in the arguments as if it were a single function call.

{\bfseries Arguments}


\begin{DoxyItemize}
\item {\ttfamily fns} -\/ the asynchronous functions to all call with the same arguments
\item {\ttfamily args...} -\/ any number of separate arguments to pass to the function
\item {\ttfamily callback} -\/ the final argument should be the callback, called when all functions have completed processing
\end{DoxyItemize}

{\bfseries Example}


\begin{DoxyCode}
async.applyEach([enableSearch, updateSchema], 'bucket', callback);

// partial application example:
async.each(
    buckets,
    async.applyEach([enableSearch, updateSchema]),
    callback
);
\end{DoxyCode}


{\bfseries Related}


\begin{DoxyItemize}
\item apply\+Each\+Series(tasks, args..., \mbox{[}callback\mbox{]}) 


\end{DoxyItemize}

\label{_queue}%
 \subsubsection*{queue(worker, \mbox{[}concurrency\mbox{]})}

Creates a {\ttfamily queue} object with the specified {\ttfamily concurrency}. Tasks added to the {\ttfamily queue} are processed in parallel (up to the {\ttfamily concurrency} limit). If all {\ttfamily worker}s are in progress, the task is queued until one becomes available. Once a {\ttfamily worker} completes a {\ttfamily task}, that {\ttfamily task}\textquotesingle{}s callback is called.

{\bfseries Arguments}


\begin{DoxyItemize}
\item {\ttfamily worker(task, callback)} -\/ An asynchronous function for processing a queued task, which must call its {\ttfamily callback(err)} argument when finished, with an optional {\ttfamily error} as an argument. If you want to handle errors from an individual task, pass a callback to {\ttfamily q.\+push()}.
\item {\ttfamily concurrency} -\/ An {\ttfamily integer} for determining how many {\ttfamily worker} functions should be run in parallel. If omitted, the concurrency defaults to {\ttfamily 1}. If the concurrency is {\ttfamily 0}, an error is thrown.
\end{DoxyItemize}

{\bfseries Queue objects}

The {\ttfamily queue} object returned by this function has the following properties and methods\+:


\begin{DoxyItemize}
\item {\ttfamily length()} -\/ a function returning the number of items waiting to be processed.
\item {\ttfamily started} -\/ a function returning whether or not any items have been pushed and processed by the queue
\item {\ttfamily running()} -\/ a function returning the number of items currently being processed.
\item {\ttfamily workers\+List()} -\/ a function returning the array of items currently being processed.
\item {\ttfamily idle()} -\/ a function returning false if there are items waiting or being processed, or true if not.
\item {\ttfamily concurrency} -\/ an integer for determining how many {\ttfamily worker} functions should be run in parallel. This property can be changed after a {\ttfamily queue} is created to alter the concurrency on-\/the-\/fly.
\item {\ttfamily push(task, \mbox{[}callback\mbox{]})} -\/ add a new task to the {\ttfamily queue}. Calls {\ttfamily callback} once the {\ttfamily worker} has finished processing the task. Instead of a single task, a {\ttfamily tasks} array can be submitted. The respective callback is used for every task in the list.
\item {\ttfamily unshift(task, \mbox{[}callback\mbox{]})} -\/ add a new task to the front of the {\ttfamily queue}.
\item {\ttfamily saturated} -\/ a callback that is called when the {\ttfamily queue} length hits the {\ttfamily concurrency} limit, and further tasks will be queued.
\item {\ttfamily empty} -\/ a callback that is called when the last item from the {\ttfamily queue} is given to a {\ttfamily worker}.
\item {\ttfamily drain} -\/ a callback that is called when the last item from the {\ttfamily queue} has returned from the {\ttfamily worker}.
\item {\ttfamily paused} -\/ a boolean for determining whether the queue is in a paused state
\item {\ttfamily pause()} -\/ a function that pauses the processing of tasks until {\ttfamily resume()} is called.
\item {\ttfamily resume()} -\/ a function that resumes the processing of queued tasks when the queue is paused.
\item {\ttfamily kill()} -\/ a function that removes the {\ttfamily drain} callback and empties remaining tasks from the queue forcing it to go idle.
\end{DoxyItemize}

{\bfseries Example}


\begin{DoxyCode}
// create a queue object with concurrency 2

var q = async.queue(function (task, callback) \{
    console.log('hello ' + task.name);
    callback();
\}, 2);


// assign a callback
q.drain = function() \{
    console.log('all items have been processed');
\}

// add some items to the queue

q.push(\{name: 'foo'\}, function (err) \{
    console.log('finished processing foo');
\});
q.push(\{name: 'bar'\}, function (err) \{
    console.log('finished processing bar');
\});

// add some items to the queue (batch-wise)

q.push([\{name: 'baz'\},\{name: 'bay'\},\{name: 'bax'\}], function (err) \{
    console.log('finished processing item');
\});

// add some items to the front of the queue

q.unshift(\{name: 'bar'\}, function (err) \{
    console.log('finished processing bar');
\});
\end{DoxyCode}






\label{_priorityQueue}%
 \subsubsection*{priority\+Queue(worker, concurrency)}

The same as \href{#queue}{\tt {\ttfamily queue}} only tasks are assigned a priority and completed in ascending priority order. There are two differences between {\ttfamily queue} and {\ttfamily priority\+Queue} objects\+:


\begin{DoxyItemize}
\item {\ttfamily push(task, priority, \mbox{[}callback\mbox{]})} -\/ {\ttfamily priority} should be a number. If an array of {\ttfamily tasks} is given, all tasks will be assigned the same priority.
\item The {\ttfamily unshift} method was removed. 


\end{DoxyItemize}

\label{_cargo}%
 \subsubsection*{cargo(worker, \mbox{[}payload\mbox{]})}

Creates a {\ttfamily cargo} object with the specified payload. Tasks added to the cargo will be processed altogether (up to the {\ttfamily payload} limit). If the {\ttfamily worker} is in progress, the task is queued until it becomes available. Once the {\ttfamily worker} has completed some tasks, each callback of those tasks is called. Check out \href{https://camo.githubusercontent.com/6bbd36f4cf5b35a0f11a96dcd2e97711ffc2fb37/68747470733a2f2f662e636c6f75642e6769746875622e636f6d2f6173736574732f313637363837312f36383130382f62626330636662302d356632392d313165322d393734662d3333393763363464633835382e676966}{\tt these} \href{https://camo.githubusercontent.com/f4810e00e1c5f5f8addbe3e9f49064fd5d102699/68747470733a2f2f662e636c6f75642e6769746875622e636f6d2f6173736574732f313637363837312f36383130312f38346339323036362d356632392d313165322d383134662d3964336430323431336266642e676966}{\tt animations} for how {\ttfamily cargo} and {\ttfamily queue} work.

While \href{#queue}{\tt queue} passes only one task to one of a group of workers at a time, cargo passes an array of tasks to a single worker, repeating when the worker is finished.

{\bfseries Arguments}


\begin{DoxyItemize}
\item {\ttfamily worker(tasks, callback)} -\/ An asynchronous function for processing an array of queued tasks, which must call its {\ttfamily callback(err)} argument when finished, with an optional {\ttfamily err} argument.
\item {\ttfamily payload} -\/ An optional {\ttfamily integer} for determining how many tasks should be processed per round; if omitted, the default is unlimited.
\end{DoxyItemize}

{\bfseries Cargo objects}

The {\ttfamily cargo} object returned by this function has the following properties and methods\+:


\begin{DoxyItemize}
\item {\ttfamily length()} -\/ A function returning the number of items waiting to be processed.
\item {\ttfamily payload} -\/ An {\ttfamily integer} for determining how many tasks should be process per round. This property can be changed after a {\ttfamily cargo} is created to alter the payload on-\/the-\/fly.
\item {\ttfamily push(task, \mbox{[}callback\mbox{]})} -\/ Adds {\ttfamily task} to the {\ttfamily queue}. The callback is called once the {\ttfamily worker} has finished processing the task. Instead of a single task, an array of {\ttfamily tasks} can be submitted. The respective callback is used for every task in the list.
\item {\ttfamily saturated} -\/ A callback that is called when the {\ttfamily queue.\+length()} hits the concurrency and further tasks will be queued.
\item {\ttfamily empty} -\/ A callback that is called when the last item from the {\ttfamily queue} is given to a {\ttfamily worker}.
\item {\ttfamily drain} -\/ A callback that is called when the last item from the {\ttfamily queue} has returned from the {\ttfamily worker}.
\item {\ttfamily idle()}, {\ttfamily pause()}, {\ttfamily resume()}, {\ttfamily kill()} -\/ cargo inherits all of the same methods and event calbacks as \href{#queue}{\tt {\ttfamily queue}}
\end{DoxyItemize}

{\bfseries Example}


\begin{DoxyCode}
// create a cargo object with payload 2

var cargo = async.cargo(function (tasks, callback) \{
    for(var i=0; i<tasks.length; i++)\{
      console.log('hello ' + tasks[i].name);
    \}
    callback();
\}, 2);


// add some items

cargo.push(\{name: 'foo'\}, function (err) \{
    console.log('finished processing foo');
\});
cargo.push(\{name: 'bar'\}, function (err) \{
    console.log('finished processing bar');
\});
cargo.push(\{name: 'baz'\}, function (err) \{
    console.log('finished processing baz');
\});
\end{DoxyCode}
 



\label{_auto}%
 \subsubsection*{auto(tasks, \mbox{[}concurrency\mbox{]}, \mbox{[}callback\mbox{]})}

Determines the best order for running the functions in {\ttfamily tasks}, based on their requirements. Each function can optionally depend on other functions being completed first, and each function is run as soon as its requirements are satisfied.

If any of the functions pass an error to their callback, the {\ttfamily auto} sequence will stop. Further tasks will not execute (so any other functions depending on it will not run), and the main {\ttfamily callback} is immediately called with the error. Functions also receive an object containing the results of functions which have completed so far.

Note, all functions are called with a {\ttfamily results} object as a second argument, so it is unsafe to pass functions in the {\ttfamily tasks} object which cannot handle the extra argument.

For example, this snippet of code\+:


\begin{DoxyCode}
async.auto(\{
  readData: async.apply(fs.readFile, 'data.txt', 'utf-8')
\}, callback);
\end{DoxyCode}


will have the effect of calling {\ttfamily read\+File} with the results object as the last argument, which will fail\+:


\begin{DoxyCode}
fs.readFile('data.txt', 'utf-8', cb, \{\});
\end{DoxyCode}


Instead, wrap the call to {\ttfamily read\+File} in a function which does not forward the {\ttfamily results} object\+:


\begin{DoxyCode}
async.auto(\{
  readData: function(cb, results)\{
    fs.readFile('data.txt', 'utf-8', cb);
  \}
\}, callback);
\end{DoxyCode}


{\bfseries Arguments}


\begin{DoxyItemize}
\item {\ttfamily tasks} -\/ An object. Each of its properties is either a function or an array of requirements, with the function itself the last item in the array. The object\textquotesingle{}s key of a property serves as the name of the task defined by that property, i.\+e. can be used when specifying requirements for other tasks. The function receives two arguments\+: (1) a {\ttfamily callback(err, result)} which must be called when finished, passing an {\ttfamily error} (which can be {\ttfamily null}) and the result of the function\textquotesingle{}s execution, and (2) a {\ttfamily results} object, containing the results of the previously executed functions.
\item {\ttfamily concurrency} -\/ An optional {\ttfamily integer} for determining the maximum number of tasks that can be run in parallel. By default, as many as possible.
\item {\ttfamily callback(err, results)} -\/ An optional callback which is called when all the tasks have been completed. It receives the {\ttfamily err} argument if any {\ttfamily tasks} pass an error to their callback. Results are always returned; however, if an error occurs, no further {\ttfamily tasks} will be performed, and the results object will only contain partial results.
\end{DoxyItemize}

{\bfseries Example}


\begin{DoxyCode}
async.auto(\{
    get\_data: function(callback)\{
        console.log('in get\_data');
        // async code to get some data
        callback(null, 'data', 'converted to array');
    \},
    make\_folder: function(callback)\{
        console.log('in make\_folder');
        // async code to create a directory to store a file in
        // this is run at the same time as getting the data
        callback(null, 'folder');
    \},
    write\_file: ['get\_data', 'make\_folder', function(callback, results)\{
        console.log('in write\_file', JSON.stringify(results));
        // once there is some data and the directory exists,
        // write the data to a file in the directory
        callback(null, 'filename');
    \}],
    email\_link: ['write\_file', function(callback, results)\{
        console.log('in email\_link', JSON.stringify(results));
        // once the file is written let's email a link to it...
        // results.write\_file contains the filename returned by write\_file.
        callback(null, \{'file':results.write\_file, 'email':'user@example.com'\});
    \}]
\}, function(err, results) \{
    console.log('err = ', err);
    console.log('results = ', results);
\});
\end{DoxyCode}


This is a fairly trivial example, but to do this using the basic parallel and series functions would look like this\+:


\begin{DoxyCode}
async.parallel([
    function(callback)\{
        console.log('in get\_data');
        // async code to get some data
        callback(null, 'data', 'converted to array');
    \},
    function(callback)\{
        console.log('in make\_folder');
        // async code to create a directory to store a file in
        // this is run at the same time as getting the data
        callback(null, 'folder');
    \}
],
function(err, results)\{
    async.series([
        function(callback)\{
            console.log('in write\_file', JSON.stringify(results));
            // once there is some data and the directory exists,
            // write the data to a file in the directory
            results.push('filename');
            callback(null);
        \},
        function(callback)\{
            console.log('in email\_link', JSON.stringify(results));
            // once the file is written let's email a link to it...
            callback(null, \{'file':results.pop(), 'email':'user@example.com'\});
        \}
    ]);
\});
\end{DoxyCode}


For a complicated series of {\ttfamily async} tasks, using the \href{#auto}{\tt {\ttfamily auto}} function makes adding new tasks much easier (and the code more readable).





\label{_retry}%
 \subsubsection*{retry(\mbox{[}opts = \{times\+: 5, interval\+: 0\}$\vert$ 5\mbox{]}, task, \mbox{[}callback\mbox{]})}

Attempts to get a successful response from {\ttfamily task} no more than {\ttfamily times} times before returning an error. If the task is successful, the {\ttfamily callback} will be passed the result of the successful task. If all attempts fail, the callback will be passed the error and result (if any) of the final attempt.

{\bfseries Arguments}


\begin{DoxyItemize}
\item {\ttfamily opts} -\/ Can be either an object with {\ttfamily times} and {\ttfamily interval} or a number.
\begin{DoxyItemize}
\item {\ttfamily times} -\/ The number of attempts to make before giving up. The default is {\ttfamily 5}.
\item {\ttfamily interval} -\/ The time to wait between retries, in milliseconds. The default is {\ttfamily 0}.
\item If {\ttfamily opts} is a number, the number specifies the number of times to retry, with the default interval of {\ttfamily 0}.
\end{DoxyItemize}
\item {\ttfamily task(callback, results)} -\/ A function which receives two arguments\+: (1) a {\ttfamily callback(err, result)} which must be called when finished, passing {\ttfamily err} (which can be {\ttfamily null}) and the {\ttfamily result} of the function\textquotesingle{}s execution, and (2) a {\ttfamily results} object, containing the results of the previously executed functions (if nested inside another control flow).
\item {\ttfamily callback(err, results)} -\/ An optional callback which is called when the task has succeeded, or after the final failed attempt. It receives the {\ttfamily err} and {\ttfamily result} arguments of the last attempt at completing the {\ttfamily task}.
\end{DoxyItemize}

The \href{#retry}{\tt {\ttfamily retry}} function can be used as a stand-\/alone control flow by passing a callback, as shown below\+:


\begin{DoxyCode}
// try calling apiMethod 3 times
async.retry(3, apiMethod, function(err, result) \{
    // do something with the result
\});
\end{DoxyCode}



\begin{DoxyCode}
// try calling apiMethod 3 times, waiting 200 ms between each retry 
async.retry(\{times: 3, interval: 200\}, apiMethod, function(err, result) \{
    // do something with the result
\});
\end{DoxyCode}



\begin{DoxyCode}
// try calling apiMethod the default 5 times no delay between each retry 
async.retry(apiMethod, function(err, result) \{
    // do something with the result
\});
\end{DoxyCode}


It can also be embedded within other control flow functions to retry individual methods that are not as reliable, like this\+:


\begin{DoxyCode}
async.auto(\{
    users: api.getUsers.bind(api),
    payments: async.retry(3, api.getPayments.bind(api))
\}, function(err, results) \{
  // do something with the results
\});
\end{DoxyCode}






\label{_iterator}%
 \subsubsection*{iterator(tasks)}

Creates an iterator function which calls the next function in the {\ttfamily tasks} array, returning a continuation to call the next one after that. It\textquotesingle{}s also possible to “peek” at the next iterator with {\ttfamily iterator.\+next()}.

This function is used internally by the {\ttfamily async} module, but can be useful when you want to manually control the flow of functions in series.

{\bfseries Arguments}


\begin{DoxyItemize}
\item {\ttfamily tasks} -\/ An array of functions to run.
\end{DoxyItemize}

{\bfseries Example}


\begin{DoxyCode}
var iterator = async.iterator([
    function()\{ sys.p('one'); \},
    function()\{ sys.p('two'); \},
    function()\{ sys.p('three'); \}
]);

node> var iterator2 = iterator();
'one'
node> var iterator3 = iterator2();
'two'
node> iterator3();
'three'
node> var nextfn = iterator2.next();
node> nextfn();
'three'
\end{DoxyCode}
 



\label{_apply}%
 \subsubsection*{apply(function, arguments..)}

Creates a continuation function with some arguments already applied.

Useful as a shorthand when combined with other control flow functions. Any arguments passed to the returned function are added to the arguments originally passed to apply.

{\bfseries Arguments}


\begin{DoxyItemize}
\item {\ttfamily function} -\/ The function you want to eventually apply all arguments to.
\item {\ttfamily arguments...} -\/ Any number of arguments to automatically apply when the continuation is called.
\end{DoxyItemize}

{\bfseries Example}


\begin{DoxyCode}
// using apply

async.parallel([
    async.apply(fs.writeFile, 'testfile1', 'test1'),
    async.apply(fs.writeFile, 'testfile2', 'test2'),
]);


// the same process without using apply

async.parallel([
    function(callback)\{
        fs.writeFile('testfile1', 'test1', callback);
    \},
    function(callback)\{
        fs.writeFile('testfile2', 'test2', callback);
    \}
]);
\end{DoxyCode}


It\textquotesingle{}s possible to pass any number of additional arguments when calling the continuation\+:


\begin{DoxyCode}
node> var fn = async.apply(sys.puts, 'one');
node> fn('two', 'three');
one
two
three
\end{DoxyCode}
 



\label{_nextTick}%
 \subsubsection*{next\+Tick(callback), set\+Immediate(callback)}

Calls {\ttfamily callback} on a later loop around the event loop. In Node.\+js this just calls {\ttfamily process.\+next\+Tick}; in the browser it falls back to {\ttfamily set\+Immediate(callback)} if available, otherwise {\ttfamily set\+Timeout(callback, 0)}, which means other higher priority events may precede the execution of {\ttfamily callback}.

This is used internally for browser-\/compatibility purposes.

{\bfseries Arguments}


\begin{DoxyItemize}
\item {\ttfamily callback} -\/ The function to call on a later loop around the event loop.
\end{DoxyItemize}

{\bfseries Example}


\begin{DoxyCode}
var call\_order = [];
async.nextTick(function()\{
    call\_order.push('two');
    // call\_order now equals ['one','two']
\});
call\_order.push('one')
\end{DoxyCode}


\label{_times}%
 \subsubsection*{times(n, iterator, \mbox{[}callback\mbox{]})}

Calls the {\ttfamily iterator} function {\ttfamily n} times, and accumulates results in the same manner you would use with \href{#map}{\tt {\ttfamily map}}.

{\bfseries Arguments}


\begin{DoxyItemize}
\item {\ttfamily n} -\/ The number of times to run the function.
\item {\ttfamily iterator} -\/ The function to call {\ttfamily n} times.
\item {\ttfamily callback} -\/ see \href{#map}{\tt {\ttfamily map}}
\end{DoxyItemize}

{\bfseries Example}


\begin{DoxyCode}
// Pretend this is some complicated async factory
var createUser = function(id, callback) \{
  callback(null, \{
    id: 'user' + id
  \})
\}
// generate 5 users
async.times(5, function(n, next)\{
    createUser(n, function(err, user) \{
      next(err, user)
    \})
\}, function(err, users) \{
  // we should now have 5 users
\});
\end{DoxyCode}


{\bfseries Related}


\begin{DoxyItemize}
\item times\+Series(n, iterator, \mbox{[}callback\mbox{]})
\item times\+Limit(n, limit, iterator, \mbox{[}callback\mbox{]})
\end{DoxyItemize}

\subsection*{Utils}

\label{_memoize}%
 \subsubsection*{memoize(fn, \mbox{[}hasher\mbox{]})}

Caches the results of an {\ttfamily async} function. When creating a hash to store function results against, the callback is omitted from the hash and an optional hash function can be used.

If no hash function is specified, the first argument is used as a hash key, which may work reasonably if it is a string or a data type that converts to a distinct string. Note that objects and arrays will not behave reasonably. Neither will cases where the other arguments are significant. In such cases, specify your own hash function.

The cache of results is exposed as the {\ttfamily memo} property of the function returned by {\ttfamily memoize}.

{\bfseries Arguments}


\begin{DoxyItemize}
\item {\ttfamily fn} -\/ The function to proxy and cache results from.
\item {\ttfamily hasher} -\/ An optional function for generating a custom hash for storing results. It has all the arguments applied to it apart from the callback, and must be synchronous.
\end{DoxyItemize}

{\bfseries Example}


\begin{DoxyCode}
var slow\_fn = function (name, callback) \{
    // do something
    callback(null, result);
\};
var fn = async.memoize(slow\_fn);

// fn can now be used as if it were slow\_fn
fn('some name', function () \{
    // callback
\});
\end{DoxyCode}


\label{_unmemoize}%
 \subsubsection*{unmemoize(fn)}

Undoes a \href{#memoize}{\tt {\ttfamily memoize}}d function, reverting it to the original, unmemoized form. Handy for testing.

{\bfseries Arguments}


\begin{DoxyItemize}
\item {\ttfamily fn} -\/ the memoized function 


\end{DoxyItemize}

\label{_ensureAsync}%
 \subsubsection*{ensure\+Async(fn)}

Wrap an async function and ensure it calls its callback on a later tick of the event loop. If the function already calls its callback on a next tick, no extra deferral is added. This is useful for preventing stack overflows ({\ttfamily Range\+Error\+: Maximum call stack size exceeded}) and generally keeping \href{http://blog.izs.me/post/59142742143/designing-apis-for-asynchrony}{\tt Zalgo} contained.

{\bfseries Arguments}


\begin{DoxyItemize}
\item {\ttfamily fn} -\/ an async function, one that expects a node-\/style callback as its last argument
\end{DoxyItemize}

Returns a wrapped function with the exact same call signature as the function passed in.

{\bfseries Example}


\begin{DoxyCode}
function sometimesAsync(arg, callback) \{
  if (cache[arg]) \{
    return callback(null, cache[arg]); // this would be synchronous!!
  \} else \{
    doSomeIO(arg, callback); // this IO would be asynchronous
  \}
\}

// this has a risk of stack overflows if many results are cached in a row
async.mapSeries(args, sometimesAsync, done);

// this will defer sometimesAsync's callback if necessary,
// preventing stack overflows
async.mapSeries(args, async.ensureAsync(sometimesAsync), done);
\end{DoxyCode}
 



\label{_constant}%
 \subsubsection*{constant(values...)}

Returns a function that when called, calls-\/back with the values provided. Useful as the first function in a {\ttfamily waterfall}, or for plugging values in to {\ttfamily auto}.

{\bfseries Example}


\begin{DoxyCode}
async.waterfall([
  async.constant(42),
  function (value, next) \{
    // value === 42
  \},
  //...
], callback);

async.waterfall([
  async.constant(filename, "utf8"),
  fs.readFile,
  function (fileData, next) \{
    //...
  \}
  //...
], callback);

async.auto(\{
  hostname: async.constant("https://server.net/"),
  port: findFreePort,
  launchServer: ["hostname", "port", function (cb, options) \{
    startServer(options, cb);
  \}],
  //...
\}, callback);
\end{DoxyCode}
 



\label{_asyncify}%
 \label{_wrapSync}%
 \subsubsection*{asyncify(func)}

{\bfseries Alias\+:} {\ttfamily wrap\+Sync}

Take a sync function and make it async, passing its return value to a callback. This is useful for plugging sync functions into a waterfall, series, or other async functions. Any arguments passed to the generated function will be passed to the wrapped function (except for the final callback argument). Errors thrown will be passed to the callback.

{\bfseries Example}


\begin{DoxyCode}
async.waterfall([
  async.apply(fs.readFile, filename, "utf8"),
  async.asyncify(JSON.parse),
  function (data, next) \{
    // data is the result of parsing the text.
    // If there was a parsing error, it would have been caught.
  \}
], callback)
\end{DoxyCode}


If the function passed to {\ttfamily asyncify} returns a Promise, that promises\textquotesingle{}s resolved/rejected state will be used to call the callback, rather than simply the synchronous return value. Example\+:


\begin{DoxyCode}
async.waterfall([
  async.apply(fs.readFile, filename, "utf8"),
  async.asyncify(function (contents) \{
    return db.model.create(contents);
  \}),
  function (model, next) \{
    // `model` is the instantiated model object. 
    // If there was an error, this function would be skipped.
  \}
], callback)
\end{DoxyCode}


This also means you can asyncify E\+S2016 {\ttfamily async} functions.


\begin{DoxyCode}
var q = async.queue(async.asyncify(async function (file) \{
  var intermediateStep = await processFile(file);
  return await somePromise(intermediateStep)
\}));

q.push(files);
\end{DoxyCode}
 



\label{_log}%
 \subsubsection*{log(function, arguments)}

Logs the result of an {\ttfamily async} function to the {\ttfamily console}. Only works in Node.\+js or in browsers that support {\ttfamily console.\+log} and {\ttfamily console.\+error} (such as FF and Chrome). If multiple arguments are returned from the async function, {\ttfamily console.\+log} is called on each argument in order.

{\bfseries Arguments}


\begin{DoxyItemize}
\item {\ttfamily function} -\/ The function you want to eventually apply all arguments to.
\item {\ttfamily arguments...} -\/ Any number of arguments to apply to the function.
\end{DoxyItemize}

{\bfseries Example}


\begin{DoxyCode}
var hello = function(name, callback)\{
    setTimeout(function()\{
        callback(null, 'hello ' + name);
    \}, 1000);
\};
\end{DoxyCode}
 
\begin{DoxyCode}
node> async.log(hello, 'world');
'hello world'
\end{DoxyCode}
 



\label{_dir}%
 \subsubsection*{dir(function, arguments)}

Logs the result of an {\ttfamily async} function to the {\ttfamily console} using {\ttfamily console.\+dir} to display the properties of the resulting object. Only works in Node.\+js or in browsers that support {\ttfamily console.\+dir} and {\ttfamily console.\+error} (such as FF and Chrome). If multiple arguments are returned from the async function, {\ttfamily console.\+dir} is called on each argument in order.

{\bfseries Arguments}


\begin{DoxyItemize}
\item {\ttfamily function} -\/ The function you want to eventually apply all arguments to.
\item {\ttfamily arguments...} -\/ Any number of arguments to apply to the function.
\end{DoxyItemize}

{\bfseries Example}


\begin{DoxyCode}
var hello = function(name, callback)\{
    setTimeout(function()\{
        callback(null, \{hello: name\});
    \}, 1000);
\};
\end{DoxyCode}
 
\begin{DoxyCode}
node> async.dir(hello, 'world');
\{hello: 'world'\}
\end{DoxyCode}
 



\label{_noConflict}%
 \subsubsection*{no\+Conflict()}

Changes the value of {\ttfamily async} back to its original value, returning a reference to the {\ttfamily async} object. 