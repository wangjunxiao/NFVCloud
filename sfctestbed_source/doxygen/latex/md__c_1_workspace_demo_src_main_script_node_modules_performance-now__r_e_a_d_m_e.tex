Implements a function similar to {\ttfamily performance.\+now} (based on {\ttfamily process.\+hrtime}).

Modern browsers have a {\ttfamily window.\+performance} object with -\/ among others -\/ a {\ttfamily now} method which gives time in miliseconds, but with sub-\/milisecond precision. This module offers the same function based on the Node.\+js native {\ttfamily process.\+hrtime} function.

According to the \href{http://www.w3.org/TR/hr-time/}{\tt High Resolution Time specification}, the number of miliseconds reported by {\ttfamily performance.\+now} should be relative to the value of {\ttfamily performance.\+timing.\+navigation\+Start}. For this module, it\textquotesingle{}s relative to when the time when this module got loaded. Right after requiring this module for the first time, the reported time is expected to have a near-\/zero value.

Using {\ttfamily process.\+hrtime} means that the reported time will be monotonically increasing, and not subject to clock-\/drift.

\subsection*{Example usage}


\begin{DoxyCode}
var now = require("performance-now")
var start = now()
var end = now()
console.log(start.toFixed(3)) // ~ 0.05 on my system
console.log((start-end).toFixed(3)) // ~ 0.002 on my system
\end{DoxyCode}


Running the now function two times right after each other yields a time difference of a few microseconds. Given this overhead, I think it\textquotesingle{}s best to assume that the precision of intervals computed with this method is not higher than 10 microseconds, if you don\textquotesingle{}t know the exact overhead on your own system.

\subsection*{Credits}

The initial structure of this module was generated by \href{https://github.com/meryn/jumpstart}{\tt Jumpstart}, using the \href{https://github.com/meryn/jumpstart-black-coffee}{\tt Jumpstart Black Coffee} template.

\subsection*{License}

performance-\/now is released under the \href{http://opensource.org/licenses/MIT}{\tt M\+IT License}. ~\newline
Copyright (c) 2013 Meryn Stol 