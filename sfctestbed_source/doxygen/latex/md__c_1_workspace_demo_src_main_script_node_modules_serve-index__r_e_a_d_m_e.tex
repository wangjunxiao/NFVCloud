\href{https://npmjs.org/package/serve-index}{\tt } \href{https://npmjs.org/package/serve-index}{\tt } \href{https://travis-ci.org/expressjs/serve-index}{\tt } \href{https://ci.appveyor.com/project/dougwilson/serve-index}{\tt } \href{https://coveralls.io/r/expressjs/serve-index?branch=master}{\tt } \href{https://www.gratipay.com/dougwilson/}{\tt }

Serves pages that contain directory listings for a given path.

\subsection*{Install}

This is a \href{https://nodejs.org/en/}{\tt Node.\+js} module available through the \href{https://www.npmjs.com/}{\tt npm registry}. Installation is done using the \href{https://docs.npmjs.com/getting-started/installing-npm-packages-locally}{\tt {\ttfamily npm install} command}\+:


\begin{DoxyCode}
$ npm install serve-index
\end{DoxyCode}


\subsection*{A\+PI}


\begin{DoxyCode}
var serveIndex = require('serve-index')
\end{DoxyCode}


\subsubsection*{serve\+Index(path, options)}

Returns middlware that serves an index of the directory in the given {\ttfamily path}.

The {\ttfamily path} is based off the {\ttfamily req.\+url} value, so a {\ttfamily req.\+url} of `'/some/dir{\ttfamily  with a}path{\ttfamily of}\textquotesingle{}public\textquotesingle{}{\ttfamily will look at}\textquotesingle{}public/some/dir\textquotesingle{}{\ttfamily . If you are using something like}express{\ttfamily , you can change the \mbox{\hyperlink{namespace_u_r_l}{U\+RL}} \char`\"{}base\char`\"{} with}app.\+use\`{} (see the express example).

\paragraph*{Options}

Serve index accepts these properties in the options object.

\subparagraph*{filter}

Apply this filter function to files. Defaults to {\ttfamily false}. The {\ttfamily filter} function is called for each file, with the signature {\ttfamily filter(filename, index, files, dir)} where {\ttfamily filename} is the name of the file, {\ttfamily index} is the array index, {\ttfamily files} is the array of files and {\ttfamily dir} is the absolute path the file is located (and thus, the directory the listing is for).

\subparagraph*{hidden}

Display hidden (dot) files. Defaults to {\ttfamily false}.

\subparagraph*{icons}

Display icons. Defaults to {\ttfamily false}.

\subparagraph*{stylesheet}

Optional path to a C\+SS stylesheet. Defaults to a built-\/in stylesheet.

\subparagraph*{template}

Optional path to an H\+T\+ML template or a function that will render a H\+T\+ML string. Defaults to a built-\/in template.

When given a string, the string is used as a file path to load and then the following tokens are replaced in templates\+:


\begin{DoxyItemize}
\item {\ttfamily \{directory\}} with the name of the directory.
\item {\ttfamily \{files\}} with the H\+T\+ML of an unordered list of file links.
\item {\ttfamily \{linked-\/path\}} with the H\+T\+ML of a link to the directory.
\item {\ttfamily \{style\}} with the specified stylesheet and embedded images.
\end{DoxyItemize}

When given as a function, the function is called as {\ttfamily template(locals, callback)} and it needs to invoke {\ttfamily callback(error, html\+String)}. The following are the provided locals\+:


\begin{DoxyItemize}
\item {\ttfamily directory} is the directory being displayed (where {\ttfamily /} is the root).
\item {\ttfamily display\+Icons} is a Boolean for if icons should be rendered or not.
\item {\ttfamily file\+List} is a sorted array of files in the directory. The array contains objects with the following properties\+:
\begin{DoxyItemize}
\item {\ttfamily name} is the relative name for the file.
\item {\ttfamily stat} is a {\ttfamily fs.\+Stats} object for the file.
\end{DoxyItemize}
\item {\ttfamily path} is the full filesystem path to {\ttfamily directory}.
\item {\ttfamily style} is the default stylesheet or the contents of the {\ttfamily stylesheet} option.
\item {\ttfamily view\+Name} is the view name provided by the {\ttfamily view} option.
\end{DoxyItemize}

\subparagraph*{view}

Display mode. {\ttfamily tiles} and {\ttfamily details} are available. Defaults to {\ttfamily tiles}.

\subsection*{Examples}

\subsubsection*{Serve directory indexes with vanilla node.\+js http server}


\begin{DoxyCode}
var finalhandler = require('finalhandler')
var http = require('http')
var serveIndex = require('serve-index')
var serveStatic = require('serve-static')

// Serve directory indexes for public/ftp folder (with icons)
var index = serveIndex('public/ftp', \{'icons': true\})

// Serve up public/ftp folder files
var serve = serveStatic('public/ftp')

// Create server
var server = http.createServer(function onRequest(req, res)\{
  var done = finalhandler(req, res)
  serve(req, res, function onNext(err) \{
    if (err) return done(err)
    index(req, res, done)
  \})
\})

// Listen
server.listen(3000)
\end{DoxyCode}


\subsubsection*{Serve directory indexes with express}


\begin{DoxyCode}
var express    = require('express')
var serveIndex = require('serve-index')

var app = express()

// Serve URLs like /ftp/thing as public/ftp/thing
app.use('/ftp', serveIndex('public/ftp', \{'icons': true\}))
app.listen()
\end{DoxyCode}


\subsection*{License}

\mbox{[}M\+IT\mbox{]}(L\+I\+C\+E\+N\+SE). The \href{http://www.famfamfam.com/lab/icons/silk/}{\tt Silk} icons are created by/copyright of \href{http://www.famfamfam.com/}{\tt F\+A\+M\+F\+A\+M\+F\+AM}. 