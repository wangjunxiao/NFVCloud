A thing that is a lot like E\+S6 {\ttfamily Map}, but without iterators, for use in environments where {\ttfamily for..of} syntax and {\ttfamily Map} are not available.

If you need iterators, or just in general a more faithful polyfill to E\+S6 Maps, check out \href{http://npm.im/es6-map}{\tt es6-\/map}.

If you are in an environment where {\ttfamily Map} is supported, then that will be returned instead, unless {\ttfamily process.\+env.\+T\+E\+S\+T\+\_\+\+P\+S\+E\+U\+D\+O\+M\+AP} is set.

You can use any value as keys, and any value as data. Setting again with the identical key will overwrite the previous value.

Internally, data is stored on an {\ttfamily Object.\+create(null)} style object. The key is coerced to a string to generate the key on the internal data-\/bag object. The original key used is stored along with the data.

In the event of a stringified-\/key collision, a new key is generated by appending an increasing number to the stringified-\/key until finding either the intended key or an empty spot.

Note that because object traversal order of plain objects is not guaranteed to be identical to insertion order, the insertion order guarantee of {\ttfamily Map.\+prototype.\+for\+Each} is not guaranteed in this implementation. However, in all versions of Node.\+js and V8 where this module works, {\ttfamily for\+Each} does traverse data in insertion order.

\subsection*{A\+PI}

Most of the \href{https://developer.mozilla.org/en-US/docs/Web/JavaScript/Reference/Global_Objects/Map}{\tt Map A\+PI}, with the following exceptions\+:


\begin{DoxyEnumerate}
\item A {\ttfamily Map} object is not an iterator.
\item {\ttfamily values}, {\ttfamily keys}, and {\ttfamily entries} methods are not implemented, because they return iterators.
\item The argument to the constructor can be an Array of {\ttfamily \mbox{[}key, value\mbox{]}} pairs, or a {\ttfamily Map} or {\ttfamily Pseudo\+Map} object. But, since iterators aren\textquotesingle{}t used, passing any plain-\/old iterator won\textquotesingle{}t initialize the map properly.
\end{DoxyEnumerate}

\subsection*{U\+S\+A\+GE}

Use just like a regular E\+S6 Map.


\begin{DoxyCode}
var PseudoMap = require('pseudomap')

// optionally provide a pseudomap, or an array of [key,value] pairs
// as the argument to initialize the map with
var myMap = new PseudoMap()

myMap.set(1, 'number 1')
myMap.set('1', 'string 1')
var akey = \{\}
var bkey = \{\}
myMap.set(akey, \{ some: 'data' \})
myMap.set(bkey, \{ some: 'other data' \})
\end{DoxyCode}
 