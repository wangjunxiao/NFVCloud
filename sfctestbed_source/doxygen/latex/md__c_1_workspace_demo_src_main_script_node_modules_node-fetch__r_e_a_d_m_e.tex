\href{https://www.npmjs.com/package/node-fetch}{\tt } \href{https://travis-ci.org/bitinn/node-fetch}{\tt } \href{https://codecov.io/gh/bitinn/node-fetch}{\tt }

A light-\/weight module that brings {\ttfamily window.\+fetch} to Node.\+js

\section*{Motivation}

Instead of implementing {\ttfamily X\+M\+L\+Http\+Request} in Node.\+js to run browser-\/specific \href{https://github.com/github/fetch}{\tt Fetch polyfill}, why not go from native {\ttfamily http} to {\ttfamily Fetch} A\+PI directly? Hence {\ttfamily node-\/fetch}, minimal code for a {\ttfamily window.\+fetch} compatible A\+PI on Node.\+js runtime.

See Matt Andrews\textquotesingle{} \href{https://github.com/matthew-andrews/isomorphic-fetch}{\tt isomorphic-\/fetch} for isomorphic usage (exports {\ttfamily node-\/fetch} for server-\/side, {\ttfamily whatwg-\/fetch} for client-\/side).

\section*{Features}


\begin{DoxyItemize}
\item Stay consistent with {\ttfamily window.\+fetch} A\+PI.
\item Make conscious trade-\/off when following \href{https://fetch.spec.whatwg.org/}{\tt whatwg fetch spec} and \href{https://streams.spec.whatwg.org/}{\tt stream spec} implementation details, document known difference.
\item Use native promise, but allow substituting it with \mbox{[}insert your favorite promise library\mbox{]}.
\item Use native stream for body, on both request and response.
\item Decode content encoding (gzip/deflate) properly, and convert string output (such as {\ttfamily res.\+text()} and {\ttfamily res.\+json()}) to U\+T\+F-\/8 automatically.
\item Useful extensions such as timeout, redirect limit, response size limit, https\+://github.com/bitinn/node-\/fetch/blob/master/\+E\+R\+R\+O\+R-\/\+H\+A\+N\+D\+L\+I\+N\+G.\+md \char`\"{}explicit errors\char`\"{} for troubleshooting.
\end{DoxyItemize}

\section*{Difference from client-\/side fetch}


\begin{DoxyItemize}
\item See https\+://github.com/bitinn/node-\/fetch/blob/master/\+L\+I\+M\+I\+T\+S.\+md \char`\"{}\+Known Differences\char`\"{} for details.
\item If you happen to use a missing feature that {\ttfamily window.\+fetch} offers, feel free to open an issue.
\item Pull requests are welcomed too!
\end{DoxyItemize}

\section*{Install}

{\ttfamily npm install node-\/fetch -\/-\/save}

\section*{Usage}


\begin{DoxyCode}
var fetch = require('node-fetch');

// if you are on node v0.10, set a Promise library first, eg.
// fetch.Promise = require('bluebird');

// plain text or html

fetch('https://github.com/')
    .then(function(res) \{
        return res.text();
    \}).then(function(body) \{
        console.log(body);
    \});

// json

fetch('https://api.github.com/users/github')
    .then(function(res) \{
        return res.json();
    \}).then(function(json) \{
        console.log(json);
    \});

// catching network error
// 3xx-5xx responses are NOT network errors, and should be handled in then()
// you only need one catch() at the end of your promise chain

fetch('http://domain.invalid/')
    .catch(function(err) \{
        console.log(err);
    \});

// stream
// the node.js way is to use stream when possible

fetch('https://assets-cdn.github.com/images/modules/logos\_page/Octocat.png')
    .then(function(res) \{
        var dest = fs.createWriteStream('./octocat.png');
        res.body.pipe(dest);
    \});

// buffer
// if you prefer to cache binary data in full, use buffer()
// note that buffer() is a node-fetch only API

var fileType = require('file-type');
fetch('https://assets-cdn.github.com/images/modules/logos\_page/Octocat.png')
    .then(function(res) \{
        return res.buffer();
    \}).then(function(buffer) \{
        fileType(buffer);
    \});

// meta

fetch('https://github.com/')
    .then(function(res) \{
        console.log(res.ok);
        console.log(res.status);
        console.log(res.statusText);
        console.log(res.headers.raw());
        console.log(res.headers.get('content-type'));
    \});

// post

fetch('http://httpbin.org/post', \{ method: 'POST', body: 'a=1' \})
    .then(function(res) \{
        return res.json();
    \}).then(function(json) \{
        console.log(json);
    \});

// post with stream from resumer

var resumer = require('resumer');
var stream = resumer().queue('a=1').end();
fetch('http://httpbin.org/post', \{ method: 'POST', body: stream \})
    .then(function(res) \{
        return res.json();
    \}).then(function(json) \{
        console.log(json);
    \});

// post with form-data (detect multipart)

var FormData = require('form-data');
var form = new FormData();
form.append('a', 1);
fetch('http://httpbin.org/post', \{ method: 'POST', body: form \})
    .then(function(res) \{
        return res.json();
    \}).then(function(json) \{
        console.log(json);
    \});

// post with form-data (custom headers)
// note that getHeaders() is non-standard API

var FormData = require('form-data');
var form = new FormData();
form.append('a', 1);
fetch('http://httpbin.org/post', \{ method: 'POST', body: form, headers: form.getHeaders() \})
    .then(function(res) \{
        return res.json();
    \}).then(function(json) \{
        console.log(json);
    \});

// node 0.12+, yield with co

var co = require('co');
co(function *() \{
    var res = yield fetch('https://api.github.com/users/github');
    var json = yield res.json();
    console.log(res);
\});
\end{DoxyCode}


See \href{https://github.com/bitinn/node-fetch/blob/master/test/test.js}{\tt test cases} for more examples.

\section*{A\+PI}

\subsection*{fetch(url, options)}

Returns a {\ttfamily Promise}

\subsubsection*{Url}

Should be an absolute url, eg {\ttfamily \href{http://example.com/}{\tt http\+://example.\+com/}}

\subsubsection*{Options}

default values are shown, note that only {\ttfamily method}, {\ttfamily headers}, {\ttfamily redirect} and {\ttfamily body} are allowed in {\ttfamily window.\+fetch}, others are node.\+js extensions.


\begin{DoxyCode}
\{
    method: 'GET'
    , headers: \{\}        // request header. format \{a:'1'\} or \{b:['1','2','3']\}
    , redirect: 'follow' // set to `manual` to extract redirect headers, `error` to reject redirect
    , follow: 20         // maximum redirect count. 0 to not follow redirect
    , timeout: 0         // req/res timeout in ms, it resets on redirect. 0 to disable (OS limit applies)
    , compress: true     // support gzip/deflate content encoding. false to disable
    , size: 0            // maximum response body size in bytes. 0 to disable
    , body: empty        // request body. can be a string, buffer, readable stream
    , agent: null        // http.Agent instance, allows custom proxy, certificate etc.
\}
\end{DoxyCode}


\section*{License}

M\+IT

\section*{Acknowledgement}

Thanks to \href{https://github.com/github/fetch}{\tt github/fetch} for providing a solid implementation reference. 