\begin{quote}
Compile E\+S2015 classes to E\+S5 \end{quote}


\subsection*{Caveats}

Built-\/in classes such as {\ttfamily Date}, {\ttfamily Array}, {\ttfamily D\+OM} etc cannot be properly subclassed due to limitations in E\+S5 (for the \href{http://babeljs.io/docs/plugins/transform-es2015-classes}{\tt es2015-\/classes} plugin). You can try to use \href{https://github.com/loganfsmyth/babel-plugin-transform-builtin-extend}{\tt babel-\/plugin-\/transform-\/builtin-\/extend} based on {\ttfamily Object.\+set\+Prototype\+Of} and {\ttfamily Reflect.\+construct}, but it also has some limitations.

\subsection*{Installation}


\begin{DoxyCode}
npm install --save-dev babel-plugin-transform-es2015-classes
\end{DoxyCode}


\subsection*{Usage}

\subsubsection*{Via {\ttfamily .babelrc} (Recommended)}

$\ast$$\ast$.babelrc$\ast$$\ast$


\begin{DoxyCode}
// without options
\{
  "plugins": ["transform-es2015-classes"]
\}

// with options
\{
  "plugins": [
    ["transform-es2015-classes", \{
      "loose": true
    \}]
  ]
\}
\end{DoxyCode}


\subsubsection*{Via C\+LI}


\begin{DoxyCode}
babel --plugins transform-es2015-classes script.js
\end{DoxyCode}


\subsubsection*{Via Node A\+PI}


\begin{DoxyCode}
require("babel-core").transform("code", \{
  plugins: ["transform-es2015-classes"]
\});
\end{DoxyCode}


\subsection*{Options}

\subsubsection*{{\ttfamily loose}}

{\ttfamily boolean}, defaults to {\ttfamily false}.

\paragraph*{Method enumerability}

Please note that in loose mode class methods {\bfseries are} enumerable. This is not in line with the spec and you may run into issues.

\paragraph*{Method assignment}

Under loose mode, methods are defined on the class prototype with simple assignments instead of being defined. This can result in the following not working\+:


\begin{DoxyCode}
class Foo \{
  set bar() \{
    throw new Error("foo!");
  \}
\}

class Bar extends Foo \{
  bar() \{
    // will throw an error when this method is defined
  \}
\}
\end{DoxyCode}


When {\ttfamily Bar.\+prototype.\+foo} is defined it triggers the setter on {\ttfamily Foo}. This is a case that is very unlikely to appear in production code however it\textquotesingle{}s something to keep in mind. 