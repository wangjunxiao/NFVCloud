This will let you identify and transform various git hosts U\+R\+Ls between protocols. It also can tell you what the \mbox{\hyperlink{namespace_u_r_l}{U\+RL}} is for the raw path for particular file for direct access without git.

\subsection*{Example}


\begin{DoxyCode}
var hostedGitInfo = require("hosted-git-info")
var info = hostedGitInfo.fromUrl("git@github.com:npm/hosted-git-info.git", opts)
/* info looks like:
\{
  type: "github",
  domain: "github.com",
  user: "npm",
  project: "hosted-git-info"
\}
*/
\end{DoxyCode}


If the \mbox{\hyperlink{namespace_u_r_l}{U\+RL}} can\textquotesingle{}t be matched with a git host, {\ttfamily null} will be returned. We can match git, ssh and https urls. Additionally, we can match ssh connect strings ({\ttfamily git@github.\+com\+:npm/hosted-\/git-\/info}) and shortcuts (eg, {\ttfamily github\+:npm/hosted-\/git-\/info}). Github specifically, is detected in the case of a third, unprefixed, form\+: {\ttfamily npm/hosted-\/git-\/info}.

If it does match, the returned object has properties of\+:


\begin{DoxyItemize}
\item info.\+type -- The short name of the service
\item info.\+domain -- The domain for git protocol use
\item info.\+user -- The name of the user/org on the git host
\item info.\+project -- The name of the project on the git host
\end{DoxyItemize}

\subsection*{Version Contract}

The major version will be bumped any time…


\begin{DoxyItemize}
\item The constructor stops accepting U\+R\+Ls that it previously accepted.
\item A method is removed.
\item A method can no longer accept the number and type of arguments it previously accepted.
\item A method can return a different type than it currently returns.
\end{DoxyItemize}

Implications\+:


\begin{DoxyItemize}
\item I do not consider the specific format of the urls returned from, say {\ttfamily .https()} to be a part of the contract. The contract is that it will return a string that can be used to fetch the repo via H\+T\+T\+PS. But what that string looks like, specifically, can change.
\item Dropping support for a hosted git provider would constitute a breaking change.
\end{DoxyItemize}

\subsection*{Usage}

\subsubsection*{var info = hosted\+Git\+Info.\+from\+Url(git\+Specifier\mbox{[}, options\mbox{]})}


\begin{DoxyItemize}
\item {\itshape git\+Specifer} is a \mbox{\hyperlink{namespace_u_r_l}{U\+RL}} of a git repository or a S\+C\+P-\/style specifier of one.
\item {\itshape options} is an optional object. It can have the following properties\+:
\begin{DoxyItemize}
\item {\itshape no\+Committish} — If true then committishes won\textquotesingle{}t be included in generated U\+R\+Ls.
\item {\itshape no\+Git\+Plus} — If true then {\ttfamily git+} won\textquotesingle{}t be prefixed on U\+R\+Ls.
\end{DoxyItemize}
\end{DoxyItemize}

\subsection*{Methods}

All of the methods take the same options as the {\ttfamily from\+Url} factory. Options provided to a method override those provided to the constructor.


\begin{DoxyItemize}
\item info.\+file(path, opts)
\end{DoxyItemize}

Given the path of a file relative to the repository, returns a \mbox{\hyperlink{namespace_u_r_l}{U\+RL}} for directly fetching it from the githost. If no committish was set then {\ttfamily master} will be used as the default.

For example {\ttfamily hosted\+Git\+Info.\+from\+Url(\char`\"{}git@github.\+com\+:npm/hosted-\/git-\/info.\+git\#v1.\+0.\+0\char`\"{}).file(\char`\"{}package.\+json\char`\"{})} would return {\ttfamily \href{https://raw.githubusercontent.com/npm/hosted-git-info/v1.0.0/package.json}{\tt https\+://raw.\+githubusercontent.\+com/npm/hosted-\/git-\/info/v1.\+0.\+0/package.\+json}}


\begin{DoxyItemize}
\item info.\+shortcut(opts)
\end{DoxyItemize}

eg, {\ttfamily github\+:npm/hosted-\/git-\/info}


\begin{DoxyItemize}
\item info.\+browse(opts)
\end{DoxyItemize}

eg, {\ttfamily \href{https://github.com/npm/hosted-git-info/tree/v1.2.0}{\tt https\+://github.\+com/npm/hosted-\/git-\/info/tree/v1.\+2.\+0}}


\begin{DoxyItemize}
\item info.\+bugs(opts)
\end{DoxyItemize}

eg, {\ttfamily \href{https://github.com/npm/hosted-git-info/issues}{\tt https\+://github.\+com/npm/hosted-\/git-\/info/issues}}


\begin{DoxyItemize}
\item info.\+docs(opts)
\end{DoxyItemize}

eg, {\ttfamily \href{https://github.com/npm/hosted-git-info/tree/v1.2.0#readme}{\tt https\+://github.\+com/npm/hosted-\/git-\/info/tree/v1.\+2.\+0\#readme}}


\begin{DoxyItemize}
\item info.\+https(opts)
\end{DoxyItemize}

eg, {\ttfamily git+https\+://github.com/npm/hosted-\/git-\/info.\+git}


\begin{DoxyItemize}
\item info.\+sshurl(opts)
\end{DoxyItemize}

eg, {\ttfamily git+ssh\+://git@github.\+com/npm/hosted-\/git-\/info.git}


\begin{DoxyItemize}
\item info.\+ssh(opts)
\end{DoxyItemize}

eg, {\ttfamily git@github.\+com\+:npm/hosted-\/git-\/info.\+git}


\begin{DoxyItemize}
\item info.\+path(opts)
\end{DoxyItemize}

eg, {\ttfamily npm/hosted-\/git-\/info}


\begin{DoxyItemize}
\item info.\+tarball(opts)
\end{DoxyItemize}

eg, {\ttfamily \href{https://github.com/npm/hosted-git-info/archive/v1.2.0.tar.gz}{\tt https\+://github.\+com/npm/hosted-\/git-\/info/archive/v1.\+2.\+0.\+tar.\+gz}}


\begin{DoxyItemize}
\item info.\+get\+Default\+Representation()
\end{DoxyItemize}

Returns the default output type. The default output type is based on the string you passed in to be parsed


\begin{DoxyItemize}
\item info.\+to\+String(opts)
\end{DoxyItemize}

Uses the get\+Default\+Representation to call one of the other methods to get a \mbox{\hyperlink{namespace_u_r_l}{U\+RL}} for this resource. As such {\ttfamily hosted\+Git\+Info.\+from\+Url(url).to\+String()} will give you a normalized version of the \mbox{\hyperlink{namespace_u_r_l}{U\+RL}} that still uses the same protocol.

Shortcuts will still be returned as shortcuts, but the special case github form of {\ttfamily org/project} will be normalized to {\ttfamily github\+:org/project}.

S\+SH connect strings will be normalized into {\ttfamily git+ssh} U\+R\+Ls.

\subsection*{Supported hosts}

Currently this supports Github, Bitbucket and Gitlab. Pull requests for additional hosts welcome. 