This package includes some utilities used by \href{https://github.com/facebookincubator/create-react-app}{\tt Create React App}.~\newline
 Please refer to its documentation\+:


\begin{DoxyItemize}
\item \href{https://github.com/facebookincubator/create-react-app/blob/master/README.md#getting-started}{\tt Getting Started} – How to create a new app.
\item https\+://github.com/facebookincubator/create-\/react-\/app/blob/master/packages/react-\/scripts/template/\+R\+E\+A\+D\+M\+E.\+md \char`\"{}\+User Guide\char`\"{} – How to develop apps bootstrapped with Create React App.
\end{DoxyItemize}

\subsection*{Usage in Create React App Projects}

These utilities come by default with \href{https://github.com/facebookincubator/create-react-app}{\tt Create React App}, which includes it by default. {\bfseries You don’t need to install it separately in Create React App projects.}

\subsection*{Usage Outside of Create React App}

If you don’t use Create React App, or if you \href{https://github.com/facebookincubator/create-react-app/blob/master/packages/react-scripts/template/README.md#npm-run-eject}{\tt ejected}, you may keep using these utilities. Their development will be aligned with Create React App, so major versions of these utilities may come out relatively often. Feel free to fork or copy and paste them into your projects if you’d like to have more control over them, or feel free to use the old versions. Not all of them are React-\/specific, but we might make some of them more React-\/specific in the future.

\subsubsection*{Entry Points}

There is no single entry point. You can only import individual top-\/level modules.

\paragraph*{{\ttfamily new Interpolate\+Html\+Plugin(replacements\+: \{\mbox{[}key\+:string\mbox{]}\+: string\})}}

This Webpack plugin lets us interpolate custom variables into {\ttfamily index.\+html}.~\newline
 It works in tandem with \href{https://github.com/ampedandwired/html-webpack-plugin}{\tt Html\+Webpack\+Plugin} 2.\+x via its \href{https://github.com/ampedandwired/html-webpack-plugin#events}{\tt events}.


\begin{DoxyCode}
var path = require('path');
var HtmlWebpackPlugin = require('html-dev-plugin');
var InterpolateHtmlPlugin = require('react-dev-utils/InterpolateHtmlPlugin');

// Webpack config
var publicUrl = '/my-custom-url';

module.exports = \{
  output: \{
    // ...
    publicPath: publicUrl + '/'
  \},
  // ...
  plugins: [
    // Makes the public URL available as %PUBLIC\_URL% in index.html, e.g.:
    // <link rel="shortcut icon" href="%PUBLIC\_URL%/favicon.ico">
    new InterpolateHtmlPlugin(\{
      PUBLIC\_URL: publicUrl
      // You can pass any key-value pairs, this was just an example.
      // WHATEVER: 42 will replace %WHATEVER% with 42 in index.html.
    \}),
    // Generates an `index.html` file with the <script> injected.
    new HtmlWebpackPlugin(\{
      inject: true,
      template: path.resolve('public/index.html'),
    \}),
    // ...
  ],
  // ...
\}
\end{DoxyCode}


\paragraph*{{\ttfamily new Module\+Scope\+Plugin(app\+Src\+: string)}}

This Webpack plugin ensures that relative imports from app\textquotesingle{}s source directory don\textquotesingle{}t reach outside of it.


\begin{DoxyCode}
var path = require('path');
var ModuleScopePlugin = require('react-dev-utils/ModuleScopePlugin');


module.exports = \{
  // ...
  resolve: \{
    // ...
    plugins: [
      new ModuleScopePlugin(paths.appSrc),
      // ...
    ],
    // ...
  \},
  // ...
\}
\end{DoxyCode}


\paragraph*{{\ttfamily new Watch\+Missing\+Node\+Modules\+Plugin(node\+Modules\+Path\+: string)}}

This Webpack plugin ensures {\ttfamily npm install $<$library$>$} forces a project rebuild.~\newline
 We’re not sure why this isn\textquotesingle{}t Webpack\textquotesingle{}s default behavior.~\newline
 See \href{https://github.com/facebookincubator/create-react-app/issues/186}{\tt \#186} for details.


\begin{DoxyCode}
var path = require('path');
var WatchMissingNodeModulesPlugin = require('react-dev-utils/WatchMissingNodeModulesPlugin');

// Webpack config
module.exports = \{
  // ...
  plugins: [
    // ...
    // If you require a missing module and then `npm install` it, you still have
    // to restart the development server for Webpack to discover it. This plugin
    // makes the discovery automatic so you don't have to restart.
    // See https://github.com/facebookincubator/create-react-app/issues/186
    new WatchMissingNodeModulesPlugin(path.resolve('node\_modules'))
  ],
  // ...
\}
\end{DoxyCode}


\paragraph*{{\ttfamily check\+Required\+Files(files\+: Array$<$string$>$)\+: boolean}}

Makes sure that all passed files exist.~\newline
 Filenames are expected to be absolute.~\newline
 If a file is not found, prints a warning message and returns {\ttfamily false}.


\begin{DoxyCode}
var path = require('path');
var checkRequiredFiles = require('react-dev-utils/checkRequiredFiles');

if (!checkRequiredFiles([
  path.resolve('public/index.html'),
  path.resolve('src/index.js')
])) \{
  process.exit(1);
\}
\end{DoxyCode}


\paragraph*{{\ttfamily clear\+Console()\+: void}}

Clears the console, hopefully in a cross-\/platform way.


\begin{DoxyCode}
var clearConsole = require('react-dev-utils/clearConsole');

clearConsole();
console.log('Just cleared the screen!');
\end{DoxyCode}


\paragraph*{{\ttfamily eslint\+Formatter(results\+: Object)\+: string}}

This is our custom E\+S\+Lint formatter that integrates well with Create React App console output.~\newline
 You can use the default one instead if you prefer so.


\begin{DoxyCode}
const eslintFormatter = require('react-dev-utils/eslintFormatter');

// In your webpack config:
// ...
module: \{
   rules: [
     \{
        test: /\(\backslash\).(js|jsx)$/,
        include: paths.appSrc,
        enforce: 'pre',
        use: [
          \{
            loader: 'eslint-loader',
            options: \{
              // Pass the formatter:
              formatter: eslintFormatter,
            \},
          \},
        ],
      \}
   ]
\}
\end{DoxyCode}


\paragraph*{{\ttfamily File\+Size\+Reporter}}

\subparagraph*{{\ttfamily measure\+File\+Sizes\+Before\+Build(build\+Folder\+: string)\+: Promise$<$Opaque\+File\+Sizes$>$}}

Captures JS and C\+SS asset sizes inside the passed {\ttfamily build\+Folder}. Save the result value to compare it after the build.

\subparagraph*{{\ttfamily print\+File\+Sizes\+After\+Build(webpack\+Stats\+: Webpack\+Stats, previous\+File\+Sizes\+: Opaque\+File\+Sizes)}}

Prints the JS and C\+SS asset sizes after the build, and includes a size comparison with {\ttfamily previous\+File\+Sizes} that were captured earlier using {\ttfamily measure\+File\+Sizes\+Before\+Build()}.


\begin{DoxyCode}
var \{
  measureFileSizesBeforeBuild,
  printFileSizesAfterBuild,
\} = require('react-dev-utils/FileSizeReporter');

measureFileSizesBeforeBuild(buildFolder).then(previousFileSizes => \{
  return cleanAndRebuild().then(webpackStats => \{
    printFileSizesAfterBuild(webpackStats, previousFileSizes);
  \});
\});
\end{DoxyCode}


\paragraph*{{\ttfamily format\+Webpack\+Messages(\{errors\+: Array$<$string$>$, warnings\+: Array$<$string$>$\})\+: \{errors\+: Array$<$string$>$, warnings\+: Array$<$string$>$\}}}

Extracts and prettifies warning and error messages from webpack \href{https://github.com/webpack/docs/wiki/node.js-api#stats}{\tt stats} object.


\begin{DoxyCode}
var webpack = require('webpack');
var config = require('../config/webpack.config.dev');
var formatWebpackMessages = require('react-dev-utils/formatWebpackMessages');

var compiler = webpack(config);

compiler.plugin('invalid', function() \{
  console.log('Compiling...');
\});

compiler.plugin('done', function(stats) \{
  var rawMessages = stats.toJson(\{\}, true);
  var messages = formatWebpackMessages(rawMessages);
  if (!messages.errors.length && !messages.warnings.length) \{
    console.log('Compiled successfully!');
  \}
  if (messages.errors.length) \{
    console.log('Failed to compile.');
    messages.errors.forEach(e => console.log(e));
    return;
  \}
  if (messages.warnings.length) \{
    console.log('Compiled with warnings.');
    messages.warnings.forEach(w => console.log(w));
  \}
\});
\end{DoxyCode}


\paragraph*{{\ttfamily get\+Process\+For\+Port(port\+: number)\+: string}}

Finds the currently running process on {\ttfamily port}. Returns a string containing the name and directory, e.\+g.,


\begin{DoxyCode}
create-react-app
in /Users/developer/create-react-app
\end{DoxyCode}



\begin{DoxyCode}
var getProcessForPort = require('react-dev-utils/getProcessForPort');

getProcessForPort(3000);
\end{DoxyCode}


\paragraph*{{\ttfamily launch\+Editor(file\+Name\+: string, line\+Number\+: number)\+: void}}

On mac\+OS, tries to find a known running editor process and opens the file in it. It can also be explicitly configured by {\ttfamily R\+E\+A\+C\+T\+\_\+\+E\+D\+I\+T\+OR}, {\ttfamily V\+I\+S\+U\+AL}, or {\ttfamily E\+D\+I\+T\+OR} environment variables. For example, you can put {\ttfamily R\+E\+A\+C\+T\+\_\+\+E\+D\+I\+T\+OR=atom} in your {\ttfamily .env.\+local} file, and Create React App will respect that.

\paragraph*{{\ttfamily noop\+Service\+Worker\+Middleware()\+: Express\+Middleware}}

Returns Express middleware that serves a {\ttfamily /service-\/worker.js} that resets any previously set service worker configuration. Useful for development.

\paragraph*{{\ttfamily open\+Browser(url\+: string)\+: boolean}}

Attempts to open the browser with a given \mbox{\hyperlink{namespace_u_r_l}{U\+RL}}.~\newline
 On Mac OS X, attempts to reuse an existing Chrome tab via Apple\+Script.~\newline
 Otherwise, falls back to \href{https://github.com/sindresorhus/opn}{\tt opn} behavior.


\begin{DoxyCode}
var path = require('path');
var openBrowser = require('react-dev-utils/openBrowser');

if (openBrowser('http://localhost:3000')) \{
  console.log('The browser tab has been opened!');
\}
\end{DoxyCode}


\paragraph*{{\ttfamily print\+Hosting\+Instructions(app\+Package\+: Object, public\+Url\+: string, public\+Path\+: string, build\+Folder\+: string, use\+Yarn\+: boolean)\+: void}}

Prints hosting instructions after the project is built.

Pass your parsed {\ttfamily package.\+json} object as {\ttfamily app\+Package}, your the \mbox{\hyperlink{namespace_u_r_l}{U\+RL}} where you plan to host the app as {\ttfamily public\+Url}, {\ttfamily output.\+public\+Path} from your Webpack configuration as {\ttfamily public\+Path}, the {\ttfamily build\+Folder} name, and whether to {\ttfamily use\+Yarn} in instructions.


\begin{DoxyCode}
const appPackage = require(paths.appPackageJson);
const publicUrl = paths.publicUrl;
const publicPath = config.output.publicPath;
printHostingInstructions(appPackage, publicUrl, publicPath, 'build', true);
\end{DoxyCode}


\paragraph*{{\ttfamily Webpack\+Dev\+Server\+Utils}}

\subparagraph*{{\ttfamily choose\+Port(host\+: string, default\+Port\+: number)\+: Promise$<$number $\vert$ null$>$}}

Returns a Promise resolving to either {\ttfamily default\+Port} or next available port if the user confirms it is okay to do. If the port is taken and the user has refused to use another port, or if the terminal is not interactive and can’t present user with the choice, resolves to {\ttfamily null}.

\subparagraph*{{\ttfamily create\+Compiler(webpack\+: Function, config\+: Object, app\+Name\+: string, urls\+: Object, use\+Yarn\+: boolean)\+: Webpack\+Compiler}}

Creates a Webpack compiler instance for Webpack\+Dev\+Server with built-\/in helpful messages. Takes the `require(\textquotesingle{}webpack'){\ttfamily entry point as the first argument. To provide the}urls{\ttfamily argument, use}prepare\+Urls()\`{} described below.

\subparagraph*{{\ttfamily prepare\+Proxy(proxy\+Setting\+: string, app\+Public\+Folder\+: string)\+: Object}}

Creates a Webpack\+Dev\+Server {\ttfamily proxy} configuration object from the {\ttfamily proxy} setting in {\ttfamily package.\+json}.

\subparagraph*{{\ttfamily prepare\+Urls(protocol\+: string, host\+: string, port\+: number)\+: Object}}

Returns an object with local and remote U\+R\+Ls for the development server. Pass this object to {\ttfamily create\+Compiler()} described above.

\paragraph*{{\ttfamily webpack\+Hot\+Dev\+Client}}

This is an alternative client for \href{https://github.com/webpack/webpack-dev-server}{\tt Webpack\+Dev\+Server} that shows a syntax error overlay.

It currently supports only Webpack 1.\+x.


\begin{DoxyCode}
// Webpack development config
module.exports = \{
  // ...
  entry: [
    // You can replace the line below with these two lines if you prefer the
    // stock client:
    // require.resolve('webpack-dev-server/client') + '?/',
    // require.resolve('webpack/hot/dev-server'),
    'react-dev-utils/webpackHotDevClient',
    'src/index'
  ],
  // ...
\}
\end{DoxyCode}
 