Enforce that all elements that require alternative text have meaningful information to relay back to the end user. This is a critical component of accessibility for screenreader users in order for them to understand the content\textquotesingle{}s purpose on the page. By default, this rule checks for alternative text on the following elements\+: {\ttfamily $<$img$>$}, {\ttfamily $<$area$>$}, {\ttfamily $<$input type=\char`\"{}image\char`\"{}$>$}, and {\ttfamily $<$object$>$}.

\paragraph*{Resources}


\begin{DoxyEnumerate}
\item \href{https://dequeuniversity.com/rules/axe/2.1/object-alt}{\tt a\+Xe object-\/alt}
\item \href{https://dequeuniversity.com/rules/axe/2.1/image-alt}{\tt a\+Xe image-\/alt}
\item \href{https://dequeuniversity.com/rules/axe/2.1/input-image-alt}{\tt a\+Xe input-\/image-\/alt}
\item \href{https://dequeuniversity.com/rules/axe/2.1/area-alt}{\tt a\+Xe area-\/alt}
\end{DoxyEnumerate}

\subsection*{How to resolve}

\subsubsection*{{\ttfamily $<$img$>$}}

An {\ttfamily $<$img$>$} must have the {\ttfamily alt} prop set with meaningful text or as an empty string to indicate that it is an image for decoration.

For images that are being used as icons for a button or control, the {\ttfamily alt} prop should be set to an empty string ({\ttfamily alt=\char`\"{}\char`\"{}}).


\begin{DoxyCode}
<button>
  <img src="icon.png" alt="" />
  Save
</button>
\end{DoxyCode}
 The content of an {\ttfamily alt} attribute is used to calculate the accessible label of an element, whereas the text content is used to produce a label for the element. For this reason, adding a label to an icon can produce a confusing or duplicated label on a control that already has appropriate text content.

\subsubsection*{{\ttfamily $<$object$>$}}

Add alternative text to all embedded {\ttfamily $<$object$>$} elements using either inner text, setting the {\ttfamily title} prop, or using the {\ttfamily aria-\/label} or {\ttfamily aria-\/labelledby} props.

\subsubsection*{{\ttfamily $<$input type=\char`\"{}image\char`\"{}$>$}}

All {\ttfamily $<$input type=\char`\"{}image\char`\"{}$>$} elements must have a non-\/empty {\ttfamily alt} prop set with a meaningful description of the image or have the {\ttfamily aria-\/label} or {\ttfamily aria-\/labelledby} props set.

\subsubsection*{{\ttfamily $<$area$>$}}

All clickable {\ttfamily $<$area$>$} elements within an image map have an {\ttfamily alt}, {\ttfamily aria-\/label} or {\ttfamily aria-\/labelledby} prop that describes the purpose of the link.

\subsection*{Rule details}

This rule takes one optional object argument of type object\+:


\begin{DoxyCode}
\{
    "rules": \{
        "jsx-a11y/img-has-alt": [ 2, \{
            "elements": [ "img", "object", "area", "input[type=\(\backslash\)"image\(\backslash\)"]" ],
            "img": ["Image"],
            "object": ["Object"],
            "area": ["Area"],
            "input[type=\(\backslash\)"image\(\backslash\)"]": ["InputImage"]
          \}],
    \}
\}
\end{DoxyCode}


The {\ttfamily elements} option is a whitelist for D\+OM elements to check for alternative text. If an element is removed from the default set of elements (noted above), any custom components for that component will also be ignored. In order to indicate any custom wrapper components that should be checked, you can map the D\+OM element to an array of J\+SX custom components. This is a good use case when you have a wrapper component that simply renders an {\ttfamily img} element, for instance (like in React)\+:


\begin{DoxyCode}
// Image.js
const Image = props => \{
  const \{
    alt,
    ...otherProps
  \} = props;

  return (
    <img alt=\{alt\} \{...otherProps\} />
  );
\}

...

// Header.js (for example)
...
return (
  <header>
    <Image alt="Logo" src="logo.jpg" />
  </header>
);
\end{DoxyCode}


Note that passing props as spread attribute without explicitly the necessary accessibility props defined will cause this rule to fail. Explicitly pass down the set of props needed for rule to pass. Use {\ttfamily Image} component above as a reference for destructuring and applying the prop. {\bfseries It is a good thing to explicitly pass props that you expect to be passed for self-\/documentation.} For example\+:

\#\#\#\# Bad 
\begin{DoxyCode}
function Foo(props) \{
  return <img \{...props\} />
\}
\end{DoxyCode}


\#\#\#\# Good 
\begin{DoxyCode}
function Foo(\{ alt, ...props\}) \{
    return <img alt=\{alt\} \{...props\} />
\}

// OR

function Foo(props) \{
    const \{
        alt,
        ...otherProps
    \} = props;

   return <img alt=\{alt\} \{...otherProps\} />
\}
\end{DoxyCode}


\#\#\# Succeed 
\begin{DoxyCode}
<img src="foo" alt="Foo eating a sandwich." />
<img src="foo" alt=\{"Foo eating a sandwich."\} />
<img src="foo" alt=\{altText\} />
<img src="foo" alt=\{`$\{person\} smiling`\} />
<img src="foo" alt="" />

<object aria-label="foo" />
<object aria-labelledby="id1" />
<object>Meaningful description</object>
<object title="An object" />

<area aria-label="foo" />
<area aria-labelledby="id1" />
<area alt="This is descriptive!" />

<input type="image" alt="This is descriptive!" />
<input type="image" aria-label="foo" />
<input type="image" aria-labelledby="id1" />
\end{DoxyCode}


\subsubsection*{Fail}


\begin{DoxyCode}
<img src="foo" />
<img \{...props\} />
<img \{...props\} alt /> // Has no value
<img \{...props\} alt=\{undefined\} /> // Has no value
<img \{...props\} alt=\{`$\{undefined\}`\} /> // Has no value
<img src="foo" role="presentation" /> // Avoid ARIA if it can be achieved without
<img src="foo" role="none" /> // Avoid ARIA if it can be achieved without

<object \{...props\} />

<area \{...props\} />

<input type="image" \{...props\} />
\end{DoxyCode}
 