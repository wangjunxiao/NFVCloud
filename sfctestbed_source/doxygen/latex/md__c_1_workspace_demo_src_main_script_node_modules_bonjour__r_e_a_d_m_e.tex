A Bonjour/\+Zeroconf protocol implementation in pure Java\+Script. Publish services on the local network or discover existing services using multicast D\+NS.

\href{https://travis-ci.org/watson/bonjour}{\tt } \href{https://github.com/feross/standard}{\tt }

\subsection*{Installation}


\begin{DoxyCode}
npm install bonjour
\end{DoxyCode}


\subsection*{Usage}


\begin{DoxyCode}
var bonjour = require('bonjour')()

// advertise an HTTP server on port 3000
bonjour.publish(\{ name: 'My Web Server', type: 'http', port: 3000 \})

// browse for all http services
bonjour.find(\{ type: 'http' \}, function (service) \{
  console.log('Found an HTTP server:', service)
\})
\end{DoxyCode}


\subsection*{A\+PI}

\subsubsection*{Initializing}


\begin{DoxyCode}
var bonjour = require('bonjour')([options])
\end{DoxyCode}


The {\ttfamily options} are optional and will be used when initializing the underlying multicast-\/dns server. For details see \href{https://github.com/mafintosh/multicast-dns#mdns--multicastdnsoptions}{\tt the multicast-\/dns documentation}.

\subsubsection*{Publishing}

\paragraph*{{\ttfamily var service = bonjour.\+publish(options)}}

Publishes a new service.

Options are\+:


\begin{DoxyItemize}
\item {\ttfamily name} (string)
\item {\ttfamily host} (string, optional) -\/ defaults to local hostname
\item {\ttfamily port} (number)
\item {\ttfamily type} (string)
\item {\ttfamily subtypes} (array of strings, optional)
\item {\ttfamily protocol} (string, optional) -\/ {\ttfamily udp} or {\ttfamily tcp} (default)
\item {\ttfamily txt} (object, optional) -\/ a key/value object to broadcast as the T\+XT record
\end{DoxyItemize}

I\+A\+NA maintains a \href{http://www.iana.org/assignments/service-names-port-numbers/service-names-port-numbers.xhtml}{\tt list of official service types and port numbers}.

\paragraph*{{\ttfamily bonjour.\+unpublish\+All(\mbox{[}callback\mbox{]})}}

Unpublish all services. The optional {\ttfamily callback} will be called when the services have been unpublished.

\paragraph*{{\ttfamily bonjour.\+destroy()}}

Destroy the mdns instance. Closes the udp socket.

\subsubsection*{Browser}

\paragraph*{{\ttfamily var browser = bonjour.\+find(options\mbox{[}, onup\mbox{]})}}

Listen for services advertised on the network. An optional callback can be provided as the 2nd argument and will be added as an event listener for the {\ttfamily up} event.

Options (all optional)\+:


\begin{DoxyItemize}
\item {\ttfamily type} (string)
\item {\ttfamily subtypes} (array of strings)
\item {\ttfamily protocol} (string) -\/ defaults to {\ttfamily tcp}
\item {\ttfamily txt} (object) -\/ passed into \href{https://github.com/watson/dns-txt}{\tt dns-\/txt module} contructor. Set to {\ttfamily \{ binary\+: true \}} if you want to keep the T\+XT records in binary
\end{DoxyItemize}

\paragraph*{{\ttfamily var browser = bonjour.\+find\+One(options\mbox{[}, callback\mbox{]})}}

Listen for and call the {\ttfamily callback} with the first instance of a service matching the {\ttfamily options}. If no {\ttfamily callback} is given, it\textquotesingle{}s expected that you listen for the {\ttfamily up} event. The returned {\ttfamily browser} will automatically stop it self after the first matching service.

Options are the same as given in the {\ttfamily browser.\+find} function.

\paragraph*{{\ttfamily Event\+: up}}

Emitted every time a new service is found that matches the browser.

\paragraph*{{\ttfamily Event\+: down}}

Emitted every time an existing service emmits a goodbye message.

\paragraph*{{\ttfamily browser.\+services}}

An array of services known by the browser to be online.

\paragraph*{{\ttfamily browser.\+start()}}

Start looking for matching services.

\paragraph*{{\ttfamily browser.\+stop()}}

Stop looking for matching services.

\paragraph*{{\ttfamily browser.\+update()}}

Broadcast the query again.

\subsubsection*{Service}

\paragraph*{{\ttfamily Event\+: up}}

Emitted when the service is up.

\paragraph*{{\ttfamily Event\+: error}}

Emitted if an error occurrs while publishing the service.

\paragraph*{{\ttfamily service.\+stop(\mbox{[}callback\mbox{]})}}

Unpublish the service. The optional {\ttfamily callback} will be called when the service have been unpublished.

\paragraph*{{\ttfamily service.\+start()}}

Publish the service.

\paragraph*{{\ttfamily service.\+name}}

The name of the service, e.\+g. {\ttfamily Apple TV}.

\paragraph*{{\ttfamily service.\+type}}

The type of the service, e.\+g. {\ttfamily http}.

\paragraph*{{\ttfamily service.\+subtypes}}

An array of subtypes. Note that this property might be {\ttfamily null}.

\paragraph*{{\ttfamily service.\+protocol}}

The protocol used by the service, e.\+g. {\ttfamily tcp}.

\paragraph*{{\ttfamily service.\+host}}

The hostname or ip address where the service resides.

\paragraph*{{\ttfamily service.\+port}}

The port on which the service listens, e.\+g. {\ttfamily 5000}.

\paragraph*{{\ttfamily service.\+fqdn}}

The fully qualified domain name of the service. E.\+g. if given the name {\ttfamily Foo Bar}, the type {\ttfamily http} and the protocol {\ttfamily tcp}, the {\ttfamily service.\+fqdn} property will be {\ttfamily Foo Bar.\+\_\+http.\+\_\+tcp.\+local}.

\paragraph*{{\ttfamily service.\+txt}}

The T\+XT record advertised by the service (a key/value object). Note that this property might be {\ttfamily null}.

\paragraph*{{\ttfamily service.\+published}}

A boolean indicating if the service is currently published.

\subsection*{License}

M\+IT 