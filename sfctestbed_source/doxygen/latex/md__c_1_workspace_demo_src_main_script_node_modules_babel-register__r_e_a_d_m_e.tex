\begin{quote}
The require hook will bind itself to node\textquotesingle{}s require and automatically compile files on the fly. \end{quote}


One of the ways you can use Babel is through the require hook. The require hook will bind itself to node\textquotesingle{}s {\ttfamily require} and automatically compile files on the fly. This is equivalent to Coffee\+Script\textquotesingle{}s \href{http://coffeescript.org/documentation/docs/register.html}{\tt coffee-\/script/register}.

\subsection*{Install}


\begin{DoxyCode}
npm install babel-register --save-dev
\end{DoxyCode}


\subsection*{Usage}


\begin{DoxyCode}
require("babel-register");
\end{DoxyCode}


All subsequent files required by node with the extensions {\ttfamily .es6}, {\ttfamily .es}, {\ttfamily .jsx} and {\ttfamily .js} will be transformed by Babel.

\begin{quote}
\paragraph*{Polyfill not included}



You must include the \href{https://babeljs.io/docs/usage/polyfill/}{\tt polyfill} separately when using features that require it, like generators. 

\end{quote}


\subsubsection*{Ignores {\ttfamily node\+\_\+modules} by default}

{\bfseries N\+O\+TE\+:} By default all requires to {\ttfamily node\+\_\+modules} will be ignored. You can override this by passing an ignore regex via\+:


\begin{DoxyCode}
require("babel-register")(\{
  // This will override `node\_modules` ignoring - you can alternatively pass
  // an array of strings to be explicitly matched or a regex / glob
  ignore: false
\});
\end{DoxyCode}


\subsection*{Specifying options}

\`{}\`{}\`{}javascript require(\char`\"{}babel-\/register\char`\"{})(\{ // Optional ignore regex -\/ if any filenames {\bfseries do} match this regex then they // aren\textquotesingle{}t compiled. ignore\+: /regex/,

// Ignore can also be specified as a function. ignore\+: function(filename) \{ if (filename === \char`\"{}/path/to/es6-\/file.\+js\char`\"{}) \{ return false; \} else \{ return true; \} \},

// Optional only regex -\/ if any filenames {\bfseries don\textquotesingle{}t} match this regex then they // aren\textquotesingle{}t compiled only\+: /my\+\_\+es6\+\_\+folder/,

// Setting this will remove the currently hooked extensions of .es6, {\ttfamily .es}, {\ttfamily .jsx} // and .js so you\textquotesingle{}ll have to add them back if you want them to be used again. extensions\+: \mbox{[}\char`\"{}.\+es6\char`\"{}, \char`\"{}.\+es\char`\"{}, \char`\"{}.\+jsx\char`\"{}, \char`\"{}.\+js\char`\"{}\mbox{]} \}); 
\begin{DoxyCode}
You can pass in all other [options](https://babeljs.io/docs/usage/api/#options) as well,
including `plugins` and `presets`. But note that the closest
       [`.babelrc`](https://babeljs.io/docs/usage/babelrc/)
to each file still applies, and takes precedence over any options you pass in here.

## Environment variables

By default `babel-node` and `babel-register` will save to a json cache in your
temporary directory.

This will heavily improve with the startup and compilation of your files. There
are however scenarios where you want to change this behaviour and there are
environment variables exposed to allow you to do this.

### BABEL\_CACHE\_PATH

Specify a different cache location.

```sh
BABEL\_CACHE\_PATH=/foo/my-cache.json babel-node script.js
\end{DoxyCode}


\subsubsection*{B\+A\+B\+E\+L\+\_\+\+D\+I\+S\+A\+B\+L\+E\+\_\+\+C\+A\+C\+HE}

Disable the cache.


\begin{DoxyCode}
BABEL\_DISABLE\_CACHE=1 babel-node script.js
\end{DoxyCode}
 