\href{https://travis-ci.org/wycats/handlebars.js}{\tt } \href{https://saucelabs.com/u/handlebars}{\tt }

\section*{Handlebars.\+js }

Handlebars.\+js is an extension to the \href{http://mustache.github.com/}{\tt Mustache templating language} created by Chris Wanstrath. Handlebars.\+js and Mustache are both logicless templating languages that keep the view and the code separated like we all know they should be.

Checkout the official Handlebars docs site at \href{http://www.handlebarsjs.com}{\tt http\+://www.\+handlebarsjs.\+com} and the live demo at \href{http://tryhandlebarsjs.com/}{\tt http\+://tryhandlebarsjs.\+com/}.

\subsection*{Installing }

See our \href{http://handlebarsjs.com/installation.html}{\tt installation documentation}.

\subsection*{Usage }

In general, the syntax of Handlebars.\+js templates is a superset of Mustache templates. For basic syntax, check out the \href{http://mustache.github.com/mustache.5.html}{\tt Mustache manpage}.

Once you have a template, use the {\ttfamily Handlebars.\+compile} method to compile the template into a function. The generated function takes a context argument, which will be used to render the template.


\begin{DoxyCode}
var source = "<p>Hello, my name is \{\{name\}\}. I am from \{\{hometown\}\}. I have " +
             "\{\{kids.length\}\} kids:</p>" +
             "<ul>\{\{#kids\}\}<li>\{\{name\}\} is \{\{age\}\}</li>\{\{/kids\}\}</ul>";
var template = Handlebars.compile(source);

var data = \{ "name": "Alan", "hometown": "Somewhere, TX",
             "kids": [\{"name": "Jimmy", "age": "12"\}, \{"name": "Sally", "age": "4"\}]\};
var result = template(data);

// Would render:
// <p>Hello, my name is Alan. I am from Somewhere, TX. I have 2 kids:</p>
// <ul>
//   <li>Jimmy is 12</li>
//   <li>Sally is 4</li>
// </ul>
\end{DoxyCode}


Full documentation and more examples are at \href{http://handlebarsjs.com/}{\tt handlebarsjs.\+com}.

\subsection*{Precompiling Templates }

Handlebars allows templates to be precompiled and included as javascript code rather than the handlebars template allowing for faster startup time. Full details are located \href{http://handlebarsjs.com/precompilation.html}{\tt here}.

\subsection*{Differences Between Handlebars.\+js and Mustache }

Handlebars.\+js adds a couple of additional features to make writing templates easier and also changes a tiny detail of how partials work.


\begin{DoxyItemize}
\item \href{http://handlebarsjs.com/#paths}{\tt Nested Paths}
\item \href{http://handlebarsjs.com/#helpers}{\tt Helpers}
\item \href{http://handlebarsjs.com/#block-expressions}{\tt Block Expressions}
\item \href{http://handlebarsjs.com/#literals}{\tt Literal Values}
\item \href{http://handlebarsjs.com/#comments}{\tt Delimited Comments}
\end{DoxyItemize}

Block expressions have the same syntax as mustache sections but should not be confused with one another. Sections are akin to an implicit {\ttfamily each} or {\ttfamily with} statement depending on the input data and helpers are explicit pieces of code that are free to implement whatever behavior they like. The \href{http://mustache.github.io/mustache.5.html}{\tt mustache spec} defines the exact behavior of sections. In the case of name conflicts, helpers are given priority.

\subsubsection*{Compatibility}

There are a few Mustache behaviors that Handlebars does not implement.
\begin{DoxyItemize}
\item Handlebars deviates from Mustache slightly in that it does not perform recursive lookup by default. The compile time {\ttfamily compat} flag must be set to enable this functionality. Users should note that there is a performance cost for enabling this flag. The exact cost varies by template, but it\textquotesingle{}s recommended that performance sensitive operations should avoid this mode and instead opt for explicit path references.
\item The optional Mustache-\/style lambdas are not supported. Instead Handlebars provides its own lambda resolution that follows the behaviors of helpers.
\item Alternative delimiters are not supported.
\end{DoxyItemize}

\subsection*{Supported Environments }

Handlebars has been designed to work in any E\+C\+M\+A\+Script 3 environment. This includes


\begin{DoxyItemize}
\item Node.\+js
\item Chrome
\item Firefox
\item Safari 5+
\item Opera 11+
\item IE 6+
\end{DoxyItemize}

Older versions and other runtimes are likely to work but have not been formally tested. The compiler requires {\ttfamily J\+S\+O\+N.\+stringify} to be implemented natively or via a polyfill. If using the precompiler this is not necessary.

\href{https://saucelabs.com/u/handlebars}{\tt }

\subsection*{Performance }

In a rough performance test, precompiled Handlebars.\+js templates (in the original version of Handlebars.\+js) rendered in about half the time of Mustache templates. It would be a shame if it were any other way, since they were precompiled, but the difference in architecture does have some big performance advantages. Justin Marney, a.\+k.\+a. \href{http://github.com/gotascii}{\tt gotascii}, confirmed that with an \href{http://sorescode.com/2010/09/12/benchmarks.html}{\tt independent test}. The rewritten Handlebars (current version) is faster than the old version, with many \href{https://travis-ci.org/wycats/handlebars.js/builds/33392182#L538}{\tt performance tests} being 5 to 7 times faster than the Mustache equivalent.

\subsection*{Upgrading }

See https\+://github.com/wycats/handlebars.\+js/blob/master/release-\/notes.md \char`\"{}release-\/notes.\+md\char`\"{} for upgrade notes.

\subsection*{Known Issues }

See https\+://github.com/wycats/handlebars.\+js/blob/master/\+F\+AQ.md \char`\"{}\+F\+A\+Q.\+md\char`\"{} for known issues and common pitfalls.

\subsection*{Handlebars in the Wild }


\begin{DoxyItemize}
\item \href{http://assemble.io}{\tt Assemble}, by \href{https://github.com/jonschlinkert}{\tt } and \href{https://github.com/doowb}{\tt }, is a static site generator that uses Handlebars.\+js as its template engine.
\item \href{https://github.com/leo/cory}{\tt Cory}, by \href{https://github.com/leo}{\tt }, is another tiny static site generator
\item \href{http://coschedule.com}{\tt Co\+Schedule} An editorial calendar for Word\+Press that uses Handlebars.\+js
\item \href{https://github.com/pismute/dashbars}{\tt dashbars} A modern helper library for Handlebars.\+js.
\item \href{http://www.emberjs.com}{\tt Ember.\+js} makes Handlebars.\+js the primary way to structure your views, also with automatic data binding support.
\item \href{https://ghost.org/}{\tt Ghost} Just a blogging platform.
\item \href{http://github.com/leshill/handlebars_assets}{\tt handlebars\+\_\+assets}\+: A Rails Asset Pipeline gem from Les Hill ().
\item \href{https://github.com/assemble/handlebars-helpers}{\tt handlebars-\/helpers} is an extensive library with 100+ handlebars helpers.
\item \href{https://github.com/shannonmoeller/handlebars-layouts}{\tt handlebars-\/layouts} is a set of helpers which implement extendible and embeddable layout blocks as seen in other popular templating languages.
\item \href{http://github.com/donpark/hbs}{\tt hbs}\+: An Express.\+js view engine adapter for Handlebars.\+js, from Don Park.
\item \href{https://github.com/jwilm/koa-hbs}{\tt koa-\/hbs}\+: \href{https://github.com/koajs/koa}{\tt koa} generator based renderer for Handlebars.\+js.
\item \href{http://github.com/jblotus}{\tt jblotus} created \href{http://tryhandlebarsjs.com}{\tt http\+://tryhandlebarsjs.\+com} for anyone who would like to try out Handlebars.\+js in their browser.
\item \href{http://71104.github.io/jquery-handlebars/}{\tt j\+Query plugin}\+: allows you to use Handlebars.\+js with \href{http://jquery.com/}{\tt j\+Query}.
\item \href{http://walmartlabs.github.io/lumbar}{\tt Lumbar} provides easy module-\/based template management for handlebars projects.
\item \href{https://github.com/hashchange/marionette.handlebars}{\tt Marionette.\+Handlebars} adds support for Handlebars and Mustache templates to Marionette.
\item \href{http://github.com/quirkey/sammy}{\tt sammy.\+js} by Aaron Quint, a.\+k.\+a. quirkey, supports Handlebars.\+js as one of its template plugins.
\item \href{http://www.sproutcore.com}{\tt Sprout\+Core} uses Handlebars.\+js as its main templating engine, extending it with automatic data binding support.
\item \href{http://yuilibrary.com/yui/docs/handlebars/}{\tt Y\+UI} implements a port of handlebars
\item \href{https://github.com/elving/swag}{\tt Swag} by \href{https://github.com/elving}{\tt } is a growing collection of helpers for handlebars.\+js. Give your handlebars.\+js templates some swag son!
\item \href{https://github.com/blakeembrey/dombars}{\tt D\+O\+M\+Bars} is a D\+O\+M-\/based templating engine built on the Handlebars parser and runtime {\bfseries D\+E\+P\+R\+E\+C\+A\+T\+ED}
\item \href{https://github.com/nknapp/promised-handlebars}{\tt promised-\/handlebars} is a wrapper for Handlebars that allows helpers to return Promises.
\item \href{https://github.com/leapfrogtechnology/just-handlebars-helpers}{\tt just-\/handlebars-\/helpers} A fully tested lightweight package with common Handlebars helpers.
\end{DoxyItemize}

\subsection*{External Resources }


\begin{DoxyItemize}
\item \href{https://gist.github.com/2287070}{\tt Gist about Synchronous and asynchronous loading of external handlebars templates}
\end{DoxyItemize}

Have a project using Handlebars? Send us a \href{https://github.com/wycats/handlebars.js/pull/new/master}{\tt pull request}!

\subsection*{License }

Handlebars.\+js is released under the M\+IT license. 