\href{https://travis-ci.org/jergason/recursive-readdir}{\tt }

Recursively list all files in a directory and its subdirectories. It does not list the directories themselves.

Because it uses fs.\+readdir, which calls \href{http://linux.die.net/man/3/readdir}{\tt readdir} under the hood on OS X and Linux, the order of files inside directories is \href{http://stackoverflow.com/questions/8977441/does-readdir-guarantee-an-order}{\tt not guaranteed}.

\subsection*{Installation}

\begin{DoxyVerb}npm install recursive-readdir
\end{DoxyVerb}


\subsection*{Usage}


\begin{DoxyCode}
var recursive = require("recursive-readdir");

recursive("some/path", function (err, files) \{
  // `files` is an array of absolute file paths
  console.log(files);
\});
\end{DoxyCode}


It can also take a list of files to ignore.


\begin{DoxyCode}
var recursive = require("recursive-readdir");

// ignore files named "foo.cs" or files that end in ".html".
recursive("some/path", ["foo.cs", "*.html"], function (err, files) \{
  console.log(files);
\});
\end{DoxyCode}


You can also pass functions which are called to determine whether or not to ignore a file\+:

\`{}\`{}\`{}javascript var recursive = require(\char`\"{}recursive-\/readdir\char`\"{});

function ignore\+Func(file, stats) \{ // {\ttfamily file} is the absolute path to the file, and {\ttfamily stats} is an {\ttfamily fs.\+Stats} // object returned from {\ttfamily fs.\+lstat()}. return stats.\+is\+Directory() \&\& path.\+basename(file) == \char`\"{}test\char`\"{}; \}

// Ignore files named \char`\"{}foo.\+cs\char`\"{} and descendants of directories named test recursive(\char`\"{}some/path\char`\"{}, \mbox{[}\char`\"{}foo.\+cs\char`\"{}, ignore\+Func\mbox{]}, function (err, files) \{ console.\+log(files); \}); 
\begin{DoxyCode}
## Promises
You can omit the callback and return a promise instead.

```javascript
readdir("some/path").then(
  function(files) \{
    console.log("files are", files);
  \},
  function(error) \{
    console.error("something exploded", error);
  \}
);
\end{DoxyCode}


The ignore strings support Glob syntax via \href{https://github.com/isaacs/minimatch}{\tt minimatch}. 