\href{http://unshift.io}{\tt }\href{http://browsenpm.org/package/requires-port}{\tt }\href{https://travis-ci.org/unshiftio/requires-port}{\tt }\href{https://david-dm.org/unshiftio/requires-port}{\tt }\href{https://coveralls.io/r/unshiftio/requires-port?branch=master}{\tt }\href{http://webchat.freenode.net/?channels=unshift}{\tt }

The module name says it all, check if a protocol requires a given port.

\subsection*{Installation}

This module is intended to be used with browserify or Node.\+js and is distributed in the public npm registry. To install it simply run the following command from your C\+LI\+:


\begin{DoxyCode}
npm install --save requires-port
\end{DoxyCode}


\subsection*{Usage}

The module exports it self as function and requires 2 arguments\+:


\begin{DoxyEnumerate}
\item The port number, can be a string or number.
\item Protocol, can be {\ttfamily http}, {\ttfamily http\+:} or even {\ttfamily \href{https://yomoma.com}{\tt https\+://yomoma.\+com}}. We just split it at {\ttfamily \+:} and use the first result. We currently accept the following protocols\+:
\begin{DoxyItemize}
\item {\ttfamily http}
\item {\ttfamily https}
\item {\ttfamily ws}
\item {\ttfamily wss}
\item {\ttfamily ftp}
\item {\ttfamily gopher}
\item {\ttfamily file}
\end{DoxyItemize}
\end{DoxyEnumerate}

It returns a boolean that indicates if protocol requires this port to be added to your \mbox{\hyperlink{namespace_u_r_l}{U\+RL}}.


\begin{DoxyCode}
'use strict';

var required = require('requires-port');

console.log(required('8080', 'http')) // true
console.log(required('80', 'http'))   // false
\end{DoxyCode}


\section*{License}

M\+IT 