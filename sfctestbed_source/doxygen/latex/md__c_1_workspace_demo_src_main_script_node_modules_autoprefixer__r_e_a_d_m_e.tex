

\href{https://github.com/postcss/postcss}{\tt Post\+C\+SS} plugin to parse C\+SS and add vendor prefixes to C\+SS rules using values from \href{http://caniuse.com/}{\tt Can I Use}. It is \href{https://developers.google.com/web/tools/setup/setup-buildtools#dont_trip_up_with_vendor_prefixes}{\tt recommended} by Google and used in Twitter and Taobao.

Write your C\+SS rules without vendor prefixes (in fact, forget about them entirely)\+:


\begin{DoxyCode}
:fullscreen a \{
    display: flex
\}
\end{DoxyCode}


Autoprefixer will use the data based on current browser popularity and property support to apply prefixes for you. You can try the \href{http://autoprefixer.github.io/}{\tt interactive demo} of Autoprefixer.


\begin{DoxyCode}
:-webkit-full-screen a \{
    display: -webkit-box;
    display: flex
\}
:-moz-full-screen a \{
    display: flex
\}
:-ms-fullscreen a \{
    display: -ms-flexbox;
    display: flex
\}
:fullscreen a \{
    display: -webkit-box;
    display: -ms-flexbox;
    display: flex
\}
\end{DoxyCode}


Twitter account for news and releases\+: \href{https://twitter.com/autoprefixer}{\tt }.

\href{https://evilmartians.com/?utm_source=autoprefixer}{\tt }

\subsection*{Features}

\subsubsection*{Write Pure C\+SS}

Working with Autoprefixer is simple\+: just forget about vendor prefixes and write normal C\+SS according to the latest \+W3\+C specs. You don’t need a special language (like Sass) or remember where you must use mixins.

Autoprefixer supports selectors (like {\ttfamily \+:fullscreen} and {\ttfamily \+::selection}), unit function ({\ttfamily calc()}), at‑rules ({\ttfamily @supports} and {\ttfamily @keyframes}) and properties.

Because Autoprefixer is a postprocessor for C\+SS, you can also use it with preprocessors such as \+Sass, Stylus or \+L\+E\+SS.

\subsubsection*{Flexbox, Filters, etc.}

Just write normal C\+SS according to the latest W3C specs and \+Autoprefixer will produce the code for old browsers.


\begin{DoxyCode}
a \{
    display: flex;
\}
\end{DoxyCode}


compiles to\+:


\begin{DoxyCode}
a \{
    display: -webkit-box;
    display: -webkit-flex;
    display: -ms-flexbox;
    display: flex
\}
\end{DoxyCode}


Autoprefixer has \href{https://github.com/postcss/autoprefixer/tree/master/lib/hacks}{\tt 27 special hacks} to fix web browser differences.

\subsubsection*{Only Actual Prefixes}

Autoprefixer utilizes the most recent data from \href{http://caniuse.com/}{\tt Can I Use} to add only necessary vendor prefixes.

It also removes old, unnecessary prefixes from your C\+SS (like {\ttfamily border-\/radius} prefixes, produced by many \+C\+SS libraries).


\begin{DoxyCode}
a \{
    -webkit-border-radius: 5px;
            border-radius: 5px;
\}
\end{DoxyCode}


compiles to\+:


\begin{DoxyCode}
a \{
    border-radius: 5px;
\}
\end{DoxyCode}


\subsection*{Browsers}

Autoprefixer uses \href{https://github.com/ai/browserslist}{\tt Browserslist}, so you can specify the browsers you want to target in your project by queries like {\ttfamily last 2 versions} or {\ttfamily $>$ 5\%}.

The best way to provide browsers is {\ttfamily browserslist} config or {\ttfamily package.\+json} with {\ttfamily browserslist} key. Put it in your project root.

We recommend to avoid Autoprefixer option and use {\ttfamily browserslist} config or {\ttfamily package.\+json}. In this case browsers will be shared with other tools like \href{https://github.com/babel/babel-preset-env}{\tt babel-\/preset-\/env} or \href{http://stylelint.io/}{\tt Stylelint}.

See \href{https://github.com/ai/browserslist#queries}{\tt Browserslist docs} for queries, browser names, config format, and default value.

\subsection*{Outdated Prefixes}

By default, Autoprefixer also removes outdated prefixes.

You can disable this behavior with the {\ttfamily remove\+: false} option. If you have no legacy code, this option will make Autoprefixer about 10\% faster.

Also, you can set the {\ttfamily add\+: false} option. Autoprefixer will only clean outdated prefixes, but will not add any new prefixes.

Autoprefixer adds new prefixes between any unprefixed properties and already written prefixes in your C\+SS. If it will break the expected prefixes order, you can clean all prefixes from your C\+SS and then add the necessary prefixes again\+:


\begin{DoxyCode}
var cleaner  = postcss([ autoprefixer(\{ add: false, browsers: [] \}) ]);
var prefixer = postcss([ autoprefixer ]);

cleaner.process(css).then(function (cleaned) \{
    return prefixer.process(cleaned.css)
\}).then(function (result) \{
    console.log(result.css);
\});
\end{DoxyCode}


\subsection*{F\+AQ}

\paragraph*{No prefixes in production}

Many other tools contain Autoprefixer. For example, webpack uses Autoprefixer to minify C\+SS by cleaning unnecessary prefixes.

If you set browsers list to Autoprefixer by {\ttfamily browsers} option, only first Autoprefixer will know your browsers. Autoprefixer inside webpack will use default browsers list. As result, webpack will remove prefixes, that first Autoprefixer added.

You need to put your browsers to \href{https://github.com/ai/browserslist#config-file}{\tt {\ttfamily browserslist} config} in project root — as result all tools (Autoprefixer, cssnano, doiuse, cssnext) will use same browsers list.

\paragraph*{Does it add polyfills?}

No. Autoprefixer only adds prefixes.

Most new C\+SS features will require client side Java\+Script to handle a new behavior correctly.

Depending on what you consider to be a “polyfill”, you can take a look at some other tools and libraries. If you are just looking for syntax sugar, you might take a look at\+:


\begin{DoxyItemize}
\item \href{https://github.com/jonathantneal/oldie}{\tt Oldie}, a Post\+C\+SS plugin that handles some IE hacks (opacity, rgba, etc).
\item \href{https://github.com/luisrudge/postcss-flexbugs-fixes}{\tt postcss-\/flexbugs-\/fixes}, a Post\+C\+SS plugin to fix flexbox issues.
\item \href{https://github.com/MoOx/postcss-cssnext}{\tt cssnext}, a tool that allows you to write standard C\+SS syntax non-\/implemented yet in browsers (custom properties, custom media, color functions, etc).
\end{DoxyItemize}

\paragraph*{Why doesn’t Autoprefixer add prefixes to {\ttfamily border-\/radius}?}

Developers are often surprised by how few prefixes are required today. If Autoprefixer doesn’t add prefixes to your C\+SS, check if they’re still required on \href{http://caniuse.com/}{\tt Can I Use}.

There is a \href{https://github.com/postcss/autoprefixer/wiki/support-list}{\tt list with all supported} properties, values, and selectors.

\paragraph*{Why Autoprefixer uses unprefixed properties in {\ttfamily @-\/webkit-\/keyframes}?}

Browser teams can remove some prefixes before others. So we try to use all combinations of prefixed/unprefixed values.

\paragraph*{How to work with legacy {\ttfamily -\/webkit-\/} only code?}

Autoprefixer needs unprefixed property to add prefixes. So if you only wrote {\ttfamily -\/webkit-\/gradient} without W3\+C’s {\ttfamily gradient}, Autoprefixer will not add other prefixes.

But \href{https://github.com/postcss/postcss}{\tt Post\+C\+SS} has a plugins to convert C\+SS to unprefixed state. Use \href{https://github.com/gucong3000/postcss-unprefix}{\tt postcss-\/unprefix} before Autoprefixer.

\paragraph*{Does Autoprefixer add {\ttfamily -\/epub-\/} prefix?}

No, Autoprefixer works only with browsers prefixes from Can I Use. But you can use \href{https://github.com/Rycochet/postcss-epub}{\tt postcss-\/epub} for prefixing e\+Pub3 properties.

\paragraph*{Why doesn’t Autoprefixer transform generic font-\/family {\ttfamily system-\/ui}?}

{\ttfamily system-\/ui} is technically not a prefix and the transformation is not future-\/proof. But you can use \href{https://github.com/JLHwung/postcss-font-family-system-ui}{\tt postcss-\/font-\/family-\/system-\/ui} to transform {\ttfamily system-\/ui} to a practical font-\/family list.

\subsection*{Usage}

\subsubsection*{Gulp}

In Gulp you can use \href{https://github.com/postcss/gulp-postcss}{\tt gulp-\/postcss} with {\ttfamily autoprefixer} npm package.


\begin{DoxyCode}
gulp.task('autoprefixer', function () \{
    var postcss      = require('gulp-postcss');
    var sourcemaps   = require('gulp-sourcemaps');
    var autoprefixer = require('autoprefixer');

    return gulp.src('./src/*.css')
        .pipe(sourcemaps.init())
        .pipe(postcss([ autoprefixer() ]))
        .pipe(sourcemaps.write('.'))
        .pipe(gulp.dest('./dest'));
\});
\end{DoxyCode}


With {\ttfamily gulp-\/postcss} you also can combine Autoprefixer with \href{https://github.com/postcss/postcss#plugins}{\tt other Post\+C\+SS plugins}.

\subsubsection*{Webpack}

In \href{http://webpack.github.io/}{\tt webpack} you can use \href{https://github.com/postcss/postcss-loader}{\tt postcss-\/loader} with {\ttfamily autoprefixer} and \href{https://github.com/postcss/postcss#plugins}{\tt other Post\+C\+SS plugins}.


\begin{DoxyCode}
module.exports = \{
    module: \{
        loaders: [
            \{
                test:   /\(\backslash\).css$/,
                loader: "style-loader!css-loader!postcss-loader"
            \}
        ]
    \}
\}
\end{DoxyCode}


And create a {\ttfamily postcss.\+config.\+js} with\+:


\begin{DoxyCode}
module.exports = \{
  plugins: [
    require('autoprefixer')
  ]
\}
\end{DoxyCode}


\subsubsection*{Grunt}

In Grunt you can use \href{https://github.com/nDmitry/grunt-postcss}{\tt grunt-\/postcss} with {\ttfamily autoprefixer} npm package.


\begin{DoxyCode}
module.exports = function(grunt) \{
    grunt.loadNpmTasks('grunt-postcss');

    grunt.initConfig(\{
        postcss: \{
            options: \{
                map: true,
                processors: [
                    require('autoprefixer')
                ]
            \},
            dist: \{
                src: 'css/*.css'
            \}
        \}
    \});

    grunt.registerTask('default', ['postcss:dist']);
\};
\end{DoxyCode}


With {\ttfamily grunt-\/postcss} you also can combine Autoprefixer with \href{https://github.com/postcss/postcss#plugins}{\tt other Post\+C\+SS plugins}.

\subsubsection*{Other Build Tools\+:}


\begin{DoxyItemize}
\item {\bfseries Ruby on Rails}\+: \href{https://github.com/ai/autoprefixer-rails}{\tt autoprefixer-\/rails}
\item {\bfseries Brunch}\+: \href{https://github.com/iamvdo/postcss-brunch}{\tt postcss-\/brunch}
\item {\bfseries Broccoli}\+: \href{https://github.com/jeffjewiss/broccoli-postcss}{\tt broccoli-\/postcss}
\item {\bfseries Middleman}\+: \href{https://github.com/middleman/middleman-autoprefixer}{\tt middleman-\/autoprefixer}
\item {\bfseries Mincer}\+: add {\ttfamily autoprefixer} npm package and enable it\+: `environment.\+enable(\textquotesingle{}autoprefixer'){\ttfamily }
\item {\ttfamily $\ast$$\ast$\+Jekyll$\ast$$\ast$\+: add}autoprefixer-\/rails{\ttfamily and}jekyll-\/assets{\ttfamily to}Gemfile\`{}
\end{DoxyItemize}

\subsubsection*{Preprocessors}


\begin{DoxyItemize}
\item {\bfseries Less}\+: \href{https://github.com/less/less-plugin-autoprefix}{\tt less-\/plugin-\/autoprefix}
\item {\bfseries Stylus}\+: \href{https://github.com/jenius/autoprefixer-stylus}{\tt autoprefixer-\/stylus}
\item {\bfseries Compass}\+: \href{https://github.com/ai/autoprefixer-rails#compass}{\tt autoprefixer-\/rails\#compass}
\end{DoxyItemize}

\subsubsection*{C\+S\+S-\/in-\/\+JS}

There is \href{https://github.com/postcss/postcss-js}{\tt postcss-\/js} to use Autoprefixer in React Inline Styles, \href{https://github.com/blakeembrey/free-style}{\tt Free Style}, Radium and other C\+S\+S-\/in-\/\+JS solutions.


\begin{DoxyCode}
let prefixer = postcssJs.sync([ autoprefixer ]);
let style = prefixer(\{
    display: 'flex'
\});
\end{DoxyCode}


\subsubsection*{G\+UI Tools}


\begin{DoxyItemize}
\item \href{https://codekitapp.com/help/autoprefixer/}{\tt Code\+Kit}
\item \href{https://prepros.io}{\tt Prepros}
\end{DoxyItemize}

\subsubsection*{C\+LI}

You can use the \href{https://github.com/postcss/postcss-cli}{\tt postcss-\/cli} to run Autoprefixer from C\+LI\+:


\begin{DoxyCode}
npm install --global postcss-cli autoprefixer
postcss *.css --use autoprefixer -d build/
\end{DoxyCode}


See {\ttfamily postcss -\/h} for help.

\subsubsection*{Java\+Script}

You can use Autoprefixer with \href{https://github.com/postcss/postcss}{\tt Post\+C\+SS} in your Node.\+js application or if you want to develop an Autoprefixer plugin for new environment.


\begin{DoxyCode}
var autoprefixer = require('autoprefixer');
var postcss      = require('postcss');

postcss([ autoprefixer ]).process(css).then(function (result) \{
    result.warnings().forEach(function (warn) \{
        console.warn(warn.toString());
    \});
    console.log(result.css);
\});
\end{DoxyCode}


There is also \href{https://raw.github.com/ai/autoprefixer-rails/master/vendor/autoprefixer.js}{\tt standalone build} for the browser or as a non-\/\+Node.\+js runtime.

You can use \href{https://github.com/RebelMail/html-autoprefixer}{\tt html-\/autoprefixer} to process H\+T\+ML with inlined C\+SS.

\subsubsection*{Text Editors and I\+DE}

Autoprefixer should be used in assets build tools. Text editor plugins are not a good solution, because prefixes decrease code readability and you will need to change value in all prefixed properties.

I recommend you to learn how to use build tools like \href{http://gulpjs.com/}{\tt Gulp}. They work much better and will open you a whole new world of useful plugins and automatization.

But, if you can’t move to a build tool, you can use text editor plugins\+:


\begin{DoxyItemize}
\item \href{https://github.com/sindresorhus/sublime-autoprefixer}{\tt Sublime Text}
\item \href{https://github.com/mikaeljorhult/brackets-autoprefixer}{\tt Brackets}
\item \href{https://github.com/sindresorhus/atom-autoprefixer}{\tt Atom Editor}
\item \href{http://vswebessentials.com/}{\tt Visual Studio}
\end{DoxyItemize}

\subsection*{Warnings}

Autoprefixer uses the \href{https://github.com/postcss/postcss/blob/master/docs/api.md#warning-class}{\tt Post\+C\+SS warning A\+PI} to warn about really important problems in your C\+SS\+:


\begin{DoxyItemize}
\item Old direction syntax in gradients.
\item Old unprefixed {\ttfamily display\+: box} instead of {\ttfamily display\+: flex} by latest specification version.
\end{DoxyItemize}

You can get warnings from {\ttfamily result.\+warnings()}\+:


\begin{DoxyCode}
result.warnings().forEach(function (warn) \{
    console.warn(warn.toString());
\});
\end{DoxyCode}


Every Autoprefixer runner should display this warnings.

\subsection*{Disabling}

Autoprefixer was designed to have no interface – it just works. If you need some browser specific hack just write a prefixed property after the unprefixed one.


\begin{DoxyCode}
a \{
    transform: scale(0.5);
    -moz-transform: scale(0.6);
\}
\end{DoxyCode}


If some prefixes were generated in a wrong way, please create an issue on Git\+Hub.

Autoprefixer has 4 features, which can be enabled or disabled by options\+:


\begin{DoxyItemize}
\item {\ttfamily supports\+: false} will disable {\ttfamily @supports} parameters prefixing.
\item {\ttfamily flexbox\+: false} will disable flexbox properties prefixing. Or {\ttfamily flexbox\+: \char`\"{}no-\/2009\char`\"{}} will add prefixes only for final and IE versions of specification.
\item {\ttfamily remove\+: false} will disable cleaning outdated prefixes.
\item {\ttfamily grid\+: true} will enable Grid Layout prefixes for IE.
\end{DoxyItemize}

If you do not need Autoprefixer in some part of your C\+SS, you can use control comments to disable \+Autoprefixer.


\begin{DoxyCode}
a \{
    transition: 1s; /* it will be prefixed */
\}

b \{
    /* autoprefixer: off */
    transition: 1s; /* it will not be prefixed */
\}
\end{DoxyCode}


Control comments disable Autoprefixer within the whole rule in which you place it. In the above example, Autoprefixer will be disabled in the entire {\ttfamily b} rule scope, not only after the comment.

You can also use comments recursively\+:


\begin{DoxyCode}
/* autoprefixer: off */
@supports (transition: all) \{
    /* autoprefixer: on */
    a \{
        /* autoprefixer: off */
    \}
\}
\end{DoxyCode}


In Sass/\+S\+C\+SS you can use all the disable options above, add an exclamation mark in the start of comment\+: {\ttfamily /$\ast$! autoprefixer\+: off $\ast$/}.

\subsection*{Options}

Function {\ttfamily autoprefixer(options)} returns new Post\+C\+SS plugin. See \href{http://api.postcss.org}{\tt Post\+C\+SS A\+PI} for plugin usage documentation.


\begin{DoxyCode}
var plugin = autoprefixer(\{ cascade: false \});
\end{DoxyCode}


There are 8 options\+:


\begin{DoxyItemize}
\item {\ttfamily browsers} (array)\+: list of browsers query (like {\ttfamily last 2 versions}), which are supported in your project. We recommend to use {\ttfamily browserslist} config or {\ttfamily browserslist} key in {\ttfamily package.\+json}, rather than this option to share browsers with other tools. See \href{https://github.com/ai/browserslist#queries}{\tt Browserslist docs} for available queries and default value.
\item {\ttfamily env} (string)\+: environment for Browserslist.
\item {\ttfamily cascade} (boolean)\+: should Autoprefixer use Visual Cascade, if C\+SS is uncompressed. Default\+: {\ttfamily true}
\item {\ttfamily add} (boolean)\+: should Autoprefixer add prefixes. Default is {\ttfamily true}.
\item {\ttfamily remove} (boolean)\+: should Autoprefixer \mbox{[}remove outdated\mbox{]} prefixes. Default is {\ttfamily true}.
\item {\ttfamily supports} (boolean)\+: should Autoprefixer add prefixes for {\ttfamily @supports} parameters. Default is {\ttfamily true}.
\item {\ttfamily flexbox} (boolean$\vert$string)\+: should Autoprefixer add prefixes for flexbox properties. With {\ttfamily \char`\"{}no-\/2009\char`\"{}} value Autoprefixer will add prefixes only for final and IE versions of specification. Default is {\ttfamily true}.
\item {\ttfamily grid} (boolean)\+: should Autoprefixer add IE prefixes for Grid Layout properties. Default is {\ttfamily false}.
\item {\ttfamily stats} (object)\+: custom \href{https://github.com/ai/browserslist#custom-usage-data}{\tt usage statistics} for {\ttfamily $>$ 10\% in my stats} browsers query.
\end{DoxyItemize}

Plugin object has {\ttfamily info()} method for debugging purpose.

You can use Post\+C\+SS processor to process several C\+SS files to increase performance.

\subsection*{Debug}

You can check which browsers are selected and which properties will be prefixed\+:


\begin{DoxyCode}
var info = autoprefixer().info();
console.log(info);
\end{DoxyCode}
 