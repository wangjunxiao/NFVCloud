\href{https://www.npmjs.com/package/espree}{\tt } \href{https://travis-ci.org/eslint/espree}{\tt } \href{https://www.npmjs.com/package/espree}{\tt } \href{https://www.bountysource.com/trackers/9348450-eslint?utm_source=9348450&utm_medium=shield&utm_campaign=TRACKER_BADGE}{\tt }

\section*{Espree}

Espree started out as a fork of \href{http://esprima.org}{\tt Esprima} v1.\+2.\+2, the last stable published released of Esprima before work on E\+C\+M\+A\+Script 6 began. Espree is now built on top of \href{https://github.com/ternjs/acorn}{\tt Acorn}, which has a modular architecture that allows extension of core functionality. The goal of Espree is to produce output that is similar to Esprima with a similar A\+PI so that it can be used in place of Esprima.

\subsection*{Usage}

Install\+:


\begin{DoxyCode}
npm i espree --save
\end{DoxyCode}


And in your Node.\+js code\+:


\begin{DoxyCode}
var espree = require("espree");

var ast = espree.parse(code);
\end{DoxyCode}


There is a second argument to {\ttfamily parse()} that allows you to specify various options\+:


\begin{DoxyCode}
var espree = require("espree");

var ast = espree.parse(code, \{

    // attach range information to each node
    range: true,

    // attach line/column location information to each node
    loc: true,

    // create a top-level comments array containing all comments
    comment: true,

    // attach comments to the closest relevant node as leadingComments and
    // trailingComments
    attachComment: true,

    // create a top-level tokens array containing all tokens
    tokens: true,

    // set to 3, 5 (default), 6, 7, or 8 to specify the version of ECMAScript syntax you want to use. 
    // You can also set to 2015 (same as 6), 2016 (same as 7), or 2017 (same as 8) to use the year-based
       naming.
    ecmaVersion: 5,

    // specify which type of script you're parsing (script or module, default is script)
    sourceType: "script",

    // specify additional language features
    ecmaFeatures: \{

        // enable JSX parsing
        jsx: true,

        // enable return in global scope
        globalReturn: true,

        // enable implied strict mode (if ecmaVersion >= 5)
        impliedStrict: true,

        // allow experimental object rest/spread
        experimentalObjectRestSpread: true
    \}
\});
\end{DoxyCode}


\subsection*{Esprima Compatibility Going Forward}

The primary goal is to produce the exact same A\+ST structure and tokens as Esprima, and that takes precedence over anything else. (The A\+ST structure being the \href{https://github.com/estree/estree}{\tt E\+S\+Tree} A\+PI with J\+SX extensions.) Separate from that, Espree may deviate from what Esprima outputs in terms of where and how comments are attached, as well as what additional information is available on A\+ST nodes. That is to say, Espree may add more things to the A\+ST nodes than Esprima does but the overall A\+ST structure produced will be the same.

Espree may also deviate from Esprima in the interface it exposes.

\subsection*{Contributing}

Issues and pull requests will be triaged and responded to as quickly as possible. We operate under the \href{http://eslint.org/docs/developer-guide/contributing}{\tt E\+S\+Lint Contributor Guidelines}, so please be sure to read them before contributing. If you\textquotesingle{}re not sure where to dig in, check out the \href{https://github.com/eslint/espree/issues}{\tt issues}.

Espree is licensed under a permissive B\+SD 2-\/clause license.

\subsection*{Build Commands}


\begin{DoxyItemize}
\item {\ttfamily npm test} -\/ run all linting and tests
\item {\ttfamily npm run lint} -\/ run all linting
\item {\ttfamily npm run browserify} -\/ creates a version of Espree that is usable in a browser
\end{DoxyItemize}

\subsection*{Differences from Espree 2.\+x}


\begin{DoxyItemize}
\item The {\ttfamily tokenize()} method does not use {\ttfamily ecma\+Features}. Any string will be tokenized completely based on E\+C\+M\+A\+Script 6 semantics.
\item Trailing whitespace no longer is counted as part of a node.
\item {\ttfamily let} and {\ttfamily const} declarations are no longer parsed by default. You must opt-\/in using {\ttfamily ecma\+Features.\+block\+Bindings}.
\item The {\ttfamily esparse} and {\ttfamily esvalidate} binary scripts have been removed.
\item There is no {\ttfamily tolerant} option. We will investigate adding this back in the future.
\end{DoxyItemize}

\subsection*{Known Incompatibilities}

In an effort to help those wanting to transition from other parsers to Espree, the following is a list of noteworthy incompatibilities with other parsers. These are known differences that we do not intend to change.

\subsubsection*{Esprima 1.\+2.\+2}


\begin{DoxyItemize}
\item Esprima counts trailing whitespace as part of each A\+ST node while Espree does not. In Espree, the end of a node is where the last token occurs.
\item Espree does not parse {\ttfamily let} and {\ttfamily const} declarations by default.
\item Error messages returned for parsing errors are different.
\item There are two addition properties on every node and token\+: {\ttfamily start} and {\ttfamily end}. These represent the same data as {\ttfamily range} and are used internally by Acorn.
\end{DoxyItemize}

\subsubsection*{Esprima 2.\+x}


\begin{DoxyItemize}
\item Esprima 2.\+x uses a different comment attachment algorithm that results in some comments being added in different places than Espree. The algorithm Espree uses is the same one used in Esprima 1.\+2.\+2.
\end{DoxyItemize}

\subsection*{Frequently Asked Questions}

\subsubsection*{Why another parser}

\href{http://eslint.org}{\tt E\+S\+Lint} had been relying on Esprima as its parser from the beginning. While that was fine when the Java\+Script language was evolving slowly, the pace of development increased dramatically and Esprima had fallen behind. E\+S\+Lint, like many other tools reliant on Esprima, has been stuck in using new Java\+Script language features until Esprima updates, and that caused our users frustration.

We decided the only way for us to move forward was to create our own parser, bringing us inline with J\+S\+Hint and J\+S\+Lint, and allowing us to keep implementing new features as we need them. We chose to fork Esprima instead of starting from scratch in order to move as quickly as possible with a compatible A\+PI.

With Espree 2.\+0.\+0, we are no longer a fork of Esprima but rather a translation layer between Acorn and Esprima syntax. This allows us to put work back into a community-\/supported parser (Acorn) that is continuing to grow and evolve while maintaining an Esprima-\/compatible parser for those utilities still built on Esprima.

\subsubsection*{Have you tried working with Esprima?}

Yes. Since the start of E\+S\+Lint, we\textquotesingle{}ve regularly filed bugs and feature requests with Esprima and will continue to do so. However, there are some different philosophies around how the projects work that need to be worked through. The initial goal was to have Espree track Esprima and eventually merge the two back together, but we ultimately decided that building on top of Acorn was a better choice due to Acorn\textquotesingle{}s plugin support.

\subsubsection*{Why don\textquotesingle{}t you just use Acorn?}

Acorn is a great Java\+Script parser that produces an A\+ST that is compatible with Esprima. Unfortunately, E\+S\+Lint relies on more than just the A\+ST to do its job. It relies on Esprima\textquotesingle{}s tokens and comment attachment features to get a complete picture of the source code. We investigated switching to Acorn, but the inconsistencies between Esprima and Acorn created too much work for a project like E\+S\+Lint.

We are building on top of Acorn, however, so that we can contribute back and help make Acorn even better.

\subsubsection*{What E\+C\+M\+A\+Script 6 features do you support?}

All of them.

\subsubsection*{What E\+C\+M\+A\+Script 7/2016 features do you support?}

There is only one E\+C\+M\+A\+Script 7 syntax change\+: the exponentiation operator. Espree supports this.

\subsubsection*{What E\+C\+M\+A\+Script 2017 features do you support?}

Because E\+C\+M\+A\+Script 2017 is still under development, we are implementing features as they are finalized. Currently, Espree supports\+:


\begin{DoxyItemize}
\item {\ttfamily async} functions
\item Trailing commas in function declarations and calls (including arrow functions and concise methods)
\end{DoxyItemize}

\subsubsection*{How do you determine which experimental features to support?}

In general, we do not support experimental Java\+Script features. We may make exceptions from time to time depending on the maturity of the features. 