\href{https://npmjs.org/package/sw-precache-webpack-plugin}{\tt } \href{https://npmjs.org/package/sw-precache-webpack-plugin}{\tt } \href{https://circleci.com/gh/goldhand/sw-precache-webpack-plugin}{\tt }

\+\_\+\+\_\+{\ttfamily S\+W\+Precache\+Webpack\+Plugin}\+\_\+\+\_\+ is a \href{http://webpack.github.io/}{\tt webpack} plugin for using https\+://github.com/goldhand/notes/blob/master/notes/service\+\_\+workers.\+md \char`\"{}service workers\char`\"{} to cache your external project dependencies. It will generate a service worker file using \href{https://github.com/GoogleChrome/sw-precache}{\tt sw-\/precache} and add it to your build directory.

\#\# Install 
\begin{DoxyCode}
npm install --save-dev sw-precache-webpack-plugin
\end{DoxyCode}


\subsection*{Basic Usage}

A simple configuration example that will work well in most production environments. Based on the configuration used in \href{https://github.com/facebookincubator/create-react-app/blob/e91648a9bb55230fa15a7867fd5b730d7e1a5808/packages/react-scripts/config/webpack.config.prod.js#L308}{\tt create-\/react-\/app}. 
\begin{DoxyCode}
var path = require('path');
var SWPrecacheWebpackPlugin = require('sw-precache-webpack-plugin');

const PUBLIC\_PATH = 'https://www.my-project-name.com/';  // webpack needs the trailing slash for
       output.publicPath

module.exports = \{

  entry: \{
    main: path.resolve(\_\_dirname, 'src/index'),
  \},

  output: \{
    path: path.resolve(\_\_dirname, 'src/bundles/'),
    filename: '[name]-[hash].js',
    publicPath: PUBLIC\_PATH,
  \},

  plugins: [
    new SWPrecacheWebpackPlugin(
      \{
        cacheId: 'my-project-name',
        dontCacheBustUrlsMatching: /\(\backslash\).\(\backslash\)w\{8\}\(\backslash\)./,
        filename: 'service-worker.js',
        minify: true,
        navigateFallback: PUBLIC\_PATH + 'index.html',
        staticFileGlobsIgnorePatterns: [/\(\backslash\).map$/, /asset-manifest\(\backslash\).json$/],
      \}
    ),
  ],
\}
\end{DoxyCode}


This will generate a new service worker at {\ttfamily src/bundles/my-\/service-\/worker.\+js}. Then you would just register it in your application\+:


\begin{DoxyCode}
(function() \{
  if('serviceWorker' in navigator) \{
    navigator.serviceWorker.register('/my-service-worker.js');
  \}
\})();
\end{DoxyCode}


\href{https://github.com/GoogleChrome/sw-precache/blob/5699e5d049235ef0f668e8e2aa3bf2646ba3872f/demo/app/js/service-worker-registration.js}{\tt Another example of registering a service worker is provided by Google\+Chrome/sw-\/precache}

\subsection*{Configuration}

You can pass a hash of configuration options to {\ttfamily S\+W\+Precache\+Webpack\+Plugin}\+:

{\bfseries plugin options}\+:
\begin{DoxyItemize}
\item {\ttfamily filename}\+: {\ttfamily \mbox{[}String\mbox{]}} -\/ Service worker filename, default is {\ttfamily service-\/worker.\+js}
\item {\ttfamily filepath}\+: {\ttfamily \mbox{[}String\mbox{]}} -\/ Service worker path and name, default is to use {\ttfamily webpack.\+output.\+path} + {\ttfamily options.\+filename}. This will overried {\ttfamily filename}. {\itshape Warning\+: Make the service worker available in the same directory it will be needed. This is because the scope of the service worker is defined by the directory the worker exists.}
\item {\ttfamily static\+File\+Globs\+Ignore\+Patterns}\+: {\ttfamily \mbox{[}Reg\+Exp\mbox{]}} -\/ Define an optional array of regex patterns to filter out of static\+File\+Globs (see below)
\item {\ttfamily merge\+Statics\+Config}\+: {\ttfamily \mbox{[}boolean\mbox{]}} -\/ Merge provided static\+File\+Globs and strip\+Prefix\+Multi with webpack\textquotesingle{}s config, rather than having those take precedence, default is false.
\item {\ttfamily minify}\+: {\ttfamily \mbox{[}boolean\mbox{]}} -\/ Set to true to minify and uglify the generated service-\/worker, default is false.
\end{DoxyItemize}

\href{https://github.com/GoogleChrome/sw-precache#options-parameter}{\tt \+\_\+\+\_\+{\ttfamily sw-\/precache} options\+\_\+\+\_\+}\+: Pass any option from {\ttfamily sw-\/precache} into your configuration. Some of these will be automatically be populated if you do not specify the value and a couple others will be modified to be more compatible with webpack. Options that are populated / modified\+:


\begin{DoxyItemize}
\item {\ttfamily cache\+Id}\+: {\ttfamily \mbox{[}String\mbox{]}} -\/ Not required but you should include this, it will give your service worker cache a unique name. Defaults to \char`\"{}sw-\/precache-\/webpack-\/plugin\char`\"{}.
\item {\ttfamily import\+Scripts}\+: {\ttfamily \mbox{[}Array$<$String$\vert$\+Object$>$\mbox{]}}
\begin{DoxyItemize}
\item When import\+Scripts array item is a {\ttfamily String}\+:
\begin{DoxyItemize}
\item Converts to object format `\{ filename\+: '$<$public\+Path$>$/my-\/script.js\textquotesingle{}\}{\ttfamily }
\end{DoxyItemize}
\item {\ttfamily When import\+Scripts array item is an}Object{\ttfamily \+:
\begin{DoxyItemize}
\item Looks forchunk\+Name{\ttfamily property.}
\item {\ttfamily Looks for}filename{\ttfamily property.}
\item {\ttfamily $\ast$$\ast$\+If a}chunk\+Name{\ttfamily is specified, it will override the accompanied value for}filename{\ttfamily .$\ast$$\ast$ $\ast$}replace\+Prefix{\ttfamily \+:}\mbox{[}String\mbox{]}{\ttfamily -\/ Should only be used in conjunction with}strip\+Prefix{\ttfamily  $\ast$}static\+File\+Globs{\ttfamily \+:}\mbox{[}Array$<$\+String$>$\mbox{]}` -\/ Omit this to allow the plugin to cache all your bundles' emitted assets. If {\ttfamily merge\+Statics\+Config=true}\+: this value will be merged with your bundles\textquotesingle{} emitted assets, otherwise this value is just passed to {\ttfamily sw-\/precache} and emitted assets won\textquotesingle{}t be included.
\end{DoxyItemize}}
\end{DoxyItemize}
\item {\ttfamily {\ttfamily strip\+Prefix}\+: {\ttfamily \mbox{[}String\mbox{]}} -\/ Same as `strip\+Prefix\+Multi\mbox{[}strip\+Prefix\mbox{]} = '\textquotesingle{}{\ttfamily  $\ast$}strip\+Prefix\+Multi{\ttfamily \+:}\mbox{[}Object$<$\+String,\+String$>$\mbox{]}{\ttfamily -\/ Omit this to use your webpack config\textquotesingle{}s}output.\+path + \textquotesingle{}/\textquotesingle{}\+: output.\+public\+Path$<$tt$>$. If}merge\+Statics\+Config=true{\ttfamily , this value will be merged with your webpack\textquotesingle{}s}output.\+path\+: public\+Path{\ttfamily for stripping prefixes. Otherwise this property will be passed directly to}sw-\/precache\`{} and Webpack\textquotesingle{}s output path won\textquotesingle{}t be replaced.
\end{DoxyItemize}

{\itshape Note that all configuration options are optional. {\ttfamily S\+W\+Precache\+Webpack\+Plugin} will by default use all your assets emitted by webpack\textquotesingle{}s compiler for the {\ttfamily static\+File\+Globs} parameter and your webpack config\textquotesingle{}s `\{\mbox{[}output.\+path + '/\textquotesingle{}\mbox{]}\+: output.\+public\+Path\}{\ttfamily as the}strip\+Prefix\+Multi{\ttfamily parameter. This behavior is probably what you want, all your webpack assets will be cached by your generated service worker. Just don\textquotesingle{}t pass any arguments when you initialize this plugin, and let this plugin handle generating your}sw-\/precache\`{} configuration.}

\subsection*{Examples}

See the \href{/examples/}{\tt examples documentation} or the implementation in \href{https://github.com/facebookincubator/create-react-app/blob/e91648a9bb55230fa15a7867fd5b730d7e1a5808/packages/react-scripts/config/webpack.config.prod.js#L308}{\tt create-\/react-\/app}.

\subsubsection*{Simplest Example}

No arguments are required by default, {\ttfamily S\+W\+Precache\+Webpack\+Plugin} will use information provided by webpack to generate a service worker into your build directory that caches all your webpack assets. 
\begin{DoxyCode}
module.exports = \{
  ...
  plugins: [
    new SWPrecacheWebpackPlugin(),
  ],
  ...
\}
\end{DoxyCode}


\subsubsection*{Advanced Example}

Here\textquotesingle{}s a more elaborate example with {\ttfamily merge\+Statics\+Config\+: true} and {\ttfamily static\+File\+Globs\+Ignore\+Patterns}. {\ttfamily merge\+Statics\+Config\+: true} allows you to add some additional static file globs to the emitted Service\+Worker file alongside Webpack\textquotesingle{}s emitted assets. {\ttfamily static\+File\+Globs\+Ignore\+Patterns} can be used to avoid including sourcemap file references in the generated Service\+Worker. 
\begin{DoxyCode}
plugins: [
  new SWPrecacheWebpackPlugin(\{
    cacheId: 'my-project-name',
    filename: 'my-project-service-worker.js',
    staticFileGlobs: [
      'src/static/img/**.*',
      'src/static/styles.css',
    ],
    stripPrefix: 'src/static/', // stripPrefixMulti is also supported
    mergeStaticsConfig: true, // if you don't set this to true, you won't see any webpack-emitted assets in
       your serviceworker config
    staticFileGlobsIgnorePatterns: [/\(\backslash\).map$/], // use this to ignore sourcemap files
  \}),
]
\end{DoxyCode}


\subsubsection*{{\ttfamily import\+Sripts} usage example}

Accepts an array of {\ttfamily $<$String$\vert$\+Object$>$}\textquotesingle{}s. {\ttfamily String} entries are legacy supported. Use {\ttfamily filename} instead.

If {\ttfamily import\+Scripts} item is object, there are 2 possible properties to set on this object\+:
\begin{DoxyItemize}
\item {\bfseries filename}\+: Use this if you are referencing a path that \char`\"{}you just know\char`\"{} exists. You probably don\textquotesingle{}t want to use this for named chunks.
\item {\bfseries chunk\+Name}\+: Supports named entry chunks \& dynamically imported named chunks. 
\begin{DoxyCode}
entry: \{
  main: \_\_dirname + '/src/index.js',
  sw: \_\_dirname + '/src/service-worker-entry.js'
\},
output: \{
  publicPath: '/my/public/path',
  chunkfileName: '[name].[<hash|chunkhash>].js'
\},
plugins: [
  new SWPrecacheWebpackPlugin(\{
    filename: 'my-project-service-worker.js',
    importSripts: [
      // * legacy supported
      // [chunkhash] is not supported for this usage
      // This is transformed to new object syntax:
      // \{ filename: '/my/public/path/some-known-script-path.js' \}
      'some-known-script-path.js',

      // This use case is identical to above, except
      // for excluding the .[hash] part:
      \{ filename: 'some-known-script-path.[hash].js' \},

      // When [chunkhash] is specified in filename:
      // - filename must match the format specified in output.chunkfileName
      // - If chunkName is invalid; an error will be reported
      \{ chunkName: 'sw' \},

      // Works for named entry chunks & dynamically imported named chunks:
      // For ex, if in your code is:
      // import(/* webpackChunkName: "my-named-chunk" */ './my-async-script.js');
      \{ chunkName: 'my-named-chunk' \},

      // All importSripts entries resolve to a string, therefore
      // the final output of the above input is:
      // [
      //   '/my/public/path/some-known-script-path.js',
      //   '/my/public/path/some-known-script-path.<compilation hash>.js',
      //   '/my/public/path/some-known-script-path.<chunkhash>.js',
      //   '/my/public/path/<id>.my-named-chunk.<chunkhash>.js'
      // ]
    ]
  \}),
]
\end{DoxyCode}

\end{DoxyItemize}

\subsection*{Webpack Dev Server Support}

Currently {\ttfamily S\+W\+Precache\+Webpack\+Plugin} will not work with {\ttfamily Webpack Dev Server}. If you wish to test the service worker locally, you can use simple a node server \href{/examples/}{\tt see example project} or {\ttfamily python Simple\+H\+T\+T\+P\+Server} from your build directory. I would suggest pointing your node server to a different port than your usual local development port and keeping the precache service worker out of your \href{https://github.com/goldhand/cookiecutter-webpack/blob/986151474b60dc19166eba18156a1f9dbceecb98/%7B%7Bcookiecutter.repo_name%7D%7D/webpack.local.config.js}{\tt local configuration (example)}.

There will likely never be {\ttfamily webpack-\/dev-\/server} support. {\ttfamily sw-\/precache} needs physical files in order to generate the service worker. Webpack-\/dev-\/server files are in-\/memory. It is only possible to provide {\ttfamily sw-\/precache} with globs to find these files. It will follow the glob pattern and generate a list of file names to cache.

\subsection*{Contributing}

Install node dependencies\+: 
\begin{DoxyCode}
$ npm install
\end{DoxyCode}


Or\+: 
\begin{DoxyCode}
$ yarn
\end{DoxyCode}


Add unit tests for your new feature in {\ttfamily ./test/plugin.spec.\+js}

\subsection*{Testing}

Tests are located in {\ttfamily ./test}

Run tests\+: 
\begin{DoxyCode}
$ npm t
\end{DoxyCode}
 