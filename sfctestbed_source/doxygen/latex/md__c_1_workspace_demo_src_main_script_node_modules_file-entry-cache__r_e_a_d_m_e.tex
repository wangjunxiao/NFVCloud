\begin{quote}
Super simple cache for file metadata, useful for process that work o a given series of files and that only need to repeat the job on the changed ones since the previous run of the process — Edit \end{quote}


\href{https://npmjs.org/package/file-entry-cache}{\tt } \href{https://travis-ci.org/royriojas/file-entry-cache}{\tt }

\subsection*{install}


\begin{DoxyCode}
npm i --save file-entry-cache
\end{DoxyCode}


\subsection*{Usage}

\`{}\`{}\`{}js // loads the cache, if one does not exists for the given // Id a new one will be prepared to be created var file\+Entry\+Cache = require(\textquotesingle{}file-\/entry-\/cache\textquotesingle{});

var cache = file\+Entry\+Cache.\+create(\textquotesingle{}test\+Cache\textquotesingle{});

var files = expand(\textquotesingle{}../fixtures/$\ast$.txt\textquotesingle{});

// the first time this method is called, will return all the files var o\+Files = cache.\+get\+Updated\+Files(files);

// this will persist this to disk checking each file stats and // updating the meta attributes {\ttfamily size} and {\ttfamily mtime}. // custom fields could also be added to the meta object and will be persisted // in order to retrieve them later cache.\+reconcile();

// use this if you want the non visited file entries to be kept in the cache // for more than one execution // // cache.\+reconcile( true /$\ast$ no\+Prune $\ast$/)

// on a second run var cache2 = file\+Entry\+Cache.\+create(\textquotesingle{}test\+Cache\textquotesingle{});

// will return now only the files that were modified or none // if no files were modified previous to the execution of this function var o\+Files = cache.\+get\+Updated\+Files(files);

// if you want to prevent a file from being considered non modified // something useful if a file failed some sort of validation // you can then remove the entry from the cache doing cache.\+remove\+Entry(\textquotesingle{}path/to/file\textquotesingle{}); // path to file should be the same path of the file received on {\ttfamily get\+Updated\+Files} // that will effectively make the file to appear again as modified until the validation is passed. In that // case you should not remove it from the cache

// if you need all the files, so you can determine what to do with the changed ones // you can call var o\+Files = cache.\+normalize\+Entries(files);

// o\+Files will be an array of objects like the following entry = \{ key\+: \textquotesingle{}some/name/file\textquotesingle{}, the path to the file changed\+: true, // if the file was changed since previous run meta\+: \{ size\+: 3242, // the size of the file mtime\+: 231231231, // the modification time of the file data\+: \{\} // some extra field stored for this file (useful to save the result of a transformation on the file \} \}


\begin{DoxyCode}
## Motivation for this module

I needed a super simple and dumb **in-memory cache** with optional disk persistence (write-back cache) in
       order to make
a script that will beautify files with `esformatter` to execute only on the files that were changed since
       the last run.

In doing so the process of beautifying files was reduced from several seconds to a small fraction of a
       second.

This module uses [flat-cache](https://www.npmjs.com/package/flat-cache) a super simple `key/value` cache
       storage with
optional file persistance.

The main idea is to read the files when the task begins, apply the transforms required, and if the process
       succeed,
then store the new state of the files. The next time this module request for `getChangedFiles` will return
       only
the files that were modified. Making the process to end faster.

This module could also be used by processes that modify the files applying a transform, in that case the
       result of the
transform could be stored in the `meta` field, of the entries. Anything added to the meta field will be
       persisted.
Those processes won't need to call `getChangedFiles` they will instead call `normalizeEntries` that will
       return the
entries with a `changed` field that can be used to determine if the file was changed or not. If it was not
       changed
the transformed stored data could be used instead of actually applying the transformation, saving time in
       case of only
a few files changed.

In the worst case scenario all the files will be processed. In the best case scenario only a few of them
       will be processed.

## Important notes
- The values set on the meta attribute of the entries should be `stringify-able` ones if possible,
       flat-cache uses `circular-json` to try to persist circular structures, but this should be considered experimental.
       The best results are always obtained with non circular values
- All the changes to the cache state are done to memory first and only persisted after reconcile.
- By default non visited entries are removed from the cache. This is done to prevent the file from growing
       too much. If this is not an issue and
  you prefer to do a manual pruning of the cache files, you can pass `true` to the `reconcile` call. Like
       this:

  ```javascript
  cache.reconcile( true /* noPrune */ );
\end{DoxyCode}


\subsection*{License}

M\+IT 