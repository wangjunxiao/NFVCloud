\subsection*{Installation}


\begin{DoxyCode}
$ [sudo] npm install portfinder
\end{DoxyCode}


\subsection*{Usage}

The {\ttfamily portfinder} module has a simple interface\+:


\begin{DoxyCode}
var portfinder = require('portfinder');

portfinder.getPort(function (err, port) \{
  //
  // `port` is guaranteed to be a free port
  // in this scope.
  //
\});
\end{DoxyCode}


Or with promise (if Promise are supported) \+:


\begin{DoxyCode}
const portfinder = require('portfinder');

portfinder.getPortPromise()
  .then((port) => \{
      //
      // `port` is guaranteed to be a free port
      // in this scope.
      //
  \})
  .catch((err) => \{
      //
      // Could not get a free port, `err` contains the reason.
      //
  \});
\end{DoxyCode}


If {\ttfamily portfinder.\+get\+Port\+Promise()} is called on a Node version without Promise ($<$4), it will throw an Error unless \href{http://bluebirdjs.com/docs/getting-started.html}{\tt Bluebird} or any Promise pollyfill is used.

By default {\ttfamily portfinder} will start searching from {\ttfamily 8000}. To change this simply set {\ttfamily portfinder.\+base\+Port}.

\#\# Run Tests 
\begin{DoxyCode}
$ npm test
\end{DoxyCode}


\paragraph*{Author\+: \href{http://nodejitsu.com}{\tt Charlie Robbins}}

\paragraph*{Maintainer\+: \href{https://github.com/eriktrom}{\tt Erik Trom}}

\paragraph*{License\+: M\+I\+T/\+X11}