Emojis have become a common way of communicating content to the end user. To a person using a screenreader, however, he/she may not be aware that this content is there at all. By wrapping the emoji in a {\ttfamily $<$span$>$}, giving it the {\ttfamily role=\char`\"{}img\char`\"{}}, and providing a useful description in {\ttfamily aria-\/label}, the screenreader will treat the emoji as an image in the accessibility tree with an accessible name for the end user.

\paragraph*{Resources}


\begin{DoxyEnumerate}
\item \href{http://tink.uk/accessible-emoji/}{\tt Lèonie Watson}
\end{DoxyEnumerate}

\subsection*{Rule details}

This rule takes no arguments.

\#\#\# Succeed 
\begin{DoxyCode}
<span role="img" aria-label="Snowman">&#9731;</span>
<span role="img" aria-label="Panda">🐼</span>
<span role="img" aria-labelledby="panda1">🐼</span> 
\end{DoxyCode}


\subsubsection*{Fail}


\begin{DoxyCode}
<span>🐼</span>
<i role="img" aria-label="Panda">🐼</i>
\end{DoxyCode}
 