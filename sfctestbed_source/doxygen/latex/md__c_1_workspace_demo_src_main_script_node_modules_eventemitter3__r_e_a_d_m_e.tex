\href{http://browsenpm.org/package/eventemitter3}{\tt }\href{https://travis-ci.org/primus/eventemitter3}{\tt }\href{https://david-dm.org/primus/eventemitter3}{\tt }\href{https://coveralls.io/r/primus/eventemitter3?branch=master}{\tt }\href{https://webchat.freenode.net/?channels=primus}{\tt }

\href{https://saucelabs.com/u/eventemitter3}{\tt }

Event\+Emitter3 is a high performance Event\+Emitter. It has been micro-\/optimized for various of code paths making this, one of, if not the fastest Event\+Emitter available for Node.\+js and browsers. The module is A\+PI compatible with the Event\+Emitter that ships by default with Node.\+js but there are some slight differences\+:


\begin{DoxyItemize}
\item Domain support has been removed.
\item We do not {\ttfamily throw} an error when you emit an {\ttfamily error} event and nobody is listening.
\item The {\ttfamily new\+Listener} event is removed as the use-\/cases for this functionality are really just edge cases.
\item No {\ttfamily set\+Max\+Listeners} and it\textquotesingle{}s pointless memory leak warnings. If you want to add {\ttfamily end} listeners you should be able to do that without modules complaining.
\item No {\ttfamily listener\+Count} function. Use {\ttfamily E\+E.\+listeners(event).length} instead.
\item Support for custom context for events so there is no need to use {\ttfamily fn.\+bind}.
\item {\ttfamily listeners} method can do existence checking instead of returning only arrays.
\end{DoxyItemize}

It\textquotesingle{}s a drop in replacement for existing Event\+Emitters, but just faster. Free performance, who wouldn\textquotesingle{}t want that? The Event\+Emitter is written in Ecma\+Script 3 so it will work in the oldest browsers and node versions that you need to support.

\subsection*{Installation}


\begin{DoxyCode}
$ npm install --save eventemitter3        # npm
$ component install primus/eventemitter3  # Component
$ bower install eventemitter3             # Bower
\end{DoxyCode}


\subsection*{Usage}

After installation the only thing you need to do is require the module\+:


\begin{DoxyCode}
var EventEmitter = require('eventemitter3');
\end{DoxyCode}


And you\textquotesingle{}re ready to create your own Event\+Emitter instances. For the A\+PI documentation, please follow the official Node.\+js documentation\+:

\href{http://nodejs.org/api/events.html}{\tt http\+://nodejs.\+org/api/events.\+html}

\subsubsection*{Contextual emits}

We\textquotesingle{}ve upgraded the A\+PI of the {\ttfamily Event\+Emitter.\+on}, {\ttfamily Event\+Emitter.\+once} and {\ttfamily Event\+Emitter.\+remove\+Listener} to accept an extra argument which is the {\ttfamily context} or {\ttfamily this} value that should be set for the emitted events. This means you no longer have the overhead of an event that required {\ttfamily fn.\+bind} in order to get a custom {\ttfamily this} value.


\begin{DoxyCode}
var EE = new EventEmitter()
  , context = \{ foo: 'bar' \};

function emitted() \{
  console.log(this === context); // true
\}

EE.once('event-name', emitted, context);
EE.on('another-event', emitted, context);
EE.removeListener('another-event', emitted, context);
\end{DoxyCode}


\subsubsection*{Existence}

To check if there is already a listener for a given event you can supply the {\ttfamily listeners} method with an extra boolean argument. This will transform the output from an array, to a boolean value which indicates if there are listeners in place for the given event\+:


\begin{DoxyCode}
var EE = new EventEmitter();
EE.once('event-name', function () \{\});
EE.on('another-event', function () \{\});

EE.listeners('event-name', true); // returns true
EE.listeners('unknown-name', true); // returns false
\end{DoxyCode}


\subsection*{License}

\mbox{[}M\+IT\mbox{]}(L\+I\+C\+E\+N\+SE) 