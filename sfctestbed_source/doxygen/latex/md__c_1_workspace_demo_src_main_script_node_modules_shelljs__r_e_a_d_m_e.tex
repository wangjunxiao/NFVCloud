\href{https://gitter.im/shelljs/shelljs?utm_source=badge&utm_medium=badge&utm_campaign=pr-badge&utm_content=badge}{\tt } \href{https://travis-ci.org/shelljs/shelljs}{\tt } \href{https://ci.appveyor.com/project/shelljs/shelljs/branch/master}{\tt } \href{https://codecov.io/gh/shelljs/shelljs}{\tt } \href{https://www.npmjs.com/package/shelljs}{\tt } \href{https://www.npmjs.com/package/shelljs}{\tt }

Shell\+JS is a portable $\ast$$\ast$(Windows/\+Linux/\+OS X)$\ast$$\ast$ implementation of Unix shell commands on top of the Node.\+js A\+PI. You can use it to eliminate your shell script\textquotesingle{}s dependency on Unix while still keeping its familiar and powerful commands. You can also install it globally so you can run it from outside Node projects -\/ say goodbye to those gnarly Bash scripts!

Shell\+JS is proudly tested on every node release since {\ttfamily v0.\+11}!

The project is \href{http://travis-ci.org/shelljs/shelljs}{\tt unit-\/tested} and battle-\/tested in projects like\+:


\begin{DoxyItemize}
\item \href{http://github.com/mozilla/pdf.js}{\tt P\+D\+F.\+js} -\/ Firefox\textquotesingle{}s next-\/gen P\+DF reader
\item \href{http://getfirebug.com/}{\tt Firebug} -\/ Firefox\textquotesingle{}s infamous debugger
\item \href{http://jshint.com}{\tt J\+S\+Hint} \& \href{http://eslint.org/}{\tt E\+S\+Lint} -\/ popular Java\+Script linters
\item \href{http://zeptojs.com}{\tt Zepto} -\/ j\+Query-\/compatible Java\+Script library for modern browsers
\item \href{http://yeoman.io/}{\tt Yeoman} -\/ Web application stack and development tool
\item \href{http://deployd.com}{\tt Deployd.\+com} -\/ Open source PaaS for quick A\+PI backend generation
\item And \href{https://npmjs.org/browse/depended/shelljs}{\tt many more}.
\end{DoxyItemize}

If you have feedback, suggestions, or need help, feel free to post in our \href{https://github.com/shelljs/shelljs/issues}{\tt issue tracker}.

Think Shell\+JS is cool? Check out some related projects in our \href{https://github.com/shelljs/shelljs/wiki}{\tt Wiki page}!

Upgrading from an older version? Check out our \href{https://github.com/shelljs/shelljs/wiki/Breaking-Changes}{\tt breaking changes} page to see what changes to watch out for while upgrading.

\subsection*{Command line use}

If you just want cross platform U\+N\+IX commands, checkout our new project \href{https://github.com/shelljs/shx}{\tt shelljs/shx}, a utility to expose {\ttfamily shelljs} to the command line.

For example\+:


\begin{DoxyCode}
$ shx mkdir -p foo
$ shx touch foo/bar.txt
$ shx rm -rf foo
\end{DoxyCode}


\subsection*{A quick note about the docs}

For documentation on all the latest features, check out our \href{https://github.com/shelljs/shelljs}{\tt R\+E\+A\+D\+ME}. To read docs that are consistent with the latest release, check out \href{https://www.npmjs.com/package/shelljs}{\tt the npm page} or \href{http://documentup.com/shelljs/shelljs}{\tt shelljs.\+org}.

\subsection*{Installing}

Via npm\+:


\begin{DoxyCode}
$ npm install [-g] shelljs
\end{DoxyCode}


\subsection*{Examples}


\begin{DoxyCode}
var shell = require('shelljs');

if (!shell.which('git')) \{
  shell.echo('Sorry, this script requires git');
  shell.exit(1);
\}

// Copy files to release dir
shell.rm('-rf', 'out/Release');
shell.cp('-R', 'stuff/', 'out/Release');

// Replace macros in each .js file
shell.cd('lib');
shell.ls('*.js').forEach(function (file) \{
  shell.sed('-i', 'BUILD\_VERSION', 'v0.1.2', file);
  shell.sed('-i', /^.*REMOVE\_THIS\_LINE.*$/, '', file);
  shell.sed('-i', /.*REPLACE\_LINE\_WITH\_MACRO.*\(\backslash\)n/, shell.cat('macro.js'), file);
\});
shell.cd('..');

// Run external tool synchronously
if (shell.exec('git commit -am "Auto-commit"').code !== 0) \{
  shell.echo('Error: Git commit failed');
  shell.exit(1);
\}
\end{DoxyCode}


\subsection*{Global vs. Local}

We no longer recommend using a global-\/import for Shell\+JS (i.\+e. `require(\textquotesingle{}shelljs/global')\`{}). While still supported for convenience, this pollutes the global namespace, and should therefore only be used with caution.

Instead, we recommend a local import (standard for npm packages)\+:


\begin{DoxyCode}
var shell = require('shelljs');
shell.echo('hello world');
\end{DoxyCode}


\subsection*{Command reference}

All commands run synchronously, unless otherwise stated. All commands accept standard bash globbing characters ({\ttfamily $\ast$}, {\ttfamily ?}, etc.), compatible with the \href{https://github.com/isaacs/node-glob}{\tt node glob module}.

For less-\/commonly used commands and features, please check out our \href{https://github.com/shelljs/shelljs/wiki}{\tt wiki page}.

\subsubsection*{cat(file \mbox{[}, file ...\mbox{]})}

\subsubsection*{cat(file\+\_\+array)}

Examples\+:


\begin{DoxyCode}
var str = cat('file*.txt');
var str = cat('file1', 'file2');
var str = cat(['file1', 'file2']); // same as above
\end{DoxyCode}


Returns a string containing the given file, or a concatenated string containing the files if more than one file is given (a new line character is introduced between each file).

\subsubsection*{cd(\mbox{[}dir\mbox{]})}

Changes to directory {\ttfamily dir} for the duration of the script. Changes to home directory if no argument is supplied.

\subsubsection*{chmod(\mbox{[}options,\mbox{]} octal\+\_\+mode $\vert$$\vert$ octal\+\_\+string, file)}

\subsubsection*{chmod(\mbox{[}options,\mbox{]} symbolic\+\_\+mode, file)}

Available options\+:


\begin{DoxyItemize}
\item {\ttfamily -\/v}\+: output a diagnostic for every file processed
\item {\ttfamily -\/c}\+: like verbose but report only when a change is made
\item {\ttfamily -\/R}\+: change files and directories recursively
\end{DoxyItemize}

Examples\+:


\begin{DoxyCode}
chmod(755, '/Users/brandon');
chmod('755', '/Users/brandon'); // same as above
chmod('u+x', '/Users/brandon');
chmod('-R', 'a-w', '/Users/brandon');
\end{DoxyCode}


Alters the permissions of a file or directory by either specifying the absolute permissions in octal form or expressing the changes in symbols. This command tries to mimic the P\+O\+S\+IX behavior as much as possible. Notable exceptions\+:


\begin{DoxyItemize}
\item In symbolic modes, \textquotesingle{}a-\/r\textquotesingle{} and \textquotesingle{}-\/r\textquotesingle{} are identical. No consideration is given to the umask.
\item There is no \char`\"{}quiet\char`\"{} option since default behavior is to run silent.
\end{DoxyItemize}

\subsubsection*{cp(\mbox{[}options,\mbox{]} source \mbox{[}, source ...\mbox{]}, dest)}

\subsubsection*{cp(\mbox{[}options,\mbox{]} source\+\_\+array, dest)}

Available options\+:


\begin{DoxyItemize}
\item {\ttfamily -\/f}\+: force (default behavior)
\item {\ttfamily -\/n}\+: no-\/clobber
\item {\ttfamily -\/u}\+: only copy if source is newer than dest
\item {\ttfamily -\/r}, {\ttfamily -\/R}\+: recursive
\item {\ttfamily -\/L}\+: follow symlinks
\item {\ttfamily -\/P}\+: don\textquotesingle{}t follow symlinks
\end{DoxyItemize}

Examples\+:


\begin{DoxyCode}
cp('file1', 'dir1');
cp('-R', 'path/to/dir/', '~/newCopy/');
cp('-Rf', '/tmp/*', '/usr/local/*', '/home/tmp');
cp('-Rf', ['/tmp/*', '/usr/local/*'], '/home/tmp'); // same as above
\end{DoxyCode}


Copies files.

\subsubsection*{pushd(\mbox{[}options,\mbox{]} \mbox{[}dir $\vert$ \textquotesingle{}-\/N\textquotesingle{} $\vert$ \textquotesingle{}+N\textquotesingle{}\mbox{]})}

Available options\+:


\begin{DoxyItemize}
\item {\ttfamily -\/n}\+: Suppresses the normal change of directory when adding directories to the stack, so that only the stack is manipulated.
\end{DoxyItemize}

Arguments\+:


\begin{DoxyItemize}
\item {\ttfamily dir}\+: Makes the current working directory be the top of the stack, and then executes the equivalent of {\ttfamily cd dir}.
\item {\ttfamily +N}\+: Brings the Nth directory (counting from the left of the list printed by dirs, starting with zero) to the top of the list by rotating the stack.
\item {\ttfamily -\/N}\+: Brings the Nth directory (counting from the right of the list printed by dirs, starting with zero) to the top of the list by rotating the stack.
\end{DoxyItemize}

Examples\+:


\begin{DoxyCode}
// process.cwd() === '/usr'
pushd('/etc'); // Returns /etc /usr
pushd('+1');   // Returns /usr /etc
\end{DoxyCode}


Save the current directory on the top of the directory stack and then cd to {\ttfamily dir}. With no arguments, pushd exchanges the top two directories. Returns an array of paths in the stack.

\subsubsection*{popd(\mbox{[}options,\mbox{]} \mbox{[}\textquotesingle{}-\/N\textquotesingle{} $\vert$ \textquotesingle{}+N\textquotesingle{}\mbox{]})}

Available options\+:


\begin{DoxyItemize}
\item {\ttfamily -\/n}\+: Suppresses the normal change of directory when removing directories from the stack, so that only the stack is manipulated.
\end{DoxyItemize}

Arguments\+:


\begin{DoxyItemize}
\item {\ttfamily +N}\+: Removes the Nth directory (counting from the left of the list printed by dirs), starting with zero.
\item {\ttfamily -\/N}\+: Removes the Nth directory (counting from the right of the list printed by dirs), starting with zero.
\end{DoxyItemize}

Examples\+:


\begin{DoxyCode}
echo(process.cwd()); // '/usr'
pushd('/etc');       // '/etc /usr'
echo(process.cwd()); // '/etc'
popd();              // '/usr'
echo(process.cwd()); // '/usr'
\end{DoxyCode}


When no arguments are given, popd removes the top directory from the stack and performs a cd to the new top directory. The elements are numbered from 0 starting at the first directory listed with dirs; i.\+e., popd is equivalent to popd +0. Returns an array of paths in the stack.

\subsubsection*{dirs(\mbox{[}options $\vert$ \textquotesingle{}+N\textquotesingle{} $\vert$ \textquotesingle{}-\/N\textquotesingle{}\mbox{]})}

Available options\+:


\begin{DoxyItemize}
\item {\ttfamily -\/c}\+: Clears the directory stack by deleting all of the elements.
\end{DoxyItemize}

Arguments\+:


\begin{DoxyItemize}
\item {\ttfamily +N}\+: Displays the Nth directory (counting from the left of the list printed by dirs when invoked without options), starting with zero.
\item {\ttfamily -\/N}\+: Displays the Nth directory (counting from the right of the list printed by dirs when invoked without options), starting with zero.
\end{DoxyItemize}

Display the list of currently remembered directories. Returns an array of paths in the stack, or a single path if +N or -\/N was specified.

See also\+: pushd, popd

\subsubsection*{echo(\mbox{[}options,\mbox{]} string \mbox{[}, string ...\mbox{]})}

Available options\+:


\begin{DoxyItemize}
\item {\ttfamily -\/e}\+: interpret backslash escapes (default)
\end{DoxyItemize}

Examples\+:


\begin{DoxyCode}
echo('hello world');
var str = echo('hello world');
\end{DoxyCode}


Prints string to stdout, and returns string with additional utility methods like {\ttfamily .to()}.

\subsubsection*{exec(command \mbox{[}, options\mbox{]} \mbox{[}, callback\mbox{]})}

Available options (all {\ttfamily false} by default)\+:


\begin{DoxyItemize}
\item {\ttfamily async}\+: Asynchronous execution. If a callback is provided, it will be set to {\ttfamily true}, regardless of the passed value.
\item {\ttfamily silent}\+: Do not echo program output to console.
\item and any option available to Node.\+js\textquotesingle{}s \href{https://nodejs.org/api/child_process.html#child_process_child_process_exec_command_options_callback}{\tt child\+\_\+process.\+exec()}
\end{DoxyItemize}

Examples\+:


\begin{DoxyCode}
var version = exec('node --version', \{silent:true\}).stdout;

var child = exec('some\_long\_running\_process', \{async:true\});
child.stdout.on('data', function(data) \{
  /* ... do something with data ... */
\});

exec('some\_long\_running\_process', function(code, stdout, stderr) \{
  console.log('Exit code:', code);
  console.log('Program output:', stdout);
  console.log('Program stderr:', stderr);
\});
\end{DoxyCode}


Executes the given {\ttfamily command} {\itshape synchronously}, unless otherwise specified. When in synchronous mode, this returns a Shell\+String (compatible with Shell\+JS v0.\+6.\+x, which returns an object of the form {\ttfamily \{ code\+:..., stdout\+:... , stderr\+:... \}}). Otherwise, this returns the child process object, and the {\ttfamily callback} gets the arguments {\ttfamily (code, stdout, stderr)}.

Not seeing the behavior you want? {\ttfamily exec()} runs everything through {\ttfamily sh} by default (or {\ttfamily cmd.\+exe} on Windows), which differs from {\ttfamily bash}. If you need bash-\/specific behavior, try out the `\{shell\+: \textquotesingle{}path/to/bash'\}\`{} option.

{\bfseries Note\+:} For long-\/lived processes, it\textquotesingle{}s best to run {\ttfamily exec()} asynchronously as the current synchronous implementation uses a lot of C\+PU. This should be getting fixed soon.

\subsubsection*{find(path \mbox{[}, path ...\mbox{]})}

\subsubsection*{find(path\+\_\+array)}

Examples\+:


\begin{DoxyCode}
find('src', 'lib');
find(['src', 'lib']); // same as above
find('.').filter(function(file) \{ return file.match(/\(\backslash\).js$/); \});
\end{DoxyCode}


Returns array of all files (however deep) in the given paths.

The main difference from `ls('-\/R\textquotesingle{}, path){\ttfamily is that the resulting file names include the base directories, e.\+g.}lib/resources/file1{\ttfamily instead of just}file1\`{}.

\subsubsection*{grep(\mbox{[}options,\mbox{]} regex\+\_\+filter, file \mbox{[}, file ...\mbox{]})}

\subsubsection*{grep(\mbox{[}options,\mbox{]} regex\+\_\+filter, file\+\_\+array)}

Available options\+:


\begin{DoxyItemize}
\item {\ttfamily -\/v}\+: Inverse the sense of the regex and print the lines not matching the criteria.
\item {\ttfamily -\/l}\+: Print only filenames of matching files
\end{DoxyItemize}

Examples\+:


\begin{DoxyCode}
grep('-v', 'GLOBAL\_VARIABLE', '*.js');
grep('GLOBAL\_VARIABLE', '*.js');
\end{DoxyCode}


Reads input string from given files and returns a string containing all lines of the file that match the given {\ttfamily regex\+\_\+filter}.

\subsubsection*{head(\mbox{[}\{\textquotesingle{}-\/n\textquotesingle{}\+: $<$num$>$\},\mbox{]} file \mbox{[}, file ...\mbox{]})}

\subsubsection*{head(\mbox{[}\{\textquotesingle{}-\/n\textquotesingle{}\+: $<$num$>$\},\mbox{]} file\+\_\+array)}

Available options\+:


\begin{DoxyItemize}
\item {\ttfamily -\/n $<$num$>$}\+: Show the first {\ttfamily $<$num$>$} lines of the files
\end{DoxyItemize}

Examples\+:


\begin{DoxyCode}
var str = head(\{'-n': 1\}, 'file*.txt');
var str = head('file1', 'file2');
var str = head(['file1', 'file2']); // same as above
\end{DoxyCode}


Read the start of a file.

\subsubsection*{ln(\mbox{[}options,\mbox{]} source, dest)}

Available options\+:


\begin{DoxyItemize}
\item {\ttfamily -\/s}\+: symlink
\item {\ttfamily -\/f}\+: force
\end{DoxyItemize}

Examples\+:


\begin{DoxyCode}
ln('file', 'newlink');
ln('-sf', 'file', 'existing');
\end{DoxyCode}


Links source to dest. Use -\/f to force the link, should dest already exist.

\subsubsection*{ls(\mbox{[}options,\mbox{]} \mbox{[}path, ...\mbox{]})}

\subsubsection*{ls(\mbox{[}options,\mbox{]} path\+\_\+array)}

Available options\+:


\begin{DoxyItemize}
\item {\ttfamily -\/R}\+: recursive
\item {\ttfamily -\/A}\+: all files (include files beginning with {\ttfamily .}, except for {\ttfamily .} and {\ttfamily ..})
\item {\ttfamily -\/L}\+: follow symlinks
\item {\ttfamily -\/d}\+: list directories themselves, not their contents
\item {\ttfamily -\/l}\+: list objects representing each file, each with fields containing {\ttfamily ls -\/l} output fields. See \href{https://nodejs.org/api/fs.html#fs_class_fs_stats}{\tt fs.\+Stats} for more info
\end{DoxyItemize}

Examples\+:


\begin{DoxyCode}
ls('projs/*.js');
ls('-R', '/users/me', '/tmp');
ls('-R', ['/users/me', '/tmp']); // same as above
ls('-l', 'file.txt'); // \{ name: 'file.txt', mode: 33188, nlink: 1, ...\}
\end{DoxyCode}


Returns array of files in the given path, or in current directory if no path provided.

\subsubsection*{mkdir(\mbox{[}options,\mbox{]} dir \mbox{[}, dir ...\mbox{]})}

\subsubsection*{mkdir(\mbox{[}options,\mbox{]} dir\+\_\+array)}

Available options\+:


\begin{DoxyItemize}
\item {\ttfamily -\/p}\+: full path (will create intermediate dirs if necessary)
\end{DoxyItemize}

Examples\+:


\begin{DoxyCode}
mkdir('-p', '/tmp/a/b/c/d', '/tmp/e/f/g');
mkdir('-p', ['/tmp/a/b/c/d', '/tmp/e/f/g']); // same as above
\end{DoxyCode}


Creates directories.

\subsubsection*{mv(\mbox{[}options ,\mbox{]} source \mbox{[}, source ...\mbox{]}, dest\textquotesingle{})}

\subsubsection*{mv(\mbox{[}options ,\mbox{]} source\+\_\+array, dest\textquotesingle{})}

Available options\+:


\begin{DoxyItemize}
\item {\ttfamily -\/f}\+: force (default behavior)
\item {\ttfamily -\/n}\+: no-\/clobber
\end{DoxyItemize}

Examples\+:


\begin{DoxyCode}
mv('-n', 'file', 'dir/');
mv('file1', 'file2', 'dir/');
mv(['file1', 'file2'], 'dir/'); // same as above
\end{DoxyCode}


Moves files.

\subsubsection*{pwd()}

Returns the current directory.

\subsubsection*{rm(\mbox{[}options,\mbox{]} file \mbox{[}, file ...\mbox{]})}

\subsubsection*{rm(\mbox{[}options,\mbox{]} file\+\_\+array)}

Available options\+:


\begin{DoxyItemize}
\item {\ttfamily -\/f}\+: force
\item {\ttfamily -\/r, -\/R}\+: recursive
\end{DoxyItemize}

Examples\+:


\begin{DoxyCode}
rm('-rf', '/tmp/*');
rm('some\_file.txt', 'another\_file.txt');
rm(['some\_file.txt', 'another\_file.txt']); // same as above
\end{DoxyCode}


Removes files.

\subsubsection*{sed(\mbox{[}options,\mbox{]} search\+\_\+regex, replacement, file \mbox{[}, file ...\mbox{]})}

\subsubsection*{sed(\mbox{[}options,\mbox{]} search\+\_\+regex, replacement, file\+\_\+array)}

Available options\+:


\begin{DoxyItemize}
\item {\ttfamily -\/i}\+: Replace contents of \textquotesingle{}file\textquotesingle{} in-\/place. {\itshape Note that no backups will be created!}
\end{DoxyItemize}

Examples\+:


\begin{DoxyCode}
sed('-i', 'PROGRAM\_VERSION', 'v0.1.3', 'source.js');
sed(/.*DELETE\_THIS\_LINE.*\(\backslash\)n/, '', 'source.js');
\end{DoxyCode}


Reads an input string from {\ttfamily files} and performs a Java\+Script {\ttfamily replace()} on the input using the given search regex and replacement string or function. Returns the new string after replacement.

Note\+:

Like unix {\ttfamily sed}, Shell\+JS {\ttfamily sed} supports capture groups. Capture groups are specified using the {\ttfamily \$n} syntax\+:


\begin{DoxyCode}
sed(/(\(\backslash\)w+)\(\backslash\)s(\(\backslash\)w+)/, '$2, $1', 'file.txt');
\end{DoxyCode}


\subsubsection*{set(options)}

Available options\+:


\begin{DoxyItemize}
\item {\ttfamily +/-\/e}\+: exit upon error ({\ttfamily config.\+fatal})
\item {\ttfamily +/-\/v}\+: verbose\+: show all commands ({\ttfamily config.\+verbose})
\item {\ttfamily +/-\/f}\+: disable filename expansion (globbing)
\end{DoxyItemize}

Examples\+:


\begin{DoxyCode}
set('-e'); // exit upon first error
set('+e'); // this undoes a "set('-e')"
\end{DoxyCode}


Sets global configuration variables

\subsubsection*{sort(\mbox{[}options,\mbox{]} file \mbox{[}, file ...\mbox{]})}

\subsubsection*{sort(\mbox{[}options,\mbox{]} file\+\_\+array)}

Available options\+:


\begin{DoxyItemize}
\item {\ttfamily -\/r}\+: Reverse the result of comparisons
\item {\ttfamily -\/n}\+: Compare according to numerical value
\end{DoxyItemize}

Examples\+:


\begin{DoxyCode}
sort('foo.txt', 'bar.txt');
sort('-r', 'foo.txt');
\end{DoxyCode}


Return the contents of the files, sorted line-\/by-\/line. Sorting multiple files mixes their content, just like unix sort does.

\subsubsection*{tail(\mbox{[}\{\textquotesingle{}-\/n\textquotesingle{}\+: $<$num$>$\},\mbox{]} file \mbox{[}, file ...\mbox{]})}

\subsubsection*{tail(\mbox{[}\{\textquotesingle{}-\/n\textquotesingle{}\+: $<$num$>$\},\mbox{]} file\+\_\+array)}

Available options\+:


\begin{DoxyItemize}
\item {\ttfamily -\/n $<$num$>$}\+: Show the last {\ttfamily $<$num$>$} lines of the files
\end{DoxyItemize}

Examples\+:


\begin{DoxyCode}
var str = tail(\{'-n': 1\}, 'file*.txt');
var str = tail('file1', 'file2');
var str = tail(['file1', 'file2']); // same as above
\end{DoxyCode}


Read the end of a file.

\subsubsection*{tempdir()}

Examples\+:


\begin{DoxyCode}
var tmp = tempdir(); // "/tmp" for most *nix platforms
\end{DoxyCode}


Searches and returns string containing a writeable, platform-\/dependent temporary directory. Follows Python\textquotesingle{}s \href{http://docs.python.org/library/tempfile.html#tempfile.tempdir}{\tt tempfile algorithm}.

\subsubsection*{test(expression)}

Available expression primaries\+:


\begin{DoxyItemize}
\item `'-\/b\textquotesingle{}, \textquotesingle{}path\textquotesingle{}{\ttfamily \+: true if path is a block device +}\textquotesingle{}-\/c\textquotesingle{}, \textquotesingle{}path\textquotesingle{}{\ttfamily \+: true if path is a character device +}\textquotesingle{}-\/d\textquotesingle{}, \textquotesingle{}path\textquotesingle{}{\ttfamily \+: true if path is a directory +}\textquotesingle{}-\/e\textquotesingle{}, \textquotesingle{}path\textquotesingle{}{\ttfamily \+: true if path exists +}\textquotesingle{}-\/f\textquotesingle{}, \textquotesingle{}path\textquotesingle{}{\ttfamily \+: true if path is a regular file +}\textquotesingle{}-\/L\textquotesingle{}, \textquotesingle{}path\textquotesingle{}{\ttfamily \+: true if path is a symbolic link +}\textquotesingle{}-\/p\textquotesingle{}, \textquotesingle{}path\textquotesingle{}{\ttfamily \+: true if path is a pipe (F\+I\+FO) +}\textquotesingle{}-\/S\textquotesingle{}, \textquotesingle{}path\textquotesingle{}\`{}\+: true if path is a socket
\end{DoxyItemize}

Examples\+:


\begin{DoxyCode}
if (test('-d', path)) \{ /* do something with dir */ \};
if (!test('-f', path)) continue; // skip if it's a regular file
\end{DoxyCode}


Evaluates expression using the available primaries and returns corresponding value.

\subsubsection*{Shell\+String.\+prototype.\+to(file)}

Examples\+:


\begin{DoxyCode}
cat('input.txt').to('output.txt');
\end{DoxyCode}


Analogous to the redirection operator {\ttfamily $>$} in Unix, but works with Shell\+Strings (such as those returned by {\ttfamily cat}, {\ttfamily grep}, etc). {\itshape Like Unix redirections, {\ttfamily to()} will overwrite any existing file!}

\subsubsection*{Shell\+String.\+prototype.\+to\+End(file)}

Examples\+:


\begin{DoxyCode}
cat('input.txt').toEnd('output.txt');
\end{DoxyCode}


Analogous to the redirect-\/and-\/append operator {\ttfamily $>$$>$} in Unix, but works with Shell\+Strings (such as those returned by {\ttfamily cat}, {\ttfamily grep}, etc).

\subsubsection*{touch(\mbox{[}options,\mbox{]} file \mbox{[}, file ...\mbox{]})}

\subsubsection*{touch(\mbox{[}options,\mbox{]} file\+\_\+array)}

Available options\+:


\begin{DoxyItemize}
\item {\ttfamily -\/a}\+: Change only the access time
\item {\ttfamily -\/c}\+: Do not create any files
\item {\ttfamily -\/m}\+: Change only the modification time
\item {\ttfamily -\/d D\+A\+TE}\+: Parse D\+A\+TE and use it instead of current time
\item {\ttfamily -\/r F\+I\+LE}\+: Use F\+I\+LE\textquotesingle{}s times instead of current time
\end{DoxyItemize}

Examples\+:


\begin{DoxyCode}
touch('source.js');
touch('-c', '/path/to/some/dir/source.js');
touch(\{ '-r': FILE \}, '/path/to/some/dir/source.js');
\end{DoxyCode}


Update the access and modification times of each F\+I\+LE to the current time. A F\+I\+LE argument that does not exist is created empty, unless -\/c is supplied. This is a partial implementation of {\itshape \href{http://linux.die.net/man/1/touch}{\tt touch(1)}}.

\subsubsection*{uniq(\mbox{[}options,\mbox{]} \mbox{[}input, \mbox{[}output\mbox{]}\mbox{]})}

Available options\+:


\begin{DoxyItemize}
\item {\ttfamily -\/i}\+: Ignore case while comparing
\item {\ttfamily -\/c}\+: Prefix lines by the number of occurrences
\item {\ttfamily -\/d}\+: Only print duplicate lines, one for each group of identical lines
\end{DoxyItemize}

Examples\+:


\begin{DoxyCode}
uniq('foo.txt');
uniq('-i', 'foo.txt');
uniq('-cd', 'foo.txt', 'bar.txt');
\end{DoxyCode}


Filter adjacent matching lines from input

\subsubsection*{which(command)}

Examples\+:


\begin{DoxyCode}
var nodeExec = which('node');
\end{DoxyCode}


Searches for {\ttfamily command} in the system\textquotesingle{}s P\+A\+TH. On Windows, this uses the {\ttfamily P\+A\+T\+H\+E\+XT} variable to append the extension if it\textquotesingle{}s not already executable. Returns string containing the absolute path to the command.

\subsubsection*{exit(code)}

Exits the current process with the given exit code.

\subsubsection*{error()}

Tests if error occurred in the last command. Returns a truthy value if an error returned and a falsy value otherwise.

{\bfseries Note}\+: do not rely on the return value to be an error message. If you need the last error message, use the {\ttfamily .stderr} attribute from the last command\textquotesingle{}s return value instead.

\subsubsection*{Shell\+String(str)}

Examples\+:


\begin{DoxyCode}
var foo = ShellString('hello world');
\end{DoxyCode}


Turns a regular string into a string-\/like object similar to what each command returns. This has special methods, like {\ttfamily .to()} and {\ttfamily .to\+End()}

\subsubsection*{env\mbox{[}\textquotesingle{}V\+A\+R\+\_\+\+N\+A\+ME\textquotesingle{}\mbox{]}}

Object containing environment variables (both getter and setter). Shortcut to process.\+env.

\subsubsection*{Pipes}

Examples\+:


\begin{DoxyCode}
grep('foo', 'file1.txt', 'file2.txt').sed(/o/g, 'a').to('output.txt');
echo('files with o\(\backslash\)'s in the name:\(\backslash\)n' + ls().grep('o'));
cat('test.js').exec('node'); // pipe to exec() call
\end{DoxyCode}


Commands can send their output to another command in a pipe-\/like fashion. {\ttfamily sed}, {\ttfamily grep}, {\ttfamily cat}, {\ttfamily exec}, {\ttfamily to}, and {\ttfamily to\+End} can appear on the right-\/hand side of a pipe. Pipes can be chained.

\subsection*{Configuration}

\subsubsection*{config.\+silent}

Example\+:


\begin{DoxyCode}
var sh = require('shelljs');
var silentState = sh.config.silent; // save old silent state
sh.config.silent = true;
/* ... */
sh.config.silent = silentState; // restore old silent state
\end{DoxyCode}


Suppresses all command output if {\ttfamily true}, except for {\ttfamily echo()} calls. Default is {\ttfamily false}.

\subsubsection*{config.\+fatal}

Example\+:


\begin{DoxyCode}
require('shelljs/global');
config.fatal = true; // or set('-e');
cp('this\_file\_does\_not\_exist', '/dev/null'); // throws Error here
/* more commands... */
\end{DoxyCode}


If {\ttfamily true} the script will throw a Javascript error when any shell.\+js command encounters an error. Default is {\ttfamily false}. This is analogous to Bash\textquotesingle{}s {\ttfamily set -\/e}

\subsubsection*{config.\+verbose}

Example\+:


\begin{DoxyCode}
config.verbose = true; // or set('-v');
cd('dir/');
rm('-rf', 'foo.txt', 'bar.txt');
exec('echo hello');
\end{DoxyCode}


Will print each command as follows\+:


\begin{DoxyCode}
cd dir/
rm -rf foo.txt bar.txt
exec echo hello
\end{DoxyCode}


\subsubsection*{config.\+glob\+Options}

Example\+:


\begin{DoxyCode}
config.globOptions = \{nodir: true\};
\end{DoxyCode}


Use this value for calls to {\ttfamily glob.\+sync()} instead of the default options.

\subsubsection*{config.\+reset()}

Example\+:


\begin{DoxyCode}
var shell = require('shelljs');
// Make changes to shell.config, and do stuff...
/* ... */
shell.config.reset(); // reset to original state
// Do more stuff, but with original settings
/* ... */
\end{DoxyCode}


Reset shell.\+config to the defaults\+:


\begin{DoxyCode}
\{
  fatal: false,
  globOptions: \{\},
  maxdepth: 255,
  noglob: false,
  silent: false,
  verbose: false,
\}
\end{DoxyCode}


\subsection*{Team}

\tabulinesep=1mm
\begin{longtabu} spread 0pt [c]{*{2}{|X[-1]}|}
\hline
\rowcolor{\tableheadbgcolor}\multicolumn{2}{|p{(\linewidth-\tabcolsep*2-\arrayrulewidth*1)*2/2}|}{\cellcolor{\tableheadbgcolor}\textbf{ \mbox{[}!\mbox{[}Nate Fi   }}\\\cline{1-2}
\endfirsthead
\hline
\endfoot
\hline
\rowcolor{\tableheadbgcolor}\multicolumn{2}{|p{(\linewidth-\tabcolsep*2-\arrayrulewidth*1)*2/2}|}{\cellcolor{\tableheadbgcolor}\textbf{ \mbox{[}!\mbox{[}Nate Fi   }}\\\cline{1-2}
\endhead
\href{https://github.com/nfischer}{\tt Nate Fischer}  &\href{http://github.com/freitagbr}{\tt Brandon Freitag}   \\\cline{1-2}
\end{longtabu}
