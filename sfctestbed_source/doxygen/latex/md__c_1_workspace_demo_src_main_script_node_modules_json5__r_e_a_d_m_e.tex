\href{https://travis-ci.org/json5/json5}{\tt }

J\+S\+ON is an excellent data format, but we think it can be better.

{\bfseries J\+S\+O\+N5 is a proposed extension to J\+S\+ON} that aims to make it easier for {\itshape humans to write and maintain} by hand. It does this by adding some minimal syntax features directly from E\+C\+M\+A\+Script 5.

J\+S\+O\+N5 remains a {\bfseries strict subset of Java\+Script}, adds {\bfseries no new data types}, and {\bfseries works with all existing J\+S\+ON content}.

J\+S\+O\+N5 is {\itshape not} an official successor to J\+S\+ON, and J\+S\+O\+N5 content may {\itshape not} work with existing J\+S\+ON parsers. For this reason, J\+S\+O\+N5 files use a new .json5 extension. $\ast$(T\+O\+DO\+: new M\+I\+ME type needed too.)$\ast$

The code here is a {\bfseries reference Java\+Script implementation} for both Node.\+js and all browsers. It’s based directly off of Douglas Crockford’s own \href{https://github.com/douglascrockford/JSON-js/blob/master/json_parse.js}{\tt J\+S\+ON implementation}, and it’s both robust and secure.

\subsection*{Why}

J\+S\+ON isn’t the friendliest to {\itshape write}. Keys need to be quoted, objects and arrays can’t have trailing commas, and comments aren’t allowed — even though none of these are the case with regular Java\+Script today.

That was fine when J\+S\+O\+N’s goal was to be a great data format, but J\+S\+O\+N’s usage has expanded beyond {\itshape machines}. J\+S\+ON is now used for writing \href{http://plovr.com/docs.html}{\tt configs}, \href{https://www.npmjs.org/doc/files/package.json.html}{\tt manifests}, even \href{http://code.google.com/p/fuzztester/wiki/JSONFileFormat}{\tt tests} — all by {\itshape humans}.

There are other formats that are human-\/friendlier, like Y\+A\+ML, but changing from J\+S\+ON to a completely different format is undesirable in many cases. J\+S\+O\+N5’s aim is to remain close to J\+S\+ON and Java\+Script.

\subsection*{Features}

The following is the exact list of additions to J\+S\+O\+N’s syntax introduced by J\+S\+O\+N5. {\bfseries All of these are optional}, and {\bfseries all of these come from E\+S5}.

\subsubsection*{Objects}


\begin{DoxyItemize}
\item Object keys can be unquoted if they’re valid \href{https://developer.mozilla.org/en/Core_JavaScript_1.5_Guide/Core_Language_Features#Variables}{\tt identifiers}. Yes, even reserved keywords (like {\ttfamily default}) are valid unquoted keys in E\+S5 \mbox{[}\href{http://es5.github.com/#x11.1.5}{\tt §11.1.\+5}, \href{http://es5.github.com/#x7.6}{\tt §7.6}\mbox{]}. (\href{https://mathiasbynens.be/notes/javascript-identifiers}{\tt More info})

$\ast$(T\+O\+DO\+: Unicode characters and escape sequences aren’t yet supported in this implementation.)$\ast$
\item Object keys can also be single-\/quoted.
\item Objects can have trailing commas.
\end{DoxyItemize}

\subsubsection*{Arrays}


\begin{DoxyItemize}
\item Arrays can have trailing commas.
\end{DoxyItemize}

\subsubsection*{Strings}


\begin{DoxyItemize}
\item Strings can be single-\/quoted.
\item Strings can be split across multiple lines; just prefix each newline with a backslash. \mbox{[}E\+S5 \href{http://es5.github.com/#x7.8.4}{\tt §7.8.\+4}\mbox{]}
\end{DoxyItemize}

\subsubsection*{Numbers}


\begin{DoxyItemize}
\item Numbers can be hexadecimal (base 16).
\item Numbers can begin or end with a (leading or trailing) decimal point.
\item Numbers can include {\ttfamily Infinity}, {\ttfamily -\/\+Infinity}, {\ttfamily NaN}, and {\ttfamily -\/\+NaN}.
\item Numbers can begin with an explicit plus sign.
\end{DoxyItemize}

\subsubsection*{Comments}


\begin{DoxyItemize}
\item Both inline (single-\/line) and block (multi-\/line) comments are allowed.
\end{DoxyItemize}

\subsection*{Example}

The following is a contrived example, but it illustrates most of the features\+:


\begin{DoxyCode}
\{
    foo: 'bar',
    while: true,

    this: 'is a \(\backslash\)
multi-line string',

    // this is an inline comment
    here: 'is another', // inline comment

    /* this is a block comment
       that continues on another line */

    hex: 0xDEADbeef,
    half: .5,
    delta: +10,
    to: Infinity,   // and beyond!

    finally: 'a trailing comma',
    oh: [
        "we shouldn't forget",
        'arrays can have',
        'trailing commas too',
    ],
\}
\end{DoxyCode}


This implementation’s own \href{package.json5}{\tt package.\+json5} is more realistic\+:


\begin{DoxyCode}
// This file is written in JSON5 syntax, naturally, but npm needs a regular
// JSON file, so compile via `npm run build`. Be sure to keep both in sync!

\{
    name: 'json5',
    version: '0.5.0',
    description: 'JSON for the ES5 era.',
    keywords: ['json', 'es5'],
    author: 'Aseem Kishore <aseem.kishore@gmail.com>',
    contributors: [
        // TODO: Should we remove this section in favor of GitHub's list?
        // https://github.com/aseemk/json5/contributors
        'Max Nanasy <max.nanasy@gmail.com>',
        'Andrew Eisenberg <andrew@eisenberg.as>',
        'Jordan Tucker <jordanbtucker@gmail.com>',
    ],
    main: 'lib/json5.js',
    bin: 'lib/cli.js',
    files: ["lib/"],
    dependencies: \{\},
    devDependencies: \{
        gulp: "^3.9.1",
        'gulp-jshint': "^2.0.0",
        jshint: "^2.9.1",
        'jshint-stylish': "^2.1.0",
        mocha: "^2.4.5"
    \},
    scripts: \{
        build: 'node ./lib/cli.js -c package.json5',
        test: 'mocha --ui exports --reporter spec',
            // TODO: Would it be better to define these in a mocha.opts file?
    \},
    homepage: 'http://json5.org/',
    license: 'MIT',
    repository: \{
        type: 'git',
        url: 'https://github.com/aseemk/json5.git',
    \},
\}
\end{DoxyCode}


\subsection*{Community}

Join the \href{http://groups.google.com/group/json5}{\tt Google Group} if you’re interested in J\+S\+O\+N5 news, updates, and general discussion. Don’t worry, it’s very low-\/traffic.

The \href{https://github.com/aseemk/json5/wiki}{\tt Git\+Hub wiki} is a good place to track J\+S\+O\+N5 support and usage. Contribute freely there!

\href{https://github.com/aseemk/json5/issues}{\tt Git\+Hub Issues} is the place to formally propose feature requests and report bugs. Questions and general feedback are better directed at the Google Group.

\subsection*{Usage}

This Java\+Script implementation of J\+S\+O\+N5 simply provides a {\ttfamily J\+S\+O\+N5} object just like the native E\+S5 {\ttfamily J\+S\+ON} object.

To use from Node\+:


\begin{DoxyCode}
npm install json5
\end{DoxyCode}



\begin{DoxyCode}
var JSON5 = require('json5');
\end{DoxyCode}


To use in the browser (adds the {\ttfamily J\+S\+O\+N5} object to the global namespace)\+:


\begin{DoxyCode}
<script src="json5.js"></script>
\end{DoxyCode}


Then in both cases, you can simply replace native {\ttfamily J\+S\+ON} calls with {\ttfamily J\+S\+O\+N5}\+:


\begin{DoxyCode}
var obj = JSON5.parse('\{unquoted:"key",trailing:"comma",\}');
var str = JSON5.stringify(obj);
\end{DoxyCode}


{\ttfamily J\+S\+O\+N5.\+parse} supports all of the J\+S\+O\+N5 features listed above ({\itshape T\+O\+DO\+: except Unicode}), as well as the native \href{https://developer.mozilla.org/en-US/docs/Web/JavaScript/Reference/Global_Objects/JSON/parse}{\tt {\ttfamily reviver} argument}.

{\ttfamily J\+S\+O\+N5.\+stringify} mainly avoids quoting keys where possible, but we hope to keep expanding it in the future (e.\+g. to also output trailing commas). It supports the native \href{https://developer.mozilla.org/en-US/docs/Web/JavaScript/Reference/Global_Objects/JSON/stringify}{\tt {\ttfamily replacer} and {\ttfamily space} arguments}, as well. $\ast$(T\+O\+DO\+: Any implemented {\ttfamily to\+J\+S\+ON} methods aren’t used today.)$\ast$

\subsubsection*{Extras}

If you’re running this on Node, you can also register a J\+S\+O\+N5 {\ttfamily require()} hook to let you {\ttfamily require()} {\ttfamily .json5} files just like you can {\ttfamily .json} files\+:


\begin{DoxyCode}
require('json5/lib/require');
require('./path/to/foo');   // tries foo.json5 after foo.js, foo.json, etc.
require('./path/to/bar.json5');
\end{DoxyCode}


This module also provides a {\ttfamily json5} executable (requires Node) for converting J\+S\+O\+N5 files to J\+S\+ON\+:


\begin{DoxyCode}
json5 -c path/to/foo.json5    # generates path/to/foo.json
\end{DoxyCode}


\subsection*{Development}


\begin{DoxyCode}
git clone git://github.com/aseemk/json5.git
cd json5
npm install
npm test
\end{DoxyCode}


As the {\ttfamily package.\+json5} file states, be sure to run {\ttfamily npm run build} on changes to {\ttfamily package.\+json5}, since npm requires {\ttfamily package.\+json}.

Feel free to \href{https://github.com/aseemk/json5/issues}{\tt file issues} and submit \href{https://github.com/aseemk/json5/pulls}{\tt pull requests} — contributions are welcome. If you do submit a pull request, please be sure to add or update the tests, and ensure that {\ttfamily npm test} continues to pass.

\subsection*{License}

M\+IT. See ./\+L\+I\+C\+E\+N\+SE.md \char`\"{}\+L\+I\+C\+E\+N\+S\+E.\+md\char`\"{} for details.

\subsection*{Credits}

\href{http://bolinfest.com/}{\tt Michael Bolin} independently arrived at and published some of these same ideas with awesome explanations and detail. Recommended reading\+: \href{http://bolinfest.com/essays/json.html}{\tt Suggested Improvements to J\+S\+ON}

\href{http://www.crockford.com/}{\tt Douglas Crockford} of course designed and built J\+S\+ON, but his state machine diagrams on the \href{http://json.org/}{\tt J\+S\+ON website}, as cheesy as it may sound, gave me motivation and confidence that building a new parser to implement these ideas this was within my reach! This code is also modeled directly off of Doug’s open-\/source \href{https://github.com/douglascrockford/JSON-js/blob/master/json_parse.js}{\tt json\+\_\+parse.\+js} parser. I’m super grateful for that clean and well-\/documented code.

\href{https://github.com/MaxNanasy}{\tt Max Nanasy} has been an early and prolific supporter, contributing multiple patches and ideas. Thanks Max!

\href{https://github.com/aeisenberg}{\tt Andrew Eisenberg} has contributed the {\ttfamily stringify} method.

\href{https://github.com/jordanbtucker}{\tt Jordan Tucker} has aligned J\+S\+O\+N5 more closely with E\+S5 and is actively maintaining this project. 