Tiny module wrapping try/catch in Java\+Script.

It\textquotesingle{}s {\itshape literally 11 lines of code}, \href{tryit.js}{\tt just read it} that\textquotesingle{}s all the documentation you\textquotesingle{}ll need.

\subsection*{install}


\begin{DoxyCode}
npm install tryit
\end{DoxyCode}


\subsection*{usage}

What you\textquotesingle{}d normally do\+: 
\begin{DoxyCode}
try \{
    dangerousThing();
\} catch (e) \{
    console.log('something');
\}
\end{DoxyCode}


With try-\/it (all it does is wrap try-\/catch) 
\begin{DoxyCode}
var tryit = require('tryit');

tryit(dangerousThing);
\end{DoxyCode}


You can also handle the error by passing a second function 
\begin{DoxyCode}
tryit(dangerousThing, function (e) \{
    if (e) \{
        console.log('do something');
    \}
\})
\end{DoxyCode}


The second function follows error-\/first pattern common in node. So if you pass a callback it gets called in both cases. But will have an error as the first argument if it fails.

\subsection*{W\+H\+AT? W\+HY DO T\+H\+I\+S!?}

Primary motivation is having a clean way to wrap things that might fail, where I don\textquotesingle{}t care if it fails. I just want to try it.

This includes stuff like blindly reading/parsing stuff from local\+Storage in the browser. If it\textquotesingle{}s not there or if parsing it fails, that\textquotesingle{}s fine. But I don\textquotesingle{}t want to leave a bunch of empty {\ttfamily catch (e) \{\}} blocks in the code.

Obviously, this is useful any time you\textquotesingle{}re going to attempt to read some unknown data structure.

In addition, my understanding is that it\textquotesingle{}s hard for JS engines to optimize code in try blocks. By actually passing the code to be executed into a re-\/used try block, we can avoid having to have more than a single try block in our app. Again, this is not a primary motivation, just a potential side benefit.

\subsection*{license}

\href{http://mit.joreteg.com/}{\tt M\+IT} 