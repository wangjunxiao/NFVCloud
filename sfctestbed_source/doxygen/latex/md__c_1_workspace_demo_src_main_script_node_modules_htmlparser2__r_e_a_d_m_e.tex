\#htmlparser2 \href{https://npmjs.org/package/htmlparser2}{\tt } \href{http://travis-ci.org/fb55/htmlparser2}{\tt } \href{https://david-dm.org/fb55/htmlparser2}{\tt }

A forgiving H\+T\+M\+L/\+X\+M\+L/\+R\+SS parser written in JS for Node\+JS. The parser can handle streams (chunked data) and supports custom handlers for writing custom D\+O\+Ms/output.

\subsection*{Installing}

npm install htmlparser2

A live demo of htmlparser2 is available at \href{http://demos.forbeslindesay.co.uk/htmlparser2/}{\tt http\+://demos.\+forbeslindesay.\+co.\+uk/htmlparser2/}

\subsection*{Usage}


\begin{DoxyCode}
var htmlparser = require("htmlparser2");
var parser = new htmlparser.Parser(\{
    onopentag: function(name, attribs)\{
        if(name === "script" && attribs.type === "text/javascript")\{
            console.log("JS! Hooray!");
        \}
    \},
    ontext: function(text)\{
        console.log("-->", text);
    \},
    onclosetag: function(tagname)\{
        if(tagname === "script")\{
            console.log("That's it?!");
        \}
    \}
\});
parser.write("Xyz <script type='text/javascript'>var foo = '<<bar>>';</ script>");
parser.end();
\end{DoxyCode}


Output (simplified)\+:


\begin{DoxyCode}
--> Xyz 
JS! Hooray!
--> var foo = '<<bar>>';
That's it?!
\end{DoxyCode}


Read more about the parser in the \href{https://github.com/fb55/htmlparser2/wiki/Parser-options}{\tt wiki}.

\subsection*{Get a D\+OM}

The {\ttfamily Dom\+Handler} (known as {\ttfamily Default\+Handler} in the original {\ttfamily htmlparser} module) produces a D\+OM (document object model) that can be manipulated using the \href{https://github.com/fb55/DomUtils}{\tt {\ttfamily Dom\+Utils}} helper.

The {\ttfamily Dom\+Handler}, while still bundled with this module, was moved to its \href{https://github.com/fb55/domhandler}{\tt own module}. Have a look at it for further information.

\subsection*{Parsing R\+S\+S/\+R\+D\+F/\+Atom Feeds}


\begin{DoxyCode}
new htmlparser.FeedHandler(function(<error> error, <object> feed)\{
    ...
\});
\end{DoxyCode}


\subsection*{Performance}

After having some artificial benchmarks for some time, {\bfseries } published his \href{https://github.com/AndreasMadsen/htmlparser-benchmark}{\tt {\ttfamily htmlparser-\/benchmark}}, which benchmarks H\+T\+ML parses based on real-\/world websites.

At the time of writing, the latest versions of all supported parsers show the following performance characteristics on \href{https://travis-ci.org/AndreasMadsen/htmlparser-benchmark/builds/10805007}{\tt Travis CI} (please note that Travis doesn\textquotesingle{}t guarantee equal conditions for all tests)\+:


\begin{DoxyCode}
gumbo-parser   : 34.9208 ms/file ± 21.4238
html-parser    : 24.8224 ms/file ± 15.8703
html5          : 419.597 ms/file ± 264.265
htmlparser     : 60.0722 ms/file ± 384.844
htmlparser2-dom: 12.0749 ms/file ± 6.49474
htmlparser2    : 7.49130 ms/file ± 5.74368
hubbub         : 30.4980 ms/file ± 16.4682
libxmljs       : 14.1338 ms/file ± 18.6541
parse5         : 22.0439 ms/file ± 15.3743
sax            : 49.6513 ms/file ± 26.6032
\end{DoxyCode}


\subsection*{How is this different from \href{https://github.com/tautologistics/node-htmlparser}{\tt node-\/htmlparser}?}

This is a fork of the {\ttfamily htmlparser} module. The main difference is that this is intended to be used only with node (it runs on other platforms using \href{https://github.com/substack/node-browserify}{\tt browserify}). {\ttfamily htmlparser2} was rewritten multiple times and, while it maintains an A\+PI that\textquotesingle{}s compatible with {\ttfamily htmlparser} in most cases, the projects don\textquotesingle{}t share any code anymore.

The parser now provides a callback interface close to \href{https://github.com/isaacs/sax-js}{\tt sax.\+js} (originally targeted at \href{https://github.com/fb55/readabilitysax}{\tt readability\+S\+AX}). As a result, old handlers won\textquotesingle{}t work anymore.

The {\ttfamily Default\+Handler} and the {\ttfamily Rss\+Handler} were renamed to clarify their purpose (to {\ttfamily Dom\+Handler} and {\ttfamily Feed\+Handler}). The old names are still available when requiring {\ttfamily htmlparser2}, so your code should work as expected. 